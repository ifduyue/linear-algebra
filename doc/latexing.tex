\documentclass[titlepage]{article}
% 1st_readme.tex
%  Instructions for Jim Hefferon's Linear Algebra book.
\usepackage[margin=1in]{geometry}
\usepackage{verbatim}
\usepackage{amsmath}
\usepackage{graphics}
%\usepackage{linalgjh,graphics}

\usepackage{url}
% adjust LaTeX's default sectioning commands
\makeatletter
\renewcommand{\section}{\@startsection{section}%
  {1}%
  {0em}%
  {5ex plus1ex minus.5ex}%
  {1ex plus .25ex minus .15ex}%
  {\large\bfseries\raggedright}}
\renewcommand{\subsection}{\@startsection{subsection}%
  {2}%
  {0em}%
  {3ex plus.25ex minus.25ex}%
  {.5ex plus .05ex minus .05ex}%
  {\bfseries\raggedright}}
\renewcommand{\subsubsection}{\@startsection{subsubsection}%
  {3}%
  {0em}%
  {1.75ex plus.25ex minus.25ex}%
  {.5ex plus .05ex}%
  {\bfseries\raggedright}}
\makeatother

\newcommand{\latexnote}{\textit{\LaTeX\ note.} }

\newif\iffancytitle\fancytitletrue % change to ..true for fancy title page
\iffancytitle
  \usepackage{color}
    \definecolor{titlegray}{rgb}{0.87,0.87,0.87}
  \DeclareFontFamily{OT1}{bigandbold}{}
  \DeclareFontShape{OT1}{bigandbold}{m}{n}{<-> cmb10 }{}
  \usepackage{pbsi}
\else
  \title{\LaTeX-ing \textit{Linear Algebra}}
  \author{Jim Hef{}feron}
  % Date used in both book and answers for current version
2021-Oct-12
\fi

\begin{document}
\iffancytitle
  \thispagestyle{empty}
  \vspace*{3in}
  \begin{center}
    \begin{picture}(0,0)(0,0)
      \put(0,40){\makebox[0pt]{\color{titlegray}\fontsize{100}{130}\usefont{OT1}{bigandbold}{m}{n}\selectfont\LaTeX-ing}}
      \put(68,0){\makebox[0pt]{\Huge\bsifamily Linear Algebra}}  
      \put(0,-350){\makebox[0pt]{\large\scshape\begin{tabular}{c} Jim Hef{}feron \\[0.75ex] % Date used in both book and answers for current version
2021-Oct-12\end{tabular}}}
    \end{picture}
  \end{center}
  \clearpage\setcounter{page}{1}
\else
  \maketitle
\fi

This document contains instructions for using the macros from 
\textit{Linear Algebra}, 
an undergraduate text by
to Jim Hef{}feron.




\section{Summary of \textit{Linear Algebra}.}
\textit{Linear Algebra}
covers the material that is usually done in a three or four
hour a week, one semester US course:
solving linear systems,
vector spaces, linear maps, determinants, and eigenvalues and eigenvectors.

The pedagogical approach is to help a student make the
transition from Calculus to upper division mathematics.
The presentation stresses careful motivation,
many examples, clarity in the proofs, and 
developmental exercise sets.
See the Preface for more detail.

To have a look,
go to \url{https://hefferon.net/linearalgebra}.
Get both the book and the answers, and put them in the same directory.
That way,  
clicking on a question number takes you to its answer 
and clicking on the answer number
takes you back to the question.






\subsection{It is Free}
These materials are freely available, in both output and source form.

First, you are free to download the book.
If you are an
instructor then you are also free to have students in your class 
use paper or electronic copies. 
This includes the freedom to print out copies and sell them
from your bookstore.
But note that hard copies are for sale from major distributers; 
see \url{https://hefferon.net/linearalgebra}.

Besides being free to use the output form of the text, you are
also free to use the source.
For example, 
an instructor can add a few favorite exercises 
here and there, perhaps include a new Topic, adjust the development
in a subsection, or even rewrite a portion entirely.
For this, some people may want to modify the source.



\section{Use the Source}
If you want to modify the materials then 
you need to know that the files here
use the \LaTeX\ macro system for the text formatter \TeX. 
It is the 
standard system for writing mathematics.

So you need a
distribution of the software;
I recommend \TeX{}Live, \texttt{http://www.tug.org/texlive}. 

If you are unfamiliar with \LaTeX{} then
this might not be the right job on which to learn.
Just saying.
(If you are just starting working with \LaTeX{} then I recommend finding the 
\textit{Short Guide to \LaTeX2e} at
\url{http://www.ctan.org/}, which might be enough
for you to get through making small changes in the materials.)

The graphics here are in MetaPost and Asymptote.
These should come with a \TeX{} distribution.




\subsection{Compiling from source}

For compilation instructions see the \verb!INSTALL! file.
Basically, there is a \texttt{Makefile}, but in particular 
you cannot just run \texttt{latex book}.
If you hit a snag, I'm sorry, I am unlikely to be able to help.

The book's source comes in a number of separate files, all of 
which live below the \texttt{src} directory.
The first chapter is in the subdirectory \texttt{src/gr}, and the files
are called 
\texttt{gr1.tex}, \texttt{gr2.tex}, etc.
The topics have names that do not follow any particular pattern.
The second chapter is in \texttt{src/vs}, 
the third chapter is in \texttt{src/map},
the fourth chapter is in \texttt{src/det}, 
and the fifth chapter is in
\texttt{src/jc}.
Besides those, here are some more.
\begin{center}
  \begin{tabular}{r|l}
    \textit{file name}  &\textit{what it is}  \\
    \hline
    \texttt{src/book.tex}     &Main file \\
    \texttt{sty/bookjhconcrete.sty}   &Formatting instructions  \\
    \texttt{src/sty/linalgjh.sty} &Macros for Linear Algebra \\
    \texttt{src/sty/jhanswer.tex} &Main file for the answers \\
    \texttt{src/cover/cover.tex} &Make the front, back, and spine \\
  \end{tabular}
\end{center}





\subsection{Slides}

I have some slides for a classroom projector.
These are keyed to the text in a couple of ways.

First, they reference numbers for theorems, lemmas, etc.
These references are generated automatically, meaning that I did not input
them by hand, so in theory they are accurate.
Second, I have changed almost all the examples from the text.
This way students see twice as many.
Third, the system is set so that the source of the statements of
definitions, theorems, etc., is the text.
Thus there is guaranteed, in theory, to be no discrepency with the 
text. 

These are generated by the Makefile.
One caution is that they use Asymptote for graphics, which can 
give you trouble involving the Ghostscript version that you have.
See \texttt{slides/README.opacity}.






\section{Guide to the Macros}
You can use the same
macros to write your materials as are used in the text.
For example, when you make up an exam and you want to
refer to the vector space of polynomials of degree three, 
use \verb!$\polyspace_3$!,
avoiding students' confusion and also simplifying your day.  
This section describes how to use those macros.


\subsection{Linear algebra macros}
The macros that seemed to me to be specific to typesetting the
linear algebra material are in the file 
\texttt{src/sty/linalgjh.sty}.
For instance, here are macros for displaying systems of linear equations,
matrices, row and column vectors, special vector spaces, linear maps,
etc.

Note that the book uses the \textit{amsmath} package, 
including the commutative diagram extension \textit{amscd},
so all of those
wonderful macros are available.
In particular, displayed equations are shown with 
\verb|equation*| (the \texttt{*} makes them not-numbered) and all of the 
alignment structures like \verb|align| and matrix structures 
are available.

Another style that the book loads is \textit{mathrsfs}, for the 
Ralph Smith Formal Script fonts to make, for instance, the script R
for the range space.


\begin{description}
\item[linsys] 
  Make a linear system, in such a way that the columns line up.    
  Here is an example of a three-unknowns system; you want to
  do this only as a displayed equation.
\begin{verbatim}
  \begin{equation*}
    \begin{linsys}{3}
       x &+ &3y &+ &a &= &7 \\
       x &- &3y &+ &a &= &7 
    \end{linsys}
  \end{equation*}
\end{verbatim}
  If a row is without some of the variables, be sure to nonetheless
  add the appropriate \&'s.
\begin{verbatim}
  \begin{equation*}
    \begin{linsys}{3}
       x &+ &3y &+ &a &= &7 \\
         &  &3y &  &  &= &7 
    \end{linsys}
  \end{equation*}
\end{verbatim}
  In the special case that a row starts with a negative sign, do not 
  use \&$-$.
  That is, do not start the second line below with \verb! &- &3y!.
\begin{verbatim}
  \begin{equation*}
    \begin{linsys}{3}
       x &+ &3y &+ &a &= &7 \\
         &  &-3y&+ &a &= &7 
    \end{linsys}
  \end{equation*}
\end{verbatim}

  \textit{Remarks.}
  \begin{enumerate}
    \item 
      In the exercises I might have three or four linear systems on 
      a horizontal line and to get them to line up (to be \texttt{t}-aligned)
      I included an optional argument governing that vertical alignment.
    \item 
      \latexnote
      The variables are put in the columns right-justified, while the  
      additions or subtractions are centered.
      Between the columns is put $4/18$-ths of an em (\TeX book, p.~167--170;
      that's a medmuskip).
    \item
      \latexnote 
      The argument saying how many variables (which in the examples above
      is the $3$)
      is there instead of some quite large number because it isn't
      as simple as $\text{\textit{arg}}+1$ times \texttt{rc}.
      But I wasn't sure, and might be convinced to change this by someone
      who knows what they are doing here.
  \end{enumerate}

\item[grstep]
  Denote a step of Gauss's Method.
\begin{verbatim}
  \begin{equation*}
    \begin{linsys}{2}
      x  &+  &y  &=  &0  \\
      x  &-  &y  &=  &1
    \end{linsys}
    \grstep{-\rho_1+\rho_2}
    \begin{linsys}{2}
      x  &+  &y  &=  &0  \\
         &   &-2y&=  &1
    \end{linsys}
  \end{equation*}
\end{verbatim}
  Show more than one row operation at a time with
  \verb|\grstep[2\rho_5]{\rho_1+\rho_3}| for two row operations,
  or \verb|\grstep[2\rho_5 \\ 3\rho_6]{\rho_1+\rho_3}| for three.
  Swap two rows with \verb|\rho_1\swap\rho_2|.
  
\item[matrix structures]
Generic matrices are made with \verb|mat|. 
\begin{verbatim}
  \begin{equation*}
    \begin{mat}
      1  &2  &3  \\
      4  &5  &6 
    \end{mat}
  \end{equation*}
\end{verbatim}
  Note that there is no need to specify the number of columns;
  see the \textit{amsmath} documentation.

  There is an optional argument to make the columns right-aligned.
\begin{verbatim}
  \begin{equation*}
    \begin{mat}[r]
      1  &2  &13  \\
      4  &5  &6 
    \end{mat}
  \end{equation*}
\end{verbatim}
I restrict its use to those matrices (and column vectors) that contain
only numbers, no variables at all.

  There are places where I needed something a little different.
  In particular, I needed augmented matrices:
\begin{verbatim}
  \begin{equation*}
    \begin{amat}{2}
      1  &2  &3  \\
      4  &5  &6 
    \end{amat}
  \end{equation*}
\end{verbatim}
  produces a matrix that is two-by-three, with a vertical bar between
  the final and next-to-final columns. 
  The argument \texttt{2} means that there
  are two columns before the vertical bar.
  (This also has an optional argument to right-align.)
  I also sometime need matrices partitioned into columns:
\begin{verbatim}
  \begin{equation*}
    \begin{pmat}{c|c|c}
      1  &2  &3  \\
      4  &5  &6 
    \end{pmat}
  \end{equation*}
\end{verbatim}
  produces a two-by-three matrix that has vertical bars separating the
  columns.

  I make displayed determinant arrays with \verb!vmat! 
\begin{verbatim}
  \begin{equation*}
    \begin{vmat}
      a  &c  \\
      b  &d   
    \end{vmat}
    =ad-bc
  \end{equation*}
\end{verbatim}
  and in-line  determinants with \verb!\deter{A}!.
  (Again, there is an optional argument to right-align.)

\item[vectors]
  Make column vectors with \verb|\colvec{1 \\ 2 \\ 3}|.
  (There is an optonal argument to right-align that I use 
  for number-only vectors \verb|\colvec[r]{1 \\ 2 \\ 3}|)
  Make row vectors with \verb|\rowvec{1  &2  &3}|.

\item[decimal point-aligned columns]
  I use the \textit{dcolumn} package:
\begin{verbatim}
  \begin{equation*}
    \begin{aligncolondecimal}{3}
      15.12  &0.345    
    \end{aligncolondecimal}
  \end{equation*}
\end{verbatim}
  makes a column vector aligned on the decimal with room for at most 
  three decimal places on the right.


\item[strings]
  A digit $3$ in a square is \verb!\digitinsq{3}!.
  Not surprisingly, a digit $3$ in a circle is \verb!\digitincirc{3}!.
  A sequence of strings is shown this way.
\begin{verbatim}
  \begin{equation*}
    \begin{strings}{ccccc}
           \vec{e}_1 &\mapsto &\vec{e}_2 &\mapsto &\zero \\
           \vec{e}_3 &\mapsto &\zero
    \end{strings}
  \end{equation*}
\end{verbatim}
  There are five \verb!c!'s because the longest line
  is of length five.

\item[names for things]
  Note that there is a page (inside the book's cover) 
  covering the notation conventions.

  I tried to remember to make up macros to name things, rather than retype the
  thing each time I ran across it.
  Here is a list.
  \begin{enumerate}
    \item The reals \verb|\Re|, the rationals \verb|\Q|,
      the complex numbers \verb|\C|, the integers \verb|\Z|,
      and the natural numbers \verb|\N| come out in the traditional
      ``blackboard bold''.
    \item A vector is \verb|\vec{v}_j|.
      The zero vector is \verb|\zero|.
      The length of a vector is \verb|\norm{\vec{v}}|, and the 
      absolute value of a number is \verb|\absval{r}|.
      An angle can be expressed in degrees as \verb!$53\degs$!.
      The distance between two vectors is \verb|\dist (\vec{v}_1,\vec{v}_2)|.
    \item The dot product of two vectors \verb|\vec{v}\dotprod\vec{w}|.
      Please note that (as it is set up coming from me) 
      this is different than a \verb!\cdot!.
    \item The vector space of degree $n$ polynomials \verb|\polyspace_n|,
      and the vector space of $n$-by-$m$ matrices 
      \verb|\matspace_{\nbym{n}{m}}|.
      The vector space of linear maps from $V$ to $W$
      \verb|\linmaps{V}{W}|. 
    \item The span of a set $S$ of vectors
      \verb|\spanof{S}|.
    \item The row space of a matrix is \verb!\rowspace{M}! and
      the columnspace is \verb!\colspace{M}!.
    \item A set is \verb!\set{\colvec{x \\ y}\suchthat 2x+y=0}!.
      The union and intersection of the sets $S$ and $T$ is
      \verb|S\union T| and \verb|S\intersection T|.
      The complement of a set is \verb!\complement{S}!.
    \item The empty set is \verb!\emptyset!.
    \item A sequence, such as a string, is \verb!\sequence{s_0,s_1,\dots,s_n}!.
      The concatenation of two sequences is \verb!\cat{B_1}{B_2}!.
    \item A basis is \verb!\basis{\vec{\beta}_1,\dots,\vec{\beta}_n}!.
      The standard basis for real $n$-space is
      \verb|\stdbasis_n|. 
    \item Isomorphic spaces is \verb!V\isomorphicto W!.
    \item The matrix representing a linear map $h$ with respect to 
      the bases $B$ and $D$ is \verb|\rep{h}{B,D}|.
    \item The size of a general matrix is \verb|\nbym{n}{m}| 
      while the special case of a square matrix is \verb|\nbyn{n}|.
    \item A map $h$'s range space \verb|\rangespace{h}| 
      and null space \verb|\nullspace{h}|, 
      and generalized range space \verb|\genrangespace{h}|
      and generalized null space \verb|\gennullspace{h}|
    \item The direct sum of two subspaces \verb|V\directsum W|.
    \item The function, as specified by its domain and codomain,
      is described by \verb!\map{f}{D}{C}!.
      Its action on a single element $x$ is 
      \verb|x\mapsunder{f} f(x)|.
      Two maps are composed with \verb!\composed{g}{f}!.
      The identity map is \verb|\identity|.       
    \item The projection of a vector into a subspace
      \verb|\proj{\vec{v}}{S}|. 
    \item The restriction of a map to some subdomain is
      \verb!\restrictionmap{f}{S}!.
    \item The rank of a matrix \verb|\rank (A)| and the nullity  of a
      matrix \verb|\nullity (A)|. 
      The transpose of a matrix \verb|\trans{A}|.
      The trace of a matrix \verb|\trace (A)|.
      The adjoint of a matrix is \verb|\adj (A)|.
    \item The size of a box is \verb!\size (B)!.
    \item The signum of a permutation $\phi$ is \verb|\sgn (\phi)|.
    \item For the Topic on voting, voter preferences are indicated
      by \verb|\votepreflist{7}{1}{5}| for the column vector,
      and \verb|\voteprefloop{1}{2}{3}| for the circle (for this one,
      the first argument appears at the ten o'clock position, the second
      argument at six o'clock, and the third argument at two o'clock).
    \item A generic field is \verb!\F!.
  \end{enumerate}


\item[aligned vdots]
  To make \verb!\vdots! come out inside a bunch of aligned equations,
  I use \verb!\vdotswithin{}! from the \textit{mathtools} package.
\begin{verbatim}
  \begin{align*}
    a_{1,1}x_1+\cdots+a_{1,n}x_n &= d_1         \\
                                 &\vdotswithin{=} \\
    a_{m,1}x_1+\cdots+a_{m,n}x_n &= d_m         
  \end{align*}
\end{verbatim}
  Otherwise the three vertical dots are not centered on the equals sign.


\end{description}





\subsection{Book layout macros}
The macros that do chapter and section headings, or cross references,
or how the exercises are laid out, are in the file 
\texttt{bookjh.sty}.


\begin{description}
  \item[theorem-like structures]
    I have already declared the
    \texttt{theorem}, \texttt{lemma}, 
    \texttt{definition}, and \texttt{corollary}
    environments.
    These will come out shaded.
    To change this, or the color of the shading or its border, see
    also \textit{bookjh.sty}
\begin{verbatim}
  \begin{definition}
    A \definend{big} vector space is one where you can't  see the end when 
    you are standing at the zero vector, even with your glasses on. 
  \end{definition}
\end{verbatim}
    I have also declared (not shaded):~ \texttt{example} and 
    \texttt{remark}.

    By the way, note the \verb!\definend!; I use this for the
    term being defined (I though I would automate indexing,
    but I've not got around to that).

  \item[proofs]
    Use this.
\begin{verbatim}
  \begin{proof}
     Because I said so; that's why.
  \end{proof}
\end{verbatim}
   The end-of-proof symbol can be changed as
   \verb!\renewcommand{\qedsymbol}{$\box$}!.
   I avoid ever having a proof end with a displayed equation
   so the question of
   where to put the end-of-proof symbol in that case never arises.

  \item[footnotes]
    Footnotes come out on the same page as the reference.
    I only use footnotes 
    to suggest that a reader can look for more in the Appendix
    \verb!\appendrefs{codomains}!.

  \item[graphics]
    I am using MetaPost for all of the graphics and all of the 
    output drivers that I am interested in handle \textit{.mp}'s straight off. 
    Instead I say this.
\begin{verbatim}
  \begin{center}
     \includegraphics{gr2.7}
  \end{center}
\end{verbatim}
   Putting two graphics side-by-side, or putting parallel text, are
   all a bit messy.
   I use \verb!vcenteredhbox{}!.
   Just mimic what I did somewhere.

  \item[exercises and answers]
    The exercise portions of a subsection looks like you might expect.
\begin{verbatim}
 \begin{exercises}
   \item 
     First Exercise
     \begin{answer}
       Answer to the first exercise.
     \end{answer}
   \item 
     Second Exercise
     \begin{answer}
       Answer to the second.
     \end{answer}
   \recommended \item 
     Third Exercise, recommended.
     \begin{answer}
        Answer to the third.
     \end{answer}
 \end{exercises}
\end{verbatim}
    Exercises are, of course, numbered automatically, 
    in the same sequence as the
    other numbered parts of the text.

    My goal is to have all exercises have answers, even the proof exercises.
    Of course, the answers are not written directly to the text;
    they are written to one or two separate files.
    Use \textit{jhanswer.tex} to print them.
  
    \textit{Remarks.}
    \begin{enumerate}
      %\item If you dislike having the exercises numbered as though they form
      %  an integral part of the material of the subsection (even though
      %  they \emph{do} form an integral part), comment out the line
      %  in the definition of the environment \verb|exerciselist| that says
      %  \verb|\setcounter{exercisecounter}{#1}|.
      %  You also probably want to adjust the definitions of 
      %  \verb|\exercise@makelabel| and \verb|\exercise@deflabel|
      %  (these are defined three times, once for default, once in the 
      %  definition of a \verb|topic|, and once in the definition of a
      %  \verb|chapter|).
      \item 
        Some of the exercises have parts.
        I use the environment \verb|exparts|. 
        The items in such an
        environment are \verb|\partsitem|'s.
\begin{verbatim}
\recommended\item Do this one right away.
  \begin{exparts}
    \partsitem Do this first.
    \partsitem Do this second.
  \end{exparts}
  \begin{answer}
    \begin{exparts}
      \partsitem Answer to the first.
      \partsitem Answer to the second.
    \end{exparts}
  \end{answer}
\end{verbatim}
        I sometimes use \verb|exparts*|, which leaves the items in a 
        horizontal list, but it doesn't work very well (it sometimes
        causes \TeX's \texttt{Underfull hbox} errors; this is not
        as easy to fix as using a paralist because I don't want the
        beginning of a part on one line and the end on another).
      \item
        Any desired changes to the spacing, etc., of exercises is 
        in \verb|exerciselist|.
        See also the style for the answers in \textit{bookans.sty}.
      \item
        You can redefine \verb|\recommendationmark| to change 
        the mark used to denote recommended exercises.
        You can also mark puzzles with \verb!\puzzle\item!.
        This can be combined with \verb!\recommended!.
      \item If you are putting in an exercise without an answer, your
        best bet is to put an answer environment that is blank.
        But really, you shouldn't do that. 
        I have found that doing the answers to all of the exercises
        improved both the questions and the presentation greatly,
        since I often found that I needed a detail here, or
        some tweak there.  
        (If what you want is that students can't read the answer in their
        answer list, that is a different matter.
        See the next two remarks.)
      \item
         These answers are typically quite detailed, giving the work
         and not just the answer; I approached the answers 
         expecting that the reader wants to learn
         and did not take the
         approach that the reader is trying to cheat.
         (\textit{Remark}.
         As a practicing teacher, I am very aware that it is not a simple
         issue.
         Nonetheless, in my opinion, a reader is 
         entitled to enough exercise and answer pairs that,
         having tried a good selection and not just read the questions 
         and peeked at the answers,
         they can move on well-prepared to the next subsection.
         I figure that a dozen such pairs in each subsection is
         about right.
         Then, to provide a selection and also to allow someone 
         a second crack at a problem type that gave them trouble the 
         first time, I aimed for two dozen
         questions in each subsection.
         If this question/answer situation bothers you, 
         I would very much welcome additional submissions of 
         exercise sets for which students do not have access
         to the answers.)  
     \item 
         I used to have it set up so that 
         you can have the answers written to either 
         a single file,
         or separate files for the answers to recommended exercises and
         the answers to the others.
         Although I no longer do that, I haven't deleted the code so maybe 
         you could get it to work.
         (In this Internet age, trying to control access to such files
         quickly proved to be folly.
         For a while I only released the entire answers to people who emailed
         me with a claim to being an instructor, but soon the silliness of that
         became too much to even hold a pretense.
         I'll just say that every twenty year old
         on the planet can come up in minutes 
         with a download address for the answers to exercises for any
         popular math text.
         It is just a fact.)
      \item
         \latexnote
         You can use \verb|\ref|'s inside answers, even though they
         are written to other files.
         The style for the answers reads in the reference files.
      \item
         An answer that is taken from a cited source may not blend 
         stylistically with the others (insofar as I have a style).
         I use the disclaimer \verb|\answerasgiven|.
    \end{enumerate}


  \item[tfae]
    A small environment for The Following are Equivalent-type lists.
 
  \item[clearemptydoublepage]
    Taken straight from the \textit{Companion}.
    It does what it says.

  \item[cross references]
    I use \verb!\nearbydefinition{def:BigThm}!-like references.
    This has the disadvantage that the referrer has to know what type
    of textual element it is referring to (in this case a definition),
    but \LaTeX\ doesn't give you a way to know in this case (that I can see).
    Note that I try to make the \verb!\label{def:BigThm}! moderately
    descriptive.

    I have set up nearby cross references (meaning there is no need
    to reference the chapter number and the section number) for:
    \verb!\nearbydefinition!, 
    \verb!\nearbyfigure!, 
    \verb!\nearbylemma!, 
    \verb!\nearbyexample!, 
    \verb!\nearbycounterexample!, 
    \verb!\nearbytheorem!, 
    \verb!\nearbycorollary!, 
    \verb!\nearbyexercise!, 
    \verb!\nearbyremark!, 
    \verb!\nearbynotice!, and 
    \verb!\nearbynote!.
    (I don't use most of these, but they are there anyway.)


  \item[topics]
   Begin a new
   topic with \verb|\topic{My Favorite Topic}|.
   The exercises come out looking a little differently automatically, 
   because I couldn't think of a common treatment that was both 
   convienent and consistent.

  \item[optional subsections]
    A subsection is marked as optional with \verb|\subsectionoptional|
    instead of \LaTeX's standard \verb|\subsection|.
    This asterisk's the name in the Table of Contents.
    I have also followed the practice of beginning each optional 
    subsection with a
    italicised comment that it is optional, but this is not part of the macro.


  \item[page headings]
    This work is all done through the wonderful \textit{fancyhdr} style.
    See the documentation for that package.

  \item[hyperlink style]
    I have set the links to come out in blue.
    Change that by altering the 
\begin{verbatim}
  \usepackage{
              ..,
              linkcolor=blue,
              ..}{hyperref}
\end{verbatim}
    entry.


\end{description}



\section{Making Your Exams}
I have included an exam style that does what you'd think it would do,
number the exercises automatically, automatically put headers and page
numbers, and optionally allow exercises to have attached answers that are
printed to a separate file.  

Here is a sample exam, without answers.
See the web page for more extensive examples.

\begin{verbatim}
\begin{exam}
  \item \pts{15}
    How much wood could a woodchuck chuck?
    \begin{enumerate}
      \item If he would chuck wood?
      \item If not?
    \end{enumerate}
  \item \pts{85}
    Justify the Axiom of Choice, since we all know it is true anyway.
\end{exam}
\end{verbatim}


\section{Bugs}

See the \verb!TODO.org! file in the source.

% These are known to me; I'd appreciate any help, of course.
% \begin{description}
%   \item[2001-Oct-08]
%     I can't figure out how to get \textit{hyperref.sty} to 
%     have the indexing links both work and show the correct page number
%     (they are advanced by ten, which I presume is length of the front matter).
%     I've disabled the indexing links with \verb!hyperindex=false!. 
% \end{description}




% \section{Extra: Installing the Brushscript Fonts}

% These are the steps that I followed to install the brushscript fonts
% on a fresh Debian Linux file system.
% Note that TeX~Live now ships with these installed.

% I first downloaded the source files from \url{http://www.ctan.org}.
% To do that, I clicked on \verb!search!, typed \textit{brushscript}
% in the box, and elected to download the entire directory as a .zip file.

% Then, with the source files on my machine,
% as the superuser, I first made \verb!/usr/local/share/texmf! for my
% non-te\TeX{} files.
% Having a local tree ensures that I won't overwrite the material if 
% I install a new standard tree.

% Then I followed these steps.
% \begin{enumerate}
% \item
%   \verb!mkdir /usr/local/share/texmf/fonts! 
%   just as in \verb!/usr/share/texmf!  
% \item
%   \verb!mkdir /usr/local/share/texmf/fonts/source!
% \item 
%   \verb!mkdir /usr/local/share/texmf/fonts/source/public! 
% \item
%   \verb!cp brushscr.zip /usr/local/share/texmf/fonts/source/public! 
%   to put it in place  
% \item
%   \verb!cd /usr/local/share/texmf/fonts/source/public! 
% \item
%   \verb!unzip brushscr!
% \item
%   \verb!cd brushscr!
% \item
%   \verb!make! as the brushscripts directions in the directory say
% \item
%   \verb!mkdir /usr/local/share/texmf/fonts/tfm! for \TeX{} font metrics
% \item
%   \verb!mkdir /usr/local/share/texmf/fonts/tfm/public!
% \item
%   \verb!mkdir /usr/local/share/texmf/fonts/tfm/public/brushscr!
% \item
%   \verb!cp *.tfm /usr/local/share/texmf/fonts/tfm/public/brushscr!
% \item
%   \verb!mkdir /usr/local/share/texmf/fonts/vf! for virtual fonts
% \item
%   \verb!mkdir /usr/local/share/texmf/fonts/vf/public!
% \item
%   \verb!mkdir /usr/local/share/texmf/fonts/vf/public/brushscr!
% \item
%   \verb!cp *.vf /usr/local/share/texmf/fonts/vf/public/brushscr!
% \item
%   \verb!mkdir /usr/local/share/texmf/tex/latex/brushscr! \LaTeX{} font definition files
% \item
%   \verb!cp *.fd /usr/local/share/texmf/tex/latex/brushscr!
% \item
%   \verb!cp *.sty /usr/local/share/texmf/tex/latex/brushscr! and \LaTeX{} styles
% \item
%   \verb!mkdir /usr/local/share/texmf/fonts/type1! Postscript outline fonts
% \item
%   \verb!mkdir /usr/local/share/texmf/fonts/type1/brushscr!
% \item
%   \verb!cp *.pfa /usr/local/share/texmf/fonts/type1/brushscr! 
%   ASCII-encoded Postscript fonts
% \item
%   \verb!mkdir /usr/local/share/texmf/dvips! tell the dvi drivers
% \item
%   \verb!mkdir /usr/local/share/texmf/dvips/config!
% \item
%   \verb!cp /usr/share/texmf/dvips/config/config.ps /usr/local/share/texmf/dvips/config/config.ps!
% \item
%   \verb!cp pbsi.map /usr/local/share/texmf/dvips/config!
% \item
%   \verb!emacs /usr/local/share/texmf/dvips/config/config.ps! edit the config file
% \item
%   I then found the lines that say, 
%   ``This shows how to add your own map file. 
%   Remove the comment and adjust the name:'' 
%   and the next line says \verb!p +myfonts.map! 
%   (but it is commented out with a percent sign). 
%   I added a line just below that to say (without the percent sign) 
%   \verb!p +pbsi.map!.
% \end{enumerate}

% Finally, I ran the \verb!texhash! command to tell \TeX{}'s path searching
% mechanism about the new files. 
\end{document}



