% \documentclass[noanswers, nolegalese, 11pt]{examjh}
\documentclass[answers, nolegalese, 11pt]{examjh}

\usepackage{../../../sty/conc}
\usepackage{../../../sty/linalgjh}

\setlength{\parindent}{0em}\setlength{\parskip}{0.5ex}
\pagestyle{empty}
\begin{document}\thispagestyle{empty}
\makebox[\textwidth]{Handin 5\hfill  From \textit{Linear Algebra}, by Hef{}feron}\vspace{-1ex}
% \makebox[\textwidth]{\hbox{}\hrulefill\hbox{}}

\begin{center}
  \fbox{\parbox{6.5in}{\it
  When you are working on this material, you must work entirely on your 
  own.
  You may use the book or your notes.
  But you may not work with other people, either in the class our outside of it,
  or use other books or the Internet.
  If you have any questions, email me.
  }}
  \end{center}


\begin{questions}
\question
Consider this map $\map{h}{\matspace_{\nbyn{2}}}{\polyspace_2}$.
\begin{equation*}
  \begin{mat}
    a  &b  \\
    c  &d
  \end{mat}
  \mapsto
  (a+d)x^2+(b+c)x+(2a+2b)
\end{equation*}
\begin{parts}
\item
Verify that this is a linear map.
\begin{solution}
We can do the single check that it preserves linear combinations.
\begin{align*}
  h(n_1\vec{v}_1+n_2\vec{v_2})
  &=h(
    n_1\cdot\begin{mat}
      a_1  &b_1  \\
      c_1  &d_1
    \end{mat}
    +
    n_2\cdot\begin{mat}
      a_2  &b_2  \\
      c_2  &d_2
    \end{mat})                        \\
  &=h(
    n_1\cdot\begin{mat}
      n_1a_1+n_2a_2  &n_1b_1+n_2b_2  \\
      n_1c_1+n_2c_2  &n_1d_1+n_2d_2
    \end{mat})                         \\
  &=((n_1a_1+n_2a_2)+(n_1d_1+n_2d_2))\cdot x^2          
   +((n_1b_1+n_2b_2)+(n_1c_1+n_2c_2))\cdot x    \\
  &\qquad  +(2(n_1a_1+n_2a_2)+2(n_1b_1+n_2b_2))      \\
  &=n_1\cdot\bigl((a_1+d_1)x^2+(b_1+c_1)x+(2a_1+2b_1)\bigr)    \\
  &\qquad +n_2\cdot\bigl((a_2+d_2)x^2+(b_2+c_2)x+(2a_2+2b_2)\bigr)   \\
  &=n_1\cdot h(\begin{mat}
      a_1  &b_1  \\
      c_1  &d_1
    \end{mat})
    +
    n_2\cdot h(\begin{mat}
      a_2  &b_2  \\
      c_2  &d_2
    \end{mat})                           \\
  &=n_1h(\vec{v}_1)+n_2\cdot h(\vec{v}_2)
\end{align*}

\end{solution}

\item
Find the set $h^{-1}(3x^2+4x+8)$.
\begin{solution}
We have these three equations.
\begin{equation*}
\begin{linsys}{4}
a  &   &   &   &   &+  &d  &=  &3 \\
   &   &b  &+  &c  &   &   &=  &4 \\
2a &+  &2b &   &   &   &   &=  &8 \\
\end{linsys}
\end{equation*}
Gauss's method
\begin{equation*}
\begin{amat}{4}
  1  &0  &0  &1  &3  \\ 
  0  &1  &1  &0  &4  \\ 
  2  &2  &0  &0  &8  \\ 
\end{amat}
\grstep{-2\rho_{1}+\rho_{3}}
\begin{amat}{4}
  1  &0  &0  &1  &3  \\ 
  0  &1  &1  &0  &4  \\ 
  0  &2  &0  &-2  &2  \\ 
\end{amat}
\grstep{-2\rho_{2}+\rho_{3}}
\begin{amat}{4}
  1  &0  &0  &1  &3  \\ 
  0  &1  &1  &0  &4  \\ 
  0  &0  &-2  &-2  &-6  \\ 
\end{amat}
\end{equation*}
and back substitution give 
$c=3-d$, $b=1+d$, and $a=3-d$.
So the inverse image is this $1$-dimensional subset of $\matspace_{\nbyn{2}}$.
% sage: M = matrix(QQ, [[1,0,0,1], [0,1,1,0], [2,2,0,0]])
% sage: v = vector(QQ, [3,4,8])
% sage: M_prime = M.augment(v, subdivide=True)
% sage: gauss_method(M_prime, latex=True)
\begin{equation*}
  h^{-1}(3x^2+4x+8)
  =
  \set{
    \begin{mat}
      3-d  &1+d  \\
      3-d  &d
    \end{mat}
    \suchthat d\in \R}
  =
  \set{
    \begin{mat}
      3  &1  \\
      3  &0
    \end{mat}
    +
    \begin{mat}
      -1  &1  \\
      -1  &1
    \end{mat}\cdot d
    \suchthat d\in \R}
\end{equation*}
\end{solution}

\end{parts}
\end{questions}
\end{document}
