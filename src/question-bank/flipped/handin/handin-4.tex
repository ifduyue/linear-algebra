% \documentclass[noanswers, nolegalese, 11pt]{examjh}
\documentclass[answers, nolegalese, 11pt]{examjh}

\usepackage{../../../sty/conc}
\usepackage{../../../sty/linalgjh}

\setlength{\parindent}{0em}\setlength{\parskip}{0.5ex}
\pagestyle{empty}
\begin{document}\thispagestyle{empty}
\makebox[\textwidth]{Handin 4\hfill  From \textit{Linear Algebra}, by Hef{}feron}\vspace{-1ex}
% \makebox[\textwidth]{\hbox{}\hrulefill\hbox{}}

\begin{center}
  \fbox{\parbox{6.5in}{\it
  When you are working on this material, you must work entirely on your 
  own.
  You may use the book or your notes.
  But you may not work with other people, either in the class our outside of it,
  or use other books or the Internet.
  If you have any questions, email me.
  }}
  \end{center}


\begin{questions}
\question
Show that $\map{h}{\polyspace_2}{\R^3}$
\begin{equation*}
  ax^2+bx+c \mapsto \colvec{a-b \\ b \\ a+c}
\end{equation*}
is an isomorphism.
\begin{parts}
\part Verify that it is one-to-one.
\begin{solution}
Assume that $h(\vec{v}_1)=h(\vec{v}_2)$, so that
$h(a_1x^2+b_1x+c_1)=h(a_2x^2+b_2x+c_2)$.
Then
\begin{equation*}
  \colvec{a_1-b_1 \\ b_1 \\ a_1+c_1}
  =
  \colvec{a_2-b_2 \\ b_2 \\ a_2+c_2}
\end{equation*}
and the middle entries show that $b_1=b_2$.
With that, the top entries show that $a_1=a_2$, and then the bottom
entries give $c_1=c_2$.
The conclusion is that $\vec{v}_1=\vec{v}_2$, 
and so $h$ is a one-to-one function.
\end{solution}

\part Verify that it is onto.
\begin{solution}
(\textit{Hint:} try this first with numbers, that is, try two particular
members of $\R^3$ to see the pattern for the general case.)
Fix a member of the codomain~$\R^3$.
\begin{equation*}
  \vec{w}=\colvec{p \\ q \\ r}\in\R^3
\end{equation*}
We must find $\vec{v}\in\polyspace_2$ so that $h(\vec{v})=\vec{w}$.
Consider $ax^2+bx+c$ where $b=q$, where $a=p+q$, and where $c=r-(p+q)$.
The verification that $\vec{v}$ is correct is this.
\begin{equation*}
  (p+q)x^2+qx+(r-(p+q))\mapsto \colvec{(p+q)-q \\ q \\ (p+q)+r-(p+q)}
                               =\colvec{p \\ q \\ r}
\end{equation*}
\end{solution}

\part Verify that it preserves vector addition and scalar multiplication.
\begin{solution}
This shows that it preserves addition.
\begin{align*}
  h(\vec{v}_1+\vec{v}_2)
  &=
  h(\;(a_1x^2+b_1x+c_1)+(a_2x^2+b_2x+c_2)\;)      \\
  &=
  h(\;(a_1+a_2)x^2+(b_1+b_2)x+(c_1+c_2)\;)      \\
  &=
  \colvec{(a_1+a_2)-(b_1+b_2) \\ b_1+b_2  \\ (a_1+a_2)+(c_1+c_2)}  \\
  &=
  \colvec{(a_1-b_1)+(a_2-b_2) \\ b_1+b_2  \\ (a_1+c_1)+(a_2+c_2)}  \\
  &=
  \colvec{a_1-b_1 \\ b_1 \\ a_1+c_1}+\colvec{a_2-b_2 \\ b_2 \\ a_2+c_2}  \\
  &=
  h(a_1x^2+b_1x+c_1)+h(a_2x^2+b_2x+c_2)  
  =h(\vec{v}_1)+h(\vec{v}_2)
\end{align*}
And this is the similar calculation for preservation of scalar multiplication.
\begin{align*}
  h(r\cdot\vec{v})
  &=
  h(\;r\cdot(ax^2+bx+c)\;)           \\
  &=
  h(\;(ra)x^2+(rb)x+(rc)\;)           \\
  &=
  \colvec{ra-rb \\ rb \\ ra+rc}       \\
  &=
  r\cdot\colvec{a-b \\ b \\ a+c}       \\
  &=r\cdot h(ax^2+bx+c)  =r\cdot h(\vec{v})
\end{align*}
\end{solution}
\end{parts}
\end{questions}
\end{document}
