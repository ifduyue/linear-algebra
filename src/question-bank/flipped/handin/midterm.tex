% \documentclass[noanswers, nolegalese, 11pt]{examjh}
\documentclass[answers, nolegalese, 11pt]{examjh}

\usepackage{../../../sty/conc}
\usepackage{../../../sty/linalgjh}

\setlength{\parindent}{0em}\setlength{\parskip}{0.5ex}
\pagestyle{empty}
\begin{document}\thispagestyle{empty}
\makebox[\textwidth]{Midterm, 2021-Spring\hfill  From \textit{Linear Algebra}, by Hef{}feron}\vspace{-1ex}
% \makebox[\textwidth]{\hbox{}\hrulefill\hbox{}}

\begin{center}
  \fbox{\parbox{6.5in}{\it
  When you are working on this material, you must work entirely on your 
  own.
  You may use the book or your notes.
  But you may not work with other people, either in the class our outside of it,
  or use other books or the Internet.
  If you have any questions, email me.
  }}
  \end{center}

{\large \textbf{Instructions: Do six of these eight questions.}}

\begin{questions}

\item Determine if each set is linearly independent (in the natural vector 
space).
  \begin{parts}
  \part $\set{\colvec{1 \\ 2 \\ 0}, 
              \colvec{-1 \\ 1 \\ 0}}$
\begin{solution}
  The natural vector space is $\Re^3$.
  We set up the equation
  \begin{equation*}
    \colvec{0 \\ 0 \\ 0}=c_1\colvec{1 \\ 2 \\ 0}
                         +c_2\colvec{-1 \\ 1 \\ 0}
  \end{equation*}
  and consider the resulting homogeneous system.
  \begin{equation*}
    \begin{amat}{2}
      1  &-1 &0  \\
      2  &1  &0  \\
      0  &0  &0
    \end{amat}
    \grstep{-2\rho_1+\rho_2}
    \begin{amat}{2}
      1  &-1 &0  \\
      0  &3  &0  \\
      0  &0  &0
    \end{amat} 
  \end{equation*}
  This has the unique solution, that $c_1=0$, $c_2=0$.
  So it is linearly independent.
\end{solution}

  \part $\set{\rowvec{1 &3 &1}, \rowvec{-1 &4 &3}, \rowvec{-1 &11 &7}}$
\begin{solution}
  The natural vector space is the set of three-wide row vectors.
  The equation
  \begin{equation*}
    \rowvec{0 &0 &0}=c_1\rowvec{1 &3 &1}
                     +c_2\rowvec{-1 &4 &3}
                     +c_3\rowvec{-1 &11 &7}
  \end{equation*}
  gives rise to a linear system
  \begin{equation*}
    \begin{amat}{3}
      1  &-1  &-1  &0  \\
      3  &4   &11  &0  \\
      1  &3   &7   &0
    \end{amat}
    \grstep[-\rho_1+\rho_3]{-3\rho_1+\rho_2}
    \begin{amat}{3}
      1  &-1  &-1  &0  \\
      0  &7   &14  &0  \\
      0  &4   &8   &0
    \end{amat}
    \grstep{(4/7)\rho_2+\rho_3}
    \begin{amat}{3}
      1  &-1  &-1  &0  \\
      0  &7   &14  &0  \\
      0  &0   &0   &0
    \end{amat}
  \end{equation*}
  with infinitely many solutions, that is, more than just the trivial solution.
  \begin{equation*}
    \set{\colvec{c_1 \\ c_2 \\ c_3}
          =\colvec{-1 \\ -2 \\ 1}c_3
          \suchthat c_3\in\Re}
  \end{equation*}
  So the set is linearly dependent.
  One dependence comes from setting $c_3=2$, giving 
  $c_1=-2$ and $c_2=-4$.
\end{solution}
  \end{parts}


\question
Solve this system.
State whether it has a unique solution, no solutions, or infinitely many
solutions.
Give the solution set, in vector form.
\begin{equation*}
    \begin{linsys}{3}
      x  &+ &y  &+ &z &= &4 \\
      2x &- &2y &- &z &= &-1 \\
      4x &+ &y  &+ &2z &= &10
    \end{linsys}
\end{equation*}
\begin{solution}
% sage: load('../inclass/gauss_method.sage')
% sage: M = matrix(QQ, [[1,1,1], [2,-2,-1], [4,1,2]])
% sage: v = vector(QQ, [4,-1,10])
% sage: M_prime = M.augment(v, subdivide=True)
% sage: gauss_method(M_prime, latex=True)
\begin{align*}
  \begin{amat}{3}
  1  &1  &1  &4  \\ 
  2  &-2  &-1  &-1  \\ 
  4  &1  &2  &10  \\ 
\end{amat}
& \grstep[-4\rho_{1}+\rho_{3}]{-2\rho_{1}+\rho_{2}}
\begin{amat}{3}
  1  &1  &1  &4  \\ 
  0  &-4  &-3  &-9  \\ 
  0  &-3  &-2  &-6  \\ 
\end{amat}                            \\
&\grstep{(-3/4)\rho_{2}+\rho_{3}}
\begin{amat}{3}
  1  &1  &1  &4  \\ 
  0  &-4  &-3  &-9  \\ 
  0  &0  &1/4  &3/4  \\ 
\end{amat}
\end{align*}
The solution is that $z=3$, $y=0$, $x=1$.
\end{solution}

\question
Consider the map $\map{h}{\polyspace_2}{\Re^2}$ defined by this.
\begin{equation*}
  ax^2+bx+c\mapsto \colvec{a+b \\ a+c}
\end{equation*}
Prove that it is a homomorphism.
\begin{solution}
We show that it preserves linear combinations.
Suppose that $\vec{v}_1,\vec{v}_2\in\polyspace_2$.
\begin{align*}
  h(r_1\vec{v}_1+r_2\vec{v}_2)
  &=
  h(r_1(a_1x^2+b_1x+c_1)+r_2(a_2x^2+b_2x+c_2))     \\
  &=
  h((r_1a_1+r_2a_2)x^2+(r_1b_1+r_2b_2)x+(r_1c_1+r_2c_2))    \\
  &=
  \colvec{(r_1a_1+r_2a_2)+(r_1b_1+r_2b_2) \\ (r_1a_1+r_2a_2)+(r_1c_1+r_2c_2)} \\
  &=
  r_1\colvec{a_1+b_1 \\ a_1+c_1}
  +
  r_2\colvec{a_2+b_2 \\ a_2+c_2} \\
  &=r_1h(\vec{v}_1)+r_2h(\vec{v}_2)
\end{align*}
\end{solution}

\question
Find a basis for this vector space.
\begin{equation*}
  V=
  \set{
    \begin{mat}
      a  &b  \\
      c  &d
    \end{mat}
  \suchthat
  \text{$a+2b-d=0$ and $3b+d=0$}}
\end{equation*}
What is its dimension?
\begin{solution}
From the linear system
\begin{equation*}
\begin{linsys}{4}
  a  &+  &2b  &   &   &-  &d  &=  &0  \\
     &   &3b  &   &   &+  &d  &=  &0  \\
\end{linsys}
\end{equation*}
we get that $b=-(1/3)d$ and that 
$a=(5/3)d$.
Parametrizing gives this.
\begin{equation*}
  V=\set{
    \begin{mat}
      (5/3)d  &-(1/3)d  \\
      c       &d
    \end{mat}
    \suchthat c,d\in\R}
  =\set{
    c\cdot\begin{mat}
      0  &0  \\
      1  &0
    \end{mat}
    +d\cdot
    \begin{mat}
      5/3  &-1/3  \\   
      0    &1
    \end{mat}
    \suchthat c,d\in\R}
\end{equation*}
The two matrices form a spanning set.
This is linearly independent because the second element is cleaerly not a
multiple of the first.
\begin{equation*}
  B=
  \sequence{
    \begin{mat}
      0  &0  \\
      1  &0
    \end{mat},
    \begin{mat}
      5/3  &-1/3  \\   
      0    &1
    \end{mat}}
\end{equation*}
\end{solution}



\question
  Consider this subset of $\Re^3$.
  \begin{equation*}
    M=\set{\colvec{x \\ y \\ z}\suchthat 2x-y=0}
  \end{equation*}
    Show that it is closed under addition and 
    scalar multiplication, and is therefore a subspace.
    \begin{solution}
      It is closed under addition because the sum of two vectors that are 
      members of the subset
      \begin{equation*}
        \colvec{x_1 \\ y_1 \\ z_1}+\colvec{x_2 \\ y_2 \\ z_2}
        =\colvec{x_1+x_2 \\ y_1+y_2 \\ z_1+z_2}
      \end{equation*}
      is also a member of~$M$ as twice its first component minus its
      second component is $2(x_1+x_2)-(y_1+y_2)=(2x_1-y_1)+(2x_2-y_2)=0+0=0$.
      It is closed under scalar multiplication because
      \begin{equation*}
        r\cdot\colvec{x \\ y \\ z}
        =\colvec{rx \\ ry \\ rz}
      \end{equation*}
      and twice the first component of that vector minus its
      second component is 
      $2rx-ry=r(2x-y)=r\cdot 0=0$.
    \end{solution}

\question
Verify that this map $\map{f}{\R^3}{\matspace_{\nbyn{2}}}$
is one-to-one.
(You do \emph{not} have to also do onto.)
\begin{equation*}
  f(\colvec{a \\ b \\ c})
  =
  \begin{mat}
    a+2b  &-c \\
    a-b   &0
  \end{mat}
\end{equation*}
\begin{solution}
Assume that $f(\vec{v}_1)=f(\vec{v}_2)$.
\begin{align*}
  f(\colvec{a_1 \\ b_1 \\ c_1})
  &=
  f(\colvec{a_2 \\ b_2 \\ c_2}) \\
  \begin{mat}
    a_1+2b_1  &-c_1 \\
    a_1-b_1   &0
  \end{mat}
  &=
  \begin{mat}
    a_2+2b_2  &-c_2 \\
    a_2-b_2   &0
  \end{mat}
\end{align*}
Comparing the upper right entries gives that $c_11=c_2$.
Comparing the upper left and lower left entries gives this.
\begin{equation*}
\begin{linsys}{4}
  a_1 &+  &2b_1 &-  &a_2  &-  &2b_2  &=  &0  \\
  a_1 &-  &b_1  &-  &a_2  &+  &b_2   &=  &0  
\end{linsys}
\grstep{-\rho_1+\rho_2}
\begin{linsys}{4}
  a_1 &+  &2b_1   &-  &a_2  &-  &2b_2  &=  &0  \\
      &   &-3b_1  &   &     &+  &3b_2  &=  &0  
\end{linsys}
\end{equation*}
Thus $b_1=b_2$, and so $a_1=a_2$.
Therefore $\vec{v}_1=\vec{v}_2$.
\end{solution}



\question
What is the row rank of this matrix?
\begin{equation*}
\begin{mat}
2   &1  &0  \\
-1  &3  &1  \\
2   &15 &4
\end{mat}
\end{equation*}
\begin{solution}
% sage: M=matrix(QQ, [[2,1,0], [-1,3,1], [2,15,4]])
% sage: gauss_method(M, latex=True)
This reduction
\begin{equation*}
\begin{mat}
  2  &1  &0  \\ 
  -1  &3  &1  \\ 
  2  &15  &4  \\ 
\end{mat}
\grstep[-1\rho_{1}+\rho_{3}]{(1/2)\rho_{1}+\rho_{2}}
\begin{mat}
  2  &1  &0  \\ 
  0  &7/2  &1  \\ 
  0  &14  &4  \\ 
\end{mat}
\grstep{-4\rho_{2}+\rho_{3}}
\begin{mat}
  2  &1  &0  \\ 
  0  &7/2  &1  \\ 
  0  &0  &0  \\ 
\end{mat}
\end{equation*}
shows that
the row rank is~$2$.
\end{solution}


\question
Consider the vector space $\polyspace_2$.
Represent $\vec{v}=2x^2-3x-1$
with respect to this basis.
\begin{equation*}
  B=\sequence{x+1, x-1, x^2+1}
\end{equation*}
\begin{solution}
The equation
\begin{equation*}
  2x^2-3x-1=c_1\cdot (x+1)+c_2\cdot (x-1)+c_3\cdot (x^2+1)
\end{equation*}
gives this linear system.
\begin{equation*}
\begin{linsys}{3}
         &   &    &   &c_3  &=  &2  \\
    c_1  &+  &c_2 &   &     &=  &-3  \\   
    c_1  &-  &c_2 &+  &c_3  &=  &-1  \\   
\end{linsys}
\end{equation*}
Gauss's method
\begin{align*}
  \begin{amat}{3}
  0  &0  &1  &2  \\ 
  1  &1  &0  &-3  \\ 
  1  &-1  &1  &-1  \\ 
\end{amat}
&\grstep{\rho_{1}\leftrightarrow \rho_{2}}
\begin{amat}{3}
  1  &1  &0  &-3  \\ 
  0  &0  &1  &2  \\ 
  1  &-1  &1  &-1  \\ 
\end{amat}                              \\
&\grstep{-1\rho_{1}+\rho_{3}}
\begin{amat}{3}
  1  &1  &0  &-3  \\ 
  0  &0  &1  &2  \\ 
  0  &-2  &1  &2  \\ 
\end{amat}                               \\
&\grstep{\rho_{2}\leftrightarrow \rho_{3}}
\begin{amat}{3}
  1  &1  &0  &-3  \\ 
  0  &-2  &1  &2  \\ 
  0  &0  &1  &2  \\ 
\end{amat}
\end{align*}
gives $c_3=2$, $c_2=0$, and $c_1=-3$.
\begin{equation*}
  \rep{2x^2-3x-1}{B}=\colvec{-3 \\ 0 \\ 2}
\end{equation*}
\end{solution}

\end{questions}
\end{document}
