% \documentclass[noanswers, nolegalese, 11pt]{examjh}
\documentclass[answers, nolegalese, 11pt]{examjh}

\usepackage{../../../sty/conc}
\usepackage{../../../sty/linalgjh}

\setlength{\parindent}{0em}\setlength{\parskip}{0.5ex}
\pagestyle{empty}
\begin{document}\thispagestyle{empty}
\makebox[\textwidth]{Handin 7\hfill  From \textit{Linear Algebra}, by Hef{}feron}\vspace{-1ex}
% \makebox[\textwidth]{\hbox{}\hrulefill\hbox{}}

\begin{center}
  \fbox{\parbox{6.5in}{\it
  When you are working on this material, you must work entirely on your 
  own.
  You may use the book or your notes.
  But you may not work with other people, either in the class our outside of it,
  or use other books or the Internet.
  If you have any questions, email me.
  }}
  \end{center}


\begin{questions}
\question
Consider the vector space $V=\matspace_{\nbyn{2}}$
as well as this basis
\begin{equation*}
B=\sequence{
\begin{mat}
 1  &0  \\
 0  &1
\end{mat},
\begin{mat}
 1  &1  \\
 0  &0
\end{mat},
\begin{mat}
 1  &0  \\
 1  &0
\end{mat},
\begin{mat}
 0  &0  \\
 0  &1
\end{mat}
}
\end{equation*}
and this one.
\begin{equation*}
D=\sequence{
\begin{mat}
 1  &0  \\
 0  &0
\end{mat},
\begin{mat}
 1  &2  \\
 0  &0
\end{mat},
\begin{mat}
 1  &2  \\
 2  &0
\end{mat},
\begin{mat}
 1  &2  \\
 2  &1
\end{mat}
}
\end{equation*}

\begin{parts}
\part
Find the change of basis matrix for converting representations with
respect to~$B$ into representations with respect to~$D$.
\begin{solution}
First determine the effect of the identity map on each vector from~$B$.
\begin{align*}
\begin{mat}
 1  &0  \\
 0  &1
\end{mat}
  &\mapsunder{\identity}\begin{mat}
 1  &0  \\
 0  &1
\end{mat}  \\
\begin{mat}
 1  &1  \\
 0  &0
\end{mat}
  &\mapsunder{\identity} \begin{mat}
 1  &1  \\
 0  &0
\end{mat} \\
\begin{mat}
 1  &0  \\
 1  &0
\end{mat}
  &\mapsunder{\identity} \begin{mat}
 1  &0  \\
 1  &0
\end{mat} \\
\begin{mat}
 0  &0  \\
 0  &1
\end{mat}
  &\mapsunder{\identity} \begin{mat}
 0  &0  \\
 0  &1
\end{mat}  
\end{align*}
Then find the representation of each of those with respect to~$D$.
For instance here is the setup for the first.
\begin{equation*}
\begin{mat}
 1  &0  \\
 0  &1
\end{mat}
=c_1\cdot \begin{mat}
 1  &0  \\
 0  &0
\end{mat}
+c_2\cdot \begin{mat}
 1  &2  \\
 0  &0
\end{mat}
+c_3\cdot \begin{mat}
 1  &2  \\
 2  &0
\end{mat}
+c_4\cdot \begin{mat}
 1  &2  \\
 2  &1
\end{mat}
\end{equation*}
The representation is this.
\begin{equation*}
\rep{\begin{mat}
 1  &0  \\
 0  &1
\end{mat}}{D}
=\colvec{1 \\ 0 \\ -1 \\ 1}
\end{equation*}
The other three are similar.
\begin{equation*}
\rep{\begin{mat}
 1  &1  \\
 0  &0
\end{mat}}{D}
=\colvec{1/2 \\ 1/2 \\ 0 \\ 0}
\qquad
\rep{\begin{mat}
 1  &0  \\
 1  &0
\end{mat}}{D}
=\colvec{1 \\ -1/2 \\ 1/2 \\ 0}
\qquad
\rep{\begin{mat}
 0  &0  \\
 0  &1
\end{mat}}{D}
=\colvec{0 \\ 0 \\ -1 \\ 1}
\end{equation*}
Put them together to get the matrix.
\begin{equation*}
\rep{\identity}{B,D}=
\begin{mat}
 1  &1/2 &1    &0  \\
 0  &1/2 &-1/2 &0  \\
-1  &0   &1/2  &-1 \\
 1  &0   &0    &1
\end{mat}
\end{equation*}
\end{solution}

\part
Represent this vector with respect to~$B$.
\begin{equation*}
  \vec{v}=
  \begin{mat}
    1  &2  \\
    3  &4
  \end{mat}
\end{equation*}
\begin{solution}
Solve
\begin{equation*}
  \begin{mat}
    1  &2  \\
    3  &4
  \end{mat}
  =c_1\cdot \begin{mat}
 1  &0  \\
 0  &1
\end{mat}
+c_2\cdot \begin{mat}
 1  &1  \\
 0  &0
\end{mat}
+c_3\cdot \begin{mat}
 1  &0  \\
 1  &0
\end{mat}
+c_4\cdot \begin{mat}
 0  &0  \\
 0  &1
\end{mat}
\end{equation*}
By eye, $c_2=2$ and $c_3=3$, and $c_1=-4$, and $c_4=8$.
\begin{equation*}
\rep{\begin{mat}
    1  &2  \\
    3  &4
  \end{mat}}{B}
=\colvec{-4 \\ 2 \\ 3 \\ 8}
\end{equation*}
\end{solution}

\part
Use the change of basis matrix to convert it to a representation with 
respect to~$D$.
\begin{solution}
Do the matrix-vector multiplication.
\begin{equation*}
\begin{mat}
 1  &1/2 &1    &0  \\
 0  &1/2 &-1/2 &0  \\
-1  &0   &1/2  &-1 \\
 1  &0   &0    &1
\end{mat}
\colvec{-4 \\ 2 \\ 3 \\ 8}
=
\colvec{0 \\ -1/2 \\ -5/2 \\ 4}
=\rep{\begin{mat}
    1  &2  \\
    3  &4
  \end{mat}}{D}
\end{equation*}

\end{solution}
\end{parts}
\end{questions}
\end{document}
