\documentclass[noanswers, nolegalese, 11pt]{examjh}
% \documentclass[answers, nolegalese, 11pt]{examjh}

\usepackage{../../../sty/conc}
\usepackage{../../../sty/linalgjh}

\setlength{\parindent}{0em}\setlength{\parskip}{0.5ex}
\pagestyle{empty}
\begin{document}\thispagestyle{empty}
\makebox[\textwidth]{Handin 2\hfill  From \textit{Linear Algebra}, by Hef{}feron}\vspace{-1ex}
% \makebox[\textwidth]{\hbox{}\hrulefill\hbox{}}

\begin{center}
  \fbox{\parbox{6.5in}{\it
  When you are working on this material, you must work entirely on your 
  own.
  You may use the book or your notes.
  But you may not work with other people, either in the class our outside of it,
  or use other books or the Internet.
  If you have any questions, email me.
  }}
  \end{center}


\begin{questions}
\question
Decide if these two matrices are row-equivalent.
\begin{equation*}
\begin{mat}
  3 &0  &6 &0 &9  \\
  2 &-2 &4 &1 &10 \\
  2 &0  &6 &1 &10
\end{mat}
\qquad
\begin{mat}
  0  &3  &3  &1  &4  \\
  0  &2  &2  &0  &0  \\
 -4  &0  &-8 &0  &-12  
\end{mat}
\end{equation*}
\begin{solution}
This is the first matrix put in reduced echelon form.
% sage: M = matrix(QQ, [[3,0,6,0,9], [2,-2,4,1,10], [2,0,6,1,10]])
% sage: gauss_jordan(M, latex=True)
\begin{align*}
\begin{mat}
  3  &0  &6  &0  &9  \\ 
  2  &-2  &4  &1  &10  \\ 
  2  &0  &6  &1  &10  \\ 
\end{mat}
&\grstep[(-2/3)\rho_{1}+\rho_{3}]{(-2/3)\rho_{1}+\rho_{2}}
\begin{mat}
  3  &0  &6  &0  &9  \\ 
  0  &-2  &0  &1  &4  \\ 
  0  &0  &2  &1  &4  \\ 
\end{mat}                                                 \\
&\grstep[(-1/2)\rho_{2} \\ (1/2)\rho_{3}]{(1/3)\rho_{1}}
\begin{mat}
  1  &0  &2  &0  &3  \\ 
  0  &1  &0  &-1/2  &-2  \\ 
  0  &0  &1  &1/2  &2  \\ 
\end{mat}                                                  \\
&\grstep{-2\rho_{3}+\rho_{1}}
\begin{mat}
  1  &0  &0  &-1  &-1  \\ 
  0  &1  &0  &-1/2  &-2  \\ 
  0  &0  &1  &1/2  &2  \\ 
\end{mat}
\end{align*}
And here is the second one.
% sage: N = matrix(QQ, [[0,3,3,1,4], [0,2,2,0,0], [-4,0,-8,0,-12]])
% sage: gauss_jordan(N,latex=True)
\begin{align*}
 \begin{mat}
  0  &3  &3  &1  &4  \\ 
  0  &2  &2  &0  &0  \\ 
  -4  &0  &-8  &0  &-12  \\ 
\end{mat}
&\grstep{\rho_{1}\leftrightarrow \rho_{3}}
\begin{mat}
  -4  &0  &-8  &0  &-12  \\ 
  0  &2  &2  &0  &0  \\ 
  0  &3  &3  &1  &4  \\ 
\end{mat}                                      \\
&\grstep{(-3/2)\rho_{2}+\rho_{3}}
\begin{mat}
  -4  &0  &-8  &0  &-12  \\ 
  0  &2  &2  &0  &0  \\ 
  0  &0  &0  &1  &4  \\ 
\end{mat}                                       \\
&\grstep[(1/2)\rho_{2}]{(-1/4)\rho_{1}}
\begin{mat}
  1  &0  &2  &0  &3  \\ 
  0  &1  &1  &0  &0  \\ 
  0  &0  &0  &1  &4  \\ 
\end{mat} 
\end{align*}
The two reduced echelon forms are unequal. 
So, the starting matrices are not row-equivalent.
\end{solution}

\question
Show that this set is not closed under vector addition.
\begin{equation*}
  S=\set{
    \begin{mat}
      a  &b \\
      c  &d
    \end{mat}
    \suchthat
    a+b+3-c-d=0}
\end{equation*}
\begin{solution}
Here are two members of $S$.
\begin{equation*}
\begin{mat}
  -3  &0 \\
  0   &0
\end{mat}
\qquad
\begin{mat}
  1 &2  \\
  3  &3
\end{mat}
\end{equation*}
Their sum
\begin{equation*}
\begin{mat}
  -2  &2  \\
   3  &3
\end{mat}
\end{equation*}
is not a member of~$S$
because it does not satisfy the criteria.
\end{solution}
\end{questions}
\end{document}
