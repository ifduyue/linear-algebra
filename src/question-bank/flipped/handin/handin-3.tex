% \documentclass[noanswers, nolegalese, 11pt]{examjh}
\documentclass[answers, nolegalese, 11pt]{examjh}

\usepackage{../../../sty/conc}
\usepackage{../../../sty/linalgjh}

\setlength{\parindent}{0em}\setlength{\parskip}{0.5ex}
\pagestyle{empty}
\begin{document}\thispagestyle{empty}
\makebox[\textwidth]{Handin 3\hfill  From \textit{Linear Algebra}, by Hef{}feron}\vspace{-1ex}
% \makebox[\textwidth]{\hbox{}\hrulefill\hbox{}}

\begin{center}
  \fbox{\parbox{6.5in}{\it
  When you are working on this material, you must work entirely on your 
  own.
  You may use the book or your notes.
  But you may not work with other people, either in the class our outside of it,
  or use other books or the Internet.
  If you have any questions, email me.
  }}
  \end{center}


\begin{questions}
\question
Find a spanning set for each.
\begin{parts}
\part
$S_1=\set{ax^3+bx^2+cx+d\in\polyspace_3\suchthat a-b+2d=0}$
\begin{solution}[3in]
Think of the restriction as a one-equation linear system.
We get $a=b-2d$ and $c=c$.
\begin{align*}
  S_1
  &=\set{(b-2d)x^3+bx^2+cx+d\in\polyspace_3\suchthat b,c,d\in\R}    \\
  &=\set{b\cdot(x^3+x^2)+c\cdot x+d\cdot (-2x^3+1)\in\polyspace_3\suchthat b,c,d\in\R}    
\end{align*}
One spanning set is $\set{x^3+x^2,x,-2x^3+1}$.
\end{solution}

\part
$S_2=\set{\colvec{x \\ y \\ z}\suchthat \text{$2x+z=0$ and $x-2y+z=0$}}$
\begin{solution}[3in]
The restrictions make a two-equation linear system.
\begin{equation*}
\begin{linsys}{3}
  2x  &  &  &+  &z  &=  &0  \\
   x  &- &2y &+  &z  &=  &0
\end{linsys}
\grstep{-(1/2)\rho_1+\rho_2}
\begin{linsys}{3}
  2x  &  &    &+  &z       &=  &0  \\
      &  &-2y &+  &(1/2)z  &=  &0
\end{linsys}
\end{equation*}
Parametrizing with the free variable~$z$ gives 
$y=(1/2)z$ and $x=-(1/2)z$.
\begin{align*}
  S_2
  &=\set{\colvec{-(1/2)z \\ (1/4)z \\ z}\suchthat z\in\R}  \\
  &=\set{\colvec{-1/2 \\ 1/4 \\ 1}\cdot z\suchthat z\in\R}  
\end{align*}
This single-element set spans~$S_2$.
\begin{equation*}
  \set{\colvec{-1/2 \\ 1/4 \\ 1}}
\end{equation*}
\end{solution}
\end{parts}
\end{questions}
\end{document}
