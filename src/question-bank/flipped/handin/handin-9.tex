% \documentclass[noanswers, nolegalese, 11pt]{examjh}
\documentclass[answers, nolegalese, 11pt]{examjh}

\usepackage{../../../sty/conc}
\usepackage{../../../sty/linalgjh}

\setlength{\parindent}{0em}\setlength{\parskip}{0.5ex}
\pagestyle{empty}
\begin{document}\thispagestyle{empty}
\makebox[\textwidth]{Handin 9\hfill  From \textit{Linear Algebra}, by Hef{}feron}\vspace{-1ex}
% \makebox[\textwidth]{\hbox{}\hrulefill\hbox{}}

\begin{center}
  \fbox{\parbox{6.5in}{\it
  When you are working on this material, you must work entirely on your 
  own.
  You may use the book or your notes.
  But you may not work with other people, either in the class our outside of it,
  or use other books or the Internet.
  If you have any questions, email me.
  }}
  \end{center}


\begin{questions}
\question
Use Gauss's method (keeping track of row swaps, if any) 
to find the determinant of each matrix.
Is it nonsingular?
\begin{parts}
\part
$
\begin{mat}
  2  &1  &0  &0  \\
  0  &3  &0  &-1  \\
  0  &1  &1  &0  \\
  2  &-1  &0  &2  \\
\end{mat}
$
\begin{solution}
% sage: load("gauss_method.sage")
% sage: M = matrix(QQ, [[2,1,0,0], [0,3,0,-1], [0,1,1,0], [2,-1,0,2]])
% sage: gauss_method(M, latex=True)
\begin{equation*}
\begin{mat}
  2  &1  &0  &0  \\ 
  0  &3  &0  &-1  \\ 
  0  &1  &1  &0  \\ 
  2  &-1  &0  &2  \\ 
\end{mat}
\grstep{-1\rho_{1}+\rho_{4}}
\begin{mat}
  2  &1  &0  &0  \\ 
  0  &3  &0  &-1  \\ 
  0  &1  &1  &0  \\ 
  0  &-2  &0  &2  \\ 
\end{mat}
\grstep[(2/3)\rho_{2}+\rho_{4}]{(-1/3)\rho_{2}+\rho_{3}}
\begin{mat}
  2  &1  &0  &0  \\ 
  0  &3  &0  &-1  \\ 
  0  &0  &1  &1/3  \\ 
  0  &0  &0  &4/3  \\ 
\end{mat}
\end{equation*}
Multiplying down the diagonal gives $8$.
Yes, the starting matrix is nonsingular.
\end{solution}

\part
$
\begin{mat}
  1  &0  &1  &0  \\
  0  &3  &0  &-1  \\
  0  &1  &4  &1  \\
 -1  &6  &-1  &-2  \\
\end{mat}
$
\begin{solution}
\begin{equation*}
\begin{mat}
  1  &0  &1  &0  \\ 
  0  &3  &0  &-1  \\ 
  0  &1  &4  &1  \\ 
  -1  &6  &-1  &-2  \\ 
\end{mat}
\grstep{1\rho_{1}+\rho_{4}}
\begin{mat}
  1  &0  &1  &0  \\ 
  0  &3  &0  &-1  \\ 
  0  &1  &4  &1  \\ 
  0  &6  &0  &-2  \\ 
\end{mat}
\grstep[-2\rho_{2}+\rho_{4}]{(-1/3)\rho_{2}+\rho_{3}}
\begin{mat}
  1  &0  &1  &0  \\ 
  0  &3  &0  &-1  \\ 
  0  &0  &4  &4/3  \\ 
  0  &0  &0  &0  \\ 
\end{mat}
\end{equation*}
Multiplying down the diagonal gives~$0$.
The starting matrix is singular.
\end{solution}

\end{parts}
\end{questions}
\end{document}
