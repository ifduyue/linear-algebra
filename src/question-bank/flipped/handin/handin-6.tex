% \documentclass[noanswers, nolegalese, 11pt]{examjh}
\documentclass[answers, nolegalese, 11pt]{examjh}

\usepackage{../../../sty/conc}
\usepackage{../../../sty/linalgjh}

\setlength{\parindent}{0em}\setlength{\parskip}{0.5ex}
\pagestyle{empty}
\begin{document}\thispagestyle{empty}
\makebox[\textwidth]{Handin 6\hfill  From \textit{Linear Algebra}, by Hef{}feron}\vspace{-1ex}
% \makebox[\textwidth]{\hbox{}\hrulefill\hbox{}}

\begin{center}
  \fbox{\parbox{6.5in}{\it
  When you are working on this material, you must work entirely on your 
  own.
  You may use the book or your notes.
  But you may not work with other people, either in the class our outside of it,
  or use other books or the Internet.
  If you have any questions, email me.
  }}
  \end{center}


\begin{questions}
\question
Consider the $\map{h}{\matspace_{\nbyn{2}}}{\polyspace_2}$ defined here.
\begin{equation*}
  h(\begin{mat}
    a  &b  \\
    c  &d
  \end{mat})
  =
  (a+d)x^2+(b+c)x+(2a+2b)
\end{equation*}
Find the matrix $H$ representing $h$ with respect to these bases.
\begin{equation*}
  B=\sequence{
    \begin{mat}
      1  &0 \\
      0  &0
    \end{mat},
    \begin{mat}
      1  &1 \\
      0  &0
    \end{mat},
    \begin{mat}
      1  &1 \\
      1  &0
    \end{mat},
    \begin{mat}
      1  &1 \\
      1  &1
    \end{mat}
    }
  \qquad
  D=\sequence{2,
              2+x,
              2+x-x^2
  }
\end{equation*}
You must show your work.
\begin{solution}
The action of $h$ on elements of~$B$ is here.
\begin{align*}
    \begin{mat}
      1  &0 \\
      0  &0
    \end{mat}
    &\mapsto 1\cdot x^2+0\cdot x+2  \\
    \begin{mat}
      1  &1 \\
      0  &0
    \end{mat}
    &\mapsto 1\cdot x^2+1\cdot x+4  \\
    \begin{mat}
      1  &1 \\
      1  &0
    \end{mat}
    &\mapsto 1\cdot x^2+2\cdot x+4  \\
    \begin{mat}
      1  &1 \\
      1  &1
    \end{mat}
    &\mapsto 2\cdot x^2+2\cdot x+4  
\end{align*}
The representation of those values is here.
\begin{multline*}
  \rep{x^2+2}{D}=\colvec{1 \\ 1 \\ -1}
  \quad
  \rep{x^2+x+4}{D}=\colvec{1 \\ 2 \\ -1}
  \\
  \rep{x^2+2x+4}{D}=\colvec{0 \\ 3 \\ -1}
  \quad
  \rep{2x^2+2x+4}{D}=\colvec{0 \\ 4 \\ -2}
\end{multline*}
So this is the matrix.
\begin{equation*}
  H=\rep{h}{B,D}=
  \begin{mat}
    1  &1  &0  &0 \\
    1  &2  &3  &4 \\
   -1  &-1 &-1 &-2 \\
  \end{mat}
\end{equation*}
\end{solution}


\end{questions}
\end{document}
