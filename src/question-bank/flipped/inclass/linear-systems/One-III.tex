\documentclass[noanswers, nolegalese, 11pt]{examjh}
% \documentclass[answers, nolegalese, 11pt]{examjh}
\usepackage{../../../../sty/conc}
\usepackage{../../../../sty/linalgjh}

\setlength{\parindent}{0em}\setlength{\parskip}{0.5ex}
\pagestyle{empty}
\begin{document}\thispagestyle{empty}
\makebox[\textwidth]{Questions for One.III\hfill  From \textit{Linear Algebra}, by Hef{}feron}\vspace{-1ex}
\makebox[\textwidth]{\hbox{}\hrulefill\hbox{}}


\begin{questions}
\question
Find the reduced echelon form of this matrix.
\begin{equation*}
\begin{mat}
  2  &1  &0  \\
  1  &0  &-1 \\
  1  &-1 &-3
\end{mat}
\end{equation*}
\begin{solution}
\begin{align*}
\begin{mat}
  2  &1  &0  \\ 
  1  &0  &-1  \\ 
  1  &-1  &-3  \\ 
\end{mat}
&\grstep{(-1/2)\rho_{1}+\rho_{3}}
\begin{mat}
  2  &1  &0  \\ 
  0  &-1/2  &-1  \\ 
  0  &-3/2  &-3  \\ 
\end{mat}                         \\
&\grstep{-3\rho_{2}+\rho_{3}}
\begin{mat}
  2  &1  &0  \\ 
  0  &-1/2  &-1  \\ 
  0  &0  &0  \\ 
\end{mat}                        \\
&\grstep[-2\rho_{2}]{(1/2)\rho_{1}}
\begin{mat}
  1  &1/2  &0  \\ 
  0  &1  &2  \\ 
  0  &0  &0  \\ 
\end{mat}                            \\
&\grstep{(-1/2)\rho_{2}+\rho_{1}}
\begin{mat}
  1  &0  &-1  \\ 
  0  &1  &2  \\ 
  0  &0  &0  \\ 
\end{mat}
\end{align*}
\end{solution}


\question
Show that this matrix is row equivalent to the one in the prior question.
\begin{equation*}
\begin{mat}
  3  &2  &1  \\
  0  &2  &4  \\
  2  &0  &-2
\end{mat}
\end{equation*}
% sage: v1 = vector(QQ, [1, 0, -1]) 
% sage: v2 = vector(QQ, [0, 1, 2])
% sage: 3*v1+2*v2
% (3, 2, 1)
% sage: -2*v1
% (-2, 0, 2)
\begin{solution}
\begin{align*}
\begin{mat}
  3  &2  &1  \\ 
  0  &2  &4  \\ 
  2  &0  &-2  \\ 
\end{mat}
&\grstep{(-2/3)\rho_{1}+\rho_{3}}
\begin{mat}
  3  &2  &1  \\ 
  0  &2  &4  \\ 
  0  &-4/3  &-8/3  \\ 
\end{mat}                      \\
&\grstep{(2/3)\rho_{2}+\rho_{3}}
\begin{mat}
  3  &2  &1  \\ 
  0  &2  &4  \\ 
  0  &0  &0  \\ 
\end{mat}                     \\
&\grstep[(1/2)\rho_{2}]{(1/3)\rho_{1}}
\begin{mat}
  1  &2/3  &1/3  \\ 
  0  &1  &2  \\ 
  0  &0  &0  \\ 
\end{mat}                     \\
&\grstep{(-2/3)\rho_{2}+\rho_{1}}
\begin{mat}
  1  &0  &-1  \\ 
  0  &1  &2  \\ 
  0  &0  &0  \\ 
\end{mat} 
\end{align*}
\end{solution}

\question
Find what combination of these rows
\begin{equation*}
  \rowvec{2 &1 &0}
  \quad
  \rowvec{1 &0 &-1}
  \quad
  \rowvec{1 &-1 &-3}
\end{equation*}
makes the row $\rowvec{3 &2 &1}$.
\begin{solution}
We want to find $c_1,c_2,c_3$ such that this holds.
\begin{equation*}
  c_1\cdot \rowvec{2 &1 &0}
  +c_2\cdot\rowvec{1 &0 &-1}
  +c_3\cdot\rowvec{1 &-1 &-3}
  =\rowvec{3 &2 &1}
\end{equation*}
We get a linear system and solve it.
\begin{equation*}
\begin{linsys}{3}
  2c_1 &+  &c_2  &+  &c_3  &=  &3  \\
  1c_1 &   &     &-  &c_3  &=  &2  \\
       &   &-c_2 &-  &3c_3  &= &1  \\
\end{linsys}
\end{equation*}
Gauss's method gives this.
\begin{equation*}
\begin{amat}{3}
  2  &1  &1  &3  \\ 
  1  &0  &-1  &2  \\ 
  0  &-1  &-3  &1  \\ 
\end{amat}
\grstep{(-1/2)\rho_{1}+\rho_{2}}
\begin{amat}{3}
  2  &1  &1  &3  \\ 
  0  &-1/2  &-3/2  &1/2  \\ 
  0  &-1  &-3  &1  \\ 
\end{amat}
\grstep{-2\rho_{2}+\rho_{3}}
\begin{amat}{3}
  2  &1  &1  &3  \\ 
  0  &-1/2  &-3/2  &1/2  \\ 
  0  &0  &0  &0  \\ 
\end{amat}
\end{equation*}
There are infinitely many solutions.
One is to take $c_3=0$ and get 
$c_2=-1$ and $c_1=2$.
\end{solution}

\end{questions}
\end{document}
