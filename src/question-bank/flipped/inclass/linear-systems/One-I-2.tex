\documentclass[noanswers, nolegalese, 11pt]{examjh}
% \documentclass[answers, nolegalese, 11pt]{examjh}
\usepackage{../../../../sty/conc}
\usepackage{../../../../sty/linalgjh}

\setlength{\parindent}{0em}\setlength{\parskip}{0.5ex}
\pagestyle{empty}
\begin{document}\thispagestyle{empty}
\makebox[\textwidth]{Questions for One.I.2\hfill  From \textit{Linear Algebra}, by Hef{}feron}\vspace{-1ex}
\makebox[\textwidth]{\hbox{}\hrulefill\hbox{}}


\begin{questions}

\question Describe of the solution set of each using vector notation.
\begin{parts}
\part
\begin{equation*}
 \begin{linsys}{3}
   x  &+  &y  &-  &z  &=   &5  \\
      &   &2y &+  &3z  &=  &9  \\
      &   &   &   &0  &=  &0  \\
 \end{linsys}
\end{equation*}

\part
\begin{equation*}
 \begin{linsys}{4}
   2x  &-  &y  &+  &z  &+  &w  &=   &1  \\
       &   &-y &   &   &+  &2w &=   &-1  \\
       &   &   &   &   &   &0  &=   &0  \\
 \end{linsys}
\end{equation*}

\end{parts}
\begin{solution}
\begin{parts}
\item
Leading are $x$ and~$y$.
Free is $z$.
The second equation gives $y=(9/2)-(3/2)z$.
Substitution gives $x+((9/2)-(3/2)z)-z = 5$,
leading to $x-(5/2)z=1/2$, so $x=(1/2)+(5/2)z$. 
\begin{equation*}
  \set{\colvec{x \\ y \\ z}=
          \colvec{1/2 \\ 9/2  \\  0}+\colvec{5/2 \\ -3/2 \\ 1}\cdot z\suchthat z\in\R}
\end{equation*}
It is a line through the origin.

\part
Leading are $x$ and~$y$.
Free are $z$ and~$w$.
Expressing the leading variables in terms of those that are free
gives the second equation as $-y=-1-2w$, so $y=1+2w$.
Substituting into the first equation yields 
$2x-(1+2w)+z+w=1$, so $2x-w+z=2$, and we get $x=1-(1/2)z+(1/2)w$.
\begin{equation*}
  \set{\colvec{x  \\ y \\ z \\ w}=
           \colvec{1 \\ 1 \\ 0 \\ 0}+\colvec{-1/2 \\ 0 \\ 1 \\ 0}\cdot z+ \colvec{1/2 \\ 2 \\ 0 \\ 1}\cdot w
           \suchthat z,w\in\R}
\end{equation*}
This is a plane through the origin.
\end{parts}
\end{solution}


\question Use Gauss's method to solve each system
using matrix notation.
Give the solution set description in vector notation.
\begin{parts}
\item
\begin{equation*}
 \begin{linsys}{3}
   x  &-  &2y &   &    &=   &3  \\
      &   &2y &-  &3z  &=   &4  \\
  2x  &-  &2y &-  &3z  &=   &10  \\
 \end{linsys}
\end{equation*}

\item
\begin{equation*}
 \begin{linsys}{4}
    a  &-  &b  &-  &c  &   &   &=   &3  \\
   -a  &+  &b  &+  &c  &-  &d  &=   &0  \\
   2a  &-  &2b &+  &c  &-  &2d &=   &-1  \\
 \end{linsys}
\end{equation*}
\end{parts}
\begin{solution}
\begin{parts}
\item
Here is Gauss's method.
\begin{equation*}
  \begin{amat}{3}
    1  &-2  &0  &3  \\
    0  &2   &-3 &4  \\
    2  &-2  &-3 &10
  \end{amat}
  \grstep{-2\rho_1+\rho_3}
  \begin{amat}{3}
    1  &-2  &0  &3  \\
    0  &2   &-3 &4  \\
    2  &2   &-3 &4
  \end{amat}
  \grstep{-\rho_2+\rho_3}
  \begin{amat}{3}
    1  &-2  &0  &3  \\
    0  &2   &-3 &4  \\
    0  &0   &0  &0
  \end{amat}
\end{equation*}
Leading variables are $x$ and~$y$.
Free is~$z$.
To express $y$ in terms of~$z$ we get $2y=4+3z$,
which gives $y=2+(3/2)z$.
Then to express $x$ in terms of~$z$, substitute
$x-2(2+(3/2)z)=3$, giving 
$x=7+3z$.
\begin{equation*}
  \set{\colvec{x \\ y \\ z}=
         \colvec{7 \\ 2 \\ 0}+\colvec{3 \\ 3/2 \\ 1}\cdot z\suchthat z\in\R}.
\end{equation*}

\part
Gauss's method gives this.
\begin{align*}
  \begin{amat}{4}
    1  &-1 &-1  &0  &3 \\
   -1  &1  &1   &-1 &0 \\
    2  &-2 &1   &-2 &1 \\ 
  \end{amat}
  &\grstep[-2\rho_1+\rho_3]{\rho_1+\rho_2}
  \begin{amat}{4}
    1  &-1 &-1  &0  &3 \\
    0  &0  &0   &-1 &3 \\
    0  &0  &3   &-2 &7 \\ 
  \end{amat}                                   \\
  &\grstep{\rho_2\leftrightarrow\rho_3}
  \begin{amat}{4}
    1  &-1 &-1  &0  &3 \\
    0  &0  &3   &-2 &7 \\ 
    0  &0  &0   &-1 &3 \\
  \end{amat}
\end{align*}
Leading are $a$, $c$, and~$d$.
Free is~$b$.
Expressing $d$ in terms of~$b$ just gives $d=-3$.
Substituting to express $c$ in terms of~$b$ gives 
$3c-2(-3)=7$, and so $c=1/3$.
Finally, to express $a$ in terms of~$b$, substitute
$a-b-(1/3)=3$, and solve: $a=(10/3)+b$.
\begin{equation*}
  \set{\colvec{a \\ b \\ c \\ d}=
         \colvec{10/3 \\ 0 \\ 1/3 \\ -3}+\colvec{1 \\ 1 \\ 0 \\ 0}\cdot b
         \suchthat b\in\R}
\end{equation*}
\end{parts}
\end{solution}


\question
The prior question shows that the solution set of 
\begin{equation*}
 \begin{linsys}{3}
   x  &-  &2y &   &    &=   &3  \\
      &   &2y &-  &3z  &=   &4  \\
  2x  &-  &2y &-  &3z  &=   &10  \\
 \end{linsys}
\end{equation*}
is this.
\begin{equation*}
  \set{\colvec{x \\ y \\ z}=
         \colvec{7 \\ 2 \\ 0}+\colvec{3 \\ 3/2 \\ 1}\cdot z\suchthat z\in\R}.
\end{equation*}
\begin{parts}
\part Give the associated homogenous system.
\part Describe its solution set using vector notation.
\part Find the solution of the original system when $z=1$.
\part Use it to give a different description of the solution set.
\end{parts}
\begin{solution}
\begin{parts}
\part
\begin{equation*}
 \begin{linsys}{3}
   x  &-  &2y &   &    &=   &0  \\
      &   &2y &-  &3z  &=   &0  \\
  2x  &-  &2y &-  &3z  &=   &0  \\
 \end{linsys}
\end{equation*}
\part
\begin{equation*}
  \set{\colvec{x \\ y \\ z}=
         \colvec{3 \\ 3/2 \\ 1}\cdot z\suchthat z\in\R}.
\end{equation*}
\part
\begin{equation*}
  \colvec{7 \\ 2 \\ 0}+\colvec{3 \\ 3/2 \\ 1}\cdot 1=\colvec{10 \\ 5/2 \\ 1}
\end{equation*}
\part
\begin{equation*}
  \set{\colvec{x \\ y \\ z}=
         \colvec{10 \\ 5/2 \\ 1}+\colvec{3 \\ 3/2 \\ 1}\cdot z\suchthat z\in\R}.
\end{equation*}
\end{parts}
\end{solution}


\end{questions}
\end{document}
