\documentclass[noanswers, nolegalese, 11pt]{examjh}
% \documentclass[answers, nolegalese, 11pt]{examjh}
\usepackage{../../../../sty/conc}
\usepackage{../../../../sty/linalgjh}

\setlength{\parindent}{0em}\setlength{\parskip}{0.5ex}
\pagestyle{empty}
\begin{document}\thispagestyle{empty}
\makebox[\textwidth]{Questions for One.I.3\hfill  From \textit{Linear Algebra}, by Hef{}feron}\vspace{-1ex}
\makebox[\textwidth]{\hbox{}\hrulefill\hbox{}}


\begin{questions}
\question
\begin{parts}
\part
What geometric shape is described by this set?
\begin{equation*}
  S=\set{\colvec{x \\ y}=\colvec{3 \\ 2}+\colvec{1 \\ 4}\cdot t\suchthat t\in\R}
\end{equation*}
\part
Set $t=2$ to get another point in this set.
\part 
Use that point to get an alternate description for $S$
(use the letter $\hat{t}$ for the parameter).
\end{parts}
\begin{solution}
\begin{parts}
\part
It is a one-parameter linear set, so it is a line.
\part
$\colvec{5 \\ 10}$
\part
Same slope, just starting at a different place.
\begin{equation*}
  S=\set{\colvec{x \\ y}=\colvec{5 \\ 10}+\colvec{1 \\ 4}\cdot \hat{t}
         \suchthat \hat{t}\in\R}
\end{equation*}
\end{parts}
\end{solution}

\question
\begin{parts}
\part
Use vector notation to describe set of points $(x,y)$ in the line $y=2x+1$.
\part 
The $y$-intercept is $(0,1)$.
In the parametrization from the prior part, 
set $x=1$ to find another point on the line.  
Get another description of the same set by 
replacing the $y$-intercept with this new point.
\end{parts}
\begin{solution}
\begin{parts}
\part
We have this.
\begin{equation*}
  L=\set{\colvec{x \\ y}=\colvec{0 \\ 1}+\colvec{1 \\ 2}\cdot x 
       \suchthat x\in\R}
\end{equation*}
\part
Taking $x=1$ gives $y=3$.
\begin{equation*}
  L=\set{\colvec{x \\ y}=\colvec{1 \\ 3}+\colvec{1 \\ 2}\cdot \hat{x} 
       \suchthat \hat{x}\in\R}
\end{equation*}
\end{parts}
\end{solution}

\question
Gauss's method on this system
\begin{equation*}
 \begin{linsys}{3}
   x  &-  &2y &   &    &=   &3  \\
      &   &2y &-  &3z  &=   &4  \\
  2x  &-  &2y &-  &3z  &=   &10  \\
 \end{linsys}
\end{equation*}
gives this solution set.
\begin{equation*}
  \set{\colvec{x \\ y \\ z}=
         \colvec{7 \\ 2 \\ 0}+\colvec{3 \\ 3/2 \\ 1}\cdot z\suchthat z\in\R}.
\end{equation*}
\begin{parts}
\part Give the associated homogenous system.
\part Describe its solution set using vector notation.
\part Find the solution of the original system when $z=1$.
\part Use it to give a different description of the solution set.
\end{parts}
\begin{solution}
\begin{parts}
% \part
% Here is Gauss's method.
% \begin{equation*}
%   \begin{amat}{3}
%     1  &-2  &0  &3  \\
%     0  &2   &-3 &4  \\
%     2  &-2  &-3 &10
%   \end{amat}
%   \grstep{-2\rho_1+\rho_3}
%   \begin{amat}{3}
%     1  &-2  &0  &3  \\
%     0  &2   &-3 &4  \\
%     2  &2   &-3 &4
%   \end{amat}
%   \grstep{-\rho_2+\rho_3}
%   \begin{amat}{3}
%     1  &-2  &0  &3  \\
%     0  &2   &-3 &4  \\
%     0  &0   &0  &0
%   \end{amat}
% \end{equation*}
% Leading variables are $x$ and~$y$.
% Free is~$z$.
% To express $y$ in terms of~$z$ we get $2y=4+3z$,
% which gives $y=2+(3/2)z$.
% Then to express $x$ in terms of~$z$, substitute
% $x-2(2+(3/2)z)=3$, giving 
% $x=7+3z$.
% \begin{equation*}
%   \set{\colvec{x \\ y \\ z}=
%          \colvec{7 \\ 2 \\ 0}+\colvec{3 \\ 3/2 \\ 1}\cdot z\suchthat z\in\R}.
% \end{equation*}
\part
\begin{equation*}
 \begin{linsys}{3}
   x  &-  &2y &   &    &=   &0  \\
      &   &2y &-  &3z  &=   &0  \\
  2x  &-  &2y &-  &3z  &=   &0  \\
 \end{linsys}
\end{equation*}
\part
\begin{equation*}
  \set{\colvec{x \\ y \\ z}=
         \colvec{3 \\ 3/2 \\ 1}\cdot z\suchthat z\in\R}.
\end{equation*}
\part
\begin{equation*}
  \colvec{7 \\ 2 \\ 0}+\colvec{3 \\ 3/2 \\ 1}\cdot 1=\colvec{10 \\ 5/2 \\ 1}
\end{equation*}
\part
\begin{equation*}
  \set{\colvec{x \\ y \\ z}=
         \colvec{10 \\ 5/2 \\ 1}+\colvec{3 \\ 3/2 \\ 1}\cdot z\suchthat z\in\R}.
\end{equation*}
\end{parts}
\end{solution}


\end{questions}
\end{document}
