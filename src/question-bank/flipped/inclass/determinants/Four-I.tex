% \documentclass[noanswers, nolegalese, 11pt]{examjh}
\documentclass[answers, nolegalese, 11pt]{examjh}
\usepackage{../../../../sty/conc}
\usepackage{../../../../sty/linalgjh}
\usepackage{graphicx}

\setlength{\parindent}{0em}\setlength{\parskip}{0.5ex}
\pagestyle{empty}
\begin{document}\thispagestyle{empty}
\makebox[\textwidth]{Worksheet for Four.I\hfill  From \textit{Linear Algebra}, by Hef{}feron}\vspace{-1ex}
\makebox[\textwidth]{\hbox{}\hrulefill\hbox{}}

The determinant of a $\nbyn{2}$ matrix is
\begin{equation*}
\begin{vmat}
  a  &b  \\
  c  &d
\end{vmat}
=ad-bc
\end{equation*}
and this is the determinant of a $\nbyn{3}$ matrix.
\begin{equation*}
     \begin{vmat}
           a  &b  &c  \\
           d  &e  &f  \\
           g  &h  &i
     \end{vmat}
  =
    aei+bfg+cdh-hfa-idb-gec
\end{equation*}

\begin{questions}
\question
\begin{parts}
\part
Perform Gauss's method on the matrix, 
decide if it is nonsingular,
and then compute the determinant.
\begin{equation*}
\begin{mat}
  1  &3  \\
  2  &4
\end{mat}
\end{equation*}
\begin{solution}
Here is Gauss's method.
\begin{equation*}
\begin{mat}
  1  &3  \\
  2  &4
\end{mat}
\grstep{-2\rho_1+\rho_2}
\begin{mat}
  1  &3  \\
  0  &-2
\end{mat}
\end{equation*}
Because there are no zero rows, the starting matrix is nonsingular.
Here is the determinant. 
\begin{equation*}
\begin{vmat}
  1  &3  \\
  2  &4
\end{vmat}
=1\cdot 4-3\cdot 2=4-6=-2
\end{equation*}
\end{solution}

\part
Do the same to this matrix.
\begin{equation*}
\begin{mat}
  1  &3  \\
  2  &6
\end{mat}
\end{equation*}
\begin{solution}
Here is Gauss's method.
\begin{equation*}
\begin{mat}
  1  &3  \\
  2  &6
\end{mat}
\grstep{-2\rho_1+\rho_2}
\begin{mat}
  1  &3  \\
  0  &0
\end{mat}
\end{equation*}
Because the is a zero row, the starting matrix is singular.
Here is the determinant. 
\begin{equation*}
\begin{vmat}
  1  &3  \\
  2  &6
\end{vmat}
=1\cdot 6-3\cdot 2=0
\end{equation*}
\end{solution}

\part
Perform Gauss's method on the matrix, 
decide if it is nonsingular,
and then compute the determinant.
\begin{equation*}
\begin{mat}
  1  &2  &3 \\
  4  &5  &6 \\
  7  &8  &9
\end{mat}
\end{equation*}
\begin{solution}
Gauss's method is straightforward.
\begin{equation*}
\begin{mat}
  1  &2  &3 \\
  4  &5  &6 \\
  7  &8  &9
\end{mat}
\grstep[-7\rho_1+\rho_3]{-4\rho_1+\rho_2}
\begin{mat}
  1  &2  &3 \\
  0  &-3 &-6 \\
  0  &-6 &-12
\end{mat}
\grstep{-2\rho_2+\rho_3}
\begin{mat}
  1  &2  &3 \\
  0  &-3 &-6 \\
  0  &0  &0
\end{mat}
\end{equation*}
There are zero rows so the starting matrix is singular.
Here is the determinant. 
\begin{equation*}
\begin{vmat}
  1  &2  &3 \\
  4  &5  &6 \\
  7  &8  &9
\end{vmat}
=1\cdot 5\cdot 9+2\cdot 6\cdot 7 + 3\cdot 4 \cdot 8
-7\cdot 5\cdot 3-8\cdot 6\cdot 1-9\cdot 4\cdot 2=0
\end{equation*}
\end{solution}

\part
Do the same to this matrix.
\begin{equation*}
\begin{mat}
  1  &2  &3 \\
  4  &5  &6 \\
  7  &8  &10
\end{mat}
\end{equation*}
\begin{solution}
Gauss's method is much the same as in the prior answer.
\begin{equation*}
\begin{mat}
  1  &2  &3 \\
  4  &5  &6 \\
  7  &8  &10
\end{mat}
\grstep[-7\rho_1+\rho_3]{-4\rho_1+\rho_2}
\begin{mat}
  1  &2  &3 \\
  0  &-3 &-6 \\
  0  &-6 &-11
\end{mat}
\grstep{-2\rho_2+\rho_3}
\begin{mat}
  1  &2  &3 \\
  0  &-3 &-6 \\
  0  &0  &1
\end{mat}
\end{equation*}
There are zero rows so the starting matrix is singular.
Here is the determinant. 
\begin{equation*}
\begin{vmat}
  1  &2  &3 \\
  4  &5  &6 \\
  7  &8  &10
\end{vmat}
=1\cdot 5\cdot 10+2\cdot 6\cdot 7 + 3\cdot 4 \cdot 8
-7\cdot 5\cdot 3-8\cdot 6\cdot 1-10\cdot 4\cdot 2=-3
\end{equation*}
\end{solution}
\end{parts}

\question
Find the determinant of each matrix by reducing it to echelon form, 
keeping track of any row swaps,
and then
multiplying down the diagonal.
\begin{parts}
\part
\begin{equation*}
\begin{mat}
  1  &2  &3 \\
  4  &5  &6 \\
  7  &8  &9
\end{mat}
\end{equation*}
\begin{solution}
Gauss's method is straightforward.
\begin{equation*}
\begin{mat}
  1  &2  &3 \\
  4  &5  &6 \\
  7  &8  &9
\end{mat}
\grstep[-7\rho_1+\rho_3]{-4\rho_1+\rho_2}
\begin{mat}
  1  &2  &3 \\
  0  &-3 &-6 \\
  0  &-6 &-12
\end{mat}
\grstep{-2\rho_2+\rho_3}
\begin{mat}
  1  &2  &3 \\
  0  &-3 &-6 \\
  0  &0  &0
\end{mat}
\end{equation*}
Multiplying down the diagonal gives $0$.
\end{solution}

\part
\begin{equation*}
\begin{mat}
  1  &0  &2  &-1 \\
  0  &2  &-1 &0  \\
  2  &1  &0  &2  \\
  -1 &2  &0  &0  
\end{mat}
\end{equation*}
\begin{solution}
Gauss's method is straightforward.
\begin{align*}
\begin{mat}
  1  &0  &2  &-1 \\
  0  &2  &-1 &0  \\
  2  &1  &0  &2  \\
  -1 &2  &0  &0  
\end{mat}
&\grstep[\rho_1+\rho_4]{-2\rho_1+\rho_3}
\begin{mat}
  1  &0  &2  &-1 \\
  0  &2  &-1 &0  \\
  0  &1  &-4 &4  \\
  0  &2  &2  &-1  
\end{mat}                                     \\
&\grstep[-\rho_2+\rho_4]{-(1/2)\rho_2+\rho_3}
\begin{mat}
  1  &0  &2  &-1 \\
  0  &2  &-1 &0  \\
  0  &0  &-7/2  &4  \\
  0  &0  &3  &-1  
\end{mat}                                    \\
&\grstep{(6/7)\rho_3+\rho_4}
\begin{mat}
  1  &0  &2  &-1 \\
  0  &2  &-1 &0  \\
  0  &0  &-7/2  &4  \\
  0  &0  &0  &17/7  
\end{mat}
\end{align*}
Multiplying down the diagonal gives $-17$.
\end{solution}

\end{parts}

\end{questions}
\end{document}
