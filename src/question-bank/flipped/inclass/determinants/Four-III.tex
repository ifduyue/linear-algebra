% \documentclass[noanswers, nolegalese, 11pt]{examjh}
\documentclass[answers, nolegalese, 11pt]{examjh}
\usepackage{../../../../sty/conc}
\usepackage{../../../../sty/linalgjh}
\usepackage{graphicx}

\setlength{\parindent}{0em}\setlength{\parskip}{0.5ex}
\pagestyle{empty}
\begin{document}\thispagestyle{empty}
\makebox[\textwidth]{Worksheet for Four.III\hfill  From \textit{Linear Algebra}, by Hef{}feron}\vspace{-1ex}
\makebox[\textwidth]{\hbox{}\hrulefill\hbox{}}

To find the roots of a quadratic polynomial, use the 
quadratic formula.
\begin{equation*}
  p(x)=a\cdot x^2+b\cdot x +c
  \qquad
  \frac{-b\pm\sqrt{b^2-4ac}}{2a}
\end{equation*}
The \textit{multiplicity} of a root is the number of times it is repeated.

\begin{questions}
\question
Consider this matrix.
\begin{equation*}
M=
\begin{mat}
  1  &0  &3  \\
  4  &2  &6  \\
  8  &7  &9
\end{mat}
\end{equation*}
\begin{parts}
\part Find the determinant by doing the Laplace expansion along the first
row.
\begin{solution}
\begin{align*}
\begin{vmat}
  1  &0  &3  \\
  4  &2  &6  \\
  8  &7  &9
\end{vmat}
&=1\cdot 
\begin{vmat}
  2  &6  \\
  7  &9 
\end{vmat}
-0\cdot
\begin{vmat}
  4  &6  \\
  8  &9       
\end{vmat}
+3\cdot
\begin{vmat}
  4 &2  \\
  8  &7      
\end{vmat}          \\
&=
1\cdot (18-42)
-0\cdot (36-48)
+3\cdot (28-16)      \\
&=1\cdot (-24)
-0\cdot (-12)
+3\cdot (12)
=12
\end{align*}
\end{solution}

\part Expand along the second row.
\begin{solution}
\begin{align*}
\begin{vmat}
  1  &0  &3  \\
  4  &2  &6  \\
  8  &7  &9
\end{vmat}
&=-4\cdot 
\begin{vmat}
  0  &3  \\
  7  &9 
\end{vmat}
+2\cdot
\begin{vmat}
  1  &3  \\
  8  &9       
\end{vmat}
-6\cdot
\begin{vmat}
  1  &0  \\
  8  &7      
\end{vmat}          \\
&=
-4\cdot (0\cdot 9-3\cdot 7)
+2\cdot (1\cdot 9-3\cdot 8)
-6\cdot (1\cdot 7-0\cdot 8)      \\
&=-4\cdot (-21)
+2\cdot (-15)
-6\cdot (7)
=12
\end{align*}
\end{solution}

\part Expand down the second column.
\begin{solution}
\begin{align*}
\begin{vmat}
  1  &0  &3  \\
  4  &2  &6  \\
  8  &7  &9
\end{vmat}
&=-0\cdot 
\begin{vmat}
  4  &6  \\
  8  &9 
\end{vmat}
+2\cdot
\begin{vmat}
  1  &3  \\
  8  &9       
\end{vmat}
-7\cdot
\begin{vmat}
  1  &3  \\
  4  &6      
\end{vmat}          \\
&=-0\cdot (-12)
+2\cdot (-15)
-7\cdot (-6)
=12
\end{align*}
\end{solution}
\end{parts}


\question
For these polynomials, 
find the roots and their multiplicities.
Use the complex numbers, $\C$, where needed.
\begin{parts}
\part $p_1(x)=(x-5)(x-2)(x+3)$
\begin{solution}
\begin{center}
\begin{tabular}{r|ccc}
  \textit{Root}         &$5$ &$2$ &$-3$ \\
   \cline{2-4}
  \textit{Multiplicity} &$1$ &$1$ &$1$ 
\end{tabular}
\end{center}
\end{solution}

\part $p_2(x)=9\cdot (x-5)(x+4)$
\begin{solution}
\begin{center}
\begin{tabular}{r|cc}
  \textit{Root}         &$5$ &$-4$  \\
   \cline{2-3}
  \textit{Multiplicity} &$1$ &$1$  
\end{tabular}
\end{center}
\end{solution}

\part $p_3(x)=4\cdot (x-(2/3))(x+3)^2$
\begin{solution}
\begin{center}
\begin{tabular}{r|cc}
  \textit{Root}         &$2/3$ &$-3$  \\
   \cline{2-3}
  \textit{Multiplicity} &$1$ &$2$ 
\end{tabular}
\end{center}
\end{solution}

\part $p_4(x)=x^2-7x+10$
\begin{solution}
It factors as $(x-5)(x-2)$.
\begin{center}
\begin{tabular}{r|cc}
  \textit{Root}         &$2$ &$5$  \\
   \cline{2-3}
  \textit{Multiplicity} &$1$ &$1$ 
\end{tabular}
\end{center}
\end{solution}

\part $p_5(x)=x^2+6x+9$
\begin{solution}
It factors into $(x+3)^2$
\begin{center}
\begin{tabular}{r|c}
  \textit{Root}         &$-3$  \\
   \cline{2-2}
  \textit{Multiplicity} &$2$  
\end{tabular}
\end{center}
\end{solution}

\part $p_6(x)=3x^2+x+1$
\begin{solution}
The quadratic formula gives two roots.
\begin{center}
\begin{tabular}{r|cc}
  \textit{Root}         &$\frac{-1+\sqrt{-11}}{6}$ &$\frac{-1-\sqrt{-11}}{6}$  \\
   \cline{2-3}
  \textit{Multiplicity} &$1$ &$1$ 
\end{tabular}
\end{center}
\end{solution}

\end{parts}

\question
Perform the indicated action in $\C$.
\begin{parts}
\part $(3+i)(1+i)$
\begin{solution}
Start by multiplying as you would if $i$ were a variable, to get this.
\begin{equation*}
3\cdot 1+3\cdot i+1\cdot i+i\cdot i
=3+4i+(-1)=2+4i
\end{equation*}
Finish by remembering that $i=\sqrt{-1}$, so $i\cdot i=-1$.
\begin{equation*}
3\cdot 1+3\cdot i+1\cdot i+i\cdot i
=3+4i+(-1)=2+4i
\end{equation*}
\end{solution}

\part $(2-i)(2+i)$
\begin{solution}
Again multiply as you would if $i$ were a variable.
\begin{equation*}
2\cdot 2+2\cdot i+2\cdot(-i)+i\cdot(-i)
\end{equation*}
Finish by using that $i\cdot i=-1$.
\begin{equation*}
=4+2i-2i-(-1)=5
\end{equation*}
\end{solution}

\part $(2+i)^2(1-i)$
\begin{solution}
First combine the first two terms.
\begin{equation*}
(2+i)^2(1-i)=(4+2i+2i+i^2)(1-i)
=(4+4i-1)(1-i)=(3+4i)(1-i)
\end{equation*}
Finish by combining the two.
\begin{equation*}
(3+4i)(1-i)
=3-3i+4i-4i^2
=3+i-4(-1)
=7+i
\end{equation*}
\end{solution}
\end{parts}


\end{questions}
\end{document}
