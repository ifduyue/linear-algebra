\documentclass[noanswers, nolegalese, 11pt]{examjh}
% \documentclass[answers, nolegalese, 11pt]{examjh}
\usepackage{../../../../sty/conc}
\usepackage{../../../../sty/linalgjh}
\usepackage{graphicx}

\setlength{\parindent}{0em}\setlength{\parskip}{0.5ex}
\pagestyle{empty}
\begin{document}\thispagestyle{empty}
\makebox[\textwidth]{Worksheet for Four.II\hfill  From \textit{Linear Algebra}, by Hef{}feron}\vspace{-1ex}
\makebox[\textwidth]{\hbox{}\hrulefill\hbox{}}

In $\Re^n$
the \definend{box}\index{box}
(or \definend{parallelepiped}\index{parallelepiped})
formed by 
\( \sequence{\vec{v}_1,\dots,\vec{v}_n} \) 
is this set
\begin{equation*}
  \set{t_1\vec{v}_1+\dots+t_n\vec{v}_n
      \suchthat t_1,\ldots,t_n\in \closedinterval{0}{1}} 
\end{equation*}
(this is like the span but the coefficient $t_i$'s are restricted
to being in the interval $[0\ldots 1]$).

The \definend{size} of a box is the determinant of the matrix
with the vectors as columns.
The absolute value of the size is the box's volume, and the sign
is its orientation.


\begin{questions}
\question
Draw the box defined by each list of vectors.
Find its size, and its volume and orientation.
\begin{parts}
\part 
$\sequence{
  \colvec{3 \\ 2},
  \colvec{-1\\ 4}
}$
\part 
$\sequence{
  \colvec{-1\\ 4},
  \colvec{3 \\ 2}
}$
\part 
$\sequence{
  \colvec{2\\ -1},
  \colvec{1 \\-3}
}$
\part 
$\sequence{
  \colvec{2 \\ 0 \\ 2},
  \colvec{0 \\ 3 \\ 1},
  \colvec{-1\\ 0 \\ 1}
}$
\end{parts}

\question
Consider the map~$\map{t}{\R^2}{\R^2}$ 
represented with respect to the standard basis by this
matrix.
\begin{equation*}
  T=
  \begin{mat}
  1  &1  \\
  -2  &0
  \end{mat}
\end{equation*}
\begin{parts}
\part
Compute the effect of that map, $t(\vec{v}_1)=\vec{w}_1$
and $t(\vec{v}_2)=\vec{w}_2$,
on the two vectors in this sequence.
\begin{equation*}
\sequence{
  \colvec{3 \\ 2}
  \colvec{-1\\ 4}
}
\end{equation*}
\part
Do the same for this sequence.
$\sequence{
  \colvec{2\\ -1},
  \colvec{1 \\-3}
}$
\part
For the results of the two prior parts, 
compute the size of the box defined by $\sequence{\vec{w}_1,\vec{w}_2}$.
\part
For those two cases,
take the ratios $\text{size of the output box}/\text{size of input box}$.
\end{parts}

\question
By what factor does each transformation change the size of
boxes?
\begin{parts}
\part $\colvec{x \\ y}\mapsto\colvec{2x \\ 3y}$
\begin{solution}
Express the transformation with respect to the standard bases
and find the determinant.
\begin{equation*}
\begin{vmat}
  2  &0  \\
  0  &3
\end{vmat}=6
\end{equation*}
\end{solution}
\part $\colvec{x \\ y \\ z}\mapsto\colvec{x-y \\ x+y+z \\ y-2z}$
\begin{solution}
\begin{equation*}
\begin{vmat}
1  &-1  &0  \\
1  &1   &1  \\
0  &1   &-2
\end{vmat}=-5
\end{equation*}
\end{solution}
\end{parts}


\end{questions}
\end{document}
