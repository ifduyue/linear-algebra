% \documentclass[noanswers, nolegalese, 11pt]{examjh}
\documentclass[answers, nolegalese, 11pt]{examjh}
\usepackage{../../../../sty/conc}
\usepackage{../../../../sty/linalgjh}
\usepackage{graphicx}

\setlength{\parindent}{0em}\setlength{\parskip}{0.5ex}
\pagestyle{empty}
\begin{document}\thispagestyle{empty}
\makebox[\textwidth]{Worksheet for Four.II\hfill  From \textit{Linear Algebra}, by Hef{}feron}\vspace{-1ex}
\makebox[\textwidth]{\hbox{}\hrulefill\hbox{}}

In $\Re^n$
the \definend{box}\index{box}
(or \definend{parallelepiped}\index{parallelepiped})
formed by 
\( \sequence{\vec{v}_1,\dots,\vec{v}_n} \) 
is this set
\begin{equation*}
  \set{t_1\vec{v}_1+\dots+t_n\vec{v}_n
      \suchthat t_1,\ldots,t_n\in \closedinterval{0}{1}} 
\end{equation*}
(this is like the span but the coefficient $t_i$'s are restricted
to being in the interval $[0\ldots 1]$).

The \definend{size} of a box is the determinant of the matrix
with the vectors as columns.
The absolute value of the size is the box's volume, and the sign
is its orientation.


\begin{questions}
\question
Draw the box defined by each list of vectors.
Find its size, and its volume and orientation.
\begin{parts}
\part 
$\sequence{
  \colvec{3 \\ 2},
  \colvec{-1\\ 4}
}$
\begin{solution}
Here is the box.
\begin{center}
  \includegraphics{asy/det000.pdf}
\end{center}
The size is the determinant.
\begin{equation*}
\begin{vmat}
  3  &-1  \\
  2  &4
\end{vmat}=3\cdot 4-(-1)\cdot 2=14
\end{equation*}
The volume is the absolute value of the size, so $\text{volume}=14$,
and the orientation is positive, `$+$'.
Note the meaning of that orientation: to get from the first vector to the
second requires going counterclockwise, as with the blue arrow.
\end{solution}

\part 
$\sequence{
  \colvec{-1\\ 4},
  \colvec{3 \\ 2}
}$
\begin{solution}
Here is the box, and the size.
\begin{equation*}
  \vcenteredhbox{\includegraphics{asy/det001.pdf}}
  \qquad
\begin{vmat}
  -1  &3  \\
   4  &2
\end{vmat}=-14
\end{equation*}
The volume is the absolute value of the size, so $\text{volume}=14$,
and the orientation is negative, `$-$'.
Here, to get from the first vector to the
second, with the blue arrow, requires going clockwise.
\end{solution}

\part 
$\sequence{
  \colvec{2\\ -1},
  \colvec{1 \\-3}
}$
\begin{solution}
The box is different, and the size.
\begin{equation*}
  \vcenteredhbox{\includegraphics{asy/det002.pdf}}
  \qquad
\begin{vmat}
   2  &1  \\
   -1  &-3
\end{vmat}=-5
\end{equation*}
The volume is $\text{volume}=5$,
and the orientation is negative, `$-$'.
\end{solution}

\part 
$\sequence{
  \colvec{2 \\ 0 \\ 2},
  \colvec{0 \\ 3 \\ 1},
  \colvec{-1\\ 0 \\ 1}
}$
\begin{solution}
Here is the box and the size.
\begin{equation*}
  \vcenteredhbox{\includegraphics{asy/det004.pdf}}
  \qquad
\begin{vmat}
   2  &0  &-1  \\
   0  &3  &0   \\
   2  &1  &1
\end{vmat}=12
\end{equation*}
The volume is $\text{volume}=12$
and the orientation is positive, `$+$'.
\end{solution}
\end{parts}

\question
Consider the map~$\map{t}{\R^2}{\R^2}$ 
represented with respect to the standard basis by this
matrix.
\begin{equation*}
  T=
  \begin{mat}
  1  &1  \\
  -2  &0
  \end{mat}
\end{equation*}
\begin{parts}
\part
Compute the effect of that map, $t(\vec{v}_1)=\vec{w}_1$
and $t(\vec{v}_2)=\vec{w}_2$,
on the two vectors in this sequence.
\begin{equation*}
\sequence{
  \colvec{3 \\ 2}
  \colvec{-1\\ 4}
}
\end{equation*}
\begin{solution}
We are doing the computation 
$\rep{\vec{v}}{\stdbasis_2,\stdbasis_2}\cdot\rep{\vec{v}}{\stdbasis_2}=\rep{t(\vec{v})}{\stdbasis_2}$.
\begin{equation*}
\begin{mat}
  1  &1  \\
 -2  &0
\end{mat}
\colvec{3 \\ 2}
=
\colvec{5 \\ 6}
\end{equation*}
The other is similar.
\begin{equation*}
\begin{mat}
  1  &1  \\
 -2  &0
\end{mat}
\colvec{-1 \\ 4}
=
\colvec{3 \\ 2}
\end{equation*}
\end{solution}

\part
Do the same for this sequence.
$\sequence{
  \colvec{2\\ -1},
  \colvec{1 \\-3}
}$
\begin{solution}
\begin{equation*}
\begin{mat}
  1  &1  \\
 -2  &0
\end{mat}
\colvec{2 \\ -1}
=
\colvec{1 \\ -4}
\qquad
\begin{mat}
  1  &1  \\
 -2  &0
\end{mat}
\colvec{1 \\ -3}
=
\colvec{-2 \\ -2}
\end{equation*}
\end{solution}

\part
For the results of the two prior parts, 
compute the size of the box defined by $\sequence{\vec{w}_1,\vec{w}_2}$.
\begin{solution}
\begin{equation*}
\begin{vmat}
  5  &3  \\
  -6 &2
\end{vmat}=28
\qquad
\begin{vmat}
  1  &-2  \\
  -4 &-2
\end{vmat}=-10
\end{equation*}
\end{solution}

\part
For those two cases,
take the ratios $\text{size of the output box}/\text{size of input box}$.
\begin{solution}
We have this.
\begin{center}
\begin{tabular}{cc|cc}
  \multicolumn{1}{c}{\textit{Input box}} 
     &\textit{Size} &\textit{Output box} &\textit{Size} \\
  \hline
   $\sequence{
     \colvec{3 \\ 2},
     \colvec{-1 \\ 4}
     }$
   &$14$
   &$\sequence{
     \colvec{5 \\ -6},
     \colvec{3 \\ 2}
     }$
   &$28$           \\
   $\sequence{
     \colvec{2 \\ -1},
     \colvec{1 \\ -3}
     }$
   &$-5$
   &$\sequence{
     \colvec{1 \\ -4},
     \colvec{-2 \\ -2}
     }$
   &$-10$           \\
   
\end{tabular}
\end{center}
Of course, they double.
Note this.
\begin{equation*}
\begin{vmat}
  1  &1  \\
  -2 &0
\end{vmat}=2
\end{equation*}
\end{solution}
\end{parts}

% \question
% By what factor does each transformation change the size of
% boxes?
% \begin{parts}
% \part $\colvec{x \\ y}\mapsto\colvec{2x \\ 3y}$
% \begin{solution}
% Express the transformation with respect to the standard bases
% and find the determinant.
% \begin{equation*}
% \begin{vmat}
%   2  &0  \\
%   0  &3
% \end{vmat}=6
% \end{equation*}
% \end{solution}
% \part $\colvec{x \\ y \\ z}\mapsto\colvec{x-y \\ x+y+z \\ y-2z}$
% \begin{solution}
% \begin{equation*}
% \begin{vmat}
% 1  &-1  &0  \\
% 1  &1   &1  \\
% 0  &1   &-2
% \end{vmat}=-5
% \end{equation*}
% \end{solution}
% \end{parts}


\end{questions}
\end{document}
