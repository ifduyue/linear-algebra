%\documentclass[noanswers, nolegalese, 11pt]{examjh}
\documentclass[answers, nolegalese, 11pt]{examjh}
\usepackage{../../../../sty/conc}
\usepackage{../../../../sty/linalgjh}

\setlength{\parindent}{0em}\setlength{\parskip}{0.5ex}
\pagestyle{empty}
\begin{document}\thispagestyle{empty}
\makebox[\textwidth]{Worksheet for Two.II\hfill  From \textit{Linear Algebra}, by Hef{}feron}\vspace{-1ex}
\makebox[\textwidth]{\hbox{}\hrulefill\hbox{}}


\begin{questions}
\question
For each space, find a basis.
Verify that it is a basis.
\begin{parts}
\part  
$S_1=\set{ax^2+bx+c\suchthat a-c=0}$
\part
$S_2=\set{\colvec{x \\ y \\ z}\suchthat \text{$x+2y=0$ and $x+y-z=0$}}$
\end{parts}
\begin{solution}
\begin{parts}
\part
We have $S_1=\set{cx^2+bx+c\suchthat b,c\in\R}$.
That equals $\set{c\cdot(x^1+1)+b\cdot(x)\suchthat b,c\in\R}$.
A spanning set is $\set{x^2+1,x}$.

It is linearly independent
because the relationship
\begin{equation*}
  c_0(x^2+1)+c_1(x)=0x^2+0x+0
\end{equation*}
gives that $c_0=c_1=0$.
Thus it is a basis for $S_1$.
\end{parts}

\end{solution}

\end{questions}
\end{document}
