\documentclass[noanswers, nolegalese, 11pt]{examjh}
% \documentclass[answers, nolegalese, 11pt]{examjh}
\usepackage{../../../../sty/conc}
\usepackage{../../../../sty/linalgjh}

\setlength{\parindent}{0em}\setlength{\parskip}{0.5ex}
\pagestyle{empty}
\begin{document}\thispagestyle{empty}
\makebox[\textwidth]{Worksheet for Two.II\hfill  From \textit{Linear Algebra}, by Hef{}feron}\vspace{-1ex}
\makebox[\textwidth]{\hbox{}\hrulefill\hbox{}}


\begin{questions}
\question
The first chapter informally describes that the
bottom row in this matrix is ``redundant'' in a linear sense because
twice the first row plus the second equals the third.
\begin{equation*}
\begin{mat}
  1  &0  &2  &1  \\
  0  &3  &1  &0  \\
  2  &3  &5  &2
\end{mat}
\end{equation*}
We now have the formal terminology:~this set
is linearly dependent.
\begin{equation*}
  \set{
       \rowvec{1 & 0 & 2 & 1},
       \rowvec{0 & 3 & 1 & 0},
       \rowvec{2 & 3 & 5 & 2}
      }
\end{equation*}
Verify that by 
determining whether there is a linear relationship among the row vectors, 
whether there are coefficients for 
$c_1\vec{v}_1+c_2\vec{v}_2+c_3\vec{v}_3=\vec{0}$
other than the trivial ones, $c_1=c_2=c_3=0$.
\begin{solution}
From 
\begin{equation*}
       c_1\cdot\rowvec{1 & 0 & 2 & 1}
       +c_2\cdot\rowvec{0 & 3 & 1 & 0}
       +c_3\cdot\rowvec{2 & 3 & 5 & 2}
       =\rowvec{0 &0 &0 &0}
\end{equation*}
the relationship among first entries, second entries, etc., gives this. 
\begin{equation*}
\begin{linsys}{3}
  c_1  &  &      &+  &2c_3  &= &0 \\
       &  &3c_2  &+  &3c_3  &= &0 \\
 2c_1  &+ &c_2   &+  &5c_3  &= &0 \\
  c_1  &  &      &+  &2c_3  &= &0 \\
\end{linsys}
\end{equation*}
Gauss's method does not give a unique solution, so the set is linearly
dependent.
\begin{equation*}
  \begin{mat}
  1  &0  &2  \\ 
  0  &3  &3  \\ 
  2  &1  &5  \\ 
  1  &0  &2  \\ 
\end{mat}
\grstep[-1\rho_{1}+\rho_{4}]{-2\rho_{1}+\rho_{3}}
\begin{mat}
  1  &0  &2  \\ 
  0  &3  &3  \\ 
  0  &1  &1  \\ 
  0  &0  &0  \\ 
\end{mat}                                  
\grstep{(-1/3)\rho_{2}+\rho_{3}}
\begin{mat}
  1  &0  &2  \\ 
  0  &3  &3  \\ 
  0  &0  &0  \\ 
  0  &0  &0  \\ 
\end{mat}
\end{equation*}
\end{solution}

\question 
Find whether this subset of $\polyspace_2$ 
is linearly dependent or linearly independent.
\begin{equation*}
  \set{
       x^2-x+1,\, 
       3x+4,\,
       2x^2-x
      }
\end{equation*}
\begin{solution}
Looking for a linear relationship
\begin{equation*}
       c_1(x^2-x+1) 
       +c_2(3x+4)
       +c_3(2x^2-x)
       =0x^2+0x+0
\end{equation*}
leads to this linear system.
\begin{equation*}
\begin{linsys}{3}
  c_1 &  &     &+  &2c_3  &=  &0  \\
 -c_1 &+ &3c_2 &-  &c_3   &=  &0  \\
  c_1 &+ &4c_2 &   &      &=  &0
\end{linsys}
\end{equation*}
Gauss's method tells us that the only solution is the trivial one.
% sage: M = matrix(QQ, [[1,0,2], [-1,3,-1], [1,4,0]])
% sage: v = vector(QQ, [0,0,0])
% sage: M_prime =M.augment(v, subdivide=True)
% sage: M_prime
% [ 1  0  2| 0]
% [-1  3 -1| 0]
% [ 1  4  0| 0]
\begin{equation*}
\begin{amat}{3}
  1  &0  &2  &0  \\ 
  -1  &3  &-1  &0  \\ 
  1  &4  &0  &0  \\ 
\end{amat}
\grstep[-1\rho_{1}+\rho_{3}]{1\rho_{1}+\rho_{2}}
\begin{amat}{3}
  1  &0  &2  &0  \\ 
  0  &3  &1  &0  \\ 
  0  &4  &-2  &0  \\ 
\end{amat}
\grstep{(-4/3)\rho_{2}+\rho_{3}}
\begin{amat}{3}
  1  &0  &2  &0  \\ 
  0  &3  &1  &0  \\ 
  0  &0  &-10/3  &0  \\ 
\end{amat}
\end{equation*}
So the set is linearly independent.
\end{solution}

\question
Is this set linearly independent?
\begin{equation*}
  \set{
    \begin{mat}
      0 &4  \\
      -2 &0  \\
    \end{mat},
    \begin{mat}
      3 &1  \\
      1 &0  \\
    \end{mat},
    \begin{mat}
     1  &0  \\
     1  &0  \\
    \end{mat}
  }
\end{equation*}
\begin{solution}
Set up 
\begin{equation*}
    c_1\cdot\begin{mat}
      0 &4  \\
      -2 &0  \\
    \end{mat}
    +c_2\cdot\begin{mat}
      3 &1  \\
      1 &0  \\
    \end{mat}
    +c_3\cdot\begin{mat}
     1  &0  \\
     1  &0  \\
    \end{mat}
    =
    \begin{mat}
    0 &0 \\
    0 &0
    \end{mat}
\end{equation*}
to get this linear system.
\begin{equation*}
\begin{linsys}{4}
        &  &3c_2  &+  &c_3  &=  &0  \\
  4c_1  &+ &c_2   &   &     &=  &0  \\
  -2c_1 &+ &c_2   &+  &c_3  &=  &0  \\
        &  &      &   &0    &=  &0  \\
\end{linsys}
\end{equation*}
% sage: M = matrix(QQ, [[0,3,1], [4,1,0], [-2,1,1], [0,0,0]])
% sage: v = vector(QQ, [0,0,0,0])
% sage: M_prime =M.augment(v, subdivide=True)
Apply Gauss's method.
\begin{align*}
  \begin{amat}{3}
  0  &3  &1  &0  \\ 
  4  &1  &0  &0  \\ 
  -2  &1  &1  &0  \\ 
  0  &0  &0  &0  \\ 
\end{amat}
&\grstep{\rho_{1}\leftrightarrow \rho_{2}}
\begin{amat}{3}
  4  &1  &0  &0  \\ 
  0  &3  &1  &0  \\ 
  -2  &1  &1  &0  \\ 
  0  &0  &0  &0  \\ 
\end{amat}                                \\
&\grstep{(1/2)\rho_{1}+\rho_{3}}
\begin{amat}{3}
  4  &1  &0  &0  \\ 
  0  &3  &1  &0  \\ 
  0  &3/2  &1  &0  \\ 
  0  &0  &0  &0  \\ 
\end{amat}                              \\
&\grstep{(-1/2)\rho_{2}+\rho_{3}}
\begin{amat}{3}
  4  &1  &0  &0  \\ 
  0  &3  &1  &0  \\ 
  0  &0  &1/2  &0  \\ 
  0  &0  &0  &0  \\ 
\end{amat}
\end{align*}
There is a unique solution, so the set is linearly independent.
\end{solution}

\end{questions}
\end{document}
