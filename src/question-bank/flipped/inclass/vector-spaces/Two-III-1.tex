% \documentclass[noanswers, nolegalese, 11pt]{examjh}
\documentclass[answers, nolegalese, 11pt]{examjh}
\usepackage{../../../../sty/conc}
\usepackage{../../../../sty/linalgjh}
\usepackage{graphicx}

\setlength{\parindent}{0em}\setlength{\parskip}{0.5ex}
\pagestyle{empty}
\begin{document}\thispagestyle{empty}
\makebox[\textwidth]{Worksheet for Two.III.1\hfill  From \textit{Linear Algebra}, by Hef{}feron}\vspace{-1ex}
\makebox[\textwidth]{\hbox{}\hrulefill\hbox{}}


\begin{questions}
\question
For each space, find a basis.
Verify that it is a basis.
\begin{parts}
\part  
$S_1=\set{ax^2+bx+c\suchthat a-c=0}$
\part
$S_2=\set{\colvec{x \\ y \\ z}\suchthat \text{$x+2y=0$ and $x+y-z=0$}}$
\end{parts}
\begin{solution}
\begin{parts}
\part
We have $S_1=\set{cx^2+bx+c\suchthat b,c\in\R}$.
That equals $\set{c\cdot(x^1+1)+b\cdot(x)\suchthat b,c\in\R}$.
A spanning set is $\set{x^2+1,x}$.

It is linearly independent
because the relationship
\begin{equation*}
  c_0(x^2+1)+c_1(x)=0x^2+0x+0
\end{equation*}
gives that $c_0=c_1=0$.
Thus it is a basis for $S_1$.

\part
The two restrictions give this linear system.
\begin{equation*}
\begin{linsys}{3}
  x  &+  &2y  &    &   &=  &0  \\
  x  &+  &y   &-   &z   &=  &0  
\end{linsys}
\grstep{-\rho_1+\rho_2}
\begin{linsys}{3}
  x  &+  &2y  &    &   &=  &0  \\
     &   &-y  &-   &z  &=  &0  
\end{linsys}
\end{equation*}
Leading are $x$ and~$y$.
Free is~$z$.
Expressing the leading variables in terms of the free variable gives this.
\begin{equation*}
  S_2=\set{\colvec{2z  \\ -z \\ z} \suchthat z\in\R}
     =\set{\colvec{2  \\ -1 \\ 1}\cdot z \suchthat z\in\R}
\end{equation*}
So this is a spanning set.
\begin{equation*}
  \set{\colvec{2  \\ -1 \\ 1}}
\end{equation*}

To check that this set is a basis, we need only check that it is 
linearly independent.
A set with only one vector is linearly independent unless that single
vector is the zero vector, which this one isn't.  
So it is a basis.
\end{parts}
\end{solution}

\question
For the vector space
\begin{equation*}
  V=\set{ax^2+bx+c\suchthat a-c=0}
\end{equation*}
this set is a basis.
\begin{equation*}
  B=\set{x^2+1,x}
\end{equation*}
\begin{parts}
\part Where $\vec{v}=3x^2+2x+3$, find $\rep{\vec{v}}{B}$.
\part Find $\rep{-x^2+4x-1}{B}$.
\end{parts}
\begin{solution}
\begin{parts}
\part
We find what coefficients there are in this relationship.
\begin{equation*}
  3x^2+2x+3=c_1\cdot(x+2+1)+c_2(x)
\end{equation*}
Obviously, $c_1=3$ and $c_2=2$ work here.
\begin{equation*}
  \rep{\vec{v}}{B}
  =\colvec{3 \\ 2}
\end{equation*}

\part
We find what coefficients there are in this relationship.
\begin{equation*}
  -x^2+4x-1=c_1\cdot(x+2+1)+c_2(x)
\end{equation*}
By eye, $c_1=-1$ and $c_2=4$.
\begin{equation*}
  \rep{\vec{v}}{B}
  =\colvec{-1 \\ 4}
\end{equation*}
\end{parts}
\end{solution}

\question
This is a basis for the $xy$~plane.
\begin{equation*}
  B=\set{
    \colvec{1 \\ 3},
    \colvec{2 \\ 0}
  }
\end{equation*}
\begin{parts}
\part
Draw the plane, and plot the basis vectors.
Also draw the line through each. 
\part  
Draw
\begin{equation*}
  \vec{v}_1=\colvec{3 \\ 6}
\end{equation*}
and find $\rep{\vec{v}_1}{B}$.
Label $\vec{v}_1$ with the signs of the coefficients, such as ``$+,+$'' or ``$+,-$''.
\begin{solution}
From
\begin{equation*}
  \colvec{3 \\ 6}=c_1\colvec{1 \\ 3}+c_2\colvec{2 \\ 0}
\end{equation*}
we can see by eye that $c_1=2$ and $c_2=1/2$.
\begin{center}
  \includegraphics{asy/vs001.pdf}
\end{center}
\end{solution}
\part
Do the same for  
\begin{equation*}
  \vec{v}_2=\colvec{4 \\ -6}
\end{equation*}
\begin{solution}
Again by eye, we have
\begin{equation*}
  \rep{\vec{v}_2}{B}=\colvec{-2 \\ 3}
\end{equation*}
\begin{center}
  \includegraphics{asy/vs002.pdf}
\end{center}
\end{solution}
\part 
Repeat for 
\begin{equation*}
  \vec{v}_3=\colvec{-7 \\ -3}
  \qquad
  \vec{v}_4=\colvec{-1 \\ 9}
\end{equation*}
\begin{solution}
We have
\begin{equation*}
  \rep{\vec{v}_3}{B}=\colvec{-1 \\ -3}
  \qquad
  \rep{\vec{v}_4}{B}=\colvec{3 \\ -2}
\end{equation*}
\begin{center}
  \includegraphics{asy/vs003.pdf}
  \qquad
  \includegraphics{asy/vs004.pdf}
\end{center}
Overall, this basis for the plane defines a coordinate system.
\begin{center}
  \includegraphics{asy/vs005.pdf}
\end{center}
\end{solution}
\end{parts}
\end{questions}
\end{document}
