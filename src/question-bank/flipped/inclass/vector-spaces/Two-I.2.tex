% \documentclass[noanswers, nolegalese, 11pt]{examjh}
\documentclass[answers, nolegalese, 11pt]{examjh}
\usepackage{../../../../sty/conc}
\usepackage{../../../../sty/linalgjh}

\setlength{\parindent}{0em}\setlength{\parskip}{0.5ex}
\pagestyle{empty}
\begin{document}\thispagestyle{empty}
\makebox[\textwidth]{Questions for Two.I.2\hfill  From \textit{Linear Algebra}, by Hef{}feron}\vspace{-1ex}
\makebox[\textwidth]{\hbox{}\hrulefill\hbox{}}


\begin{questions}
\question
Consider this subset of the collection of all quadratic polynomials,
$\polyspace_2$.
\begin{equation*}
  S=\set{px^2+qx+r\in\polyspace_2\suchthat p+2q-r=0}
\end{equation*}
\begin{parts}
\part
List two elements and two non-elements.
\begin{solution}
Here are two elements.
They satisfy the requirement that the coefficient of~$x^2$ plus twice
the coefficient of~$x$ minus the constant term equals zero.
\begin{equation*}
  x^2+2x+5
  \qquad
  2x^2+2
\end{equation*}
Here are two non-elements.
\begin{equation*}
  x^2+2x-4
  \qquad
  x+2
\end{equation*}
\end{solution}

\part
Show that it is closed under addition and also closed under
scalar multiplication.
\begin{solution}
Fix $\vec{v},\vec{w}\in S$ and take
\begin{equation*}
  \vec{v}=v_2x^2+v_1x+v_0
  \qquad
  \vec{w}=w_2x^2+w_1x+w_0
\end{equation*}
where $v_2+2v_1-v_0=0$ and  $w_2+2w_1-w_0=0$.
The sum of the two is
\begin{equation*}
  \vec{v}+\vec{w}=(v_2+w_2)\cdot x^2
                  +(v_1+w_1)\cdot x
                  +(v_0+w_0)
\end{equation*}
and $(v_2+w_2)+2(v_1+w_1)-(v_0+w_0)=(v_2+2v_1-v_0)+(w_2+2w_1-w_0)=0$.
Thus $\vec{v}+\vec{w}\in S$.

For closure under scalar multiplication, take $r\in \R$.
This is the scalar multiple of the vector.
\begin{equation*}
  r\vec{v}=r(v_2x^2+v_1x+v_0)
          =(rv_2)\cdot x^2+(rv_1)\cdot x+(rv_0)
\end{equation*}
Because $rv_2+2rv_1-rv_0=r\cdot(v_2+2v_1-v_0)=0$, the set $S$
is closed under scalar multiplication.
\end{solution}

\part
Show that it is closed under linear combinations of two elements.
\begin{solution}
Fix $\vec{v}=v_2x^2+v_1x+v_0$ and 
$\vec{w}=w_2x^2+w_1x+w_0$,
where $v_2+2v_1-v_0=0$ and  $w_2+2w_1-w_0=0$.
Let $r,s\in\R$.
Then $r\vec{v}+s\vec{w}=(rv_2)x^2+(rv_1)x+(rv_0)+(sw_2)x^2+(sw_1)x+(sw_0)$,
and that equals $(rv_2+sw_2)x^2+(rv_1+sw_1)x+(rv_0+sw_0)$.
Now, $(rv_2+sw_2)+2(rv_1+sw_1)-(rv_0+sw_0)$ equals
$(rv_2+2rv_1-rv_0)+(sw_2+2sw_1-+sw_0)=0$,
so the linear combination is also a member of~$S$.
\end{solution}
\end{parts}

\question
Each of these is a subspace.
Parametrize each to find a spanning set.
\begin{parts}
\part This is a subspace of $\polyspace_2$.
\begin{equation*}
  S_1=\set{px^2+qx+r\in\polyspace_2\suchthat p+2q-r=0}
\end{equation*}
\begin{solution}
Consider $p+2q-r=0$ to be a linear system with one equation.
Then writing the leading variables in terms of those that are free gives
$p=-2q+r$.
We have 
\begin{align*}
  S_1 &=\set{(-2q+r)\cdot x^2+q\cdot x +r \suchthat q,r\in\R}   \\
      &=\set{(-2x^2+x)\cdot q+(x^2+1)\cdot r\suchthat q,r\in\R}
\end{align*}
The spanning set has two elements, $\set{-2x^2+x, x^2+1}$.
\end{solution}
\part This is a subspace of $\matspace_{\nbyn{2}}$.
\begin{equation*}
  S_2=\set{
    \begin{mat}
      a  &b  \\
      c  &d
    \end{mat} \suchthat \text{$a-2b=0$ and $2c+d=0$}}
\end{equation*}
\begin{solution}
The restrictions make a linear system.
\begin{equation*}
\begin{linsys}{4}
  a  &-  &2b  &  &  &  &  &=  &0  \\
     &   &    &  &2c&+ &d &=  &0  
\end{linsys}
\end{equation*}
Writing its leading variables in terms of its free variables
gives $a=2b$ and $c=(-1/2)d$.
\begin{align*}
  S_2 &= \set{
           \begin{mat}
             2b   &b  \\
             -d/2 &d \\
           \end{mat}
           \suchthat b,d\in\R }    \\
       &=\set{
         \begin{mat}
           2  &1  \\
           0  &0
         \end{mat}\cdot b
         +
         \begin{mat}
           0  &0  \\
        -1/2  &1      
         \end{mat}\cdot d
         \suchthat b,d\in\R}
\end{align*}
Here is a spanning set.
\begin{equation*}
  \set{
         \begin{mat}
           2  &1  \\
           0  &0
         \end{mat},
         \begin{mat}
           0  &0  \\
        -1/2  &1      
         \end{mat}
       }
\end{equation*}
\end{solution}

\part This is a subspace of $\R^3$.
\begin{equation*}
  S_3=\set{\colvec{x \\ y \\ z}\suchthat 2x-z=0}
\end{equation*}
\begin{solution}
Here also, consider the restriction $2x-z=0$ to be a one-equation linear
system, to get that $x=(1/2)z$.
But don't miss that also free is~$y$.
\begin{align*}
  S_3 &=\set{\colvec{z/2  \\ y \\ z}\suchthat y,z\in\R}    \\
      &=\set{\colvec{0 \\ 1 \\ 0}\cdot y+\colvec{1/2 \\ 0 \\ 1}\cdot z
        \suchthat y,z\in\R}
\end{align*}
With that, a spanning set is this.
\begin{equation*}
  \set{\colvec{0 \\ 1 \\ 0}, \colvec{1/2 \\ 0 \\ 1}}
\end{equation*}
\end{solution}

\part Another subspace of $\R^3$.
\begin{equation*}
  S_4=\set{\colvec{x \\ y \\ z}\suchthat \text{$x+y-z=0$ and $2x+2y=0$}}
\end{equation*}
\begin{solution}
Finally, a linear system that requires some of Gauss's method.
\begin{equation*}
\begin{linsys}{3}
  x &+  &y  &-  &z  &=  &0  \\
 2x &+  &2y &   &   &=  &0
\end{linsys}
\grstep{-2\rho_1+\rho_2}
\begin{linsys}{3}
  x &+  &y  &-  &z    &=  &0  \\
    &   &   &   &2z   &=  &0
\end{linsys}
\end{equation*}
Conclusion: leading are $x$ and~$z$, while $y$ is free.
The restriction is that $x=-y$ and $z=0$.
\begin{equation*}
  S_4 =  \set{\colvec{-y  \\ y  \\ 0}\suchthat y\in\R}  
      =  \set{\colvec{-1 \\ 1 \\ 0}\cdot y\suchthat y\in\R}
\end{equation*}
This set is a line through the origin. 
This is a spanning set.
\begin{equation*}
  \set{\colvec{-1 \\ 1 \\ 0}}
\end{equation*}

\end{solution}
\end{parts}


\end{questions}
\end{document}
