% \documentclass[noanswers, nolegalese, 11pt]{examjh}
\documentclass[answers, nolegalese, 11pt]{examjh}
\usepackage{../../../../sty/conc}
\usepackage{../../../../sty/linalgjh}
\usepackage{graphicx}

\setlength{\parindent}{0em}\setlength{\parskip}{0.5ex}
\pagestyle{empty}
\begin{document}\thispagestyle{empty}
\makebox[\textwidth]{Worksheet for Three.IV\hfill  From \textit{Linear Algebra}, by Hef{}feron}\vspace{-1ex}
\makebox[\textwidth]{\hbox{}\hrulefill\hbox{}}

Fix the vector spaces $V=\R^2$, and $W=\R^2$, and $X=\R^2$.
Fix this vector.
\begin{equation*}
  \vec{v}=\colvec{3 \\ 0}
\end{equation*}
Fix these bases.
\begin{equation*}
  B_V=\sequence{\colvec{1 \\ 1}, \colvec{1 \\ 0}}
  \quad
  B_W=\sequence{\colvec{1 \\ -1}, \colvec{2 \\ 0}}
  \quad
  B_X=\sequence{\colvec{0 \\  1}, \colvec{2 \\ 2}}
\end{equation*}
Also fix two linear maps.
\begin{equation*}
  f(\colvec{a \\ b}) = \colvec{a+b \\ 2b}
  \qquad
  h(\colvec{a \\ b}) = \colvec{a-b \\ a+b}
\end{equation*}


\begin{questions}
\question
We first represent the two maps with respect to the same pair of bases.
\begin{parts}
\part
Find the matrix representing $f$: 
$\rep{f}{B_V,B_W}$.
\begin{solution}
This is the action of the map on the input basis.
\begin{equation*}
  \colvec{1 \\ 1}\mapsunder{f}\colvec{2 \\ 2}
  \qquad
  \colvec{1 \\ 0}\mapsunder{f}\colvec{1 \\ 0}
\end{equation*}
We represent those results with respect to the output basis.
\begin{align*}
  \colvec{2 \\ 2}=c_1\cdot \colvec{1 \\ -1}+c_2\colvec{2 \\ 0}
  &\quad\Longrightarrow\quad
  \rep{\colvec{2 \\ 2}}{B_w}=\colvec{-2 \\ 2}       \\
  \colvec{1 \\ 0}=c_1\cdot \colvec{1 \\ -1}+c_2\colvec{2 \\ 0}
  &\quad\Longrightarrow\quad
  \rep{\colvec{1 \\ 0}}{B_w}=\colvec{0 \\ 1/2}       
\end{align*}
Consequently, this is the matrix representing the map.
\begin{equation*}
  \rep{f}{B_V,B_w}=
  \begin{mat}
    -2  &0  \\
     2  &1/2
  \end{mat}
\end{equation*}
\end{solution}

\part
Find the matrix representing $h$, that is, 
$\rep{h}{B_V,B_W}$.
\begin{solution}
Here is the action of the map on $B_V$.
\begin{equation*}
  \colvec{1 \\ 1}\mapsunder{h}\colvec{0 \\ 2}
  \qquad
  \colvec{1 \\ 0}\mapsunder{h}\colvec{1 \\ 1}
\end{equation*}
Represent those with respect to the output basis.
\begin{align*}
  \colvec{0 \\ 2}=c_1\cdot \colvec{1 \\ -1}+c_2\colvec{2 \\ 0}
  &\quad\Longrightarrow\quad
  \rep{\colvec{0 \\ 2}}{B_w}=\colvec{-2 \\ 1}       \\
  \colvec{1 \\ 1}=c_1\cdot \colvec{1 \\ -1}+c_2\colvec{2 \\ 0}
  &\quad\Longrightarrow\quad
  \rep{\colvec{1 \\ 1}}{B_w}=\colvec{-1 \\ 1}       
\end{align*}
So this is the matrix representing the map.
\begin{equation*}
  \rep{h}{B_V,B_w}=
  \begin{mat}
    -2  &-1  \\
     1  &1
  \end{mat}
\end{equation*}
\end{solution}

\part
Find the column vector representing $\vec{v}$, $\rep{\vec{v}}{B_v}$.
\begin{solution}
Solve this equation.
\begin{equation*}
  \colvec{3 \\ 0}=c_1\colvec{1 \\ 1}+c_2\colvec{1 \\ 0}
\end{equation*}
This is the result.
\begin{equation*}
  \rep{\vec{v}}{B_V}
  =
  \colvec{0 \\ 3}
\end{equation*}
\end{solution}

\part
Use $\rep{f}{B_V,B_w}$ and $\rep{\vec{v}}{B_v}$ to find
$\rep{f(\vec{v})}{B_w}$.
\begin{solution}
To get the result, find the matrix-vector product.
\begin{align*}
  \rep{f(\vec{v})}{B_W}
    &=\rep{f}{B_v,B_W}\cdot\rep{\vec{v}}{B_V}  \\
    &=
      \begin{mat}
      2  &0  \\
      2   &1/2
      \end{mat}
      \colvec{0 \\ 3}              \\
    &=\colvec{0 \\ 3/2}
\end{align*}
The point of matrix-vector multiplication is that it does the bookkeeping
needed to represent the application of the map to the vector.
Restated, the formula for $f$ gives this
\begin{equation*}
  f(\vec{v})=f(\colvec{3 \\ 0})=\colvec{3 \\ 0}
\end{equation*}
and from the representation $\rep{f(\vec{v})}{B_W}$ we get the same thing.
\begin{equation*}
  0\cdot\colvec{1 \\ 1}+(3/2)\cdot\colvec{2 \\ 0}=\colvec{3 \\ 0}
\end{equation*}
\end{solution}
\end{parts}



\question
Consider the map $\map{2f}{V}{W}$.
\begin{parts}
\part Define it.
\begin{solution}
Because the function $\map{f}{V}{W}$ is defined this way,
\begin{equation*}
  f(\colvec{a \\ b}) = \colvec{a+b \\ 2b}
\end{equation*}
the function $2f$, with the same domain and codomain, is defined this way,
\begin{equation*}
  2f\,(\colvec{a \\ b}) = \colvec{2a+2b \\ 4b}
\end{equation*}
\end{solution}

\part Represent it: find $\rep{2f}{B_V,B_w}$.
\begin{solution}
Because the representation of $f$ that we computed above is this,
\begin{equation*}
  F=\rep{f}{B_V,B_w}=
  \begin{mat}
    -2  &0  \\
     2  &1/2
  \end{mat}
\end{equation*}
the representation of $2f$ given by the results in the book is this.
\begin{equation*}
  2F=\rep{2f}{B_V,B_w}=
  \begin{mat}
    -4  &0  \\
     4  &1
  \end{mat}
\end{equation*}
The point of the defintion of multiplying a matrix by a scalar is that
it represents the function that you get by multiplyting by the scalar. 
\end{solution}

\part Apply it: find $\rep{2f\,(\vec{v})}{B_w}$.
\begin{solution}
As above, 
the point of matrix-vector multiplication is that it does the bookkeeping
needed to represent the application of the map to the vector.
\begin{align*}
  \rep{2f\,(\vec{v})}{B_W}
    &=\rep{2f}{B_v,B_W}\cdot\rep{\vec{v}}{B_V}  \\
    &=
      \begin{mat}
      4  &0  \\
      4   &1
      \end{mat}
      \colvec{0 \\ 3}              \\
    &=\colvec{0 \\ 3}
\end{align*}
\end{solution}
\end{parts}

\question
Consider the map $\map{f+h}{V}{W}$.
\begin{parts}
\part Define it.
\begin{solution}
Because the function $\map{f}{V}{W}$ is defined this way,
\begin{equation*}
  f(\colvec{a \\ b}) = \colvec{a+b \\ 2b}
\end{equation*}
and the function $\map{h}{V}{W}$ is defined this way,
\begin{equation*}
  h(\colvec{a \\ b}) = \colvec{a-b \\ a+b}
\end{equation*}
the function $f+h$ is defined this way.
\begin{equation*}
  f+h\,(\colvec{a \\ b}) = \colvec{(a+b)+(a-b) \\ 2b+(a+b)}
                        = \colvec{2a \\ a+3b}
\end{equation*}
\end{solution}

\part Represent it: find $\rep{f+h}{B_V,B_w}$.
\begin{solution}
The representation of $f$ that we computed above is this.
\begin{equation*}
  F=\rep{f}{B_V,B_w}=
  \begin{mat}
    -2  &0  \\
     2  &1/2
  \end{mat}
\end{equation*}
The representation of $h$ that we computed above is this.
\begin{equation*}
  H=\rep{h}{B_V,B_w}=
  \begin{mat}
    -2  &-1  \\
     1  &1
  \end{mat}
\end{equation*}
The representation of $f+h$ is this.
\begin{equation*}
  F+H=\rep{f+h}{B_V,B_w}=
  \begin{mat}
    -2  &0  \\
     2  &1/2
  \end{mat}
  +\begin{mat}
    -2  &-1  \\
     1  &1
  \end{mat}
  =
  \begin{mat}
    -4  &-1  \\
     3  &3/2
  \end{mat}
\end{equation*}
Briefly, we define the `$+$' operation between two matrices so that
$F+H$ represents $f+h$.
\end{solution}

\part Apply it: find $\rep{f+h\,(\vec{v})}{B_w}$.
\begin{solution}
Matrix-vector multiplication is what does the bookkeeping
needed to represent the application of the map to the vector.
\begin{align*}
  \rep{f+h\,(\vec{v})}{B_w} 
  &=\rep{f+h}{B_V,B_w}\cdot\rep{\vec{v}}{B_V}     \\
  &=
  \begin{mat}
    -4  &-1  \\
     3  &3/2
  \end{mat}
  \colvec{0 \\ 3}             \\
  &=\colvec{-3 \\ 9/2}
\end{align*}
\end{solution}
\end{parts}

\question
The bases matter.
\begin{parts}
\part
Find $\rep{h}{B_W,B_X}$
\part Contrast it with
$\rep{h}{B_V,B_w}$.
\end{parts}

\question
Consider the map $\map{\composed{h}{f}}{V}{U}$.
\begin{parts}
\part Define it.
\part Represent it: find $\rep{\composed{h}{f}}{B_V,B_U}$.
\part Apply it: find $\rep{\composed{h}{f}\,(\vec{v})}{B_U}$.
\end{parts}




\end{questions}
\end{document}
