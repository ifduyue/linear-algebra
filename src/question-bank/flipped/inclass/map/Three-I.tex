% \documentclass[noanswers, nolegalese, 11pt]{examjh}
\documentclass[answers, nolegalese, 11pt]{examjh}
\usepackage{../../../../sty/conc}
\usepackage{../../../../sty/linalgjh}
\usepackage{graphicx}

\setlength{\parindent}{0em}\setlength{\parskip}{0.5ex}
\pagestyle{empty}
\begin{document}\thispagestyle{empty}
\makebox[\textwidth]{Worksheet for Three.I\hfill  From \textit{Linear Algebra}, by Hef{}feron}\vspace{-1ex}
\makebox[\textwidth]{\hbox{}\hrulefill\hbox{}}

A function $\map{f}{V}{W}$ is an association of inputs with outputs.
We write $f(\vec{v})=\vec{w}$ or $\vec{v}\mapsto\vec{w}$.
The key property is that for each input~$\vec{v}$ there is exactly one
associate output~$\vec{w}$~--- that is, for each~$\vec{v}$
there is an output~$\vec{w}$ and there is only one such output.
We say that the function is \textit{well-defined}.
To draw a function we sometimes use a bean diagram, with $V$ on the left
and~$W$ on the right, and with arrows connecting inputs with outputs.
\begin{center}
  \vcenteredhbox{\includegraphics{asy/map000.pdf}}
\end{center}


A function is \textit{one-to-one} if for every output there is at most one
associated input.
A function is \textit{onto} if for every output there is at least one
associated input.
\begin{center}
\begin{tabular}{r|cc}
  \multicolumn{1}{r}{\ }
  &\multicolumn{1}{c}{\textit{One-to-one}}
  &\multicolumn{1}{c}{\textit{Not one-to-one}}    \\
  \cline{2-3}\rule{0em}{6ex}
  \textit{Onto} 
    &\vcenteredhbox{\includegraphics{asy/map003.pdf}}
    &\vcenteredhbox{\includegraphics{asy/map000.pdf}}  \\[1ex]
  \rule{0em}{6ex}
  \textit{Not onto} 
    &\vcenteredhbox{\includegraphics{asy/map001.pdf}}
    &\vcenteredhbox{\includegraphics{asy/map002.pdf}}  \\
\end{tabular}
\end{center}

To show that a function is one-to-one, the simplest thing is to 
assume that $f(\vec{v}_1)=f(\vec{v}_2)$
and go on to show that therefore $\vec{v}_1=\vec{v}_2$.
To show that a function is not one-to-one, 
the simplest thing is to find two different inputs $\vec{v}_1$ and~$\vec{v}_2$
that map to the same output $f(\vec{v}_1)=f(\vec{v}_2)$.

To show that a function is onto, the simplest thing is to start
with a~$\vec{w}$ and give a formula for an input~$\vec{v}$ that maps
to it, that is, such that $f(\vec{v})=\vec{w}$.   
To show that a function is not onto, the simplest thing is to produce 
a~$\vec{w}$ for which there is no input~$\vec{v}$ that maps to it,
that is, such that there is no $\vec{v}$ with $f(\vec{v})=\vec{w}$.   


\begin{questions}
\question
For the function 
$\map{f}{\matspace_{\nbyn{2}}}{\R^4}$
giving this association
\begin{equation*}
  \begin{mat}
    a  &b  \\
    c  &d
  \end{mat}
  \mapsto
  \colvec{a \\ b \\ c \\ d}
\end{equation*}
what column vector is associated with each
matrix?
\begin{parts}
\part $
\begin{mat}
  5  &2  \\
  1  &9
\end{mat}$
\begin{solution}
$\colvec{5 \\ 2 \\ 1 \ 9}$
\end{solution}

\part $
\begin{mat}
  0  &-1/2  \\
  0  &3
\end{mat}$
\begin{solution}
$\colvec{0 \\ -1/2 \\ 0 \\ 3}$
\end{solution}
\end{parts}

\question
For the function 
$\map{g}{\matspace_{\nbyn{2}}}{\R^4}$
giving this association
\begin{equation*}
  \begin{mat}
    a  &b  \\
    c  &d
  \end{mat}
  \mapsto
  \colvec{a \\ c \\ b-d \\ d-b}
\end{equation*}
what column vector is associated with each
matrix?
\begin{parts}
\part $
\begin{mat}
  5  &2  \\
  1  &9
\end{mat}$
\begin{solution}
$\colvec{5 \\ 1 \\ -7 \\ 7}$
\end{solution}

\part $
\begin{mat}
  0  &-1/2  \\
  0  &3
\end{mat}$
\begin{solution}
$\colvec{0 \\ 0 \\ -7/2 \\ 7/2}$
\end{solution}
\end{parts}

\question
Let $\map{f}{\polyspace_2}{\R^2}$ be this.
\begin{equation*}
  f(c_2x^2+c_1x+c_0)=\colvec{-c_1 \\ 3c_0}
\end{equation*}
\begin{parts}
\part Find $f(2x^2-4x+3)$.
\begin{solution}
  $\colvec{4 \\ 9}$
\end{solution}

\part Find $f(7)$.
\begin{solution}
  $\colvec{0 \\ 21}$
\end{solution}
\end{parts}

\question
Consider $\map{f}{\R^2}{\R^2}$ as here.
\begin{equation*}
  \colvec{a \\ b}\mapsto\colvec{a \\ a-b}
\end{equation*}
\begin{parts}
\part Find $f(\vec{v})$ for this vector.
\begin{equation*}
  \vec{v}=\colvec{3 \\ 1}
\end{equation*}

\begin{solution}
 $f(\colvec{3 \\ 1})=\colvec{3 \\ 2}$
\end{solution}

\part Show that $f$ is one-to-one.
\begin{solution}
Suppose that $f(\vec{v}_1)=f(\vec{v}_2)$.
Then 
\begin{equation*}
  f(\colvec{a_1 \\ b_1})=f(\colvec{a_1 \\ b_1})
\end{equation*}
and so we have this.
\begin{equation*}
  \colvec{a_1 \\ a_1-b_1}=\colvec{a_2 \\ a_2-b_2}
\end{equation*}
The top entries give that $a_1=a_2$.
The bottom entries give that $a_11-b_1=a_2-b_2$, and with the information
that $a_1=a_2$, we have that $b_1=b_2$ also.
Therefore $\vec{v}_1=\vec{v}_2$.
\end{solution}
\end{parts}


\question
Consider $\map{g}{\polyspace_2}{\polyspace_3}$ given by 
$g(ax^2+bx+c)=(a+b)x^3-cx+b$.
\begin{parts}
\part Compute $g(9x^2+3x-4)$.
\begin{solution}
$g(9x^2+3x-4)=12x^3+4x+3$
\end{solution}

\part
Show that $g$ is one-to-one.
\begin{solution}
Suppose that $g(\vec{v}_1)=g(\vec{v}_2)$.  
Writing $\vec{v}_1=a_1x^2+b_1x+c_1$ and  $\vec{v}_2=a_2x^2+b_2x+c_2$
we get this. 
\begin{equation*}
  (a_1+b_1)x^3-c_1x+b_1=(a_2+b_2)x^3-c_2x+b_2
\end{equation*}
Comparing the two constant terms gives that $b_1=b_2$.
The linear terms give that $c_1=c_2$.
The coefficients of~$x^3$ give $a_1+b_1=a_2+b_2$, and
since $b_1=b_2$ we have $a_1=a_2$ also.
Thus $\vec{v}_1=\vec{v}_2$.
\end{solution}
\end{parts}

\question
Define $\map{f}{\matspace_{\nbyn{2}}}{\R^2}$ as here.
\begin{equation*}
  f(
  \begin{mat}
    a  &b  \\
    c  &d
  \end{mat}
  )
  =
  \colvec{a  \\ a+b}
\end{equation*}
\begin{parts}
\part Find $f(\vec{v})$ for this input.
\begin{equation*}
  \vec{v}=
  \begin{mat}
    3  &2  \\
    4  &5
  \end{mat}
\end{equation*}
\begin{solution}
  $\colvec{3 \\ 5}$
\end{solution}

\item Show that $f$ is not one-to-one.
\begin{solution}
Note that $f(\vec{v}_1)=f(\vec{v}_2)$ for these two.
\begin{equation*}
  \vec{v}_1=
  \begin{mat}
    3  &2  \\
    4  &5
  \end{mat}
  \qquad
  \vec{v}_2=
  \begin{mat}
    3  &2  \\
    0  &0
  \end{mat}
\end{equation*}
\end{solution}

\end{parts}
\end{questions}
\end{document}
