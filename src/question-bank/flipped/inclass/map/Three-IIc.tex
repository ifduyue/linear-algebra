% \documentclass[noanswers, nolegalese, 11pt]{examjh}
\documentclass[answers, nolegalese, 11pt]{examjh}
\usepackage{../../../../sty/conc}
\usepackage{../../../../sty/linalgjh}
\usepackage{graphicx}

\setlength{\parindent}{0em}\setlength{\parskip}{0.5ex}
\pagestyle{empty}
\begin{document}\thispagestyle{empty}
\makebox[\textwidth]{Worksheet for Three.II (part 4)\hfill  From \textit{Linear Algebra}, by Hef{}feron}\vspace{-1ex}
\makebox[\textwidth]{\hbox{}\hrulefill\hbox{}}



\begin{questions}
\question
Fix the vector spaces $V=\R^3$ and $W=\R^2$.
Consider the map $\map{h}{V}{W}$, which is a homomorphism.
\begin{equation*}
  \colvec{x \\ y \\ z}\mapsto\colvec{x+z \\ y+z}
\end{equation*}
\begin{parts}
\part
Find the null space, $\nullspace{h}$.
\begin{solution}
We have
\begin{equation*}
h(\colvec{x \\ y \\ z})=\colvec{x+z \\ y+z}=\colvec{0 \\ 0}
\end{equation*}
which leads to $x+z=0$ and $y+z=0$.
\begin{equation*}
  \nullspace{h}
  =\set{\colvec{-z \\ -z \\ z}\suchthat z\in\R}
  =\set{\colvec{-1 \\ -1 \\ 1}\cdot z\suchthat z\in\R}
\end{equation*}
Here is a picture.
\begin{center}
  \includegraphics{asy/map013.pdf}
\end{center}
\end{solution}

\part
Give a basis $B_N$ for the null space.
\begin{solution}
\begin{equation*}
  B_N=\sequence{\colvec{-1 \\ -1 \\ 1}}
\end{equation*}
\end{solution}

\part
Find the range space, $\rangespace{h}$
\begin{solution}
The range space is all of $W=\R^2$, 
as we can see by restricting the inputs to $z=0$.
\end{solution}

\part
Expand $B_N$, that is, add vectors to it, to make a basis for the domain, $B_V$.
\begin{solution}
There are infinitely many ways, but here is one.
(I made these vectors, just plucked them out of the air, 
but in such a way that the result is linearly independent.)
\begin{equation*}
  B_V=
  \sequence{
    \colvec{-1 \\ -1 \\ 1},
    \colvec{1 \\ 0 \\ 1},
    \colvec{1 \\ 1 \\ 0}
    }
\end{equation*}
Checking that it is a basis is striaghtforward.
Here is linear independence:
\begin{equation*}
    c_1\colvec{-1 \\ -1 \\ 1}
    +c_2\colvec{1 \\ 0 \\ 1}
    +c_3\colvec{1 \\ 1 \\ 0}
    =\colvec{0 \\ 0 \\ 0}
\end{equation*}
leads to
\begin{equation*}
\begin{linsys}{3}
  -c_1  &+  &c_2  &+  &c_3  &=  &0  \\
  -c_1  &   &     &+  &c_3  &=  &0  \\
   c_1  &+  &c_2  &   &     &=  &0  \\
\end{linsys}
\end{equation*}
and Gauss's method gives that the only solution is the trivial one
$c_1=0$, $c_2=0$, $c_3=0$.
With that, because the domain $V=\R^3$ is $3$-dimensional, this linearly
independent set with three members must be a basis.
\end{solution}

\part
For all the members of $B_v$ that you've added, compute the image when
you apply $h$.
\begin{solution}
\begin{equation*}
  h(\colvec{1 \\ 0 \\ 1})=\colvec{2 \\ 1}
  \qquad
  h(\colvec{1 \\ 1 \\ 0})=\colvec{1 \\ 1}
\end{equation*}
\end{solution}

\part
Call the sequence of resulting vectors $B_R$.
Check that it
is a basis for the range, $W=\R^2$.
\begin{solution}
We have this.
\begin{equation*}
  B_R=
  \sequence{\colvec{2 \\ 1},\colvec{1 \\ 1}}
\end{equation*}
It is linearly indepencdent since the second member is not a multiple of 
the first.
Because the range $W=\R^2$ is $2$-dimensional, this two-member linearly
independent set must be a basis. 
\end{solution}

\part
Conclude that (for this map)
the nullity plus the rank equals the dimension of the domain.
\begin{solution}
The nullity is the dimension of the null space, $1$.
The rank is the dimension of the range space, $2$.
Together they add to the dimension of the domain, since
$V=\R^3$ is $3$-dimensional/
\end{solution}
\end{parts}

\end{questions}
\end{document}
