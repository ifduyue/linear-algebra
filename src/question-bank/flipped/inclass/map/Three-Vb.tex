% \documentclass[noanswers, nolegalese, 11pt]{examjh}
\documentclass[answers, nolegalese, 11pt]{examjh}
\usepackage{../../../../sty/conc}
\usepackage{../../../../sty/linalgjh}
\usepackage{graphicx}

\setlength{\parindent}{0em}\setlength{\parskip}{0.5ex}
\pagestyle{empty}
\begin{document}\thispagestyle{empty}
\makebox[\textwidth]{Worksheet for Three.V (part 3)\hfill  From \textit{Linear Algebra}, by Hef{}feron}\vspace{-1ex}
\makebox[\textwidth]{\hbox{}\hrulefill\hbox{}}

Let $V=\polyspace_2$ and consider these two bases.
\begin{equation*}
B=\sequence{1-x, 1+x+x^2, x^2}
\quad
\hat{B}=\sequence{1+x^2, 1+x, 1}
\end{equation*}
Fix this linear transformation  $\map{h}{V}{W}$
\begin{equation*}
  h(ax^2+bx+c)=(a+b)x^2+cx-b
\end{equation*}
You may want to refer to this diagram.
\begin{equation*}
  \begin{CD}
    V_{\wrt{B}}                   @>h>H>        V_{\wrt{B}}       \\
    @V{\text{\scriptsize$\identity$}} VV                @V{\text{\scriptsize$\identity$}} VV \\
    V_{\wrt{\hat{B}}}             @>h>\hat{H}>  V_{\wrt{\hat{B}}}
  \end{CD}
\end{equation*}

\begin{questions}
\question
\begin{parts}
\part
Find the change of basis matrix on the left, 
$P^{-1}=\rep{\identity}{\hat{B},B}$.
\begin{solution}
First observe that we are on the left side of the diagram.
Notices also that we are going against the arrow drawn there, so we
are representing the map that goes from the bottom left to the top left.
Also observe that this question is unchanged from the prior worksheet.

To compute $P^{-1}=\rep{\identity}{\hat{B},B}$, as to compute the
representation of any map, first compute the effect of the map
on the starting basis.
With the identity map this step is trivial.
\begin{align*}
  1+x^2  
  &\mapsunder{\identity}  1+x^2  \\
  1+x  
  &\mapsunder{\identity}  1+x  \\
  1  
  &\mapsunder{\identity}  1
\end{align*}
The second step is to represent each output with respect to the 
ending basis.
So we must compute $\rep{1+x^2}{B}$ by finding the coefficients in
\begin{equation*}
  1+x^2=c_1\cdot(1-x)+c_2\cdot (1+x+x^2)+c_3\cdot(x^2)
\end{equation*}
as in
\begin{equation*}
\begin{linsys}{3}
  c_1  &+  &c_2  &   &   &=  &1  \\
  -c_1 &+  &c_2  &   &   &=  &0  \\
       &   &c_2  &+  &c_3&=  &1  \\
\end{linsys}
\grstep{\rho_1+\rho_2}
\begin{linsys}{3}
  c_1  &+  &c_2  &   &   &=  &1  \\
       &   &2c_2 &   &   &=  &1  \\
       &   &c_2  &+  &c_3&=  &1  \\
\end{linsys}
\grstep{(-1/2)\rho_2+\rho_3}
\begin{linsys}{3}
  c_1  &+  &c_2  &   &   &=  &1  \\
       &   &2c_2 &   &   &=  &1  \\
       &   &     &   &c_3&=  &1/2  \\
\end{linsys}
\end{equation*}
giving this.
\begin{equation*}
  \rep{1+x^2}{B}=\colvec{1/2 \\ 1/2 \\ 1/2}
\end{equation*}

Similarly we compute
$\rep{1+x^2}{B}$ by finding the coefficients in
\begin{equation*}
  1+x+x^2=c_1\cdot(1-x)+c_2\cdot (1+x+x^2)+c_3\cdot(x^2)
\end{equation*}
as in
\begin{equation*}
\begin{linsys}{3}
  c_1  &+  &c_2  &   &   &=  &1  \\
  -c_1 &+  &c_2  &   &   &=  &1  \\
       &   &c_2  &+  &c_3&=  &0  \\
\end{linsys}
\grstep{\rho_1+\rho_2}
\begin{linsys}{3}
  c_1  &+  &c_2  &   &   &=  &1  \\
       &   &2c_2 &   &   &=  &2  \\
       &   &c_2  &+  &c_3&=  &0  \\
\end{linsys}
\grstep{(-1/2)\rho_2+\rho_3}
\begin{linsys}{3}
  c_1  &+  &c_2  &   &   &=  &1  \\
       &   &2c_2 &   &   &=  &2  \\
       &   &     &   &c_3&=  &-1  \\
\end{linsys}
\end{equation*}
giving this.
\begin{equation*}
  \rep{1+x}{B}=\colvec{0 \\ 1 \\ -1}
\end{equation*}


Finally, to find $\rep{1}{B}$ we do
\begin{equation*}
  1=c_1\cdot(1-x)+c_2\cdot (1+x+x^2)+c_3\cdot(x^2)
\end{equation*}
and get
\begin{equation*}
\begin{linsys}{3}
  c_1  &+  &c_2  &   &   &=  &1  \\
  -c_1 &+  &c_2  &   &   &=  &0  \\
       &   &c_2  &+  &c_3&=  &0  \\
\end{linsys}
\grstep{\rho_1+\rho_2}
\begin{linsys}{3}
  c_1  &+  &c_2  &   &   &=  &1  \\
       &   &2c_2 &   &   &=  &1  \\
       &   &c_2  &+  &c_3&=  &0  \\
\end{linsys}
\grstep{(-1/2)\rho_2+\rho_3}
\begin{linsys}{3}
  c_1  &+  &c_2  &   &   &=  &1  \\
       &   &2c_2 &   &   &=  &1  \\
       &   &     &   &c_3&=  &-1/2  \\
\end{linsys}
\end{equation*}
giving this.
\begin{equation*}
  \rep{1}{B}=\colvec{1/2 \\ 1/2 \\ -1/2}
\end{equation*}

In short, we have this matrix.
\begin{equation*}
P^{-1}=\rep{\identity}{\hat{B},B}=
\begin{mat}
  1/2  &0  &1/2  \\
  1/2  &1  &1/2  \\
  1/2  &-1 &-1/2
\end{mat}
\end{equation*}
\end{solution}

\part
Find the change of basis matrix $P=\rep{\identity}{B,\hat{B}}$
by determining the effect of the identity map on the vectors in the starting
basis, and representing them with respect to the ending basis.
\begin{solution}
First compute the effect of the map on the starting basis.
\begin{align*}
  1-x  
  &\mapsunder{\identity}  1-x  \\
  1+x+x^2  
  &\mapsunder{\identity}  1+x+x^2  \\
  x^2  
  &\mapsunder{\identity}  x^2  
\end{align*}

Next represent each output with respect to the ending basis.
For the first, solve this equation.
\begin{equation*}
  1-x = c_1\cdot(1+x^2)+c_2\cdot(1+x)+c_3\cdot(1)
\end{equation*}
By eye, this is the answer.
\begin{equation*}
  \rep{1-x}{\hat{B}}=\colvec{0 \\ -1 \\ 2}
\end{equation*}
Similarly, for the second solve this equation.
\begin{equation*}
  1+x+x^2 = c_1\cdot(1+x^2)+c_2\cdot(1+x)+c_3\cdot(1)
\end{equation*}
By eye, this is the answer.
\begin{equation*}
  \rep{1+x+x^2}{\hat{B}}=\colvec{1 \\ 1 \\ -1}
\end{equation*}
Finally, 
\begin{equation*}
  x^2 = c_1\cdot(1+x^2)+c_2\cdot(1+x)+c_3\cdot(1)
\end{equation*}
gives this.
\begin{equation*}
  \rep{x^2}{\hat{B}}=\colvec{1 \\ 0 \\ -1}
\end{equation*}
This is the matrix representation.
\begin{equation*}
P=\rep{\identity}{B,\hat{B}}=
\begin{mat}
 0 &1 &1 \\
-1 &1 &0 \\
 2 &-1&-1 
\end{mat}
\end{equation*}
\end{solution}

\part Compute the same matrix by instead finding the inverse of $P^{-1}$.
\begin{solution}
\begin{align*}
\left(\begin{array}{ccc|ccc}
  1/2  &0  &1/2  &1  &0  &0  \\ 
  1/2  &1  &1/2  &0  &1  &0  \\ 
  1/2  &-1  &-1/2  &0  &0  &1  \\ 
\end{array}\right)
&\grstep[-1\rho_{1}+\rho_{3}]{-1\rho_{1}+\rho_{2}}
\left(\begin{array}{ccc|ccc}
  1/2  &0  &1/2  &1  &0  &0  \\ 
  0  &1  &0  &-1  &1  &0  \\ 
  0  &-1  &-1  &-1  &0  &1  \\ 
\end{array}\right)                                 \\
&\grstep{1\rho_{2}+\rho_{3}}
\left(\begin{array}{ccc|ccc}
  1/2  &0  &1/2  &1  &0  &0  \\ 
  0  &1  &0  &-1  &1  &0  \\ 
  0  &0  &-1  &-2  &1  &1  \\ 
\end{array}\right)                                 \\
&\grstep[-1\rho_{3}]{2\rho_{1}}
\left(\begin{array}{ccc|ccc}
  1  &0  &1  &2  &0  &0  \\ 
  0  &1  &0  &-1  &1  &0  \\ 
  0  &0  &1  &2  &-1  &-1  \\ 
\end{array}\right)                               \\
&\grstep{-1\rho_{3}+\rho_{1}}
\left(\begin{array}{ccc|ccc}
  1  &0  &0  &0  &1  &1  \\ 
  0  &1  &0  &-1  &1  &0  \\ 
  0  &0  &1  &2  &-1  &-1  \\ 
\end{array}\right)
\end{align*}
\end{solution}

\part
Find the representation $H=\rep{h}{B,B}$.
\begin{solution}
Here the effect of the map on the starting basis~$B$ is not trivial.
\begin{align*}
  1-x
  &\mapsunder{h} -x^2+x+1  \\
  1+x+x^2
  &\mapsunder{h} 2x^2+x-1  \\
  x^2
  &\mapsunder{h} x^2  
\end{align*}

To represent the first output with respect to~$B$, set up
\begin{equation*}
  -x^2+x+1=c_1\cdot(1-x)+c_2\cdot (1+x+x^2)+c_3\cdot(x^2)
\end{equation*}
and solve
\begin{equation*}
\begin{linsys}{3}
  c_1  &+  &c_2  &   &   &=  &1  \\
  -c_1 &+  &c_2  &   &   &=  &1  \\
       &   &c_2  &+  &c_3&=  &-1  \\
\end{linsys}
\grstep{\rho_1+\rho_2}
\begin{linsys}{3}
  c_1  &+  &c_2  &   &   &=  &1  \\
       &   &2c_2 &   &   &=  &2  \\
       &   &c_2  &+  &c_3&=  &-1  \\
\end{linsys}
\grstep{(-1/2)\rho_2+\rho_3}
\begin{linsys}{3}
  c_1  &+  &c_2  &   &   &=  &1  \\
       &   &2c_2 &   &   &=  &2  \\
       &   &     &   &c_3&=  &-2  \\
\end{linsys}
\end{equation*}
giving this.
\begin{equation*}
  \rep{-x^2+x+1}{B}=\colvec{0 \\ 1 \\ -2}
\end{equation*}
To represent the second output with respect to~$B$, 
\begin{equation*}
  2x^2+x-1=c_1\cdot(1-x)+c_2\cdot (1+x+x^2)+c_3\cdot(x^2)
\end{equation*}
and solve
\begin{equation*}
\begin{linsys}{3}
  c_1  &+  &c_2  &   &   &=  &-1  \\
  -c_1 &+  &c_2  &   &   &=  &1  \\
       &   &c_2  &+  &c_3&=  &2  \\
\end{linsys}
\grstep{\rho_1+\rho_2}
\begin{linsys}{3}
  c_1  &+  &c_2  &   &   &=  &-1  \\
       &   &2c_2 &   &   &=  &0  \\
       &   &c_2  &+  &c_3&=  &2  \\
\end{linsys}
\grstep{(-1/2)\rho_2+\rho_3}
\begin{linsys}{3}
  c_1  &+  &c_2  &   &   &=  &-1  \\
       &   &2c_2 &   &   &=  &0  \\
       &   &     &   &c_3&=  &2  \\
\end{linsys}
\end{equation*}
giving this.
\begin{equation*}
  \rep{2x^2+x-1}{B}=\colvec{-1 \\ 0 \\ 2}
\end{equation*}
To represent the third, set up 
\begin{equation*}
  x^2=c_1\cdot(1-x)+c_2\cdot (1+x+x^2)+c_3\cdot(x^2)
\end{equation*}
and solve
\begin{equation*}
\begin{linsys}{3}
  c_1  &+  &c_2  &   &   &=  &0  \\
  -c_1 &+  &c_2  &   &   &=  &0  \\
       &   &c_2  &+  &c_3&=  &1  \\
\end{linsys}
\grstep{\rho_1+\rho_2}
\begin{linsys}{3}
  c_1  &+  &c_2  &   &   &=  &0  \\
       &   &2c_2 &   &   &=  &0  \\
       &   &c_2  &+  &c_3&=  &1  \\
\end{linsys}
\grstep{(-1/2)\rho_2+\rho_3}
\begin{linsys}{3}
  c_1  &+  &c_2  &   &   &=  &0  \\
       &   &2c_2 &   &   &=  &0  \\
       &   &     &   &c_3&=  &1  \\
\end{linsys}
\end{equation*}
giving this.
\begin{equation*}
  \rep{x^2}{B}=\colvec{0 \\ 0 \\ 1}
\end{equation*}
We have this matrix.
\begin{equation*}
  H=\rep{h}{B,B}=
  \begin{mat}
    0  &-1 &0  \\
    1  &0  &0  \\
   -2  &2  &1  \\
  \end{mat}
\end{equation*}
\end{solution}

\part
Find the representation $\hat{H}=\rep{h}{\hat{B},\hat{B}}$.
\begin{solution}
This is the effect of the map on the starting basis.
\begin{align*}
  1+x^2
  &\mapsunder{h} x^2+x  \\
  1+x
  &\mapsunder{h} x^2+x-1  \\
  1
  &\mapsunder{h} x  
\end{align*}

To represent the first with respect to~$\hat{B}$, set up
\begin{equation*}
x^2+x=c_1\cdot(1+x^2)+c_2\cdot(1+x)+c_3\cdot(1)
\end{equation*}
and the solution is this.
\begin{equation*}
\rep{x^2+x}{\hat{B}}=
\colvec{1 \\ 1 \\ -2}
\end{equation*}
To represent the second with respect to~$\hat{B}$, set up
\begin{equation*}
x^2+x-1=c_1\cdot(1+x^2)+c_2\cdot(1+x)+c_3\cdot(1)
\end{equation*}
and the solution is this.
\begin{equation*}
\rep{x^2+x-1}{\hat{B}}=
\colvec{1 \\ 1 \\ -3}
\end{equation*}
The third is 
\begin{equation*}
x=c_1\cdot(1+x^2)+c_2\cdot(1+x)+c_3\cdot(1)
\end{equation*}
and the solution is this.
\begin{equation*}
\rep{x^2+x-1}{\hat{B}}=
\colvec{0 \\ 1 \\ -1}
\end{equation*}

That leads to this matrix.
\begin{equation*}
  \hat{H}=\rep{h}{\hat{B},\hat{B}}
  =
  \begin{mat}
    1  &1  &0  \\
    1  &1  &1  \\
   -2  &-3 &-1
  \end{mat}
\end{equation*}
\end{solution}

\part
Compute $PHP^{-1}$
\begin{solution}
\begin{align*}
\begin{mat}
 0 &1 &1 \\
-1 &1 &0 \\
 2 &-1&-1 
\end{mat}
  \begin{mat}
    0  &-1 &0  \\
    1  &0  &0  \\
   -2  &2  &1  \\
  \end{mat}
\begin{mat}
  1/2  &0  &1/2  \\
  1/2  &1  &1/2  \\
  1/2  &-1 &-1/2
\end{mat}
&=
\begin{mat}
  -1 &2  &1 \\
  1  &1  &0 \\
  1 &-4  &-1           
\end{mat}
\begin{mat}
  1/2  &0  &1/2  \\
  1/2  &1  &1/2  \\
  1/2  &-1 &-1/2
\end{mat}                     \\
&=  
  \begin{mat}
    1  &1  &0  \\
    1  &1  &1  \\
   -2  &-3 &-1
  \end{mat}
\end{align*}
\end{solution}

\end{parts}
\end{questions}
\end{document}
