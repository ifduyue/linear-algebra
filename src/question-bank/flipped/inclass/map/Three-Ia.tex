% \documentclass[noanswers, nolegalese, 11pt]{examjh}
\documentclass[answers, nolegalese, 11pt]{examjh}
\usepackage{../../../../sty/conc}
\usepackage{../../../../sty/linalgjh}
\usepackage{graphicx}

\setlength{\parindent}{0em}\setlength{\parskip}{0.5ex}
\pagestyle{empty}
\begin{document}\thispagestyle{empty}
\makebox[\textwidth]{Worksheet for Three.I, part two\hfill  From \textit{Linear Algebra}, by Hef{}feron}\vspace{-1ex}
\makebox[\textwidth]{\hbox{}\hrulefill\hbox{}}


\begin{questions}
\question
Verify that this map $\map{h}{\matspace_{\nbyn{2}}}{\R^4}$ is an isomorphism.
\begin{equation*}
\begin{mat}
  a  &b  \\
  c  &d
\end{mat}
\mapsto
\colvec{a+b \\ a-b  \\ c \\ c+d}
\end{equation*}
\begin{solution}
We must check (1) and~(2) from the definition.
Each of these, in turn, involves checking two things.

We first check that the map~$h$ is one-to-one.
Assume $h(\vec{v}_1)=h(\vec{v}_2)$.
\begin{align*}
  h(
  \begin{mat}
    a_1  &b_1  \\
    c_1  &d_1
  \end{mat})
  &=
  h(
  \begin{mat}
    a_2  &b_2  \\
    c_2  &d_2
  \end{mat})                  \\
  \colvec{a_1+b_1 \\ a_1-b_1  \\ c_1 \\ c_1+d_1}
  &=
  \colvec{a_2+b_2 \\ a_2-b_2  \\ c_2 \\ c_2+d_2}           
  \tag{$*$}
\end{align*}
Comparing third entries gives that $c_1=c_2$, and then the fourth entries
give that $d_1=d_2$.
Comparing the first and second entries gives that $a_1+b_1=a_2+b_2$
and $a_1-b_1=a_2-b_2$.
Adding these together gives that $a_1=a_2$, and thus $b_1=b_2$.
\textit{Remark:} a person can wonder what to do if the relationships 
are too complicated to eyeball.
Equation~($*$) gives this linear system,
\begin{equation*}
\begin{linsys}{8}
  a_1  &+  &b_1  &  &    &  &    &-  &a_2  &-  &b_2  &   &    &  &    &=  &0 \\
  a_1  &-  &b_1  &  &    &  &    &-  &a_2  &+  &b_2  &   &    &  &    &=  &0 \\
       &   &     &  &c_1 &  &    &   &     &   &     &-  &c_2 &  &    &=  &0 \\
       &   &     &  &c_1 &+ &d_1 &   &     &   &     &-  &c_2 &- &d_2  &=  &0 \\
\end{linsys}
\end{equation*}
and Gauss's method
\begin{align*}
\begin{amat}{8}
  1  &1  &0  &0  &-1  &-1  &0  &0  &0  \\ 
  1  &-1  &0  &0  &-1  &1  &0  &0  &0  \\ 
  0  &0  &1  &0  &0  &0  &-1  &0  &0  \\ 
  0  &0  &1  &1  &0  &0  &-1  &-1  &0  \\ 
\end{amat}
&\grstep{-1\rho_{1}+\rho_{2}}
\begin{amat}{8}
  1  &1  &0  &0  &-1  &-1  &0  &0  &0  \\ 
  0  &-2  &0  &0  &0  &2  &0  &0  &0  \\ 
  0  &0  &1  &0  &0  &0  &-1  &0  &0  \\ 
  0  &0  &1  &1  &0  &0  &-1  &-1  &0  \\ 
\end{amat}                                     \\
&\grstep{-1\rho_{3}+\rho_{4}}
\begin{amat}{8}
  1  &1  &0  &0  &-1  &-1  &0  &0  &0  \\ 
  0  &-2  &0  &0  &0  &2  &0  &0  &0  \\ 
  0  &0  &1  &0  &0  &0  &-1  &0  &0  \\ 
  0  &0  &0  &1  &0  &0  &0  &-1  &0  \\ 
\end{amat}
\end{align*}
gives that $d_1=d_2$, $c_1=c_2$, $b_1=b_2$ and $a_1=a_2$,
and so $\vec{v}_1=\vec{v}_2$.

To finish condition~(1), we will show that the function~$h$ is onto.
For these, often a numerical example is a help.
What input does it take to produce this output?
\begin{equation*}
\begin{mat}
  a  &b  \\
  c  &d
\end{mat}
\mapsto
\colvec{a+b \\ a-b  \\ c \\ c+d}
=
\colvec{1  \\ 2 \\ 3 \\ 4}  
\end{equation*}
We need $c=3$, and $d=1$.
We also need $a=3/2$ and $b=-1/2$.
In general, to get this output
\begin{equation*}
\begin{mat}
  a  &b  \\
  c  &d
\end{mat}
\mapsto
\colvec{a+b \\ a-b  \\ c \\ c+d}
=
\colvec{x  \\ y \\ z \\ w}  
\end{equation*}
we can use $c=z$ and $d=w-z$,
and as well $a=(x+y)/2$ and $b=(x-y)/2$.

Condition~(2) is more of a computation.
We first show that this map respects addition.
\begin{align*}
  h(\vec{v}_1+\vec{v}_2)
  &=  
  h(\;\begin{mat}
  a_1  &b_1  \\
  c_1  &d_1
  \end{mat}
  +
  \begin{mat}
  a_1  &b_1  \\
  c_1  &d_1
  \end{mat}\;)                           \\
  &=  
  h(\begin{mat}
  a_1+a_2  &b_1+b_2  \\
  c_1+c_2  &d_1+d_2
  \end{mat})                           \\
  &=
  \colvec{(a_1+a_2)+(b_1+b_2)  \\ (a_1+a_2)-(b_1+b_2) \\ c_1+c_2 \\ (c_1+c_2)+(d_1+d_2)}         \\
  &=
  \colvec{(a_1+b_1)+(a_2+b_2)  \\ (a_1-b_1)+(a_2-b_2) \\ c_1+c_2 \\ (c_1+d_1)+(c_2+d_2)}         \\
  &=  
  \colvec{a_1+b_1 \\ a_1-b_1  \\ c_1 \\ c_1+d_1}
  +
  \colvec{a_2+b_2 \\ a_2-b_2  \\ c_2 \\ c_2+d_2}          \\
  &=
  h(\begin{mat}
  a_1  &b_1  \\
  c_1  &d_1
  \end{mat})
  +
  h(\begin{mat}
  a_1  &b_1  \\
  c_1  &d_1
\end{mat})                           
\end{align*}

For the other half of condition~(2), we show that the function~$h$ preserves
scalar multiplication.
\begin{align*}
  h(r\vec{v})
  &=
  h(
    r\cdot\begin{mat}
      a  &b  \\
      c  &d
    \end{mat})                      \\
  &=
  h(
    \begin{mat}
      ra  &rb  \\
      rc  &rd
    \end{mat})                  \\
  &=
  \colvec{ra+rb \\ ra-rb  \\ rc \\ rc+rd}          \\
  &=
  r\cdot\colvec{a+b \\ a-b  \\ c \\ c+d}          \\
  &=
  r\cdot h(\;
    \begin{mat}
      a  &b  \\
      c  &d
    \end{mat}\;)   
\end{align*}
\end{solution}

\question
Verify condition~(2) of the argument showing 
that this map $\map{f}{\polyspace_2}{\polyspace_2}$ is an automorphism.
\begin{equation*}
  ax^2+bx+c \mapsto (a-c)x^2+(b-c)x+c
\end{equation*}
\begin{solution}
We first show that $f$ preserves addition.
\begin{align*}
  f(\vec{v}_1+\vec{v}_2)
  &=
  f(\;(a_1x^2+b_1x+c_1)+(a_2x^2+b_2x+c_2)\;)           \\
  &=
  f(\;(a_1+a_2)x^2+(b_1+b_2)x+(c_1+c_2)\;)           \\
  &=
  (a_1+a_2-(c_1+c_2))x^2+(b_1+b_2-(c_1+c_2))x+(c_1+c_2)  \\
  &=
  \bigl((a_1-c_1)x^2+(b_1-c_1)x+c\bigr) +\bigl((a_2-c_2)x^2+(b_2-c_2)x+c_2\bigr) \\
  &=
  f(\vec{v}_1)+f(\vec{v}_2)
\end{align*}
We finish by showing that $f$ preserves scalar multiplication.
\begin{align*}
  f(r\cdot \vec{v})
  &=
  f(\;r\cdot(ax^2+bx+c)\;)        \\
  &=
  f(\;(ra)x^2+(rb)x+(rc)\;)        \\
  &=
  (ra-rc)x^2+(rb-rc)x+rc            \\
  &=
  r\cdot\bigl( (a-c)x^2+(b-c)x+c \bigr)            \\
  &=r\cdot f(\vec{v})
\end{align*}
\end{solution}
\end{questions}
\end{document}
