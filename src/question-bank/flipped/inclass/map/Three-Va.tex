% \documentclass[noanswers, nolegalese, 11pt]{examjh}
\documentclass[answers, nolegalese, 11pt]{examjh}
\usepackage{../../../../sty/conc}
\usepackage{../../../../sty/linalgjh}
\usepackage{graphicx}

\setlength{\parindent}{0em}\setlength{\parskip}{0.5ex}
\pagestyle{empty}
\begin{document}\thispagestyle{empty}
\makebox[\textwidth]{Worksheet for Three.V (part 2)\hfill  From \textit{Linear Algebra}, by Hef{}feron}\vspace{-1ex}
\makebox[\textwidth]{\hbox{}\hrulefill\hbox{}}

Let $V=\polyspace_2$ and consider these two bases.
\begin{equation*}
B=\sequence{1-x, 1+x+x^2, x^2}
\quad
\hat{B}=\sequence{1+x^2, 1+x, 1}
\end{equation*}
Let $W=\R^2$ and consider these two bases.
\begin{equation*}
D=\sequence{\colvec{2 \\ 1},\colvec{0 \\ 2}}
\quad
\hat{D}=\stdbasis_2=\sequence{\vec{e}_1=\colvec{1 \\ 0},\vec{e}_2=\colvec{0 \\ 1}}
\end{equation*}
Fix this linear map $\map{h}{V}{W}$
\begin{equation*}
  h(ax^2+bx+c)=\colvec{a+b \\ a+c}
\end{equation*}
You may want to refer to this diagram.
\begin{equation*}
  \begin{CD}
    V_{\wrt{B}}                   @>h>H>        W_{\wrt{D}}       \\
    @V{\text{\scriptsize$\identity$}} VV                @V{\text{\scriptsize$\identity$}} VV \\
    V_{\wrt{\hat{B}}}             @>h>\hat{H}>  W_{\wrt{\hat{D}}}
  \end{CD}
\end{equation*}

\begin{questions}
\question
\begin{parts}
\part
Find the change of basis matrix $Q=\rep{\identity}{\hat{B},B}$.
\begin{solution}
First observe that we are on the left side of the diagram.
Notices also that we are going against the arrow drawn there, so we
are representing the map that goes from the bottom left to the top left.

To compute $Q=\rep{\identity}{\hat{B},B}$, as to compute the
representation of any map, first compute the effect of the map
on the starting basis.
With the identity map this step is trivial.
\begin{align*}
  1+x^2  
  &\mapsunder{\identity}  1+x^2  \\
  1+x  
  &\mapsunder{\identity}  1+x  \\
  1  
  &\mapsunder{\identity}  1
\end{align*}
The second step is to represent each output with respect to the 
ending basis.
So we must compute $\rep{1+x^2}{B}$ by finding the coefficients in
\begin{equation*}
  1+x^2=c_1\cdot(1-x)+c_2\cdot (1+x+x^2)+c_3\cdot(x^2)
\end{equation*}
as in
\begin{equation*}
\begin{linsys}{3}
  c_1  &+  &c_2  &   &   &=  &1  \\
  -c_1 &+  &c_2  &   &   &=  &0  \\
       &   &c_2  &+  &c_3&=  &1  \\
\end{linsys}
\grstep{\rho_1+\rho_2}
\begin{linsys}{3}
  c_1  &+  &c_2  &   &   &=  &1  \\
       &   &2c_2 &   &   &=  &1  \\
       &   &c_2  &+  &c_3&=  &1  \\
\end{linsys}
\grstep{(-1/2)\rho_2+\rho_3}
\begin{linsys}{3}
  c_1  &+  &c_2  &   &   &=  &1  \\
       &   &2c_2 &   &   &=  &1  \\
       &   &     &   &c_3&=  &1/2  \\
\end{linsys}
\end{equation*}
giving this.
\begin{equation*}
  \rep{1+x^2}{B}=\colvec{1/2 \\ 1/2 \\ 1/2}
\end{equation*}

Similarly we compute
$\rep{1+x^2}{B}$ by finding the coefficients in
\begin{equation*}
  1+x+x^2=c_1\cdot(1-x)+c_2\cdot (1+x+x^2)+c_3\cdot(x^2)
\end{equation*}
as in
\begin{equation*}
\begin{linsys}{3}
  c_1  &+  &c_2  &   &   &=  &1  \\
  -c_1 &+  &c_2  &   &   &=  &1  \\
       &   &c_2  &+  &c_3&=  &0  \\
\end{linsys}
\grstep{\rho_1+\rho_2}
\begin{linsys}{3}
  c_1  &+  &c_2  &   &   &=  &1  \\
       &   &2c_2 &   &   &=  &2  \\
       &   &c_2  &+  &c_3&=  &0  \\
\end{linsys}
\grstep{(-1/2)\rho_2+\rho_3}
\begin{linsys}{3}
  c_1  &+  &c_2  &   &   &=  &1  \\
       &   &2c_2 &   &   &=  &2  \\
       &   &     &   &c_3&=  &-1  \\
\end{linsys}
\end{equation*}
giving this.
\begin{equation*}
  \rep{1+x}{B}=\colvec{0 \\ 1 \\ -1}
\end{equation*}


Finally, to find $\rep{1}{B}$ we do
\begin{equation*}
  1=c_1\cdot(1-x)+c_2\cdot (1+x+x^2)+c_3\cdot(x^2)
\end{equation*}
and get
\begin{equation*}
\begin{linsys}{3}
  c_1  &+  &c_2  &   &   &=  &1  \\
  -c_1 &+  &c_2  &   &   &=  &0  \\
       &   &c_2  &+  &c_3&=  &0  \\
\end{linsys}
\grstep{\rho_1+\rho_2}
\begin{linsys}{3}
  c_1  &+  &c_2  &   &   &=  &1  \\
       &   &2c_2 &   &   &=  &1  \\
       &   &c_2  &+  &c_3&=  &0  \\
\end{linsys}
\grstep{(-1/2)\rho_2+\rho_3}
\begin{linsys}{3}
  c_1  &+  &c_2  &   &   &=  &1  \\
       &   &2c_2 &   &   &=  &1  \\
       &   &     &   &c_3&=  &-1/2  \\
\end{linsys}
\end{equation*}
giving this.
\begin{equation*}
  \rep{1}{B}=\colvec{1/2 \\ 1/2 \\ -1/2}
\end{equation*}

In short, we have this matrix.
\begin{equation*}
Q=\rep{\identity}{\hat{B},B}=
\begin{mat}
  1/2  &0  &1/2  \\
  1/2  &1  &1/2  \\
  1/2  &-1 &-1/2
\end{mat}
\end{equation*}
\end{solution}

\part
Find the change of basis matrix $P=\rep{\identity}{D,\hat{D}}$.
\begin{solution}
First compute the effect of the map on the starting basis.
\begin{align*}
  \colvec{2 \\ 1}  
  &\mapsunder{\identity}  \colvec{2 \\ 1}  \\
  \colvec{0 \\ 2}  
  &\mapsunder{\identity}  \colvec{0 \\ 2}  \\
\end{align*}
Next represent each output with respect to the ending basis.
For
\begin{equation*}
  \rep{\colvec{2 \\ 1}}{\hat{D}}
\end{equation*}
we must solve this equation.
\begin{equation*}
  \colvec{2 \\ 1} = c_1\cdot\colvec{1 \\ 0}+c_2\cdot\colvec{0 \\ 1}
\end{equation*}
Obviously this is the answer.
\begin{equation*}
  \rep{\colvec{2 \\ 1}}{\hat{D}}=\colvec{2 \\ 1}
\end{equation*}
Similarly, for
\begin{equation*}
  \rep{\colvec{0 \\ 2}}{\hat{D}}
\end{equation*}
solve this equation
\begin{equation*}
  \colvec{2 \\ 1} = c_1\cdot\colvec{1 \\ 0}+c_2\cdot\colvec{0 \\ 1}
\end{equation*}
and get this.
\begin{equation*}
  \rep{\colvec{0 \\ 2}}{\hat{D}}=\colvec{0 \\ 2}
\end{equation*}
This is the matrix.
\begin{equation*}
P=\rep{\identity}{D,\hat{D}}=
\begin{mat}
2 &0 \\
1 &2 
\end{mat}
\end{equation*}
\end{solution}

\part
Find the representation $H=\rep{h}{B,D}$.
\begin{solution}
Here the effect of the map on the starting basis~$B$ is not trivial.
\begin{align*}
  1-x
  &\mapsunder{h} \colvec{-1 \\ 1}  \\
  1+x+x^2
  &\mapsunder{h} \colvec{2 \\ 2}  \\
  x^2
  &\mapsunder{h} \colvec{1 \\ 1}  
\end{align*}
To represent the first output with respect to the ending basis, set up
\begin{equation*}
  \colvec{-1 \\ 1}=c_1\cdot\colvec{2 \\ 1}+c_2\cdot\colvec{0 \\ 2}
\end{equation*}
and conclude this.
\begin{equation*}
  \rep{\colvec{-1 \\ 1}}{D}=\colvec{-1/2 \\ 3/4}
\end{equation*}

To represent the second output, set up
\begin{equation*}
  \colvec{2 \\ 2}=c_1\cdot\colvec{2 \\ 1}+c_2\cdot\colvec{0 \\ 2}
\end{equation*}
and by eye conclude this.
\begin{equation*}
  \rep{\colvec{2 \\ 2}}{D}=\colvec{1 \\ 1/2}
\end{equation*}

Finally, for the third,
\begin{equation*}
  \colvec{1 \\ 1}=c_1\cdot\colvec{2 \\ 1}+c_2\cdot\colvec{0 \\ 2}
\end{equation*}
gives this.
\begin{equation*}
  \rep{\colvec{1 \\ 1}}{D}=\colvec{1/2 \\ 1/4}
\end{equation*}

In total we have this matrix.
\begin{equation*}
  H=\rep{h}{B,D}=
  \begin{mat}
  -1/2  &1   &1/2  \\
  3/4   &1/2 &1/4
  \end{mat}
\end{equation*}
\end{solution}

\part
Find the representation $\hat{H}=\rep{h}{\hat{B},\hat{D}}$.
\begin{solution}
This is the effect of the map on the starting basis.
\begin{align*}
  1+x^2
  &\mapsunder{h} \colvec{1 \\ 2}  \\
  1+x
  &\mapsunder{h} \colvec{1 \\ 1}  \\
  1
  &\mapsunder{h} \colvec{0 \\ 1}  
\end{align*}

To represent the first with respect to the ending basis, set up
\begin{equation*}
\colvec{1 \\ 2}=c_1\colvec{1 \\ 0}+c_2\colvec{0 \\ 1}
\end{equation*}
and clearly the solution is this.
\begin{equation*}
\rep{\colvec{1 \\ 2}}{\hat{D}}=
\colvec{1 \\ 2}
\end{equation*}

The second and third are much the same.
\begin{equation*}
\rep{\colvec{1 \\ 1}}{\hat{D}}=
\colvec{1 \\ 1}
\qquad
\rep{\colvec{0 \\ 1}}{\hat{D}}=
\colvec{0 \\ 1}
\end{equation*}
That leads to this matrix.
\begin{equation*}
  \hat{H}=\rep{h}{\hat{B},\hat{D}}
  =
  \begin{mat}
    1  &1  &0  \\
    2  &1  &1
  \end{mat}
\end{equation*}
\end{solution}

\part
Compute $PHQ$
\begin{solution}
\begin{align*}
\begin{mat}
2 &0 \\
1 &2 
\end{mat}
  \begin{mat}
  -1/2  &1   &1/2  \\
  3/4   &1/2 &1/4
  \end{mat}
\begin{mat}
  1/2  &0  &1/2  \\
  1/2  &1  &1/2  \\
  1/2  &-1 &-1/2
\end{mat}
&=
\begin{mat}
 -1  &2  &1  \\
  1  &2  &1
\end{mat}
\begin{mat}
  1/2  &0  &1/2  \\
  1/2  &1  &1/2  \\
  1/2  &-1 &-1/2
\end{mat}                    \\
&=
  \begin{mat}
    1  &1  &0  \\
    2  &1  &1
  \end{mat}
\end{align*}
\end{solution}

\end{parts}
\end{questions}
\end{document}
