% \documentclass[noanswers, nolegalese, 11pt]{examjh}
\documentclass[answers, nolegalese, 11pt]{examjh}
\usepackage{../../../../sty/conc}
\usepackage{../../../../sty/linalgjh}
\usepackage{graphicx}

\setlength{\parindent}{0em}\setlength{\parskip}{0.5ex}
\pagestyle{empty}
\begin{document}\thispagestyle{empty}
\makebox[\textwidth]{Worksheet for Three.II (part 2)\hfill  From \textit{Linear Algebra}, by Hef{}feron}\vspace{-1ex}
\makebox[\textwidth]{\hbox{}\hrulefill\hbox{}}



\begin{questions}
\question
Consider the plane, $\R^2$, and its canonical basis.
\begin{equation*}
  \stdbasis_2=\sequence{\colvec{1 \\ 0}, \colvec{0 \\ 1}}
\end{equation*}
We can specify a linear map from $\R^2$ to itself, 
that is, a linear transformation
of~$\R^2$, by specifying its action on the basis vectors.
\begin{equation*}
  \colvec{1 \\ 0}\mapsunder{h}\colvec{1 \\ 1}
  \qquad
  \colvec{0 \\ 1}\mapsunder{h}\colvec{-2 \\ 3}
\end{equation*}
\begin{parts}
\part Find the value of the function on this input.
\begin{equation*}
  h(\colvec{4 \\ 5})
\end{equation*}
\begin{solution}
Expand the given vector in terms of the standard basis. 
\begin{equation*}
  h(\colvec{4 \\ 5})=h(\;4\cdot\colvec{1 \\ 0}+5\cdot\colvec{0 \\ 1}\;)
\end{equation*}
Apply that the map preserves addition
\begin{equation*}
  =h(4\cdot\colvec{1 \\ 0})+h(5\cdot\colvec{0 \\ 1})
\end{equation*}
and scalar multiplication.
\begin{equation*}
  =4\cdot h(\colvec{1 \\ 0})+5\cdot h(\colvec{0 \\ 1})
\end{equation*}
Finish by evaluating the map on the basis vectors.
\begin{equation*}
  =4\cdot \colvec{1 \\ 1}+5\cdot \colvec{-2 \\ 3}
  =\colvec{-6 \\ 19}
\end{equation*}
\end{solution}

\part Also find the output on this input.
\begin{equation*}
  h(\colvec{-7 \\ 0})
\end{equation*}
\begin{solution}
Go through the same steps as in the prior part.
\begin{align*}
  h(\colvec{-7 \\ 0})
  &=
  h(\;-7\cdot\colvec{1 \\ 0}+0\cdot\colvec{0 \\ 1}\;)       \\
  &=
  h(-7\cdot\colvec{1 \\ 0})+h(0\cdot\colvec{0 \\ 1})       \\
  &=
  -7\cdot h(\colvec{1 \\ 0})+0\cdot h(\colvec{0 \\ 1})       \\
  &=
  -7\cdot \colvec{1 \\ 1}+0\cdot \colvec{-2 \\ 3}       \\
  &=
  \colvec{-7 \\ -7}       \\
\end{align*}
\end{solution}

\part Find the general formula for this map.
\begin{equation*}
  h(\colvec{x \\ y})
\end{equation*}
\begin{solution}
We have this.
\begin{align*}
  h(\colvec{x \\ y})
  &=
  h(\;x\cdot\colvec{1 \\ 0}+y\cdot\colvec{0 \\ 1}\;)       \\
  &=
  h(x\cdot\colvec{1 \\ 0})+h(y\cdot\colvec{0 \\ 1})       \\
  &=
  x\cdot h(\colvec{1 \\ 0})+y\cdot h(\colvec{0 \\ 1})       \\
  &=
  x\cdot \colvec{1 \\ 1}+y\cdot \colvec{-2 \\ 3}       \\
  &=
  \colvec{x-2y \\ x+3y}       \\
\end{align*}
\end{solution}
\end{parts}



\question
Consider the vector space of quadratic polynomials.
\begin{equation*}
  \polyspace_2=\set{ax^2+bx+c \suchthat a,b,c\in\R}
\end{equation*}
This is a basis: $B=\set{x^2+1, x^2, 3x-1}$.
Define the the transformation 
$\map{f}{\polyspace_2}{\polyspace_2}$
by starting with this action on the elements of the basis~$B$,
and extending linearly.
\begin{equation*}
  x^2+1\mapsto 2x
  \qquad
  x^2\mapsto 2x
  \qquad
  3x-1\mapsto 3
\end{equation*}
\begin{parts}
\part What is $f(4x^2-2x+5)$?
\begin{solution}
First get an expression for $4x^2-2x+5$ as a linear combination of basis 
elements.
This equation
\begin{equation*}
  4x^2-2x+5=c_1(x^2+1)+c_2(x^2)+c_3(3x-1)
\end{equation*}
gives this linear system.
\begin{equation*}
\begin{linsys}{3}
  c_1  &+  &c_2  &   &     &=  &4  \\
       &   &     &   &3c_3 &=  &-2  \\
  c_1  &   &     &-  &c_3  &=  &5  \\
\end{linsys}
\end{equation*}
We can solve it by eye, giving $c_3=-2/3$, then $c_1=13/3$, and then
$c_2=-1/3$.  Restated, we have this.
\begin{equation*}
  \rep{4x^2-2x+5}{B}=\colvec{13/3 \\ -1/3 \\ -2/3}
\end{equation*}
Now proceed as in the prior question.
\begin{align*}
  f(4x^2-2x+5)
  &=
  f(\;\frac{13}{3}\cdot(x^2+1)+\frac{-1}{3}\cdot(x^2)+\frac{-2}{3}\cdot(3x-1) \;) \\
  &=
  f(\frac{13}{3}\cdot(x^2+1))+f(\frac{-1}{3}\cdot(x^2))+f(\frac{-2}{3}\cdot(3x-1)) \\
  &=
  \frac{13}{3}\cdot f(x^2+1)+\frac{-1}{3}\cdot f(x^2)+\frac{-2}{3}\cdot f(3x-1)  \\
  &=
  \frac{13}{3}\cdot (2x)+\frac{-1}{3}\cdot (2x)+\frac{-2}{3}\cdot (3)  \\
  &=
  8x-2
\end{align*}
\end{solution}

\part $f(-3x+2)$?
\begin{solution}
First,
\begin{equation*}
  -3x+2=c_1(x^2+1)+c_2(x^2)+c_3(3x-1)
\end{equation*}
gives this.
\begin{equation*}
\begin{linsys}{3}
  c_1  &+  &c_2  &   &     &=  &0  \\
       &   &     &   &3c_3 &=  &-3  \\
  c_1  &   &     &-  &c_3  &=  &2  \\
\end{linsys}
\end{equation*}
We get $c_3=-1$, then $c_1=1$, and then
$c_2=-1$. 
\begin{equation*}
  \rep{-3x+2}{B}=\colvec{1 \\ -1 \\ -1}
\end{equation*}
Now just apply linearity.
\begin{align*}
  f(-3x+2)
  &=
  f(\;1\cdot(x^2+1)-1\cdot(x^2)-1\cdot(3x-1) \;) \\
  &=
  f(1\cdot(x^2+1))+f(-1\cdot (x^2))+f(-1\cdot(3x-1)) \\
  &=
  f(x^2+1)-f(x^2)- f(3x-1)  \\
  &=
  (2x)- (2x)+- (3)  \\
  &=
  -3
\end{align*}
\end{solution}

\part Next we find the formula for $f(ax^2+bx+c)$.
First, represent $ax^2+bx+c$ with respect to the basis.
\begin{solution}
From
\begin{equation*}
  ax^2+bx+c=c_1(x^2+1)+c_2(x^2)+c_3(3x-1)
\end{equation*}
we get this.
\begin{equation*}
\begin{linsys}{3}
  c_1  &+  &c_2  &   &     &=  &a  \\
       &   &     &   &3c_3 &=  &b  \\
  c_1  &   &     &-  &c_3  &=  &c  \\
\end{linsys}
\end{equation*}
We get $c_3=b/3$, then $c_1=(b/3)+c$, and then
$c_2=a-(b/3)-c$. 
\begin{equation*}
  \rep{-3x+1}{B}=\colvec{(b/3)+c \\ a-(b/3)-c \\ b/3}
\end{equation*}
\end{solution}

\part Finish by applying linearity.
\begin{solution}
Using the prior part we have this.
\begin{align*}
  f(ax^2+bx+c)
  &=
  f(\;((b/3)+c)\cdot(x^2+1)+(a-(b/3)-c)\cdot(x^2)+(b/3)\cdot(3x-1) \;) \\
  &=
  ((b/3)+c)\cdot f(x^2+1)+(a-(b/3)-c)\cdot f(x^2)+(b/3)\cdot f(3x-1) \\
  &=
  ((b/3)+c)\cdot (2x)+(a-(b/3)-c)\cdot (2x)+(b/3)\cdot (3) \\
  &=
  2a\cdot x+b
\end{align*}
\end{solution}
\part What name do you know this map by?
\begin{solution}
This is the derivative.
\end{solution}

\end{parts}


\end{questions}
\end{document}
