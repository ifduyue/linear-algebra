% \documentclass[noanswers, nolegalese, 11pt]{examjh}
\documentclass[answers, nolegalese, 11pt]{examjh}
\usepackage{../../../../sty/conc}
\usepackage{../../../../sty/linalgjh}
\usepackage{graphicx}

\setlength{\parindent}{0em}\setlength{\parskip}{0.5ex}
\pagestyle{empty}
\begin{document}\thispagestyle{empty}
\makebox[\textwidth]{Worksheet for Three.II (part 3)\hfill  From \textit{Linear Algebra}, by Hef{}feron}\vspace{-1ex}
\makebox[\textwidth]{\hbox{}\hrulefill\hbox{}}



\begin{questions}
\question
Fix the vector spaces $V=\R^3$ and $W=\R^2$.
Consider the projection map $\map{f}{V}{W}$, which is a homomorphism.
\begin{equation*}
  \colvec{x \\ y \\ z}\mapsto\colvec{x \\ y}
\end{equation*}
\begin{parts}
\part
What members of the domain are mapped to this~$\vec{w}$?
\begin{equation*}
  \zero=\colvec{0 \\ 0}
\end{equation*}
\begin{solution}
This is the inverse image of the zero vector.
\begin{equation*}
  f^{-1}(\zero)=\set{\colvec{0 \\ 0 \\ z}\suchthat z \in\R}
\end{equation*}
In general, the inverse image of $\zero\in W$ is a subspace of~$V$.
It is the null space.
Here is a drawing.
\begin{center}
  \includegraphics{asy/map004.pdf}
\end{center}
\end{solution}

\part
What members of the domain are mapped 
to this member of $W$?
\begin{equation*}
  \vec{w}_1=\colvec{3 \\ 4}
\end{equation*}
Name one such vector, $\vec{v}_1$
\begin{solution}
This is the inverse image of~$\vec{w}_1$.
\begin{equation*}
  f^{-1}(\colvec{3 \\ 4})=\set{\colvec{3 \\ 4 \\ z}\suchthat z \in\R}
\end{equation*}
Here is a drawing.
\begin{center}
  \includegraphics{asy/map006.pdf}
\end{center}
One member of that set is this.
\begin{equation*}
  \vec{v_1}=\colvec{3 \\ 4 \\ 1}
\end{equation*}
\end{solution}

\part
What members of the domain are mapped 
to this member of $W$?
\begin{equation*}
  \vec{w}_2=\colvec{2 \\ 1}
\end{equation*}
Produce one such vector, $\vec{v}_2$.
\begin{solution}
This is the inverse image of~$\vec{w}_2$.
\begin{equation*}
  f^{-1}(\colvec{2 \\ 1})=\set{\colvec{2 \\ 1 \\ z}\suchthat z \in\R}
\end{equation*}
Here is a drawing.
\begin{center}
  \includegraphics{asy/map005.pdf}
\end{center}
One member of that set is this.
\begin{equation*}
  \vec{v_2}=\colvec{2 \\ 1 \\ 4}
\end{equation*}
\end{solution}

\part
In general, 
what members of the domain are mapped 
to this vector?
\begin{equation*}
  \vec{w}=\colvec{a \\ b}
\end{equation*}
\begin{solution}
This is the inverse image of~$\vec{w}$.
\begin{equation*}
  f^{-1}(\vec{w})=\set{\colvec{a \\ b \\ z}\suchthat z \in\R}
\end{equation*}
One member of that set is this.
\begin{equation*}
  \vec{v_2}=\colvec{a \\ b \\ -3}
\end{equation*}
\end{solution}

\part
Observe that $\vec{w}_1+\vec{w}_2$ gives this.
\begin{equation*}
  \vec{w}_3=\colvec{5 \\ 5}
\end{equation*}
What members of the domain are mapped 
to this~$\vec{w}_3$?
\begin{solution}
This is the inverse image of~$\vec{w}_3$.
\begin{equation*}
  f^{-1}(\colvec{5 \\ 5})=\set{\colvec{5 \\ 5 \\ z}\suchthat z \in\R}
\end{equation*}
Here is a drawing.
\begin{center}
  \includegraphics{asy/map007.pdf}
\end{center}
One member of that set is this.
\begin{equation*}
  \vec{v}_3=\colvec{5 \\ 5 \\ 5}
\end{equation*}
\end{solution}

\part
Add $\vec{v}_1+\vec{v}_2$.
Observe that it is a member of the
inverse image of $\vec{w}_3$.
\begin{solution}
A blue vector plus a red vector makes a purple vector.
\begin{equation*}
  \vec{v}_1+\vec{v}_2=\colvec{3 \\ 4 \\ 1}+\colvec{2 \\ 1 \\ 4}
          =\colvec{5 \\ 5 \\ 5}=\vec{v}_3
\end{equation*}
\end{solution}

\part
Observe that $6\cdot \vec{w}_2$ gives this.
\begin{equation*}
  6\vec{w}_2=\vec{w}_4=\colvec{12 \\ 6}
\end{equation*}
What members of the domain are mapped 
to this~$\vec{w}_4$?
\begin{solution}
This is the inverse image of~$\vec{w}_4$.
\begin{equation*}
  f^{-1}(\colvec{12 \\ 6})=\set{\colvec{12 \\ 6 \\ z}\suchthat z \in\R}
\end{equation*}
Here is a drawing.
\begin{center}
  \includegraphics{asy/map008.pdf}
\end{center}
One member of that set is this.
\begin{equation*}
  \vec{v}_6=\colvec{12 \\ 6 \\ -1}
\end{equation*}
\end{solution}

\part
Find $6\cdot\vec{v}_2$.
Observe that it is a member of the
inverse image of $\vec{w}_6$.
\begin{solution}
We have 
\begin{equation*}
  6\vec{v_2}=6\cdot\colvec{2 \\ 1 \\ 4}=\colvec{12 \\ 6 \\ 24}
\end{equation*}
and it projects to this vector.
\begin{equation*}
  f(6\vec{v_2})=f(\colvec{12 \\ 6 \\ 24})=\colvec{12 \\ 6}
\end{equation*}
\end{solution}
\end{parts}



\question
We can go through the same steps for this linear map $\map{h}{V}{W}$.
\begin{equation*}
  \colvec{x \\ y \\ z}\mapsto\colvec{x+y \\ x+z}
\end{equation*}
\begin{parts}
\part
What members of the domain are mapped to this~$\vec{w}$?
\begin{equation*}
  \zero=\colvec{0 \\ 0}
\end{equation*}
\begin{solution}
This is the inverse image of the zero vector  $\zero\in W$.
\begin{equation*}
  h^{-1}(\zero)=\set{\colvec{x \\ y \\ z}\suchthat \text{$x+y=0$ and $x+z=0$}}
\end{equation*}
We get this linear system.
\begin{equation*}
\begin{linsys}{3}
 x  &+  &y  &    &     &=   &0  \\
 x  &   &   &+   &z    &=   &0  \\
\end{linsys}
\grstep{-\rho_1+\rho_2}
\begin{linsys}{3}
 x  &+  &y  &    &     &=   &0  \\
    &   &-y &+   &z    &=   &0  \\
\end{linsys}
\end{equation*}
So we have that the null space is this.
\begin{equation*}
  \nullspace{h}=
  \set{\colvec{-z \\ z \\ z}\suchthat z\in\R}
     =\set{\colvec{-1 \\ 1 \\ 1}\cdot z \suchthat z\in\R}
\end{equation*}
Here is the picture.
\begin{center}
  \includegraphics{asy/map009.pdf}
\end{center}
\end{solution}

\part
What members of the domain are mapped 
to this member of $W$?
\begin{equation*}
  \vec{w}_1=\colvec{3 \\ 4}
\end{equation*}
\begin{solution}
To find the inverse image of~$\vec{w}_1$,
we get this linear system.
\begin{equation*}
\begin{linsys}{3}
 x  &+  &y  &    &     &=   &3  \\
 x  &   &   &+   &z    &=   &4  \\
\end{linsys}
\grstep{-\rho_1+\rho_2}
\begin{linsys}{3}
 x  &+  &y  &    &     &=   &3  \\
    &   &-y &+   &z    &=   &1  \\
\end{linsys}
\end{equation*}
So $y=-1+z$ and $x=4-z$.
\begin{equation*}
  h^{-1}(\colvec{3 \\ 4})=\set{\colvec{4-z \\ -1+z \\ z}\suchthat z \in\R}
        =\set{\colvec{4 \\ -1 \\ 0}+\colvec{-1 \\ 1 \\ 1}\cdot z\suchthat z \in\R}
\end{equation*}
Here is a drawing.
\begin{center}
  \includegraphics{asy/map010.pdf}
\end{center}
\end{solution}

\part
What members of the domain are mapped 
to this member of $W$?
\begin{equation*}
  \vec{w}_2=\colvec{2 \\ 1}
\end{equation*}
\begin{solution}
To find the inverse image of~$\vec{w}_1$,
we get this linear system.
\begin{equation*}
\begin{linsys}{3}
 x  &+  &y  &    &     &=   &2  \\
 x  &   &   &+   &z    &=   &1  \\
\end{linsys}
\grstep{-\rho_1+\rho_2}
\begin{linsys}{3}
 x  &+  &y  &    &     &=   &2  \\
    &   &-y &+   &z    &=   &-1  \\
\end{linsys}
\end{equation*}
So $y=1+z$ and $x=1-z$.
\begin{equation*}
  h^{-1}(\colvec{2 \\ 1})=\set{\colvec{1-z \\ 1+z \\ z}\suchthat z \in\R}
        =\set{\colvec{1 \\ 1 \\ 0}+\colvec{-1 \\ 1 \\ 1}\cdot z\suchthat z \in\R}
\end{equation*}
Here is a drawing.
\begin{center}
  \includegraphics{asy/map011.pdf}
\end{center}
\end{solution}

\part
In general, 
what members of the domain are mapped 
to this vector?
\begin{equation*}
  \vec{w}=\colvec{a \\ b}
\end{equation*}
\begin{solution}
We get this linear system.
\begin{equation*}
\begin{linsys}{3}
 x  &+  &y  &    &     &=   &a  \\
 x  &   &   &+   &z    &=   &b  \\
\end{linsys}
\grstep{-\rho_1+\rho_2}
\begin{linsys}{3}
 x  &+  &y  &    &     &=   &a  \\
    &   &-y &+   &z    &=   &-a+b  \\
\end{linsys}
\end{equation*}
So $y=(a-b)+z$ and $x=b-z$.
\begin{equation*}
  h^{-1}(\colvec{a \\ b})=\set{\colvec{b-z \\ (a-b)+z \\ z}\suchthat z \in\R}
        =\set{\colvec{b \\ a-b \\ 0}+\colvec{-1 \\ 1 \\ 1}\cdot z\suchthat z \in\R}
\end{equation*}
Here is a drawing.
\begin{center}
  \includegraphics{asy/map012.pdf}
\end{center}
\end{solution}

\end{parts}

\end{questions}
\end{document}
