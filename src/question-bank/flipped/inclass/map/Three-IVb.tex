% \documentclass[noanswers, nolegalese, 11pt]{examjh}
\documentclass[answers, nolegalese, 11pt]{examjh}
\usepackage{../../../../sty/conc}
\usepackage{../../../../sty/linalgjh}
\usepackage{graphicx}

\setlength{\parindent}{0em}\setlength{\parskip}{0.5ex}
\pagestyle{empty}
\begin{document}\thispagestyle{empty}
\makebox[\textwidth]{Worksheet for Three.IV (part 3)\hfill  From \textit{Linear Algebra}, by Hef{}feron}\vspace{-1ex}
\makebox[\textwidth]{\hbox{}\hrulefill\hbox{}}

Fix these matrices.
\begin{equation*}
  M=
  % \begin{mat}
  %   1  &2  &3  \\
  %   4  &5  &6  \\
  %   7  &8  &9
  % \end{mat}
  \begin{mat}
    1  &2  &0  \\
    0  &4  &0  \\
   3   &6  &1
  \end{mat}
 \qquad
 I=
 \begin{mat}
 1  &0  &0  \\
 0  &1  &0  \\
 0  &0  &1
 \end{mat}
\end{equation*}



\begin{questions}
% sage: M = matrix(QQ, [[1,2,0], [0,4,0], [3,6,1]])
% sage: gauss_jordan(M)
% [1 2 0]
% [0 4 0]
% [3 6 1]
%  take -3 times row 1 plus row 3
% [1 2 0]
% [0 4 0]
% [0 0 1]
%  take 1/4 times row 2
% [1 2 0]
% [0 1 0]
% [0 0 1]
%  take -2 times row 2 plus row 1
% [1 0 0]
% [0 1 0]
% [0 0 1]

\question
We will perform Gauss-Jordan reduction on the matrix in two ways.
\begin{parts}
\part
Perform the step $-3\rho_1+\rho_3$ on $M$ to get a matrix~$A$.
\part
Instead, multiply $M$ from the left by this matrix.
\begin{equation*}
R_1=\begin{mat}
  1  &0  &0  \\
  0  &1  &0  \\
  -3 &0  &1
\end{mat}
\end{equation*}
Note that this matrix is the result of starting with~$I$ 
and performing $-3\rho_1+\rho_3$.
% sage: M = matrix(QQ, [[1,2,0], [0,4,0], [3,6,1]])
% sage: R1 = matrix(QQ, [[1,0,0], [0,1,0], [-3,0,1]])
% sage: A = R1*M
% sage: A
% [1 2 0]
% [0 4 0]
% [0 0 1]
\part 
Next, perform the step $(1/4)\rho_2$ on~$A$ to get a matrix~$B$.
\part
Instead, multiply $A$ from the left by this matrix.
\begin{equation*}
R_2=\begin{mat}
  1  &0  &0  \\
  0  &1/4  &0  \\
  0 &0  &1
\end{mat}
\end{equation*}
Note that this matrix is the result of starting with~$I$ 
and performing $(1/4)\rho_2$.
% sage: R2 = matrix(QQ, [[1,0,0], [0,1/4,0], [0,0,1]])
% sage: B = R2*A
% sage: B
% [1 2 0]
% [0 1 0]
% [0 0 1]
\part 
Finally, perform the step $-2\rho_2+\rho_1$ on~$B$.
\part
Instead, multiply $B$ from the left by this matrix.
\begin{equation*}
R_3=\begin{mat}
  1  &-2 &0  \\
  0  &1  &0  \\
  0  &0  &1
\end{mat}
% sage: R3 = matrix(QQ, [[1,-2,0], [0,1,0], [0,0,1]])
% sage: C = R3*B
% sage: C
% [1 0 0]
% [0 1 0]
% [0 0 1]
\end{equation*}
Note that this matrix is the result of starting with~$I$ 
and performing $-2\rho_2+\rho_1$.
\part
So we have $R_3R_2R_1\cdot M=I$.
Multiply $R_2R_1$, then find $R_3(R_2R_1)$.
% sage: R3*R2*R1
% [   1 -1/2    0]
% [   0  1/4    0]
% [  -3    0    1]
\end{parts}

\question
Combine the two matrices $M$ and $I$
by writing the numbers from $M$, and next to those writing 
the numbers from $I$, 
with a vertical line 
separating the two.
\begin{parts}
\part
Perform $-3\rho_1+\rho_3$ on the combination.
\part
On the result, perform $(1/4)\rho_2$.
\part
Finish by doing $-2\rho_2+\rho_1$.
Compare the right hand side to $R_3R_2R_1$ from the prior question.
% sage: M^(-1)
% [   1 -1/2    0]
% [   0  1/4    0]
% [  -3    0    1]
\end{parts}

\end{questions}
\end{document}
