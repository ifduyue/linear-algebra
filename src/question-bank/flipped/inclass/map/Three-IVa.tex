% \documentclass[noanswers, nolegalese, 11pt]{examjh}
\documentclass[answers, nolegalese, 11pt]{examjh}
\usepackage{../../../../sty/conc}
\usepackage{../../../../sty/linalgjh}
\usepackage{graphicx}

\setlength{\parindent}{0em}\setlength{\parskip}{0.5ex}
\pagestyle{empty}
\begin{document}\thispagestyle{empty}
\makebox[\textwidth]{Worksheet for Three.IV (part 2)\hfill  From \textit{Linear Algebra}, by Hef{}feron}\vspace{-1ex}
\makebox[\textwidth]{\hbox{}\hrulefill\hbox{}}

\begin{questions}
\question
Consider the plane~$\R^2$.
Its canonical basis is this.
\begin{equation*}
  \stdbasis_2=\sequence{\vec{e}_1,\vec{e}_2}
      =\sequence{\colvec{1 \\ 0},
                  \colvec{0 \\ 1}}
\end{equation*}
\begin{parts}
\part
Draw a picture of the plane, and of $\vec{e}_1$ rotated through
an angle of $\theta$ counterclockwise.
\part
Do the same for $\vec{e}_2$.
\part 
Represent the linear map of rotation through an angle of $\theta$
counterclockwise, with respect to the standard bases 
$\stdbasis_2,\stdbasis_2$. 
\end{parts}


\question
Fix this matrix.
\begin{equation*}
M=
\begin{mat}
  1  &2  &3  \\
  4  &5  &6  \\
  7  &8  &9
\end{mat}
\end{equation*}
\begin{parts}
\part
Multiply $M$ from the left by these two.
\begin{equation*}
\begin{mat}
  0  &0  &0  \\
  0  &0  &1  \\
  0  &0  &0  
\end{mat}
 \qquad
\begin{mat}
  0  &1  &0  \\
  0  &0  &0
\end{mat}
\end{equation*}
\part
Multiply $M$ from the right by these two.
\begin{equation*}
\begin{mat}
  0  &0  &0  \\
  0  &0  &1  \\
  0  &0  &0  
\end{mat}
\qquad
\begin{mat}
  0  &1    \\
  0  &0    \\
  0  &0
\end{mat}
 \qquad
\end{equation*}
\part
Multiply it from the left by this permutation matrix.
\begin{equation*}
P=\begin{mat}
  0  &1  &0  \\
  0  &0  &1  \\
  1  &0  &0  
\end{mat}
\end{equation*}
\part Multiply $M$ from the right by the same matrix.
\end{parts}

\question
Fix these two matrices.
\begin{equation*}
F=
\begin{mat}
  1  &2  &3  \\
  4  &5  &6  \\
  7  &8  &9
\end{mat}
\qquad
G=
\begin{mat}
  1  &0  &2  \\
  0  &1  &3  \\
  2  &-1  &0
\end{mat}
\end{equation*}
\begin{parts}
\part
Compute $FG$.
\part
Compute these three.
\begin{equation*}
F\colvec{1 \\ 0 \\ 2}
\qquad
F\colvec{0 \\ 1 \\ -1}
\qquad
F\colvec{2 \\ 3 \\ 0}
\end{equation*}
\part
Compute these three.
\begin{equation*}
\begin{mat}
  1  &2  &3  
\end{mat}
G
\quad
\begin{mat}
  4  &5  &6  
\end{mat}
G
\quad
\begin{mat}
  7  &8  &9  
\end{mat}
G
\end{equation*}
\end{parts}


\end{questions}
\end{document}
