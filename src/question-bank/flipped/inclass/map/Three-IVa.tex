% \documentclass[noanswers, nolegalese, 11pt]{examjh}
\documentclass[answers, nolegalese, 11pt]{examjh}
\usepackage{../../../../sty/conc}
\usepackage{../../../../sty/linalgjh}
\usepackage{graphicx}

\setlength{\parindent}{0em}\setlength{\parskip}{0.5ex}
\pagestyle{empty}
\begin{document}\thispagestyle{empty}
\makebox[\textwidth]{Worksheet for Three.IV (part 2)\hfill  From \textit{Linear Algebra}, by Hef{}feron}\vspace{-1ex}
\makebox[\textwidth]{\hbox{}\hrulefill\hbox{}}

\begin{questions}
\question
Fix this matrix.
\begin{equation*}
M=
\begin{mat}
  1  &2  &3  \\
  4  &5  &6  \\
  7  &8  &9
\end{mat}
\end{equation*}
\begin{parts}
\part
Multiply $M$ from the left by these two.
\begin{equation*}
\begin{mat}
  0  &0  &0  \\
  0  &0  &1  \\
  0  &0  &0  
\end{mat}
 \qquad
\begin{mat}
  0  &1  &0  \\
  0  &0  &0
\end{mat}
\end{equation*}
\begin{solution}
We have this
\begin{equation*}
\begin{mat}
  0  &0  &0  \\
  0  &0  &1  \\
  0  &0  &0  
\end{mat}
\begin{mat}
  1  &2  &3  \\
  4  &5  &6  \\
  7  &8  &9
\end{mat}
=
\begin{mat}
  0  &0  &0  \\
  7  &8  &9  \\
  0  &0  &0
\end{mat}
\end{equation*}
and also this.
\begin{equation*}
\begin{mat}
  0  &1  &0  \\
  0  &0  &0
\end{mat}
\begin{mat}
  1  &2  &3  \\
  4  &5  &6  \\
  7  &8  &9
\end{mat}
=
\begin{mat}
  4  &5  &6  \\
  0  &0  &0
\end{mat}
\end{equation*}
\end{solution}

\part
Multiply $M$ from the right by these two.
\begin{equation*}
\begin{mat}
  0  &0  &0  \\
  0  &0  &1  \\
  0  &0  &0  
\end{mat}
\qquad
\begin{mat}
  0  &1    \\
  0  &0    \\
  0  &0
\end{mat}
\end{equation*}
\begin{solution}
The first is
\begin{equation*}
\begin{mat}
  1  &2  &3  \\
  4  &5  &6  \\
  7  &8  &9
\end{mat}
\begin{mat}
  0  &0  &0  \\
  0  &0  &1  \\
  0  &0  &0  
\end{mat}
=
\begin{mat}
  0  &0  &2  \\
  0  &0  &5  \\
  0  &0  &8
\end{mat}
\end{equation*}
and the second is this.
\begin{equation*}
\begin{mat}
  1  &2  &3  \\
  4  &5  &6  \\
  7  &8  &9
\end{mat}
\begin{mat}
  0  &1    \\
  0  &0    \\
  0  &0
\end{mat}
=
\begin{mat}
  0  &1  \\
  0  &4  \\
  0  &7
\end{mat}
\end{equation*}
\end{solution}

\part
Multiply it from the left by this permutation matrix.
\begin{equation*}
P=\begin{mat}
  0  &1  &0  \\
  0  &0  &1  \\
  1  &0  &0  
\end{mat}
\end{equation*}
\begin{solution}
\begin{equation*}
PM=\begin{mat}
  0  &1  &0  \\
  0  &0  &1  \\
  1  &0  &0  
\end{mat}
\begin{mat}
  1  &2  &3  \\
  4  &5  &6  \\
  7  &8  &9
\end{mat}
=
\begin{mat}
  4  &5  &6  \\
  7  &8  &9  \\
  1  &2  &3
\end{mat}
\end{equation*}
\end{solution}

\part Multiply $M$ from the right by the same matrix.
\begin{solution}
\begin{equation*}
MP=
\begin{mat}
  1  &2  &3  \\
  4  &5  &6  \\
  7  &8  &9
\end{mat}
\begin{mat}
  0  &1  &0  \\
  0  &0  &1  \\
  1  &0  &0  
\end{mat}
=
\begin{mat}
  3  &1  &2  \\
  6  &4  &5  \\
  9  &7  &8
\end{mat}
\end{equation*}
\end{solution}
\end{parts}

\question
Fix these two matrices.
\begin{equation*}
F=
\begin{mat}
  1  &2  &3  \\
  4  &5  &6  \\
  7  &8  &9
\end{mat}
\qquad
G=
\begin{mat}
  1  &0  &2  \\
  0  &1  &3  \\
  2  &-1  &0
\end{mat}
\end{equation*}
\begin{parts}
\part
Compute $FG$.
\begin{solution}
\begin{equation*}
FG=
\begin{mat}
  1  &2  &3  \\
  4  &5  &6  \\
  7  &8  &9
\end{mat}
\begin{mat}
  1  &0  &2  \\
  0  &1  &3  \\
  2  &-1  &0
\end{mat}
=
\begin{mat}
  7  &-1  &8  \\
  16 &-1  &23 \\
  25 &-1  &38
\end{mat}
\end{equation*}
\end{solution}

\part
Compute these three.
\begin{equation*}
F\colvec{1 \\ 0 \\ 2}
\qquad
F\colvec{0 \\ 1 \\ -1}
\qquad
F\colvec{2 \\ 3 \\ 0}
\end{equation*}
\begin{solution}
This is the first.
\begin{equation*}
\begin{mat}
  1  &2  &3  \\
  4  &5  &6  \\
  7  &8  &9
\end{mat}
\colvec{1 \\ 0 \\ 2}
=
\colvec{7  \\ 16  \\ 25}
\end{equation*}
This is the second
\begin{equation*}
\begin{mat}
  1  &2  &3  \\
  4  &5  &6  \\
  7  &8  &9
\end{mat}
\colvec{0 \\ 1 \\ -1}
=
\colvec{-1  \\ -1  \\ -1}
\end{equation*}
and the third.
\begin{equation*}
\begin{mat}
  1  &2  &3  \\
  4  &5  &6  \\
  7  &8  &9
\end{mat}
\colvec{2 \\ 3 \\ 0}
=
\colvec{8  \\ 23  \\ 38}
\end{equation*}
\end{solution}

% \part
% Compute these three.
% \begin{equation*}
% \begin{mat}
%   1  &2  &3  
% \end{mat}
% G
% \quad
% \begin{mat}
%   4  &5  &6  
% \end{mat}
% G
% \quad
% \begin{mat}
%   7  &8  &9  
% \end{mat}
% G
% \end{equation*}
\end{parts}



\end{questions}
\end{document}
