% \documentclass[noanswers, nolegalese, 11pt]{examjh}
\documentclass[answers, nolegalese, 11pt]{examjh}
\usepackage{../../../../sty/conc}
\usepackage{../../../../sty/linalgjh}
\usepackage{graphicx}

\setlength{\parindent}{0em}\setlength{\parskip}{0.5ex}
\pagestyle{empty}
\begin{document}\thispagestyle{empty}
\makebox[\textwidth]{Worksheet for Three.V (part 1)\hfill  From \textit{Linear Algebra}, by Hef{}feron}\vspace{-1ex}
\makebox[\textwidth]{\hbox{}\hrulefill\hbox{}}


\begin{questions}
\question
Consider $\vec{v}\in\polyspace_2$ given by $\vec{v}=x^2+2x+3$.
\begin{parts}
\part
Represent this vector with respect to 
$B=\sequence{1-x, 1+x+x^2, x^2}$.
\begin{solution}
To find the coefficients in this equation.
\begin{equation*}
  x^2+2x+3=c_1\cdot(1-x)+c_2\cdot(1+x+x^2)+c_3\cdot (x^2)
\end{equation*}
we can do Gauss's method on this linear system.
\begin{align*}
  \begin{linsys}{3}
    c_1  &+  &c_2  &   &   &=  &3  \\
   -c_1  &+  &c_2  &   &   &=  &2  \\
         &   &c_2  &+  &c_3&=  &1  \\
  \end{linsys}
&\grstep{\rho_1+\rho_2}
  \begin{linsys}{3}
    c_1  &+  &c_2  &   &   &=  &3  \\
         &   &2c_2 &   &   &=  &5  \\
         &   &c_2  &+  &c_3&=  &1  \\
  \end{linsys}                             \\
&\grstep{(-1/2)\rho_2+\rho_3}
  \begin{linsys}{3}
    c_1  &+  &c_2  &   &   &=  &3  \\
         &   &2c_2 &   &   &=  &5  \\
         &   &     &   &c_3&=  &-3/2  \\
  \end{linsys}                            
\end{align*}
We get this.
\begin{equation*}
  \rep{\vec{v}}{B}=\colvec{1/2 \\ 5/2 \\ -3/2}
\end{equation*}
\end{solution}

\part
Represent it with respect to 
$D=\sequence{1+x^2, 1+x, 1}$.
\begin{solution}
From 
\begin{equation*}
  x^2+2x+3 =c_1\cdot(1+x^2)+c_2\cdot(1+x)+c_3\cdot (1)
\end{equation*}
by eye
the equation of $x^2$'s gives $c_1=1$, 
the equation of $x$'s gives $c_2=2$, and then from 
the equation of constants $c_1+c_2+c_3=3$ we have $c_3=0$.
\begin{equation*}
  \rep{\vec{v}}{D}=\colvec{1 \\ 2 \\ 0}
\end{equation*}
\end{solution}

\part
Represent $B$'s basis vectors $\vec{\beta}_1$, 
$\vec{\beta}_2$, and $\vec{\beta}_3$ with respect to~$D$.
\begin{solution}
For $\rep{\vec{\beta}_1}{D}$, solve this.
\begin{equation*}
  1-x = c_1\cdot (1+x^2)+c_2\cdot (1+x)+c_3\cdot (1)
\end{equation*}
By eye we get $c_1=0$, $c_2=1$, and $c_3=2$.
\begin{equation*}
  \rep{1-x}{D}=\colvec{0 \\ -1 \\ 2}
\end{equation*}

For $\rep{\vec{\beta}_2}{D}$, solve this.
\begin{equation*}
  1+x+x^2 = c_1\cdot (1+x^2)+c_2\cdot (1+x)+c_3\cdot (1)
\end{equation*}
By eye, $c_1=1$, $c_2=1$, and $c_3=-1$.
\begin{equation*}
  \rep{1+x+x^2}{D}=\colvec{1 \\ 1 \\ -1}
\end{equation*}

Finally, for $\rep{\vec{\beta}_2}{D}$
\begin{equation*}
  x^2 = c_1\cdot (1+x^2)+c_2\cdot (1+x)+c_3\cdot (1)
\end{equation*}
we get this.
\begin{equation*}
  \rep{x^2}{D}=\colvec{1 \\ 0 \\ -1}
\end{equation*}
\end{solution}

\part
Find the change of basis matrix $\rep{\identity}{B,D}$
\begin{solution}
To find the matrix representing any map, not just the identity,
we first determine what the map does to the basis vectors in the domain.
In this case we have this.
\begin{equation*}
  \vec{\beta_1}\mapsunder{\identity}\vec{\beta_1}
  \quad
  \vec{\beta_2}\mapsunder{\identity}\vec{\beta_2}
  \quad
  \vec{\beta_3}\mapsunder{\identity}\vec{\beta_3}
\end{equation*}
Next, represent those outputs with respect to the basis vectors in the 
codomain, the output space.
We did this work in the prior item.
\begin{equation*}
  \rep{1-x}{D}=\colvec{0 \\ -1 \\ 2}
  \quad
  \rep{1+x+x^2}{D}=\colvec{1 \\ 1 \\ -1}
  \quad
  \rep{x^2}{D}=\colvec{1 \\ 0 \\ -1}
\end{equation*}
Then the answer puts the columns together.
\begin{equation*}
  \rep{\identity}{B,D}
  =
  \begin{mat}
    0  &1  &1  \\
   -1  &1  &0  \\
    2  &-1 &-1
  \end{mat}
\end{equation*}
\end{solution}

\part 
Use it to convert $\rep{\vec{v}}{B}$ to 
$\rep{\vec{v}}{D}$.
\begin{solution}
Do the matrix vector multiplication.
\begin{equation*}
  \rep{\identity}{B,D}\rep{\vec{v}}{B}
=
  \begin{mat}
    0  &1  &1  \\
   -1  &1  &0  \\
    2  &-1 &-1
  \end{mat}
\colvec{1/2 \\ 5/2 \\ -3/2}
=
\colvec{1 \\ 2 \\ 0}
=  \rep{\vec{v}}{D}
\end{equation*}
\end{solution}
\end{parts}

\question
Find the change of basis matrix to convert from 
representations with respect to the first 
basis to representations with respect to the second.
\begin{equation*}
  B=\sequence{\colvec{1 \\ 1}, \colvec{2 \\ 0}}
  \qquad
  D=\sequence{\colvec{0 \\ 4}, \colvec{-1 \\ 1}}
\end{equation*}
\begin{solution}
First,
\begin{equation*}
  \colvec{1 \\ 1}=c_1\colvec{0 \\ 4} + c_2\colvec{-1 \\ 1}
\end{equation*}
gives this. 
\begin{equation*}
  \rep{\vec{\beta}_1}{D}=\colvec{1/2 \\ -1}
\end{equation*}
In addition,
\begin{equation*}
  \colvec{2 \\ 0}=c_1\colvec{0 \\ 4} + c_2\colvec{-1 \\ 1}
\end{equation*}
gives this. 
\begin{equation*}
  \rep{\vec{\beta}_2}{D}=\colvec{1/2 \\ -2}
\end{equation*}
So the change of basis matrix is this.
\begin{equation*}
\rep{\identity}{B,D}
=
\begin{mat}
  1/2  &1/2  \\
   -1  &-2
\end{mat}
\end{equation*}
\end{solution}

\question
Which of these matrices could be change of basis matrices?
\begin{parts}
\part 
$
\begin{mat}
  1  &2 \\
  3  &4
\end{mat}
$
\begin{solution}
This matrix is nonsingular, for instance since Gauss's method 
gives an echelon form matrix without any zero rows.
So it could be used to change bases. 
\end{solution}

\part
$
\begin{mat}
  1  &2  &3  \\
  4  &5  &6  \\
  7  &8  &9
\end{mat}
$
\begin{solution}
This matrix is singular; Gauss's method 
gives an echelon form matrix with a zero row.
So it could not be used as a change of basis matrix. 
\end{solution}
\end{parts}



\end{questions}
\end{document}
