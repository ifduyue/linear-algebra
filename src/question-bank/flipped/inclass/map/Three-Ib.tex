% \documentclass[noanswers, nolegalese, 11pt]{examjh}
\documentclass[answers, nolegalese, 11pt]{examjh}
\usepackage{../../../../sty/conc}
\usepackage{../../../../sty/linalgjh}
\usepackage{graphicx}

\setlength{\parindent}{0em}\setlength{\parskip}{0.5ex}
\pagestyle{empty}
\begin{document}\thispagestyle{empty}
\makebox[\textwidth]{Worksheet for Three.I, part three\hfill  From \textit{Linear Algebra}, by Hef{}feron}\vspace{-1ex}
\makebox[\textwidth]{\hbox{}\hrulefill\hbox{}}


Isomorphisms are correspondences between bases.

\begin{questions}
\question
Consider $\polyspace_2$.
\begin{parts}
\part 
We could verify that 
$B=\sequence{x^2+x, x^2, x^2+1}$ is a basis (but we won't, just in the
interests of time).
What is the dimension of $\polyspace_2$?
\begin{solution}
There are three members of the basis, so $V$ is $3$~dimenstional.
\end{solution}

\part
Represent $3x^2+2x+1$ in terms of that basis.
That is, find $c_1$, $c_2$, $c_3$ in this equation.
\begin{equation*}
  c_1(x^2+x)+c_2(x^2)+c_3(x^2+1)=3x^2+2x+1
\end{equation*}
\begin{solution}
For the values of $c_1$, $c_2$, and~$c_3$ in this equation
\begin{equation*}
  c_1(x^2+x)+c_2(x^2)+c_3(x^2+1)=3x^2+2x+1
\end{equation*}
we get a linear system, where the first equation finds the 
coefficient of~$x^2$ on both sides, etc.
\begin{equation*}
\begin{linsys}{3}
  c_1  &+  &c_2  &+  &c_3  &=  &3  \\
  c_1  &   &     &   &     &=  &2  \\
       &   &     &   &c_3  &=  &1  \\
\end{linsys}
\end{equation*}
By eye, we get $c_1=2$, $c_3=1$, and $c_2=0$.
\begin{equation*}
  \rep{3x^2+2x+1}{B}=\colvec{2 \\ 0 \\ 1}
\end{equation*}
\end{solution}

\part
Represent $ax^2+bx+c$ in terms of that basis.
That is, find $c_1$, $c_2$, $c_3$ in this equation.
\begin{equation*}
  c_1(x^2+x)+c_2(x^2)+c_3(x^2+1)=ax^2+bx+c
\end{equation*}
\begin{solution}
We are given $a$, $b$, and $c$. 
We want the values of $c_1$, $c_2$, and~$c_3$ in this equation.
\begin{equation*}
  c_1(x^2+x)+c_2(x^2)+c_3(x^2+1)=ax^2+bx+c
\end{equation*}
As in the prior problem, get the linear system.
\begin{equation*}
\begin{linsys}{3}
  c_1  &+  &c_2  &+  &c_3  &=  &a  \\
  c_1  &   &     &   &     &=  &b  \\
       &   &     &   &c_3  &=  &c  \\
\end{linsys}
\end{equation*}
Also as in the prior problem, by eye we get $c_1=b$, $c_3=c$, and $c_2=a-b-c$.
\begin{equation*}
  b\cdot(x^2+x)+(a-b-c)\cdot(x^2)+c\cdot(x^2+1)=ax^2+bx+c
\end{equation*}

\end{solution}
\end{parts}

\question
Consider this subspace of $\Re^4$.
\begin{equation*}
  W=\set{\colvec{p \\ q \\ r \\ s}\suchthat p-q+2s=0}
\end{equation*}
\begin{parts}
\part
Find a basis~$D$ for the space~$W$
(because of time we won't verify that it is linearly independent).
What is the dimension of~$W$?
\begin{solution}
We have this.
\begin{align*}
  \set{\colvec{p \\ q \\ r \\ s}\suchthat p-q+2s=0}
  &=
  \set{\colvec{q-2s \\ q \\ r \\ s}\suchthat q,r,s\in\R}       \\
  &=
  \set{q\cdot\colvec{1 \\ 1 \\ 0 \\ 0}
       +r\cdot\colvec{0 \\ 0 \\ 1 \\ 0}
       +s\cdot\colvec{-2 \\ 0 \\ 0 \\ 1}
       \suchthat q,r,s\in\R}  
\end{align*}
So we take this to be the basis.
\begin{equation*}
  D=\sequence{\colvec{1 \\ 1 \\ 0 \\ 0},
              \colvec{0 \\ 0 \\ 1 \\ 0},
              \colvec{-2 \\ 0 \\ 0 \\ 1}
             }
\end{equation*}
This basis has three elements, so $W$ is $3$~dimensional.
\end{solution}
\end{parts}

\question
The dimensions are the same so we can make an isomorphism 
by making a correspondence between basis elements.
$\map{f}{V}{W}$.
\begin{parts}
\part
Associate the first member of~$B$, the polynomial $x^2+x$, with the
second member of~$D$.
Associate the polynomial $x^2$ with the
third member of~$D$.
Associate $x^2+1$ with the
first member of~$D$.
\begin{solution}
\begin{equation*}
  x^2+x\longleftrightarrow\colvec{0 \\ 0 \\ 1 \\ 0}
  \qquad
  x^2\longleftrightarrow\colvec{-2 \\ 0 \\ 0 \\ 1}
  \qquad
  x^2+1\longleftrightarrow\colvec{1 \\ 1 \\ 0 \\ 0}
\end{equation*}
\end{solution}

\part
Extend linearly, that is, give the formula for $f(ax^2+bx+c)$ by
using the first question, 
and that the function preserves addition and scalar multiplication.
\begin{solution}
\begin{align*}
  f(ax^2+bx+c)
  &=
  f(\;b\cdot(x^2+x)+(a-b-c)\cdot(x^2)+c\cdot(x^2+1)\;)       \\
  &=
  f(\;b\cdot(x^2+x)\;)+f(\;(a-b-c)\cdot(x^2)\;)+f(\;c\cdot(x^2+1)\;)       \\
  &=
  b\cdot f(x^2+x)+(a-b-c)\cdot f(x^2)+c\cdot f(x^2+1)        \\
  &=
  b\cdot \colvec{0 \\ 0 \\ 1 \\ 0}
  +(a-b-c)\cdot \colvec{-2 \\ 0 \\ 0 \\ 1}
  +c\cdot\colvec{1 \\ 1 \\ 0 \\ 0}                  \\
  &=
  % \colvec{-2\cdot(a-b-c)+c \\ c \\ b \\ a-b-c}
  % =
  \colvec{-2a+2b+3c \\ c \\ b \\ a-b-c}
\end{align*}
\end{solution}
\end{parts}

\end{questions}
\end{document}
