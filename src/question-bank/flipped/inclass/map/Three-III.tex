% \documentclass[noanswers, nolegalese, 11pt]{examjh}
\documentclass[answers, nolegalese, 11pt]{examjh}
\usepackage{../../../../sty/conc}
\usepackage{../../../../sty/linalgjh}
\usepackage{graphicx}

\setlength{\parindent}{0em}\setlength{\parskip}{0.5ex}
\pagestyle{empty}
\begin{document}\thispagestyle{empty}
\makebox[\textwidth]{Worksheet for Three.III \hfill  From \textit{Linear Algebra}, by Hef{}feron}\vspace{-1ex}
\makebox[\textwidth]{\hbox{}\hrulefill\hbox{}}

Fix the vector spaces $V=\polyspace_2$ and $W=\R^2$.
Consider the map $\map{h}{V}{W}$, which is a homomorphism.
\begin{equation*}
  h(ax^2+bx+c)=\colvec{a+b \\ a+c}
\end{equation*}
Consider also the member of the domain $\vec{v}=3x^2+2x+1$.
Note that this is its image. 
\begin{equation*}
  \vec{w}=h(\vec{v})=\colvec{5 \\ 4}
\end{equation*}


\begin{questions}
\question
Fix these bases
\begin{equation*}
  B_1=\sequence{x^2+x, x+1, x}
  \qquad
  D_1=\sequence{\colvec{1 \\ 1}, \colvec{1 \\ -1}}
\end{equation*}
\begin{parts}
\part 
Represent $\vec{v}$ with respect to the first basis, $\rep{\vec{v}}{B_1}$.
\begin{solution}
Solving this
\begin{equation*}
  3x^2+2x+1=c_1\cdot(x^2+x)+c_2\cdot(x+1)+c_3\cdot(x)
\end{equation*}
gives this.
\begin{equation*}
  \rep{\vec{v}}{B_1}=\colvec{3 \\ 1 \\ -2}
\end{equation*}
\end{solution}

\part
Determine where each member of $B_1$ is mapped by~$h$.
\begin{solution}
The basis vectors are mapped as here.
\begin{align*}
  x^2+x &\mapsto \colvec{1+1 \\ 1+0}=\colvec{2 \\ 1}  \\
  x+1   &\mapsto \colvec{0+1 \\ 0+1}=\colvec{1 \\ 1}  \\
  x     &\mapsto \colvec{1 \\ 0}  
\end{align*}
\end{solution}

\part
Represent each of those two-tall results with respect to~$D_1$.
\begin{solution}
First is the calculation for $\rep{h(\vec{\beta}_1)}{D_1}$.
\begin{equation*}
  \colvec{2 \\ 1}=c_1\colvec{1 \\ 1}+c_2\colvec{1 \\ -1}
\end{equation*}
We get this.
\begin{equation*}
  \rep{\colvec{2 \\ 1}}{D_1}=\colvec{3/2 \\ 1/2}
\end{equation*}
The second one is the calculation for $\rep{h(\vec{\beta}_2)}{D_1}$.
\begin{equation*}
  \colvec{1 \\ 1}=c_1\colvec{1 \\ 1}+c_2\colvec{1 \\ -1}
\end{equation*}
We get this.
\begin{equation*}
  \rep{\colvec{1 \\ 1}}{D_1}=\colvec{1 \\ 0}
\end{equation*}
Third is the calculation for $\rep{h(\vec{\beta}_3)}{D_1}$.
\begin{equation*}
  \colvec{1 \\ 0}=c_1\colvec{1 \\ 1}+c_2\colvec{1 \\ -1}
\end{equation*}
We get this.
\begin{equation*}
  \rep{\colvec{1 \\ 0}}{D_1}=\colvec{1/2 \\ 1/2}
\end{equation*}
\end{solution}


\part
Give the matrix representing $h$ with respect to the bases, $\rep{h}{B_1,D_1}$
\begin{solution}
Take the three column vectors from the prior item and make them
the columns of a matrix, in order.
\begin{equation*}
  \rep{h}{B_1,D_1}=
  \begin{mat}
    3/2  &1  &1/2  \\
    1/2  &0  &1/2
  \end{mat}
\end{equation*}
\end{solution}

\part
Use the matrix vector product definition to compute 
$\rep{h}{B_1,D_1}\cdot \rep{\vec{v}}{B_1}$.
\begin{solution}
This matrix-vector multiplictation
\begin{equation*}
  \begin{mat}
    3/2  &1  &1/2  \\
    1/2  &0  &1/2
  \end{mat}
  \cdot
  \colvec{3 \\ 1 \\ -2}
\end{equation*}
gives this.
\begin{equation*}
  \colvec{(3/2)\cdot 3+1\cdot 1+(1/2)\cdot(-2) \\
          (1/2)\cdot 3+0\cdot 1+(1/2)\cdot (-2)}
  =
  \colvec{9/2 \\ 1/2}
\end{equation*}
\end{solution}

\part 
Verify that the result is $\rep{h(\vec{v})}{D_1}$. 
\begin{solution}
Solving
\begin{equation*}
  \colvec{5 \\ 4}=c_1\colvec{1 \\ 1}+c_2\colvec{1 \\ -1}
\end{equation*}
does indeed give $c_1=9/2$ and $c_2=1/2$.
\end{solution}
\end{parts}


\question
Now fix these bases
\begin{equation*}
  B_2=\sequence{x^2+x+1, x+1, 1}
  \qquad
  D_2=\sequence{\colvec{1 \\ 1}, \colvec{0 \\ 1}}
\end{equation*}
\begin{parts}
\part 
Find $\rep{\vec{v}}{B_2}$ and $\rep{\vec{w}}{D_2}$.
\begin{solution}
For $\rep{\vec{v}}{B_2}$ we solve
\begin{equation*}
  c_1(x^2+x+1)+c_2(x+1)+c_3(1)=3x^2+2x+1
\end{equation*}
to get this.
\begin{equation*}
  \rep{\vec{v}}{B_2}=\colvec{3 \\ -1 \\ -1}
\end{equation*}
And for the other we solve
\begin{equation*}
  c_1\colvec{1 \\ 1}+c_2\colvec{0 \\ 1}=\colvec{5 \\ 4}
\end{equation*}
leading to this.
\begin{equation*}
  \rep{\vec{w}}{D_2}=
  \colvec{5 \\ -1}
\end{equation*}
\end{solution}

\part
Determine where each member of $B_2$ is mapped by~$h$,
and represent each of those two-tall results with respect to~$D_2$.
\begin{solution}
This is the action of the map on the basis vectors from $B_2$.
\begin{align*}
  x^2+x+1 &\mapsto \colvec{2 \\ 2}  \\
      x+1 &\mapsto \colvec{1 \\ 1}  \\
  x^2+x+1 &\mapsto \colvec{0 \\ 1}  \\
\end{align*}
Representing the first with respect to $D_2$ means solving 
\begin{equation*}
  \colvec{2 \\ 2}=c_1\colvec{1 \\ 1}+c_2\colvec{0 \\ 1}
\end{equation*}
which gives this.
\begin{equation*}
  \rep{h(\vec{\beta}_1)}{D_2}=\colvec{2 \\ 0}
\end{equation*}
Representing $h(\vec{\beta}_2)$ with respect to $D_2$ means solving 
\begin{equation*}
  \colvec{1 \\ 1}=c_1\colvec{1 \\ 1}+c_2\colvec{0 \\ 1}
\end{equation*}
which gives this.
\begin{equation*}
  \rep{h(x+1)}{D_2}=\colvec{1 \\ 0}
\end{equation*}
And, representing $h(\vec{\beta}_3)$ with respect to $D_2$ means solving 
\begin{equation*}
  \colvec{0 \\ 1}=c_1\colvec{1 \\ 1}+c_2\colvec{0 \\ 1}
\end{equation*}
yielding this.
\begin{equation*}
  \rep{h(1)}{D_2}=\colvec{0 \\ 1}
\end{equation*}
\end{solution}

\part
Give $\rep{h}{B_2,D_2}$.
\begin{solution}
Take the three column vectors from the prior answer, in order,
and make them the columns of the matrix.
\begin{equation*}
  \rep{h}{B_2,D_2}=
  \begin{mat}
    2  &1 &0 \\
    0  &0 &1
  \end{mat}
\end{equation*}
\end{solution}

\part
Compute the matrix-vector product 
$\rep{h}{B_2,D_2}\cdot \rep{\vec{v}}{B_2}$.
Compare with $\rep{\vec{w}}{D_2}$. 
\begin{solution}
We have this.
\begin{equation*}
  \rep{h}{B_2,D_2}\cdot\rep{\vec{v}}{B_2}=
  \begin{mat}
    2  &1 &0 \\
    0  &0 &1
  \end{mat}
  \cdot\colvec{3 \\ -1 \\ -1}
  =
  \colvec{5 \\ -1}
\end{equation*}
\end{solution}
\end{parts}

\end{questions}
\end{document}
