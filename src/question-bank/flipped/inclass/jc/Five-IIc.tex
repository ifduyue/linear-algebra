% \documentclass[noanswers, nolegalese, 11pt]{examjh}
\documentclass[answers, nolegalese, 11pt]{examjh}
\usepackage{../../../../sty/conc}
\usepackage{../../../../sty/linalgjh}
\usepackage{graphicx}

\setlength{\parindent}{0em}\setlength{\parskip}{0.5ex}
\pagestyle{empty}
\begin{document}\thispagestyle{empty}
\makebox[\textwidth]{Worksheet for Five.II (part 3)\hfill  From \textit{Linear Algebra}, by Hef{}feron}\vspace{-1ex}
\makebox[\textwidth]{\hbox{}\hrulefill\hbox{}}


\begin{questions}

\question
    Consider this matrix.
    \begin{equation*}
      \begin{mat}[r]
        1  &1  &1  \\
        0  &0  &1  \\
        0  &0  &1
      \end{mat}
    \end{equation*}
\begin{parts}
\part Find its characteristic polynomial.
\begin{solution}
      The characteristic equation is
      \begin{equation*}
        0=
        \begin{vmatrix}
          1-x  &1   &1   \\
          0    &-x  &1   \\
          0    &0   &1-x
        \end{vmatrix}
        =(1-x)^2(-x)
      \end{equation*}
\end{solution}

\part Find its eigenvalues.
\begin{solution}
The eigenvalues are $\lambda_1=1$ (this is a repeated root
      of the equation) and $\lambda_2=0$.
\end{solution}

\part Find the eigenvectors associated with the first eigenvalue.
\begin{solution}
      Consider this system.
      \begin{equation*}
        \begin{linsys}{3}
          (1-x)\cdot b_1  &+  &b_2         &+  &b_3            &=  &0  \\
                          &   &-x\cdot b_2 &+  &b_3            &=  &0  \\
                          &   &            &   &(1-x)\cdot b_3 &= &0  
        \end{linsys}
      \end{equation*}
      For $\lambda_1=1$ we get this system.
      \begin{equation*}
        \begin{linsys}{3}
               0\cdot b_1  &+  &b_2         &+  &b_3            &=  &0  \\
                          &   &-1\cdot b_2 &+  &b_3            &=  &0  \\
                          &   &            &   &0\cdot b_3 &= &0  
        \end{linsys}
      \end{equation*}
      The sum of the first two equations gives $2b_3=0$, and so $b_3=0$
      With that, back substitution gives $b_2=0$.
      The solution set is this eigenspace.
      \begin{equation*}
        V_{1}=\set{\colvec{b_1 \\ 0 \\ 0}\suchthat b_1\in\C}
             =\set{\colvec{1 \\ 0 \\ 0}\cdot b_1\suchthat b_1\in\C}
      \end{equation*}
\end{solution}

\part Find the eigenvectors associated with the second eigenvalue.
\begin{solution}
      Again consider this system.
      \begin{equation*}
        \begin{linsys}{3}
          (1-x)\cdot b_1  &+  &b_2         &+  &b_3            &=  &0  \\
                          &   &-x\cdot b_2 &+  &b_3            &=  &0  \\
                          &   &            &   &(1-x)\cdot b_3 &= &0  
        \end{linsys}
      \end{equation*}
      When $\lambda_2=0$ we have this.
      \begin{equation*}
        \begin{linsys}{3}
                     b_1  &+  &b_2         &+  &b_3            &=  &0  \\
                          &   &            &   &b_3            &=  &0  \\
                          &   &            &   &b_3            &= &0  
        \end{linsys}
      \end{equation*}
The solution set is this eigenspace.
      \begin{equation*}
        V_{0}=\set{\colvec{-b_2 \\ b_2 \\ 0}\suchthat b_2\in\C}
             =\set{\colvec{-1 \\ 1 \\ 0}\cdot b_2\suchthat b_2\in\C}
      \end{equation*}
\end{solution}
\end{parts}


\question
     Consider the
     differentiation operator
     \( \map{d/dx}{\polyspace_3}{\polyspace_3} \).
From Calculus we know it is linear.
\begin{equation*}
  \frac{d}{dx}\,(f+g)=\frac{d}{dx}\,f+\frac{d}{dx}\, g
  \qquad
  \frac{d}{dx}\,r\cdot f
  =r\cdot \frac{d}{dx}\,f
\end{equation*}

\begin{parts}
\part Represent this map with respect to the natural basis
$B=\sequence{1,x,x^2,x^3}$.
\begin{solution}
       The map's action is $1\mapsto 0$, $x\mapsto 1$, $x^2\mapsto 2x$,
       and $x^3\mapsto 3x^2$. 
       So its representation is easy to compute.
       \begin{equation*}
         T=\rep{d/dx}{B,B}=
         \begin{mat}[r]
           0  &1  &0  &0  \\
           0  &0  &2  &0  \\
           0  &0  &0  &3  \\
           0  &0  &0  &0
         \end{mat}_{B,B}
       \end{equation*}
\end{solution}

\part Find its characteristic equation.
\begin{solution}
       The computation is easy because the matrix is already in echelon form.
       \begin{equation*}
         0=\deter{T-xI}=
         \begin{vmatrix}
           -x &1  &0  &0  \\
           0  &-x &2  &0  \\
           0  &0  &-x &3  \\
           0  &0  &0  &-x          
         \end{vmatrix}
         =x^4
       \end{equation*}
\end{solution}

\part List the eigenvalues.
\begin{solution}
The map has the single eigenvalue $\lambda=0$ (it is a repeated root
of the characteristic equation).
\end{solution}

\part Find the associated eigenspaces.
\begin{solution}
       Solve this matrix equation.
       \begin{equation*}
         \begin{mat}[r]
           0  &1  &0  &0  \\
           0  &0  &2  &0  \\
           0  &0  &0  &3  \\
           0  &0  &0  &0
         \end{mat}_{B,B}
         \colvec{b_1 \\ b_2 \\ b_3 \\ b_4}_B
         =0\cdot\colvec{b_1 \\ b_2 \\ b_3 \\ b_4}_B
       \end{equation*}
       We get this.
       \begin{equation*}
       \begin{linsys}{4}
         0b_1 &+ &1\cdot b_2 &  &     &   &     &=  &0  \\       
              &  &           &  &2\cdot b_3 &   &     &=  &0  \\       
              &  &           &  &     &   &3\cdot b_4 &=  &0  \\       
              &  &           &  &     &   &0    &=  &0        
       \end{linsys}
       \end{equation*}
       The conclusion is that $b_2=b_3=b_4=0$, and $b_1$ is free.
       This is the set of representatives.
       \begin{equation*}
         \set{\colvec{b_1 \\ 0 \\ 0 \\ 0}_B
               \suchthat b_1\in\C}
        \end{equation*}
        And here is the eigenspace associated with the eigenvalue $\lambda=1$.
       \begin{equation*}
         V_0=\set{b_1+0\cdot x+0\cdot x^2+0\cdot x^3
               \suchthat b_1\in\C}
         =\set{b_1
               \suchthat b_1\in\C}
       \end{equation*}
       Of course, we recognize from Calculus that the collection of constant
       polynomials is indeed nulled out by the derivative operation.
\end{solution}
\end{parts}


\end{questions}
\end{document}
