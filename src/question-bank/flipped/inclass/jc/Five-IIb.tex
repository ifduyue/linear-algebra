% \documentclass[noanswers, nolegalese, 11pt]{examjh}
\documentclass[answers, nolegalese, 11pt]{examjh}
\usepackage{../../../../sty/conc}
\usepackage{../../../../sty/linalgjh}
\usepackage{graphicx}

\setlength{\parindent}{0em}\setlength{\parskip}{0.5ex}
\pagestyle{empty}
\begin{document}\thispagestyle{empty}
\makebox[\textwidth]{Worksheet for Five.II (part 3)\hfill  From \textit{Linear Algebra}, by Hef{}feron}\vspace{-1ex}
\makebox[\textwidth]{\hbox{}\hrulefill\hbox{}}


\begin{questions}

\question
Recall from last time when we diagonalized this 
upper-triangular matrix
\begin{equation*}
  T=
  \begin{mat}
    2  &1 \\   
    0  &-3
  \end{mat}
\end{equation*}
that our use of the lemma led to this linear system.
\begin{equation*}
\begin{linsys}{2}
  (2-x)b_1  &+  &b_2       &=  &0  \\
            &   &(-3-x)b_2 &=  &0  \\
\end{linsys}
\end{equation*}
\begin{parts}
\item Take $x=-3$ and solve the linear system.
The set of solutions forms a subspace.
It is the \textit{eigenspace $V_{-3}$ associated with the eigenvalue 
$\lambda=-3$}. 
\begin{solution}
\begin{equation*}
\begin{linsys}{2}
  2b_1  &+  &b_2   &=  &-3b_1  \\
        &   &-3b_2 &=  &-3b_2
\end{linsys}
\end{equation*}
We conclude that $b_2$ is free.
We take it as a parameter, rewrite the first equation as $5b_1+b_2=0$, and 
the set of associated vector solutions is this.
\begin{equation*}
  V_{-3}=\set{\colvec{(-1/5)b_2 \\ b_2} \suchthat \text{$b_2\in\C$}}
\end{equation*}
\end{solution}

\item Take $x=2$ and solve the linear system to find $V_2$.
\begin{solution}
It gives a linear system.
\begin{equation*}
\begin{linsys}{2}
  2b_1  &+  &b_2   &=  &2b_1  \\
        &   &-3b_2 &=  &2b_2
\end{linsys}
\end{equation*}
We conclude that $b_2=0$ and $b_1$ is free.
That is, the set of associated vector solutions is this.
\begin{equation*}
  V_2=\set{\colvec{b_1 \\ 0} \suchthat \text{$b_1\in\C$}}
\end{equation*}
\end{solution}

\item $x=4$
\begin{solution}
Plugging $x=4$ into this system 
\begin{equation*}
\begin{linsys}{2}
  (2-x)b_1  &+  &b_2       &=  &0  \\
            &   &(-3-x)b_2 &=  &0  \\
\end{linsys}
\end{equation*}
gives this matrix equation.
\begin{equation*}
\begin{mat}
  -2  &1  \\
  0   &-7 \\
\end{mat}
\colvec{b_1 \\ b_2}
=\colvec{0 \\ 0}
\end{equation*}
The matrix is nonsingular, so there is a unique solution.
Because the system is homogeneous, that solution is  this.
\begin{equation*}
\colvec{b_1 \\ b_2}
=\colvec{0 \\ 0}
\end{equation*}
\end{solution}

\item $x=0$
\begin{solution}
Plugging $x=0$ into 
\begin{equation*}
\begin{linsys}{2}
  (2-x)b_1  &+  &b_2       &=  &0  \\
            &   &(-3-x)b_2 &=  &0  \\
\end{linsys}
\end{equation*}
gives this matrix equation.
\begin{equation*}
\begin{mat}
  2  &1  \\
  0   &7 \\
\end{mat}
\colvec{b_1 \\ b_2}
=\colvec{0 \\ 0}
\end{equation*}
The matrix is nonsingular, so there is a unique solution.
\begin{equation*}
\colvec{b_1 \\ b_2}
=\colvec{0 \\ 0}
\end{equation*}
\end{solution}
\end{parts}


\question
We will use the determinant function to solve $T\vec{v}=x\vec{v}$.
\begin{equation*}
\begin{mat}
          13 &-4 \\
          -4 &7
        \end{mat}
\end{equation*}
\begin{parts}
\item Set up the matrix equation and bring all variables to the left side.
\begin{solution}
This is the matrix equation.
\begin{align*}
  T\vec{v}
  &=x\cdot \vec{v}        \\
  \begin{mat}
  13  &-4  \\
  -4  &7
  \end{mat}
  \colvec{b_1 \\ b_2}
  &=
  x\cdot \colvec{b_1 \\ b_2}
\end{align*}
We get this linear system.
\begin{equation*}
\begin{linsys}{2}
  13b_1  &-  &4b_2  &=  &x\cdot b_1  \\
  -4b_1  &+  &7b_2  &=  &x\cdot b_2  \\
\end{linsys}
\end{equation*}
Bring all the variables to the left side.
\begin{equation*}
\begin{linsys}{2}
  (13-x)b_1  &-  &4b_2      &=  &0  \\
      -4b_1  &+  &(7-x)b_2  &=  &0  \\
\end{linsys}
\end{equation*}
This is the result.
\begin{align*}
  \begin{mat}
  13-x  &-4  \\
   -4   &7-x
  \end{mat}
  \colvec{b_1 \\ b_2}
  &=\colvec{0 \\ 0}      \\
  (T-x\cdot I)\vec{v}&=\vec{0}
\end{align*}
\end{solution}

\item Find the \textit{characteristic polynomial}, $\deter{T-xI}$.
\begin{solution}
Apply the $ad-bc$ determinant formular to get this.
\begin{equation*}
  \begin{mat}
  13-x  &-4  \\
   -4   &7-x
  \end{mat}
  =(13-x)(7-x)-(-4)(-4)
  =x^2-20x+75=(x-5)(x-15)
\end{equation*}
\end{solution}

\item The factors of the characteristic equation $\deter{T-xI}=0$
are $5$
and~$15$.
Find the eigenspace associated with $\lambda_1=5$. 
\begin{solution}
Into this equation, plug $x=5$.
\begin{equation*}
  \begin{mat}
  13-x  &-4  \\
   -4   &7-x
  \end{mat}
  \colvec{b_1 \\ b_2}
  =\colvec{0 \\ 0}      
\end{equation*}
We get a linear system.
\begin{equation*}
\begin{linsys}{2}
  8b_1  &-  &4b_2  &=  &0  \\
  -4b_1 &+  &2b_2  &=  &0
\end{linsys}
\grstep{(1/2)\rho_1+\rho_2}
\begin{linsys}{2}
  8b_1  &-  &4b_2  &=  &0  \\
            &0     &=  &0
\end{linsys}
\end{equation*}
Leading is $b_1$ and free is $b_2$, so $b_1=(1/2)\cdot b_2$.
\begin{equation*}
           V_5=\set{\colvec{1/2 \\ 1}\cdot b_2\suchthat b_2\in\C}
\end{equation*}
\end{solution}

\item
Find the eigenspace associated with $\lambda_2=15$.
\begin{solution}
Into this equation, plug $x=15$.
\begin{equation*}
  \begin{mat}
  13-x  &-4  \\
   -4   &7-x
  \end{mat}
  \colvec{b_1 \\ b_2}
  =\colvec{0 \\ 0}      
\end{equation*}
We get a linear system.
\begin{equation*}
\begin{linsys}{2}
  -2b_1 &-  &4b_2  &=  &0  \\
  -4b_1 &+  &-8b_2 &=  &0
\end{linsys}
\grstep{-2\rho_1+\rho_2}
\begin{linsys}{2}
  -2b_1  &-  &4b_2  &=  &0  \\
             &0     &=  &0
\end{linsys}
\end{equation*}
Leading is $b_1$ and free is $b_2$, so $b_1=-2\cdot b_2$.
\begin{equation*}
           V_{15}=\set{\colvec{-2 \\ 1}\cdot b_2\suchthat b_2\in\C}
\end{equation*}
\end{solution}
\end{parts}

\end{questions}
\end{document}
