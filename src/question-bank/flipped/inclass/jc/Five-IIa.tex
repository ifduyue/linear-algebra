\documentclass[noanswers, nolegalese, 11pt]{examjh}
% \documentclass[answers, nolegalese, 11pt]{examjh}
\usepackage{../../../../sty/conc}
\usepackage{../../../../sty/linalgjh}
\usepackage{graphicx}

\setlength{\parindent}{0em}\setlength{\parskip}{0.5ex}
\pagestyle{empty}
\begin{document}\thispagestyle{empty}
\makebox[\textwidth]{Worksheet for Five.II (part 2)\hfill  From \textit{Linear Algebra}, by Hef{}feron}\vspace{-1ex}
\makebox[\textwidth]{\hbox{}\hrulefill\hbox{}}


\begin{questions}

\question
This is an upper-triangular matrix.
\begin{equation*}
  T=
  \begin{mat}
    2  &1 \\   
    0  &-3
  \end{mat}
\end{equation*}
We will diagonalize it.
\begin{parts}
\part Consider that the matrix represents a transformation
$\map{t}{\C^2}{\C^2}$ with respect to the standard basis,
$\stdbasis_2, \stdbasis_2$.
\begin{equation*}
  \stdbasis=
  \sequence{
   \colvec{1  \\  0},
   \colvec{0  \\  1}
   }  
\end{equation*}
Produce the appropriate commutative diagram, where the
other basis is~$B$.
\begin{solution}
           \begin{equation*}
             \begin{CD}
               \C^2_{\wrt{\stdbasis_2}}                  @>t>T>  \C^2_{\wrt{\stdbasis_2}}       \\
               @V\scriptstyle\identity VPV      @V\scriptstyle\identity VPV \\
               \C^2_{\wrt{B}}                  @>t>\hat{T}>  \C^2_{\wrt{B}}
             \end{CD}
           \end{equation*}
\end{solution}

\part
Set up the $\lambda$ rescaling equations 
involving the vectors for the basis~$B$.
\begin{solution}
Let $B=\sequence{\vec{\beta}_1,\vec{\beta}_2}$.
By Lemma 2.4 we have that on the basis~$B$ 
the transformation rescales the basis vectors, that is,
for some constants $\lambda_1,\lambda_2\in \C$ we have this.
\begin{equation*}
  \begin{mat}
    2  &1 \\   
    0  &-3
  \end{mat}
  \vec{\beta}_1
  =\lambda_1\vec{\beta}_1
  \qquad
  \begin{mat}
    2  &1 \\   
    0  &-3
  \end{mat}
  \vec{\beta}_2
  =\lambda_2\vec{\beta}_2
\end{equation*}
\end{solution}

\part 
Produce the resulting linear system.
\begin{solution}
We are looking for scalars $x\in\C$ so that this holds.
\begin{equation*}
  \begin{mat}
    2  &1 \\   
    0  &-3
  \end{mat}
  \colvec{b_1 \\ b_2}
  =x\cdot\colvec{b_1 \\ b_2}
\end{equation*}
for some pairs $b_1,b_2\in\C$.
So: we want scalars~$x\in\C$ so that this system has  solution
\begin{equation*}
\begin{linsys}{2}
  2b_1  &+  &b_2  &=  &x\cdot b_1  \\
        &   &-3b_2&=  &x\cdot b_2  \\
\end{linsys}
\quad\Rightarrow\quad
\begin{linsys}{2}
  (2-x)b_1  &+  &b_2       &=  &0  \\
            &   &(-3-x)b_2 &=  &0  \\
\end{linsys}
\end{equation*}
for some $b_1,b_2\in\C$ that are not both zero (since the vector with 
both entries zero can not be a member of any basis).
\end{solution}

\part
Solve that linear system.
\begin{solution}
Start with 
\begin{equation*}
\begin{linsys}{2}
  (2-x)b_1  &+  &b_2       &=  &0  \\
            &   &(-3-x)b_2 &=  &0  \\
\end{linsys}
\end{equation*}
and focus first on the bottom equation.
Either $x=-3$ or $b_2=0$.

Following up on the second case, 
if $b_2=0$ then the first equation becomes $(2-x)b_1=0$.
We can't have $b_1=0$ also, since we can't have that both $b$'s are zeros.
So we get the other solution $x=2$.

We write these as $\lambda_1=2$ and $\lambda_2=-3$.
\end{solution}

\part 
Produce the vector in the resulting basis~$B$ associated with $\lambda_1=2$.
\begin{solution}
To get $\vec{\beta}_1$, the vector associated with $\lambda_1=2$, we have this.
\begin{equation*}
  \begin{mat}
    2  &1 \\   
    0  &-3
  \end{mat}
  \vec{\beta}_1
  =\lambda_1\vec{\beta}_1
  \quad\Rightarrow
  \begin{mat}
    2  &1 \\   
    0  &-3
  \end{mat}
  \colvec{b_1 \\ b_2}
  =2\cdot \vec{b_1 \\ b_2}
\end{equation*}
It gives a linear system.
\begin{equation*}
\begin{linsys}{2}
  2b_1  &+  &b_2   &=  &2b_1  \\
        &   &-3b_2 &=  &2b_2
\end{linsys}
\end{equation*}
We conclude that $b_2=0$ and $b_1$ is free.
That is, the set of associated vector solutions is this.
\begin{equation*}
  V_2=\set{\colvec{b_1 \\ 0} \suchthat \text{$b_1\in\C$}}
\end{equation*}
For the frist basis vector $\vec{\beta}_1$ we can pick any nonzero member
of that set.  
This will do.
\begin{equation*}
  \vec{\beta}_1=\colvec{1 \\ 0}
\end{equation*}
\end{solution}

\part 
Also produce the vector in~$B$ associated with $\lambda_2=-3$.
\begin{solution}
To get $\vec{\beta}_2$, the vector associated with $\lambda_2=-3$, we have this.
\begin{equation*}
  \begin{mat}
    2  &1 \\   
    0  &-3
  \end{mat}
  \vec{\beta}_2
  =\lambda_2\vec{\beta}_2
  \quad\Rightarrow
  \begin{mat}
    2  &1 \\   
    0  &-3
  \end{mat}
  \colvec{b_1 \\ b_2}
  =-3\cdot\vec{b_1 \\ b_2}
\end{equation*}
It gives a linear system.
\begin{equation*}
\begin{linsys}{2}
  2b_1  &+  &b_2   &=  &-3b_1  \\
        &   &-3b_2 &=  &-3b_2
\end{linsys}
\end{equation*}
We conclude that $b_2$ is free.
We take it as a parameter, rewrite the first equation as $5b_1+b_2=0$, and 
the set of associated vector solutions is this.
\begin{equation*}
  V_{-3}=\set{\colvec{(-1/5)b_2 \\ b_2} \suchthat \text{$b_2\in\C$}}
\end{equation*}
For $\vec{\beta}_2$ we can pick any nonzero member.  
This will do.
\begin{equation*}
  \vec{\beta}_2=\colvec{-1/5 \\ 1}
\end{equation*}
\end{solution}

\part
Exhibit the basis $B$.
\begin{solution}
\begin{equation*}
  B=
  \sequence{
    \colvec{1 \\ 0},  
    \colvec{-1/5 \\ 1}
}
\end{equation*}
\end{solution}


\part
Check the change of basis.
\begin{solution}
Here is the diagram again.
           \begin{equation*}
             \begin{CD}
               \C^2_{\wrt{\stdbasis_2}}                  @>t>T>  \C^2_{\wrt{\stdbasis_2}}       \\
               @V\scriptstyle\identity VPV      @V\scriptstyle\identity VPV \\
               \C^2_{\wrt{B}}                  @>t>\hat{T}>  \C^2_{\wrt{B}}
             \end{CD}
           \end{equation*}
We know the matrices on the top and bottom.
\begin{equation*}
  T = 
  \begin{mat}
    2  &1 \\   
    0  &-3
  \end{mat}
  \qquad
  \hat{T} = 
  \begin{mat}
    2  &  \\   
    0  &-3
  \end{mat}
\end{equation*}
The matrix on the right is $P=\rep{B,\stdbasis_2}{\identity}$.
To find it we compute the action of the function on the starting
basis
\begin{align*}
  \vec{\beta}_1=\colvec{1 \\ 0}
  &\mapsunder{\identity}
  \colvec{1 \\ 0}                   \\
  \vec{\beta}_2=\colvec{-1/5 \\ 1}
  &\mapsunder{\identity}
  \colvec{-1/5 \\ 1}                   
\end{align*}
and represent each result with respect to the ending basis,
\begin{equation*}
  \rep{\colvec{1 \\ 0}}{\stdbasis_2}=\colvec{1 \\ 0}
  \qquad
  \rep{\colvec{-1/5 \\ 1}}{\stdbasis_2}=\colvec{-1/5 \\ 1}
\end{equation*}
to give this.
\begin{equation*}
  P=
  \begin{mat}
    1  &-1/5  \\
    0  &1
  \end{mat}
\end{equation*}
For the matrix on the left side $P^{-1}$, we use the formula for the 
inverse of a $\nbyn{2}$ matrix.
\begin{equation*}
\begin{mat}
  a  &b  \\
  c  &d
\end{mat}^{-1}
=
\frac{1}{ad-bc}\cdot
\begin{mat}
  d  &-b \\
  -c &a
\end{mat}
\qquad
P^{-1}
=
\frac{1}{1}\cdot
\begin{mat}
  1  &1/5 \\
  0  &1
\end{mat}
\end{equation*}
Finally, we get a diagonal matrix.
\begin{equation*}
  \hat{T}=
  P^{-1}TP=
\begin{mat}
  1  &1/5 \\
  0  &1
\end{mat}
  \begin{mat}
    2  &1 \\   
    0  &-3
  \end{mat}
  \begin{mat}
    1  &-1/5  \\
    0  &1
  \end{mat}
=
\begin{mat}
  2  &0  \\
  0  &-3
\end{mat}
\end{equation*}
% sage: P = matrix(QQ, [[1,-1/5], [0,1]])
% sage: P^(-1)
% [  1 1/5]
% [  0   1]
% sage: T = matrix(QQ, [[2,1], [0,-3]])
% sage: T
% [ 2  1]
% [ 0 -3]
% sage: P^(-1) * T * P
% [ 2  0]
% [ 0 -3]
\end{solution}
\end{parts}


\end{questions}
\end{document}
