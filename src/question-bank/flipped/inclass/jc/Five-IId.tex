% \documentclass[noanswers, nolegalese, 11pt]{examjh}
\documentclass[answers, nolegalese, 11pt]{examjh}
\usepackage{../../../../sty/conc}
\usepackage{../../../../sty/linalgjh}
\usepackage{graphicx}

\setlength{\parindent}{0em}\setlength{\parskip}{0.5ex}
\pagestyle{empty}
\begin{document}\thispagestyle{empty}
\makebox[\textwidth]{Worksheet for Five.II (part 4)\hfill  From \textit{Linear Algebra}, by Hef{}feron}\vspace{-1ex}
\makebox[\textwidth]{\hbox{}\hrulefill\hbox{}}

Consider this linear transformation $\map{t}{\R^2}{\R^2}$.
\begin{equation*}
  t(\colvec{x \\ y})=\colvec{3x+2y \\ 3x+8y}
\end{equation*}
We will illustrate that the eigenvalues are a feature of the
map.

\begin{questions}
\question
Fix the basis 
\begin{equation*}
  B=\sequence{
         \colvec{1 \\ 0},
         \colvec{1 \\ 1}
      }
\end{equation*}
\begin{parts}
\part Represent the map $t$ with respect to $B,B$.
\begin{solution}
This is the action of the map on the basis.
\begin{align*}
  \colvec{1  \\ 0}  &\mapsunder{t} \colvec{3 \\ 3}  \\
  \colvec{1  \\ 1}  &\mapsunder{t} \colvec{5 \\ 11}  
\end{align*}
The representation of the first outcome with respect to~$B$ is this.
\begin{equation*}
\colvec{3 \\ 3}=0\cdot\colvec{1 \\ 0}
                +3\cdot\colvec{1 \\ 1}
\qquad
\rep{\colvec{3 \\ 3}}{B}=\colvec{0 \\ 3}
\end{equation*}
The second outcome gives this.
\begin{equation*}
\colvec{5 \\ 11}=-6\cdot\colvec{1 \\ 0}
                +11\cdot\colvec{1 \\ 1}
\qquad
\rep{\colvec{5 \\ 11}}{B}=\colvec{-6 \\ 11}
\end{equation*}
The matrix is this.
\begin{equation*}
\rep{t}{B,B}=
\begin{mat}
0   &-6  \\
3   &11 
\end{mat}
\end{equation*}
\end{solution}

\part Use the matrix $\rep{t}{B,B}$ to compute the 
characteristic equation.
\begin{solution}
We find $\deter{T-x\cdot I}$ as here.
\begin{equation*}
0=
\begin{vmat}
  0-x  &-6  \\
  3    &11-x
\end{vmat}
=(-x)(11-x)+18
=x^2-11x+18
\end{equation*}
\end{solution}

\part Find the eigenvalues. 
\begin{solution}
The roots are
\begin{equation*}
0=x^2-11x+18
=(x-2)(x-9)
\end{equation*}
$\lambda_1=2$ and $\lambda_2=9$.
\end{solution}
\end{parts}


\question
Fix this basis. 
\begin{equation*}
  D=\sequence{
         \colvec{2 \\ 1},
         \colvec{0 \\ 3}
      }
\end{equation*}
\begin{parts}
\part Represent the map $t$ with respect to $D,D$.
\begin{solution}
This is the action of the map on the basis.
\begin{align*}
  \colvec{2  \\ 1}  &\mapsunder{t} \colvec{8 \\ 14}  \\
  \colvec{0  \\ 3}  &\mapsunder{t} \colvec{6 \\ 24}  
\end{align*}
The representation of the first with respect to~$D$ is this.
\begin{equation*}
\colvec{8 \\ 14}=4\cdot\colvec{2 \\ 1}
                +(10/3)\cdot\colvec{0 \\ 3}
\qquad
\rep{\colvec{8 \\ 14}}{D}=\colvec{4 \\ 10/3}
\end{equation*}
The second outcome gives this.
\begin{equation*}
\colvec{6 \\ 24}=3\cdot\colvec{2 \\ 1}
                +7\cdot\colvec{0 \\ 3}
\qquad
\rep{\colvec{6 \\ 24}}{D}=\colvec{3 \\ 7}
\end{equation*}
The matrix is this.
\begin{equation*}
\rep{t}{D,D}=
\begin{mat}
4      &3  \\
10/3   &7 
\end{mat}
\end{equation*}
\end{solution}

\part Use the matrix $\rep{t}{D,D}$ to compute the 
characteristic equation.
\begin{solution}
Here it is.
\begin{equation*}
0=\deter{T-x\cdot I}
=\begin{vmat}
  4-x   &3  \\
  10/3  &7-x
\end{vmat}
=(4-x)(7-x)-10
=x^2-11x+18
\end{equation*}
\end{solution}

\part Find the eigenvalues. 
\begin{solution}
The characteristic equation $0=x^2-11x+18$ is the same so the roots
are the same:~$\lambda_1=2$ and $\lambda_2=9$.
\end{solution}

\end{parts}

\end{questions}
\end{document}
