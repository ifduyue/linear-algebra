% \documentclass[noanswers, nolegalese, 11pt]{examjh}
\documentclass[answers, nolegalese, 11pt]{examjh}
\usepackage{../../../../sty/conc}
\usepackage{../../../../sty/linalgjh}
\usepackage{graphicx}

\setlength{\parindent}{0em}\setlength{\parskip}{0.5ex}
\pagestyle{empty}
\begin{document}\thispagestyle{empty}
\makebox[\textwidth]{Worksheet for Four.III\hfill  From \textit{Linear Algebra}, by Hef{}feron}\vspace{-1ex}
\makebox[\textwidth]{\hbox{}\hrulefill\hbox{}}

We consider the special case of linear transformations,
maps where the domain and codomain are equal, 
$\map{t}{V}{V}$.
We further specialize to representations with respect to 
$B,B$ and $D,D$.
\begin{equation*}
  \begin{CD}
    V_{\wrt{B}}                   @>t>T>        V_{\wrt{B}}       \\
    @V{\scriptstyle\identity} VV              @V{\scriptstyle\identity} VV \\
    V_{\wrt{D}}                   @>t>\hat{T}>        V_{\wrt{D}}
  \end{CD}
\end{equation*}
In matrix terms, we write this.
\begin{align*} 
\rep{t}{D,D}
  &=\rep{\identity}{B,D}\cdot\rep{t}{B,B}\cdot\bigl(\rep{\identity}{B,D}\bigr)^{-1} \\
\hat{T}
  &=P\cdot T\cdot P^{-1}
\end{align*}

\begin{questions}

\question
    Consider the transformation $\map{t}{\polyspace_2}{\polyspace_2}$
    described by
    $x^2\mapsto x+1$, $x\mapsto x^2-1$, and $1\mapsto 3$.
\begin{parts}
\part What is $t(3x^2-2x-1)$?
\begin{solution}
This basis is particularly easy to work with.
Because $t$ is a linear map, we have this.
\begin{equation*}
 t(3x^2-2x-1)=3\cdot t(x^2)-2\cdot t(x)-t(1)
  =3\cdot(x-1)-2\cdot(x^2-1)-(3)
  =-2x^2+3x-4
\end{equation*}
\end{solution}

      \part Find $T=\rep{t}{B,B}$ where $B=\sequence{x^2,x,1}$.
\begin{solution}
      Because we describe $t$ with the members of $B$,
          finding the matrix representation is easy:
          \begin{equation*}
            \rep{t(x^2)}{B}=\colvec[r]{0 \\ 1 \\ 1}_B
            \quad
            \rep{t(x)}{B}=\colvec[r]{1 \\ 0 \\ -1}_B
            \quad
            \rep{t(1)}{B}=\colvec[r]{0 \\ 0 \\ 3}_B
          \end{equation*}
          gives this.
          \begin{equation*}
            \rep{t}{B,B}
            \begin{mat}[r]
              0  &1  &0  \\
              1  &0  &0  \\
              1  &-1 &3  
            \end{mat}
          \end{equation*}
\end{solution}

      \part Find $\hat{T}=\rep{t}{D,D}$ where $D=\sequence{1,1+x,1+x+x^2}$.
\begin{solution}
        We will find $t(1)$, $t(1+x)$, and $t(1+x+x^2)$,
          to find how each is represented with respect to $D$.
          We are given that $t(1)=3$, and the other two are easy to see:
          $t(1+x)=t(1)+t(x)=x^2+2$ and $t(1+x+x^2)=x^2+x+3$.
          By eye, we get the representation of each vector
          \begin{equation*}
            \rep{t(1)}{D}=\colvec[r]{3 \\ 0 \\ 0}_D
            \quad
            \rep{t(1+x)}{D}=\colvec[r]{2  \\ -1 \\  1}_D
            \quad
            \rep{t(1+x+x^2)}{D}=\colvec[r]{2 \\ 0 \\ 1}_D
          \end{equation*}
          and thus the representation of the map.
          \begin{equation*}
            \rep{t}{D,D}
            =
            \begin{mat}[r]
              3  &2  &2  \\
              0  &-1 &0  \\
              0  &1  &1
            \end{mat}
          \end{equation*}
\end{solution}

      \part Find the matrix $P$ such that $\hat{T}=PTP^{-1}$. 
      \begin{solution}
         From the diagram
           \begin{equation*}
             \begin{CD}
               V_{\wrt{B}}                  @>t>T>  V_{\wrt{B}}       \\
               @V\scriptstyle\identity VPV      @V\scriptstyle\identity VPV \\
               V_{\wrt{D}}                  @>t>\hat{T}>  V_{\wrt{D}}
             \end{CD}
           \end{equation*}
           We can easily calculate $P=\rep{\identity}{B,D}$, and 
           $P^{-1}=\rep{\identity}{D,B}$ for that matter.
           \begin{equation*}
             P=
             \begin{mat}[r]
               0  &-1  &1  \\ 
               -1  &1  &0  \\
               1  &0  &0 
             \end{mat}
             \qquad
             P^{-1}=
             \begin{mat}[r]
               0  &0  &1  \\ 
               0  &1  &1  \\
               1  &1  &1 
             \end{mat}
           \end{equation*}      
      \end{solution}
\end{parts}


\end{questions}
\end{document}
