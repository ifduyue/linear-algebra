% \documentclass[11pt]{examjh}
\documentclass[11pt,answers]{examjh}
\usepackage{../../linalgjh}
\examhead{MA 213 Hef{}feron, 2017-Fall}{Exam One}
\begin{document}
% sage: load("../../lab/gauss_method.sage")

\begin{questions}

\question
  Do Gauss-Jordan reduction on this matrix.
  \begin{equation*}
    \begin{mat}
      -1 &0  &1 &0 \\  
      1  &2  &0 &1 \\
      0  &1  &-1 &0 
    \end{mat}
  \end{equation*}
  \begin{solution}[2in]
    \begin{multline*}
      \grstep{\rho_1+\rho_2}
      \begin{mat}
        -1 &0  &1 &0 \\  
        0  &2  &1 &1 \\
        0  &1  &-1 &0 
      \end{mat}
      \grstep{(1/2)\rho_2+\rho_3}
      \begin{mat}
        -1 &0  &1    &0 \\  
        0  &2  &1    &1 \\
        0  &0  &-3/2 &-1/2 
      \end{mat}                                      \\
      \grstep[(1/2)\rho_2 \\ -(2/3)\rho_3]{-\rho_1}
      \begin{mat}
        1  &0  &-1    &0 \\  
        0  &1  &1/2   &1/2 \\
        0  &0  &1     &1/3 
      \end{mat}
      \grstep[-(1/2)\rho_3+\rho_2]{\rho_3+\rho_1}
      \begin{mat}
        1  &0  &0     &1/3 \\  
        0  &1  &0     &1/3 \\
        0  &0  &1     &1/3 
      \end{mat}
    \end{multline*}
% sage: M = matrix(QQ, [[-1,0,1,0],[1,2,0,1], [0,1,-1,0]])
% sage: gauss_jordan(M)
% [-1  0  1  0]
% [ 1  2  0  1]
% [ 0  1 -1  0]
%  take 1 times row 1 plus row 2
% [-1  0  1  0]
% [ 0  2  1  1]
% [ 0  1 -1  0]
%  take -1/2 times row 2 plus row 3
% [  -1    0    1    0]
% [   0    2    1    1]
% [   0    0 -3/2 -1/2]
%  take -1 times row 1
%  take 1/2 times row 2
%  take -2/3 times row 3
% [  1   0  -1   0]
% [  0   1 1/2 1/2]
% [  0   0   1 1/3]
%  take 1 times row 3 plus row 1
%  take -1/2 times row 3 plus row 2
% [  1   0   0 1/3]
% [  0   1   0 1/3]
% [  0   0   1 1/3]
  \end{solution}



\question
Verify that this is a subspace of $\polyspace_3$.
\begin{equation*}
  S=\set{a+bx+cx^2\suchthat a-c=0}
\end{equation*}
\begin{solution}[2in]
We can verify that $S$ is closed under linear combinations of two elements.
Two elements of $S$ are $\vec{v}_1=a_1+b_1+c_1x^2$
and~$\vec{v}_2=a_2+b_2x+c_2x^2$
subject to the restrictions that $a_1-c_1=0$ and $a_2-c_2=0$.
Where $r_1,r_2\in\R$, a linear combination of those
\begin{align*}
  r_1\cdot\vec{v}_1+r_2\cdot\vec{v_2}
  &=r_1\cdot (a_1+b_1x+c_1x^2)+r_2\cdot (a_2+b_2x+c_2x^2) \\
  &=(r_1a_1+r_2a_2)+(r_1b_1+r_2b_2)\cdot x+(r_1c_1+r_2c_2)\cdot x^2
\end{align*}
is a quadratic polynomial satisfying the restriction that its
constant coefficient minus its quadratic coefficient zero:
$(r_1a_1+r_2a_2)-(r_1c_1+r_2c_2)=r_1\cdot(a_1-c_1)-r_2\cdot(a_2-c_2)=0-0=0$.
\end{solution}



\question
Find and parametrize each solution set.
\begin{parts}
\part $
       \begin{linsys}{3}
         2x  &  &  &+  &z  &=  &3  \\
          x  &- &y &-  &z  &=  &1  \\
         3x  &- &y &   &   &=  &4  
       \end{linsys}
      $
  \begin{solution}[2in]
    \begin{equation*}
      \grstep[-(3/2)\rho_1+\rho_3]{-(1/2)\rho_1+\rho_2}
      \begin{amat}{3}
        2 &     &1    &3 \\
        0 &-1   &-3/2 &-1/2 \\
        0 &-1   &-3/2 &-1/2 
      \end{amat}
      \grstep{-\rho_2+\rho_3}
      \begin{amat}{3}
        2 &     &1    &3 \\
        0 &-1   &-3/2 &-1/2 \\
        0 &0    &0    &0 
      \end{amat}
    \end{equation*}
    The parametrization is this.
    \begin{equation*}
      S=\set{\colvec{x \\ y \\ z}=
             \colvec{3/2 \\ 1/2 \\ 0}+
             \colvec{-1/2 \\ -3/2 \\ 1}z\suchthat z\in\Re}
    \end{equation*}
  \end{solution}
% sage: M = matrix(QQ, [[2,0,1], [1,-1,-1], [3,-1,0]])
% sage: v = vector(QQ, [3,1,4])
% sage: M_prime = M.augment(v, subdivide=True)
% sage: gauss_method(M_prime)
% [ 2  0  1| 3]
% [ 1 -1 -1| 1]
% [ 3 -1  0| 4]
%  take -1/2 times row 1 plus row 2
%  take -3/2 times row 1 plus row 3
% [   2    0    1|   3]
% [   0   -1 -3/2|-1/2]
% [   0   -1 -3/2|-1/2]
%  take -1 times row 2 plus row 3
% [   2    0    1|   3]
% [   0   -1 -3/2|-1/2]
% [   0    0    0|   0]

\part $
  \begin{linsys}{2}
    2x &+ &y &= &0 \\
     x &- &y &= &-3 \\
    4x &+ &5y &= &-1 
  \end{linsys}
  $
  \begin{solution}[2in]
    \begin{equation*}
      \grstep[-2\rho_1+\rho_3]{-(1/2)\rho_1+\rho_2}
      \begin{amat}{2}
        2 &1     &0 \\
        0 &-3/2  &-3  \\
        0 &3     &-1 
      \end{amat}
      \grstep{2\rho_2+\rho_3}
      \begin{amat}{2}
        2 &1     &0 \\
        0 &-3/2  &-3  \\
        0 &0     &-7 
      \end{amat}
    \end{equation*}
    The solution set is empty $S=\set{}$.
  \end{solution}
% sage: M = matrix(QQ, [[2,1], [1,-1], [4,5]])
% sage: v = vector(QQ, [0,-3,-1])
% sage: M_prime = M.augment(v, subdivide=True)
% sage: gauss_method(M_prime)
% [ 2  1| 0]
% [ 1 -1|-3]
% [ 4  5|-1]
%  take -1/2 times row 1 plus row 2
%  take -2 times row 1 plus row 3
% [   2    1|   0]
% [   0 -3/2|  -3]
% [   0    3|  -1]
%  take 2 times row 2 plus row 3
% [   2    1|   0]
% [   0 -3/2|  -3]
% [   0    0|  -7]
\end{parts}

\question
  For this system
  \begin{equation*}
    \begin{linsys}{3}
      -x &+ &y  &  &   &= &1  \\
       x &+ &y  &+ &z  &= &2  \\
      3x &+ &y  &+ &2z &= &3  \\
    \end{linsys}
  \end{equation*}
  \begin{parts}
  \part
  Find and parametrize the general solution.
  \begin{solution}[2in]
  \begin{equation*}
  \grstep[3\rho_1+\rho_3]{\rho_1+\rho_2}
  \grstep{-2\rho_2+\rho_3} 
    \begin{linsys}{3}
      -x &+ &y  &  &   &= &1  \\
         &  &2y &+ &z  &= &3  \\
         &  &   &  &0  &= &0  
    \end{linsys}
  \end{equation*}
  The general solution is
  \begin{equation*}
  \set{\colvec{x \\ y \\ z}
    =\colvec{1/2 \\ 3/2 \\ 0}+\colvec{-1/2 \\ -1/2 \\ 1}\cdot z
    \suchthat z\in\R}
  \end{equation*}
  \end{solution}
% sage: M = matrix(QQ, [[-1,1,0],[1,1,1], [3,1,2]])
% sage: v = vector(QQ, [1,2,3])
% sage: M_prime = M.augment(v, subdivide=True)
% sage: gauss_method(M_prime)
% [-1  1  0| 1]
% [ 1  1  1| 2]
% [ 3  1  2| 3]
%  take 1 times row 1 plus row 2
%  take 3 times row 1 plus row 3
% [-1  1  0| 1]
% [ 0  2  1| 3]
% [ 0  4  2| 6]
%  take -2 times row 2 plus row 3
% [-1  1  0| 1]
% [ 0  2  1| 3]
% [ 0  0  0| 0]
% sage: eqs=[-x+y==1, x+y+z==2, 3*x+y+2*z==3]
% sage: solve(eqs,x,y)
% [[x == -1/2*z + 1/2, y == -1/2*z + 3/2]]
  
  \part
    Identify the particular solution, and the homogeneous solution.
  \begin{solution}[1in]
  They are:
  \begin{equation*}
  \text{particular:\ }\colvec{1/2 \\ 3/2 \\ 0}
  \qquad
  \text{homogeneous:\ }
  \set{\colvec{x \\ y \\ z}
    =\colvec{-1/2 \\ -1/2 \\ 1}\cdot z
    \suchthat z\in\R}
  \end{equation*}
  \end{solution}

  \part
    Find another particular solution.
    \begin{solution}[1.25in]
      We can for instance take $z=1$.
      \begin{equation*}
      \colvec{1/2 \\ 3/2 \\ 0}+\colvec{-1/2 \\ -1/2 \\ 1}\cdot 1
      =\colvec{0 \\ 1 \\ 1}
      \end{equation*}
    \end{solution}
  \end{parts}


\newpage
\question
Find a basis for each vector space.
\begin{parts}
\part $\set{\colvec{x \\ y \\ z}\suchthat x+z=0}\subset\Re^3$
\begin{solution}[1.25in]
Treat the restriction as a one-equation linear system.
Leading is $x$ and free are $y$ and~$z$.
Parametrizing
\begin{equation*}
  \colvec{x \\ y \\ z}=\colvec{0 \\ 1 \\ 0}\cdot y+\colvec{-1 \\ 0 \\ 1}\cdot z
\end{equation*}
leads to this basis.
\begin{equation*}
B=\sequence{\colvec{0 \\ 1 \\ 0},
             \colvec{-1 \\ 0 \\ 1}}
\end{equation*}
\end{solution}


\part $\set{p(x)=a+bx+cx^2\suchthat p(1)=0}$
\begin{solution}[1.5in]
The restriction is $a+b+c=0$.
Parametrizing gives
$\set{(-b-c)+bx+cx^2\suchthat b,c\in\R}
=\set{(-1+x)\cdot b+(-1+x^2)\cdot c\suchthat b,c\in\Re}$.
A basis is $\sequence{-1+x,-1+x^2}$. 
\end{solution}
\end{parts}




  

\question
  Decide if these two matrices are row~equivalent.
  \begin{equation*}
    \begin{mat}
      2 &4  &2  \\
      1 &1  &1  \\
      0 &-1 &0
    \end{mat}\quad
    \begin{mat}
      0  &2 &0  \\
      -1 &1 &-1 \\
      2  &3  &2
    \end{mat}
  \end{equation*}
  \begin{solution}[2in]
    Perform Gauss-Jordan reduction on each and see if the outcomes
    are the same.
    The first is
    \begin{equation*}
    \begin{mat}
    2 &4  &2  \\
    1 &1  &1  \\
    0 &-1 &0
    \end{mat}
    \grstep{-(1/2)\rho_1+\rho_2}
    \grstep{-\rho_2+\rho_3}
    \grstep[-\rho_2]{(1/2)\rho_1}
    \grstep{-2\rho_2+\rho_1}
    \begin{mat}
    1 &0  &1  \\
    0 &1  &0  \\
    0 &0  &0
    \end{mat}
    \end{equation*}
    and the second is this.
    \begin{equation*}
    \begin{mat}
    0  &2  &0  \\
    -1 &1  &-1  \\
    2  &3  &2
    \end{mat}
    \grstep{\rho_1\leftrightarrow\rho_2}
    \grstep{2\rho_1+\rho_3}
    \grstep{(5/2)\rho_2+\rho_3}
    \grstep[(1/2)\rho_2]{-\rho_1}
    \grstep{\rho_2+\rho_1}
    \begin{mat}
    1 &0  &1  \\
    0 &1  &0  \\
    0 &0  &0
    \end{mat}
    \end{equation*}
    So they are row equivalent.
  \end{solution}
% sage: M = matrix(QQ, [[2,4,2], [1,1,1], [0,-1,0]])
% sage: gauss_jordan(M)
% [ 2  4  2]
% [ 1  1  1]
% [ 0 -1  0]
%  take -1/2 times row 1 plus row 2
% [ 2  4  2]
% [ 0 -1  0]
% [ 0 -1  0]
%  take -1 times row 2 plus row 3
% [ 2  4  2]
% [ 0 -1  0]
% [ 0  0  0]
%  take 1/2 times row 1
%  take -1 times row 2
% [1 2 1]
% [0 1 0]
% [0 0 0]
%  take -2 times row 2 plus row 1
% [1 0 1]
% [0 1 0]
% [0 0 0]
% sage: M = matrix(QQ, [[0,2,0], [-1,1,-1], [2,3,2]])
% sage: gauss_jordan(M)
% [ 0  2  0]
% [-1  1 -1]
% [ 2  3  2]
%  swap row 1 with row 2
% [-1  1 -1]
% [ 0  2  0]
% [ 2  3  2]
%  take 2 times row 1 plus row 3
% [-1  1 -1]
% [ 0  2  0]
% [ 0  5  0]
%  take -5/2 times row 2 plus row 3
% [-1  1 -1]
% [ 0  2  0]
% [ 0  0  0]
%  take -1 times row 1
%  take 1/2 times row 2
% [ 1 -1  1]
% [ 0  1  0]
% [ 0  0  0]
%  take 1 times row 2 plus row 1
% [1 0 1]
% [0 1 0]
% [0 0 0]

  
\question
  Decide if each is set linearly independent or dependent.
  If it is independent then verify that; if dependent then state a linear
  relation among the vectors.
  \begin{parts}
    \part $\set{\colvec{1 \\ -1 \\ 2}, \colvec{3 \\ -1 \\ 0}, \colvec{-3 \\ -1 \\ 6}}$
    \begin{solution}[1.25in]
      The linear system has infinitely many solutions.
      \begin{equation*}
      \begin{linsys}{3}
        c_1  &+ &3c_2 &- &3c_3 &= &0 \\
        -c_1 &- &c_2  &- &c_3  &= &0 \\
        2c_1 &  &     &+ &6c_3 &= &0 
        \end{linsys}
        \grstep[-2\rho_1+\rho_3]{\rho_1+\rho_2}
        \grstep{3\rho_2+\rho_3}
      \begin{linsys}{3}
        c_1  &+ &3c_2 &- &3c_3 &= &0 \\
             &  &2c_2 &- &4c_3 &= &0 \\
             &  &     &  &0    &= &0 
        \end{linsys}
      \end{equation*}
      So the set of three vectors is linearly dependent.
      For an example of a dependence, take $c_3=1$.
      That leads to $c_2=2$ and $c_1=-3$.
    \end{solution}

  
    \part $\set{x^2-1,x^2+1,x+1}$
    \begin{solution}[0.75in]
      The linear relationship
      \begin{equation*}
        0+0x+0x^2=c_1\cdot(x^2-1)+c_2(x^2+1)+c_3(x+1)
      \end{equation*}
      gives $c_3=0$ by consideration of the coefficients of~$x$,
      leaving $c_1+c_2=0$ and $-c_1+c_2=0$.
      Solving that system is easy; the only solution is the trivial one.
      So the starting set is linearly independent.
    \end{solution}

  \end{parts}

% \question
%   Find a basis for the span of this set of three-wide row vectors.
%   \begin{equation*}
%     S=\set{\rowvec{2 &-1 &0}, \rowvec{1 &1 &-3}, \rowvec{5 &-1 &-3}}
%   \end{equation*}
%   (You must, as always, show the work.)
%   \begin{solution}[1.25in]
%     Make a matrix and do Gauss's Method.
%     \begin{equation*}
%       \begin{mat}
%         2 &-1 &0 \\
%         1 &1 &-3 \\ 
%         5 &-1 &-3
%       \end{mat}
%       \grstep[-(5/2)\rho_21+\rho_3]{-(1/2)\rho_1+\rho_2}
%       \begin{mat}
%         2 &-1  &0 \\
%         0 &3/2 &-3 \\ 
%         0 &3/2 &-3
%       \end{mat}
%       \grstep{-\rho_2+\rho_3}
%       \begin{mat}
%         2 &-1  &0 \\
%         0 &3/2 &-3 \\ 
%         0 &0   &0
%       \end{mat}
%     \end{equation*}
%     A basis is $B=\set{\rowvec{2 &-1  &0}, \rowvec{0 &3/2 &-3}}$.
% sage: M = matrix(QQ, [[2,-1,0], [1,1,-3], [5,-1,-3]])
% sage: gauss_method(M)
% [ 2 -1  0]
% [ 1  1 -3]
% [ 5 -1 -3]
%  take -1/2 times row 1 plus row 2
%  take -5/2 times row 1 plus row 3
% [  2  -1   0]
% [  0 3/2  -3]
% [  0 3/2  -3]
%  take -1 times row 2 plus row 3
% [  2  -1   0]
% [  0 3/2  -3]
% [  0   0   0]
%  \end{solution}
\end{questions}
\end{document}
