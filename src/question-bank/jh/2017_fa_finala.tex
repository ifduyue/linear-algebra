% \documentclass[11pt]{examjh}
\documentclass[11pt,answers]{examjh}
\usepackage{../../linalgjh}
\examhead{MA 213 Hef{}feron, 2017-Fall}{Final Exam}

\begin{document}
\begin{questions}


\question
Represent the projection map from $\Re^3$ to the $xy$-plane with
respect to the basis.
\begin{equation*}
B=\sequence{\colvec{1 \\ 1 \\ 1},
\colvec{0 \\ 1 \\ 1},
\colvec{0 \\ 0 \\ 1}}
\end{equation*}
\begin{solution}[2in]
Projection has this effect.
\begin{equation*}
\colvec{1 \\ 1 \\ 1}\mapsto\colvec{1 \\ 1 \\ 0}
\quad
\colvec{0 \\ 1 \\ 1}\mapsto\colvec{0 \\ 1 \\ 0}
\quad
\colvec{0 \\ 0 \\ 1}\mapsto\colvec{0 \\ 0 \\ 0}
\end{equation*}
Finding the representations with respect to $B$
\begin{equation*}
\rep{\colvec{1 \\ 1 \\ 0}}{B}=\colvec{1 \\ 0 \\ -1}
\quad
\rep{\colvec{0 \\ 1 \\ 0}}{B}=\colvec{0 \\ 1 \\ -1}
\quad
\rep{\colvec{0 \\ 0 \\ 0}}{B}=\colvec{0 \\ 0 \\ 0}
\quad
\end{equation*}
and so the matrix is this.
\begin{equation*}
\rep{\pi}{B,B}=
\begin{mat}
1  &0  &0 \\
0  &1  &0 \\
-1 &-1 &0
\end{mat}
\end{equation*}
\end{solution}





\question
For each, decide if it is a basis for the space.
Justify.
\begin{parts}
\item
$\set{1+x+x^2,x-x^2,2-x}\subseteq\polyspace_2$
\begin{solution}[1in]
The set is linearly independent since this
\begin{equation*}
  c_1(1+x+x^2)+c_2(x-x^2)+c_3(2-x)=0+0x+0x^2
\end{equation*}
gives a linear system
\begin{equation*}
\begin{linsys}{3}
  c_1 &- &c_2 &  &     &= &0 \\
  c_1 &- &c_2 &- &c_3  &= &0 \\
  c_1 &  &    &+ &2c_3 &= &0 
\end{linsys}
\grstep[-\rho_1+\rho_3]{-\rho_1+\rho_2}  
\begin{linsys}{3}
  c_1 &- &c_2 &  &     &= &0 \\
      &  &    &- &c_3  &= &0 \\
      &  &c_2 &+ &2c_3 &= &0 
\end{linsys}
\end{equation*}
whose only solution is $c_1=c_2=c_3=0$.
Because it has three vectors, and the dimension of $\polyspace_2$ is three,
it is a basis.
\end{solution}

\item
$
  \set{\colvec{0 \\ 1 \\ 2},
  \colvec{1 \\ 1 \\ 2},
  \colvec{0 \\ -1 \\ 0}}\subseteq\R^3
$
\begin{solution}[1.25in]
The equation
\begin{equation*}
c_1\colvec{0 \\ 1 \\ 2}
+c_2\colvec{1 \\ 1 \\ 2}
+c_3\colvec{0 \\ -1 \\ 0}
=\colvec{0 \\ 0 \\ 0}
\end{equation*}
has $c_2=0$ by the equation of first entries.
With that, the equation of third entries gives $c_1=0$.
Thus the only solution is the trivial one.
Because $\R^3$ has dimension three, this is a basis.
\end{solution}
\end{parts}





\question  % Five.II.3.20
Find the characteristic polynomial and the eigenvalues.
\begin{equation*}
\begin{mat}
1 &2  \\
4 &3
\end{mat}
\end{equation*}
\begin{solution}[2in]
The equation
\begin{equation*}
\begin{vmatrix}
1-x  &2  \\
4  &3-x
\end{vmatrix}=0
\end{equation*}
gives the characteristic equation
$0=(1-x)(3-x)-8=x^2-4x-5$.
The eigenvalues are $\lambda_1=5$ and  $\lambda_2=-1$.
\end{solution}



\question
Perform the operations, or say ``not defined.''
\begin{parts}
\item
$
5\cdot\begin{mat}
1 &3 \\
-1 &3
\end{mat}
-3\cdot
\begin{mat}
4 &0 \\
3 &1
\end{mat}
$
\begin{solution}[1in]
$
\begin{mat}
-7  &15 \\
-14 &12
\end{mat}
$
\end{solution}
\item
$
  \det(
  \begin{mat}
  3 &3  &-1 \\
  0 &0 &2
  \end{mat}
)
$
\begin{solution}[1in]
The determinant of a non-square matrix is not defined. 
\end{solution}
\item
$
\begin{mat}
1 &3 &-1 \\
2 &4 &2 
\end{mat}
\begin{mat}
3 &0 \\
2 &0 \\
1 &-1
\end{mat}
$
\begin{solution}[1in]
$
\begin{mat}
8  &1 \\
16 &-2
\end{mat}
$
\end{solution}
\end{parts}




\question
Consider the map $\map{t}{\Re^2}{\Re^2}$ that rotates vectors through an
angle of $\pi/6$ radians counterclockwise.
\begin{parts}
\item
Represent it with respect to the standard basis.
\begin{solution}[1.5in]
On the basis elements the effect of the transformation is
\begin{equation*}
\colvec{1 \\ 0}\mapsto\colvec{\cos \pi/6 \\ \sin \pi/6}
\qquad
\colvec{0 \\ 1}\mapsto\colvec{-\sin \pi/6 \\ \cos \pi/6}
\end{equation*}
and with respect to the standard basis vectors represent themselves, 
so the matrix representation is this.
\begin{equation*}
T=\rep{t}{\stdbasis_2,\stdbasis_2}=
\begin{mat}
\cos\pi/6  &-\sin\pi/6 \\
\sin\pi/6   &\cos\pi/6
\end{mat}
\end{equation*}

\end{solution}
\item Give the change of basis arrow diagram.
\begin{solution}[1in]
\begin{equation*}
  \begin{CD}
    V_{\wrt{\stdbasis_2}}                   @>t>T>        V_{\wrt{\stdbasis_2}}       \\
    @V{\scriptstyle\identity} VV              @V{\scriptstyle\identity} VV \\
    V_{\wrt{B}}                   @>t>\hat{T}>        V_{\wrt{B}}
  \end{CD}
\end{equation*}
\end{solution}

\item
Find the matrices $P$ and $Q$ that change its representation to being with
respect to this basis.
\begin{equation*}
B=\sequence{\colvec{1 \\ 1},
\colvec{1 \\ -1}}
\end{equation*}
\begin{solution}[2in]
To find the matrix for the left side of the diagram we compute the
effect of the identity map.
\begin{equation*}
\colvec{1 \\ 0}\mapsunder{\identity}\colvec{1 \\ 0}
\quad
\colvec{1 \\ 0}\mapsunder{\identity}\colvec{1 \\ 0}
\end{equation*}
and represent those with respect to $B$.
\begin{equation*}
\rep{\colvec{1 \\ 0}}{B}=\colvec{1/2 \\ 1/2}
\quad
\rep{\colvec{0 \\ 1}}{B}=\colvec{1/2 \\ -1/2}
\qquad
P=
\begin{mat}
1/2  &1/2  \\
1/2  &-1/2
\end{mat}
\end{equation*}
\end{solution}

\item
Give $\rep{t}{B,B}$.
\begin{solution}[1in]
We compute $P^{-1}TP$.
\begin{equation*}
\frac{1}{(1/2)(-1/2)-(1/2)(1/2)}
\begin{mat}
-1/2  &-1/2  \\
-1/2  &1/2
\end{mat}
\begin{mat}
\cos\pi/6  &-\sin\pi/6 \\
\sin\pi/6   &\cos\pi/6
\end{mat}
\begin{mat}
1/2  &1/2  \\
1/2  &-1/2
\end{mat}
\end{equation*}
\end{solution}
\end{parts}



\question
Find the range space and the rank of this linear map $\map{h}{\polyspace_2}{\R^2}$.
\begin{equation*}
  ax^2+bx+c\mapsto\colvec{a+b \\ a-b}
\end{equation*}
\begin{solution}[2in]
One way is to represent the map with respect to some bases.
We take these.
\begin{equation*}
B=\sequence{1,x,x^2}
\qquad
D=\stdbasis_2
\end{equation*}
We have 
\begin{equation*}
1\mapsunder{h}\colvec{0 \\ 0}
\quad
x\mapsunder{h}\colvec{1 \\ -1}
\quad
x^2\mapsunder{h}\colvec{1 \\ 1}
\end{equation*}
and so we have $\rep{h}{B,D}$.
\begin{equation*}
\rep{\colvec{0 \\ 0}}{D}=\colvec{0 \\ 0}
\quad
\rep{\colvec{1 \\ -1}}{D}=\colvec{1 \\ -1}
\quad
\rep{\colvec{1 \\ 1}}{D}=\colvec{1 \\ 1}
\qquad
\begin{mat}
0 &1  &1 \\
0 &-1 &1
\end{mat}
\end{equation*}
By eye we see that Gauss's method on that matrix gives that its rank is two.
This is also the rank of the map.
Because the codomain has dimension two, the map is onto and the range space
is all of $\R^2$.
\end{solution}




\question % One.I.3.9
Find the solution set of this linear system.
Express it as a span of a minimal set of vectors.
\begin{equation*}
\begin{linsys}{4}
  x &   &  &+  &z  &+  &w  &=  &-1  \\
  2x&-  &y &   &   &+  &w  &=  &3   \\
  x &+  &y &+  &3z &+  &2w  &=  &1
\end{linsys}
\end{equation*}
\begin{solution}[2.5in]
\begin{equation*}
  \begin{linsys}{4}
    x  &   &  &+  &z  &+ &w  &=  &-1  \\
   2x  &-  &y &   &   &+ &w  &=  &3   \\
    x  &+  &y &+  &3z &+ &2w &=  &1   
  \end{linsys}
  \grstep[-\rho_1+\rho_3]{-2\rho_1+\rho_2}
  \begin{linsys}{4}
    x  &   &  &+  &z  &+ &w  &=  &-1  \\
       &   &-y&-  &2z &- &w  &=  &5   \\
       &   &y &+  &2z &+ &w  &=  &2   
   \end{linsys}
\end{equation*}
It has no solutions because the final two equations
conflict.
The solution set is empty.
\end{solution}

\question
Find the determinant of these matrices.
State whether each is singular or nonsingular.
\begin{parts}
\item $
\begin{mat}
1 &2 \\
3 &4
\end{mat}
$
\begin{solution}[0.5in]
$(1)(4)-(2)(3)=-2$
\end{solution}

\item
$
\begin{mat}
1 &0 &3  \\
2 &1 &-1 \\
2 &0 &4
\end{mat}
$
\begin{solution}[2in]
Doing the Laplace expansion down the second column gives this.
\begin{equation*}
\begin{vmatrix}
1 &0 &3  \\
2 &1 &-1 \\
2 &0 &4
\end{vmatrix}
=-0
+1\cdot
\begin{vmatrix}
1 &3 \\
2 &4
\end{vmatrix}
-0
=-2
\end{equation*}
\end{solution}
\end{parts}



\question
Find the inverse of each matrix, or state that none exists.
\begin{parts}
\item % Three.Iv.4.15(e)
$
\begin{mat}
0 &1 &5  \\
0 &-2 &4  \\
2 &3 &-2 
\end{mat}
$
\begin{solution}[2in]
          \begin{multline*}
            \begin{pmat}{rrr|rrr}
              0  &1  &5  &1  &0  &0  \\
              0  &-2 &4  &0  &1  &0  \\ 
              2  &3  &-2 &0  &0  &1
            \end{pmat}
            \grstep{\rho_3\leftrightarrow\rho_1}\;
            \begin{pmat}{rrr|rrr}
              2  &3  &-2 &0  &0  &1  \\
              0  &-2 &4  &0  &1  &0  \\ 
              0  &1  &5  &1  &0  &0  
            \end{pmat}                                            \\
            \begin{aligned}
              &\grstep{(1/2)\rho_2+\rho_3}
              \begin{pmat}{rrr|rrr}
                2  &3  &-2 &0  &0   &1  \\
                0  &-2 &4  &0  &1   &0  \\ 
                0  &0  &7  &1  &1/2 &0  
              \end{pmat}                                             \\
              &\grstep[-(1/2)\rho_2 \\ (1/7)\rho_3]{(1/2)\rho_1}
              \begin{pmat}{rrr|rrr}
                1  &3/2  &-1 &0    &0     &1/2  \\
                0  &1    &-2 &0    &-1/2  &0    \\ 
                0  &0    &1  &1/7  &1/14  &0  
              \end{pmat}                                   \\
              &\grstep[\rho_3+\rho_1]{2\rho_3+\rho_2}
              \begin{pmat}{rrr|rrr}
                1  &3/2  &0  &1/7  &1/14  &1/2  \\
                0  &1    &0  &2/7  &-5/14 &0    \\ 
                0  &0    &1  &1/7  &1/14  &0  
              \end{pmat}                                   \\
              &\grstep{-(3/2)\rho_2+\rho_1}
              \begin{pmat}{rrr|rrr}
                1  &0    &0  &-2/7 &17/28 &1/2  \\
                0  &1    &0  &2/7  &-5/14 &0    \\ 
                0  &0    &1  &1/7  &1/14  &0  
              \end{pmat}
            \end{aligned}
          \end{multline*}
\end{solution}
\item
$
\begin{mat}
2  &6 \\
4  &12
\end{mat}
$
\begin{solution}[1in]
The determinant is zero so the matrix has no inverse.
\end{solution}
\end{parts}



\question
Diagonalize this matrix.
\begin{equation*}
\begin{mat}
3 &0 \\
8 &-1
\end{mat}
\end{equation*}
\begin{solution}[2in]
Find the eigenvalues via the characteristic equation.
\begin{equation*}
0=\begin{vmatrix}
3-x &0  \\
8   &-1-x
\end{vmatrix}
=(3-x)(-1-x)-(8)(0)
\qquad \lambda_1=3, \; \lambda_2=-1
\end{equation*}
A diagonal matrix similar to the given one is this.
\begin{equation*}
\begin{mat}
3  &0 \\
0  &-1
\end{mat}
\end{equation*}
\end{solution}
\end{questions}
\end{document}

