% \documentclass[11pt,answers]{examjh}
\documentclass[11pt,noanswers]{examjh}
\usepackage{../../linalgjh}
\examhead{MA 213 Hef{}feron, 2017-Fall}{Due: Thu 2017-Dec-07}

\setlength{\parindent}{0em}
\begin{document}
\makebox[\linewidth]{\textbf{Homework 4, MA~213}}

\vspace*{3ex}
\textit{You may work with others to figure out how to do questions, 
and you are welcome to look for answers in the book, online, by talking
to someone who had the course before, etc.
However, you must write 
the answers on your own.
You must also show your work (you may, of course, 
quote any result from the book).}

\begin{questions}

\question
  For each space find the matrix changing a vector representation with 
  respect to $B$ to one with respect to~$D$.
  \begin{enumerate}
  \item $V=\Re^3$, $B=\stdbasis_3$, 
      $D=\sequence{\colvec{1 \\ 2 \\ 3},
                   \colvec{1 \\ 1 \\ 1},
                   \colvec{0 \\ 1 \\ -1}}$
  \item $V=\polyspace_2$,
    $B=\sequence{x^2, x^2+x, x^2+x+1}$,
    $D=\sequence{2, -x, x^2}$
  \end{enumerate}

\question
  Consider the linear transformation $\map{t}{\Re^3}{\Re^3}$
  represented with respect to the 
  standard bases by this matrix.
  \begin{equation*}
    \begin{mat}
      1 &0 &-1 \\
      3 &1 &1 \\
     -1 &0 &3
    \end{mat}
  \end{equation*}
  \begin{enumerate}
    \item Compute the determinant of the matrix.
    \item Find the size of the box defined by these vectors.
      \begin{equation*}
        \colvec{1 \\ -1 \\ 2}
        \quad
        \colvec{2 \\ 0 \\ -1}
        \quad
        \colvec{1 \\ 1 \\ 0}
      \end{equation*}
    Wht is its orientation?
  \item Find the image under $t$ of the vectors in the prior item and 
    find the size of the box that they define.
  \end{enumerate}

\question
  Express this nonsingular matrix as a product of elementary reduction
  matrices.
  \begin{equation*}
    \begin{mat}
      1 &2  &0 \\
      2 &-1 &0 \\
      3 &1 &2
    \end{mat}
  \end{equation*}

\question
  Find the $P$ and~$Q$ to express $H$ via $PHQ$ as a block partial identity
  matrix.
  \begin{equation*}
    H=
    \begin{mat}
      2 &1  &1  \\
      3 &-1 &0  \\
      1 &3  &2 
    \end{mat}
  \end{equation*}


\end{questions}
\end{document}
