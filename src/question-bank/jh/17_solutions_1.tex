%\documentclass{exam}
\documentclass[answers]{exam}
\usepackage[margin=1in]{geometry}
\usepackage{../../linalgjh}

\setlength{\parindent}{0em}
\begin{document}
\makebox[\linewidth]{\textbf{Homework, MA~213}\hspace*{4in}\textbf{2017-Sep-13}}

\vspace*{3ex}
You may work with others to figure out how to do questions, 
and you are welcome to look for answers in the book, online, by talking
to someone who had the course before, etc.
\textit{However, you must write 
  the answers on your own.
  Start with a fresh paper; do not copy from notes, online, etc., when
  you are writing it up for handing in.
  And, you must show your work.
  You can only get credit for work shown.}

\begin{questions}
\question
Solve each system.
State whether it has a unique solution, no solutions, or infinitely many
solutions.
Give the solution set, in vector form.
\begin{parts}
\part
    $\begin{linsys}{3}
      x  &+ &y  &+ &z &= &4 \\
      2x &- &2y &- &z &= &-1 \\
      4x &+ &y  &+ &2z &= &10
    \end{linsys}$
    \begin{solution}
    \begin{align*}
      \begin{amat}{3}
        1 &1  &1  &4     \\
        2 &-2 &-1 &-1    \\
        4 &1  &2  &10
      \end{amat}
      &\grstep[-4\rho_1+\rho_3]{-2\rho_1+\rho_2}              
      \begin{amat}{3}
        1 &1   &1   &4     \\
        0 &-4  &-3  &-9    \\
        0 &-3  &-2  &-6
      \end{amat}                                  \\
      &\grstep{(3/4)\rho_2+\rho_3}              
      \begin{amat}{3}
        1 &1   &1    &4     \\
        0 &-4  &-3   &-9    \\
        0 &0   &1/4  &3/4
      \end{amat}
    \end{align*}
    Back substitution gives $z=3$, $y=0$, and $x=1$.
    \end{solution}
\part
    $\begin{linsys}{4}
      x  &+ &y  &- &z  &= &0 \\
      2x &+ &y  &  &   &= &1 \\
      4x &+ &3y &- &2z &= &0  \\
      x  &  &   &+ & z &= &0  
    \end{linsys}$
    \begin{solution}
    \begin{align*}
      \begin{amat}{3}
        1 &1  &-1  &0     \\
        2 &1  &0   &1    \\
        4 &3  &-2  &0    \\
        1 &0  &1   &0
      \end{amat}
      &\grstep[-4\rho_1+\rho_3\\ -1\rho_1+\rho_4]{-2\rho_1+\rho_2}              
      \begin{amat}{3}
        1 &1  &-1  &0     \\
        0 &-1  &2   &1    \\
        0 &-1  &2  &0    \\
        0 &-1  &2   &0
      \end{amat}                                                    \\
      &\grstep[-\rho_2+\rho_4]{-\rho_2+\rho_3}              
      \begin{amat}{3}
        1 &1  &-1  &0     \\
        0 &-1  &2   &1    \\
        0 &0  &0  &-1    \\
        0 &0  &0   &-1
      \end{amat} 
    \end{align*}
    Because of the contradictory equation (actually, there are two), stop
    and conclude that there is no solution.
    \end{solution}
    
\part
    $\begin{linsys}{5}
      2x  &- &y  &- &z &+ &2w &= &3 \\
       x  &+ &y  &+ &z &  &   &= &-1 
    \end{linsys}$
    \begin{solution}
    \begin{align*}
      \begin{amat}{4}
        2 &-1  &-1  &2 &3     \\
        1 &1   &1   &0 &-1    
      \end{amat}
      &\grstep{(1/2)\rho_1+\rho_2}              
      \begin{amat}{4}
        2 &-1  &-1  &2 &3     \\
        0 &3/2   &3/2   &-1 &-5/2    
      \end{amat}
    \end{align*}
    The free variables are $z$ and~$w$.
    In terms of those, the other two are 
    $y=-z+(2/3)w$ and~$x=(2/3)-(2/3)w$.
    \end{solution}
\end{parts}



\question
  For the third system in the first question, 
  give the associated homogeneous system and 
  give its solution set.
  \begin{solution}
  The associated homogeneous system
  \begin{equation*}
    \begin{linsys}{5}
      2x  &- &y  &- &z &+ &2w &= &0 \\
       x  &+ &y  &+ &z &  &   &= &0 
    \end{linsys}
  \end{equation*}
  is solved with the same linear reduction steps.
    \begin{align*}
      \begin{amat}{4}
        2 &-1  &-1  &2 &0     \\
        1 &1   &1   &0 &0    
      \end{amat}
      &\grstep{(1/2)\rho_1+\rho_2}              
      \begin{amat}{4}
        2 &-1  &-1  &2 &0     \\
        0 &3/2   &3/2   &-1 &0    
      \end{amat}
    \end{align*}
  Again $z$ and~$w$ are free.
  The solution set is this.
  \begin{equation*}
    \set{\colvec{x \\ y \\ z \\ w}
           =\colvec{0 \\ -1 \\ 1 \\ 0}z+\colvec{-2/3 \\ 2/3 \\ 0 \\ 1}w 
        \suchthat  z,w\in\R}
  \end{equation*}
  \end{solution}

  


\question
  Do Gauss-Jordan reduction.
  Describe the solution set.
  \begin{parts}
  \part
    $\begin{linsys}{3}
      x  &+ &y  &+ &z &= &4 \\
      2x &- &2y &- &z &= &-1 \\
      4x &+ &y  &+ &2z &= &10
    \end{linsys}$
    \begin{solution}
    These steps give the solution shown.
    \begin{equation*}
    \begin{amat}{3}
      1 &1  &1  &4   \\
      2 &-2 &-1 &-1  \\
      4 &1  &2  &10
    \end{amat}
    \grstep[-4\rho_1+\rho_3]{-2\rho_1+\rho_2}
    \grstep{(3/4)\rho_2+\rho_3}
    \grstep[4\rho_3]{(1/4)\rho_2}
    \grstep[-(3/4)\rho_3+\rho_2]{-\rho_3+\rho_1}
    \grstep{-\rho_2+\rho_1}
    \begin{amat}{3}
      1 &0  &0  &1   \\
      0 &1  &0  &0  \\
      0 &0  &1  &3
    \end{amat}
    \end{equation*}
    The solution set has one member, the vector whose components are
    $1$, $0$, and $3$.    
    \begin{equation*}
      \set{\colvec{1 \\ 0 \\ 3}}
    \end{equation*}
    \end{solution}

  \part
    $\begin{linsys}{3}
      x  &+ &y  &+ &2z  &= &0 \\
      2x  &- &y  &+  &z &= &1 \\
      4x &+ &y  &+ &5z &= &1  
    \end{linsys}$
  \end{parts}
  \begin{solution}
  Here there are infinitely many solutions.
  \begin{equation*}
    \begin{amat}{3}
    1 &1  &2 &0 \\
    2 &-1 &1 &1 \\
    4 &1  &5 &1  
    \end{amat}
    \grstep[-4\rho_1+\rho_3]{-2\rho_1+\rho_1}
    \grstep{-\rho_2+\rho_3}
    \grstep{(1/3)\rho_2}
    \grstep{-\rho_2+\rho_1}
    \begin{amat}{3}
    1 &0  &1 &1/3 \\
    0 &1  &1 &-1/3 \\
    0 &0  &0 &0  
    \end{amat}    
  \end{equation*}
  Back substitution gives this description of the solution set.
  \begin{equation*}
    \set{\colvec{1/3 \\ -1/3 \\ 0}+\colvec{-1 \\ -1 \\ 1}z\suchthat z\in\Re}
  \end{equation*}

  \end{solution}

\end{questions}
\end{document}


sage: load("../../lab/gauss_method.sage")
sage: M = matrix(QQ, [[1,1,1], [2,-2,-1], [4,1,2]])
sage: v = vector(QQ, [4,-1,10])
sage: M_prime = M.augment(v, subdivide=True)
sage: gauss_method(M_prime)
[ 1  1  1| 4]
[ 2 -2 -1|-1]
[ 4  1  2|10]
 take -2 times row 1 plus row 2
 take -4 times row 1 plus row 3
[ 1  1  1| 4]
[ 0 -4 -3|-9]
[ 0 -3 -2|-6]
 take -3/4 times row 2 plus row 3
[  1   1   1|  4]
[  0  -4  -3| -9]
[  0   0 1/4|3/4]
sage: var('x,y,z,w')
(x, y, z, w)
sage: eqns = [x+y+z==4, 2*x-2*y-z==-1, 4*x+y+2*z==10]
sage: solve(eqns, x, y, z)
[[x == 1, y == 0, z == 3]]
sage: M = matrix(QQ, [[1,1,-1], [2,1,0], [4,3,-2], [1,0,1]])
sage: v = vector(QQ, [0,1,0,0])
sage: M_prime = M.augment(v, subdivide=True)
sage: gauss_method(M_prime)
[ 1  1 -1| 0]
[ 2  1  0| 1]
[ 4  3 -2| 0]
[ 1  0  1| 0]
 take -2 times row 1 plus row 2
 take -4 times row 1 plus row 3
 take -1 times row 1 plus row 4
[ 1  1 -1| 0]
[ 0 -1  2| 1]
[ 0 -1  2| 0]
[ 0 -1  2| 0]
 take -1 times row 2 plus row 3
 take -1 times row 2 plus row 4
[ 1  1 -1| 0]
[ 0 -1  2| 1]
[ 0  0  0|-1]
[ 0  0  0|-1]
 take -1 times row 3 plus row 4
[ 1  1 -1| 0]
[ 0 -1  2| 1]
[ 0  0  0|-1]
[ 0  0  0| 0]
sage: M = matrix(QQ, [[2,-1,-1,2], [1,1,1,0]])
sage: v = vector(QQ, [3,-1])
sage: M_prime = M.augment(v, subdivide=True)
sage: gauss_method(M_prime)
[ 2 -1 -1  2| 3]
[ 1  1  1  0|-1]
 take -1/2 times row 1 plus row 2
[   2   -1   -1    2|   3]
[   0  3/2  3/2   -1|-5/2]
sage: eqns = [2*x-y-z+2*w==3, x+y+z==-1]
sage: solve(eqns, x, y, z)
[[x == -2/3*w + 2/3, y == -r1 + 2/3*w - 5/3, z == r1]]
sage: solve(eqns, x, y, z, w)
[[x == -2/3*r2 + 2/3, y == 2/3*r2 - r3 - 5/3, z == r3, w == r2]]
sage: M = matrix(QQ, [[1,1,1], [2,-2,-1], [4,1,2]])
sage: v = vector(QQ, [4,-1,10])
sage: M_prime = M.augment(v, subdivide=True)
sage: gauss_jordan(M_prime)
[ 1  1  1| 4]
[ 2 -2 -1|-1]
[ 4  1  2|10]
 take -2 times row 1 plus row 2
 take -4 times row 1 plus row 3
[ 1  1  1| 4]
[ 0 -4 -3|-9]
[ 0 -3 -2|-6]
 take -3/4 times row 2 plus row 3
[  1   1   1|  4]
[  0  -4  -3| -9]
[  0   0 1/4|3/4]
 take -1/4 times row 2
 take 4 times row 3
[  1   1   1|  4]
[  0   1 3/4|9/4]
[  0   0   1|  3]
 take -1 times row 3 plus row 1
 take -3/4 times row 3 plus row 2
[1 1 0|1]
[0 1 0|0]
[0 0 1|3]
 take -1 times row 2 plus row 1
[1 0 0|1]
[0 1 0|0]
[0 0 1|3]
sage: 
sage: M = matrix(QQ, [[1,1,2], [2,-1,1], [4,1,5]])
sage: v = vector(QQ, [0,1,1])
sage: M_prime = M.augment(v, subdivide=True)
sage: gauss_jordan(M_prime)
[ 1  1  2| 0]
[ 2 -1  1| 1]
[ 4  1  5| 1]
 take -2 times row 1 plus row 2
 take -4 times row 1 plus row 3
[ 1  1  2| 0]
[ 0 -3 -3| 1]
[ 0 -3 -3| 1]
 take -1 times row 2 plus row 3
[ 1  1  2| 0]
[ 0 -3 -3| 1]
[ 0  0  0| 0]
 take -1/3 times row 2
[   1    1    2|   0]
[   0    1    1|-1/3]
[   0    0    0|   0]
 take -1 times row 2 plus row 1
[   1    0    1| 1/3]
[   0    1    1|-1/3]
[   0    0    0|   0]
sage: var('x,y,z')
(x, y, z)
sage: eqns = [x+y+2*z==0, 2*x-y+z==1, 4*x+y+5*z==1]
sage: solve(eqns, x, y, z)
[[x == -r4 + 1/3, y == -r4 - 1/3, z == r4]]
sage: 
