\documentclass[11pt]{examjh}
% \documentclass[11pt,answers]{examjh}
\usepackage{../../linalgjh}
\examhead{MA 213 Hef{}feron, 2017-Fall}{Exam Two}

\begin{document}
\begin{questions}
\question
List the dimension of each space.
\begin{parts}
\item $\R^4$
\item $\matspace_{3\times 2}$
\item $\polyspace_4$
\begin{solution}[0.5in]

\end{solution}
\end{parts}



\question
Perform each matrix operation or state ``not defined.''
\begin{parts}
\item
$
  \begin{mat}
    1 &1  &2 \\
    3 &-1 &0 \\
    2 &2  &1
  \end{mat}
  \begin{mat}
    -1 &1 \\
    -3 &3 \\
     1 &0
  \end{mat}
  $
\begin{solution}[1.25in]
  \begin{equation*}
    \begin{mat}
    \end{mat}
  \end{equation*}
\end{solution}    
\item
$
5\cdot\begin{mat}
5 &1 \\
2 &-1
\end{mat}
+6\cdot
\begin{mat}
4 &0  &0  \\
2 &-1 &6
\end{mat}
$  
\begin{solution}[1in]
  \begin{equation*}
    \begin{mat}
    \end{mat}
  \end{equation*}
\end{solution}    
\item
$
\begin{mat}
1 &3 &1 \\
0 &0 &2
\end{mat}
\begin{mat}
1 &0 \\
3 &0 \\
1 &2
\end{mat}
$
\begin{solution}[1.25in]
  \begin{equation*}
    \begin{mat}
    \end{mat}
  \end{equation*}
\end{solution}    
\end{parts}



\question
Show that the map $\map{t}{\polyspace_2}{\matspace_{1\times 3}}$ given by
$t(ax^2+bx+c)=\rowvec{b-a &c &b}$ is an isomorphism.
\begin{solution}[2.5in]
To see that the map is one-to-one suppose that $t(\vec{v}_1)=t(\vec{v}_2)$,
aiming to conclude that $\vec{v}_1=\vec{v}_2$.
That is, $t(a_1x^2+b_1x+c_1)=t(a_2x^2+b_2x+c_2)$.
Then $b_1x^2-(a_1+c_1)x+a_1=b_2x^2-(a_2+c_2)x+a_2$ and because 
quadratic polynomials
are equal only if they have have the same quadratic terms, the same constant
terms, and the same linear terms we conclude that 
$b_1=b_2$, that $a_1=a_2$, and from that, $c_1=c_2$.
Therefore $a_1x^2+b_1x+c_1=a_2x^2+b_2x+c_2$ and the function is 
one-to-one.

To see that the map is onto, we suppose that we are given a member~$\vec{w}$ 
of the codomain and we find a member~$\vec{v}$ of the domain that maps to
it.
Let the member of the codomain be~$\vec{w}=px^2+qx+r$.
Observe that where $\vec{v}=rx^2+px+(-q+r)$ then $t(\vec{v})=\vec{w}$.
Thus~$t$ is onto.  

To see that the map is a homomorphism we show that it respects linear 
combinations of two elements.
\begin{multline*}
  t(r_1(a_1x^2+b_1x+c_1)+r_2(a_2x^2+b_2x+c_2))              \\ 
  \begin{split} \quad 
  &=t((r_1a_1+r_2a_2)x^2+(r_1b_1+r_2b_2)x+(r_1c_1+r_2c_2))   \\
  &=(r_1b_1+r_2b_2)x^2-((r_1a_1+r_2a_2)+(r_1c_1+r_2c_2))x+(r_1a_1+r_2a_2)  \\
  &=(r_1b_1)x^2-(r_1a_1+r_1c_1)x+r_1a_1
     +(r_2b_2)x^2-(r_2a_2+r_2c_2)x+r_2a_2                       \\
  &=r_1t(a_1x^2+b_1x+c_1)+r_2t(a_2x^2+b_2x+c_2)
  \end{split}
\end{multline*}
\end{solution}



\question
Consider the map $\map{h}{\polyspace_3}{\polyspace_2}$
given by this formula
$ax^3+bx^2+cx+d \mapsto (a+b-c)x^2-(b+c)x+(b-c)$.
\begin{parts}
\item Prove it is a linear map.
\begin{solution}[2.25in]

\end{solution}
\item Represent it with respect to these.
\begin{equation*}
   B=\sequence{x^3,x^3+x^2,x,x+1}
   \quad
  D=\sequence{x^2+1,x+1,1}
\end{equation*}
\begin{solution}[2.5in]
The action of the map on the domain's basis vectors is this.
\begin{equation*}
    \colvec{1 \\ 1 \\ 1}\mapsto\colvec{2 \\ 2}
    \quad
    \colvec{1 \\ 1 \\ 0}\mapsto\colvec{2 \\ 1}
    \quad
    \colvec{1 \\ 0 \\ 0}\mapsto\colvec{1 \\ 1}
\end{equation*}
Represent those with respect to the codomain's basis.
\begin{equation*}
  \rep{\colvec{2 \\ 2}}{D}=\colvec{2 \\ 1}_D
  \quad
  \rep{\colvec{2 \\ 1}}{D}=\colvec{2 \\ 1/2}_D
  \quad
  \rep{\colvec{1 \\ 1}}{D}=\colvec{1 \\ 1/2}_D
\end{equation*}
Concatenate them together into a matrix.
\begin{equation*}
  \rep{h}{B,D}=
  \begin{mat}
    2 &2   &1 \\
    1 &1/2 &1/2
  \end{mat}
\end{equation*}
\end{solution}
\end{parts}  



\question
Which of these spaces are isomorphic?  Briefly say why.
\begin{parts}
\item $\Re^2$
\item $\polyspace_2$
\item $\matspace_{nbyn{2}}$
\begin{solution}[0.5in]

\end{solution}
\end{parts}


\question
  Consider the map $\map{h}{\Re^3}{\Re^3}$ represented by this matrix
  with respect to the standard bases.
  \begin{equation*}
    \begin{mat}
      1 &0 &-1 \\
      2 &1 &0  \\
      2 &2 &2
    \end{mat}
  \end{equation*}
  \begin{parts}
    \item Find the range space of the map.
      What is the map's rank?
\begin{solution}[2in]
Here is the Gauss-Jordan reduction.
\begin{equation*}
  \begin{amat}{3}
      1 &0 &-1 &a \\
      2 &1 &0  &b \\
      2 &2 &2  &c  
  \end{amat}
  \grstep[-2\rho_1+\rho_3]{-2\rho_1+\rho_2}
  \begin{amat}{3}
      1 &0 &-1 &a \\
      0 &1 &2  &-2a+b \\
      0 &2 &4  &-2a+c  
  \end{amat}
  \grstep{-2\rho_2+\rho_3}
  \begin{amat}{3}
      1 &0 &-1 &a \\
      0 &1 &2  &-2a+b \\
      0 &0 &0  &2a-2b+c  
  \end{amat}
\end{equation*}
The range space is the set containing all of the members of the codomain 
for which this system has a solution.
\begin{equation*}
  \rangespace{h}=\set{\colvec{b-(1/2)c \\ b \\ c}\suchthat b,c\in\Re}
\end{equation*}
The rank is 2.
\end{solution}
    \item Describe the null space of the map.
      What is the map's nullity?
\begin{solution}[1.25in]
The null space is the set of members of the domain that map to 
$a=0$, $b=0$, and~$c=0$.
\begin{equation*}
  \nullspace{h}=\set{\colvec{z \\ -2z \\ z}\suchthat z\in\Re}
\end{equation*}
The nullity is~$1$.
\end{solution}
    \item Is the map onto?  Briefly justify.
\begin{solution}[1.0in]
  The codomain has dimension~$3$ but the map's range has only dimension~$2$, 
  so the map is not onto.
\end{solution}
    \item Is the map one-to-one?  Briefly justify.
\begin{solution}[1.0in]
  The dimension of the nullspace is not~$0$, so the map is not one-to-one.
\end{solution}
  \end{parts}


\question
Show that the map $\map{f}{\matspace_{\nbyn{2}}}{\R^2}$
\begin{equation*}
\begin{mat}
  a  &b  \\
  c  &d 
\end{mat}\mapsto
\colvec{a+b \\ 2a+2b}  
\end{equation*}
is not onto.
\begin{solution}[1in]

\end{solution}

  

\question
True or False: every matrix can be multiplied by itself, since the domains
and codomains of the underlying functions agree.  
(Give a one-sentence justification of your answer.)
\begin{solution}[0.5in]
  False.
  Here is a multiplication of a matrix with itself.
  \begin{equation*}
    \begin{mat}
      1 &2 \\
      3 &4
    \end{mat}
    \begin{mat}
      1 &2 \\
      3 &4
    \end{mat}
    =    
    \begin{mat}
      7 &10 \\
      15 &22
    \end{mat}
  \end{equation*}
\end{solution}


\question
Let the matrix $H$ represents the map $\map{f}{\Re^2}{\Re^3}$
and the matrix $G$ represent $\map{g}{\Re^3}{\Re^2}$,
\begin{equation*}
H=
\begin{mat}
2  &1  \\
3  &1  \\
4  &1
\end{mat}
\qquad
G=
\begin{mat}
0  &1 &2  \\
3  &2 &1  
\end{mat}
\end{equation*}
\begin{parts}
\item What is the representation of the composition $g\circ f$?
\begin{solution}[1.25in]

\end{solution}
\item Does the composition $f\circ g$ have a representation?
\begin{solution}[1.25in]

\end{solution}
\end{parts}

\end{questions}
\end{document}


sage: M1=matrix(QQ,[[1,1,2], [3,-1,0], [2,2,1]])
sage: M1
[ 1  1  2]
[ 3 -1  0]
[ 2  2  1]
sage: M2=matrix(QQ,[[-1,1], [-3,3], [1,0]])
sage: M2
[-1  1]
[-3  3]
[ 1  0]
sage: M1*M2
[-2  4]
[ 0  0]
[-7  8]

sage: M=matrix(QQ, [[1,0,-1], [2,1,0], [2,2,2]])
sage: load "../lab/gauss_method.sage"
/usr/lib/sagemath/local/lib/python2.7/site-packages/sage/misc/sage_extension.py:371: DeprecationWarning: Use %runfile instead of load.
See http://trac.sagemath.org/12719 for details.
  line = f(line, line_number)
sage: gauss_jordan(M)
[ 1  0 -1]
[ 2  1  0]
[ 2  2  2]
 take -2 times row 1 plus row 2
 take -2 times row 1 plus row 3
[ 1  0 -1]
[ 0  1  2]
[ 0  2  4]
 take -2 times row 2 plus row 3
[ 1  0 -1]
[ 0  1  2]
[ 0  0  0]


sage: M3=matrix(QQ,[[1,2], [3,4]])
sage: M3
[1 2]
[3 4]
sage: M3*M3
[ 7 10]
[15 22]

sage: M4=matrix(QQ,[[1,0,-1], [2,1,0], [2,0,-1]])
sage: M4
[ 1  0 -1]
[ 2  1  0]
[ 2  0 -1]
sage: M4^(-1)
[-1  0  1]
[ 2  1 -2]
[-2  0  1]
sage: gauss_jordan(M4)
[ 1  0 -1]
[ 2  1  0]
[ 2  0 -1]
 take -2 times row 1 plus row 2
 take -2 times row 1 plus row 3
[ 1  0 -1]
[ 0  1  2]
[ 0  0  1]
 take 1 times row 3 plus row 1
 take -2 times row 3 plus row 2
[1 0 0]
[0 1 0]
[0 0 1]
