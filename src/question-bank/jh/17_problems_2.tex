\documentclass[11pt,answers]{examjh}
% \documentclass[11pt,noanswers]{examjh}
\usepackage{../../linalgjh}
\examhead{MA 213 Hef{}feron, 2017-Fall}{Due: Wed 2017-Oct-18}

\setlength{\parindent}{0em}
\begin{document}
\makebox[\linewidth]{\textbf{Homework 2, MA~213}}

\vspace*{3ex}
\textit{You may work with others to figure out how to do questions, 
and you are welcome to look for answers in the book, online, by talking
to someone who had the course before, etc.
However, you must write 
the answers on your own.
You must also show your work (you may, of course, 
quote any result from the book).}

\begin{questions}
\question Represent the vector with respect to each of the two bases.
\begin{equation*}
    \vec{v}=\colvec{3  \\ -1}
    \quad
    B_1=\sequence{\colvec{1  \\ -1}, \colvec{1  \\ 1}},\;
    B_2=\sequence{\colvec{1 \\ 2}, \colvec{1 \\ 3}}
\end{equation*}
\begin{solution}
  Solving
  \begin{equation*}
    \colvec{3 \\ -1}=\colvec{1 \\ -1}\cdot c_1
                     +\colvec{1 \\ 1}\cdot c_2
  \end{equation*}
  gives $c_1=2$ and~$c_2=1$. 
  \begin{equation*}
    \rep{\colvec{3 \\ -1}}{B_1}
       =\colvec{2 \\ 1}_{B_1}
  \end{equation*}
  Similarly, solving
  \begin{equation*}
    \colvec{3 \\ -1}=\colvec{1 \\ 2}\cdot c_1
                     +\colvec{1 \\ 3}\cdot c_2
  \end{equation*}
  gives this. 
  \begin{equation*}
    \rep{\colvec{3 \\ -1}}{B_2}
       =\colvec{10 \\ -7}_{B_2}
  \end{equation*}    
\end{solution}


\question
Verify that each map is an isomorphism.
\begin{parts}
\part
  $\map{h}{\R^3}{\R^3}$ given by
  \begin{equation*}
    \colvec{a \\ b \\ c}
    \mapsto
    \colvec{a \\ a+b \\ a+b+c}
  \end{equation*}
\begin{solution}
To verify that the map is one-to-one suppose that $h(\vec{v}_1)=h(\vec{v}_2)$.
That gives this.
\begin{align*}
  h(\colvec{a_1 \\ b_1 \\ c_1}) &= h(\colvec{a_2 \\ b_2 \\ c_2})  \\
  \colvec{a_1 \\ a_1+b_1 \\ a_1+b_1+c_1}
                                &= \colvec{a_2 \\ a_2+b_2 \\ a_2+b_2+c_2}
\end{align*}
The equation of first components says that $a_1=a_2$.
With that, the equation of second components says that $b_1=b_2$,
and then the equation of third components gives $c_1=c_2$.

To verify that the map is onto, consider $\vec{w}\in\Re^3$.
\begin{equation*}
  \vec{w}=\colvec{x \\ y \\ z}
\end{equation*}
Observe that $\vec{w}=h(\vec{v})$ where
\begin{equation*}
  \vec{v}=\colvec{x \\ y-x \\ z-y}
\end{equation*}
since this is the action of $h$.
\begin{equation*}
   h(\colvec{x \\ y-x \\ z-y})=\colvec{x \\ x+(y-x) \\ x+(y-x)+(z-y)}
\end{equation*}

To check that $h$ preserves addition we can do this.
\begin{align*}
  h(\vec{v}_1+\vec{v}_2)
  &= h(\colvec{a_1 \\ b_1 \\ c_1}+\colvec{a_2 \\ b_2 \\ c_2})     \\
  &=  h(\colvec{a_1+a_2 \\ b_1+b_2 \\ c_1+c_2})          \\
  &=\colvec{(a_1+a_2) \\ (a_1+a_2)+(b_1+b_2) \\ (a_1+a_2)+(b_1+b_2)+(c_1+c_2)}  \\
  &= \colvec{a_1 \\ a_1+b_1 \\ a_1+b_1+c_1}+\colvec{a_2 \\ a_2+b_2 \\ a_2+b_2+c_2}  \\
  &= h(\colvec{a_1 \\ b_1 \\ c_1})+h(\colvec{a_2 \\ b_2 \\ c_2})     \\
  &= h(\vec{v}_1)+h(\vec{v}_2)
\end{align*}

The check for preservation of scalar multiplication is similar.
\begin{align*}
  h(r\cdot\vec{v})
  &=h(r\cdot\colvec{a \\ b \\ c})      \\
  &=h(\colvec{ra \\ rb \\ rc})          \\
  &=\colvec{ra \\  ra+rb \\ ra+rb+rc})     \\
  &=r\cdot\colvec{a \\  a+b \\ a+b+c})     \\
  &=r\cdot h(\colvec{a \\  b \\ c})     \\
  &=r\cdot h(\vec{v})
\end{align*}
\end{solution}


\part
  $\map{h}{\Re^4}{\matspace_{\nbyn{2}}}$ given by
  \begin{equation*}
    \colvec{a \\ b \\ c \\ d}
    \mapsto
    \begin{mat}
      a  &b-c \\
      b+c  &d
    \end{mat}
  \end{equation*}

\begin{solution}
This map is one-to-one because if $h(\vec{v}_1)=h(\vec{v}_2)$ then
\begin{align*}
  h(\colvec{a_1 \\ b_1 \\ c_1 \\ d_1})
  &=h(\colvec{a_2 \\ b_2 \\ c_2 \\ d_2})       \\
  \begin{mat}
      a_1  &b_1-c_1 \\
      b_1+c_1  &d_1
  \end{mat}
  &=
  \begin{mat}
      a_2  &b_2-c_2 \\
      b_2+c_2  &d_2
  \end{mat}
\end{align*}
Clearly $a_1=a_2$ and $d_1=d_2$.
From the two equation $b_1-c_1=b_2-c_2$ and
$b_1+c_1=b_2+c_2$
we get, by adding, that $2b_1=2b_2$ and so $b_1=b_2$.
Then $c_1=c_2$ follows immediately.
Thus $\vec{v}_1=\vec{v_2}$.

This map is onto.
Consider $\vec{u}\in\matspace_{\nbyn{2}}$.
\begin{equation*}
  \vec{u}=
  \begin{mat}
  x  &y  \\
  z  &w
  \end{mat}
\end{equation*}
Where
\begin{equation*}
  \vec{v}=
  \colvec{x   \\ (y+z)/2 \\ (z-y)/2 \\ w}
\end{equation*}
this calculation shows that $h(\vec{v})=\vec{u}$.
\begin{equation*}
  h(\colvec{x   \\ (y+z)/2 \\ (z-y)/2 \\ w})
  =
  \begin{mat}
  x  &((y+z)-(z-y))/2 \\
  ((z-y)+(y+z))/2  &w \\
  \end{mat}
  =\vec{u}
\end{equation*}

We combine the two remaining checks into the single check that the function
respects linear combinations.
\begin{align*}
  h(r_1\vec{v}_1+r_2\vec{v}_2)
  &=h(r_1\cdot\colvec{a_1 \\ b_1 \\ c_1  \\ d_1}
  +r_2\cdot\colvec{a_2 \\ b_2 \\ c_2 \\ d_2})             \\
  &=h(\colvec{r_1a_1+r_2a_2 \\  r_1b_1+r_2b_2 \\ r_1c_1+r_2c_2 \\ r_1d_1+r_2d_2})  \\
  &=\begin{mat}
    r_1a_1+r_2a_2     &(r_1b_1+r_2b_2)-(r_1c_1+r_2c_2) \\
    (r_1b_1+r_2b_2)+(r_1c_1+r_2c_2) &r_1d_1+r_2d_2
    \end{mat}                                               \\
  &=r_1\cdot
    \begin{mat}
    a_1     &b_1-c_1 \\
    b_1+c_1 &d_1
    \end{mat}
    +r_2\cdot
    \begin{mat}
    a_2     &b_2-c_2 \\
    b_2+c_2 &d_2
    \end{mat}                    \\
  &=r_1\cdot h(\colvec{a_1 \\ b_1 \\ c_1 \\ d_1})
    +r_2\cdot h(\colvec{a_2 \\ b_2 \\ c_2 \\ d_2})         \\
  &=r_1\cdot h(\vec{v}_1)+r_2\cdot h(\vec{v}_2)
\end{align*}



\end{solution}
\end{parts}



\question
Is the vector in the span of the set?
  \begin{equation*}
    \colvec{1 \\ 0 \\ 3}
    \quad
    \set{\colvec{2 \\ 1 \\ -1},
         \colvec{1 \\ -1 \\ 1}}
  \end{equation*}
\begin{solution}
  The equation
  \begin{equation*}
    \colvec{1 \\ 0 \\ 3} = 
             c_1\colvec{2 \\ 1 \\ -1}
             +c_2\colvec{1 \\ -1 \\ 1}
  \end{equation*}
  gives rise to a linear system
  \begin{equation*}
    \begin{amat}{2}
      2  &1  &1  \\
      1  &-1 &0  \\
      -1 &1  &3
    \end{amat}
    \grstep[(1/2)\rho_1+\rho_3]{(-1/2)\rho_1+\rho_2}
    \begin{amat}{2}
      2  &1    &1  \\
      0  &-3/2 &-1/2  \\
      0  &0  &3
    \end{amat}
  \end{equation*}
  that has no solution, so the vector is not in the span.  
\end{solution}
% sage: M = matrix(QQ, [[2,1], [1,-1], [-1,1] ])
% sage: var('c1,c2,c3,c4')
% (c1, c2, c3, c4)
% sage: v = vector(QQ, [1,0,3])
% sage: M_prime = M.augment(v, subdivide=True)
% sage: gauss_method(M_prime)
% [ 2  1| 1]
% [ 1 -1| 0]
% [-1  1| 3]
%  take -1/2 times row 1 plus row 2
%  take 1/2 times row 1 plus row 3
% [   2    1|   1]
% [   0 -3/2|-1/2]
% [   0  3/2| 7/2]
%  take 1 times row 2 plus row 3
% [   2    1|   1]
% [   0 -3/2|-1/2]
% [   0    0|   3]


\question
List three unequal spaces that are isomorphic to $\R^2$. 
\begin{solution}
There are many choices.
Three are $\polyspace_1=\set{a+bx\suchthat a,b\in\R}$,
the plane
$P=\set{c_1\colvec{1 \\ 2 \\ 3}+c_2\colvec{1 \\ 0 \\ -1}\suchthat c_1,c_2\in\R}$,
and this subspace of $\matspace_{\nbyn{2}}$.
\begin{equation*}
  Q=\set{
    \begin{mat}
    a &b \\
    a &0
    \end{mat}
    \suchthat a,b\in\Re}
  \end{equation*}
For all three finding a basis is routine.  
\end{solution}
  
\end{questions}
\end{document}
