% \documentclass[11pt]{examjh}
\documentclass[11pt,answers]{examjh}
\usepackage{../../linalgjh}
\examhead{MA 213 Hef{}feron, 2017-Fall}{Exam One}
\begin{document}
% sage: load("../../lab/gauss_method.sage")

\begin{questions}

\question
Decide if each is
a basis for the space.
If it is a basis then show the verification.
If not, show why not.
\begin{parts}
\part
$\set{\colvec{2 \\ 1}, \colvec{-1 \\ 1}}\subset\R^2$
\begin{solution}[1.5in]
This is a basis for $\Re^2$.
It is linearly independent.
\begin{equation*}
  \begin{linsys}{2}
  2c_1 &-  &c_2  &=  &0  \\
  c_1  &+  &c_2  &=  &0 
  \end{linsys}
  \grstep{-(1/2)\rho_1+\rho_2}
  \begin{linsys}{2}
  2c_1 &-  &c_2       &=  &0  \\
       &   &(3/2)c_2  &=  &0 
  \end{linsys}
  \;\Longrightarrow\;
  c_1=c_2=0
\end{equation*}
It also spans the space.
\begin{equation*}
  \begin{linsys}{2}
  2c_1 &-  &c_2  &=  &x  \\
  c_1  &+  &c_2  &=  &y 
  \end{linsys}
  \grstep{-(1/2)\rho_1+\rho_2}
  \begin{linsys}{2}
  2c_1 &-  &c_2       &=  &x  \\
       &   &(3/2)c_2  &=  &-(1/2)x+y 
  \end{linsys}
  \;\Longrightarrow\;
  \begin{array}{@{}l}
  c_2=-(1/3)x+(1/3)y  \\
  c_1=(1/3)x+(1/6)y
  \end{array}  
  \end{equation*}
\end{solution}


\part
$\set{x^2+3, x^2-1, -x^2}\subset\polyspace_2$
\begin{solution}[1.25in]
This is not a basis.
It does not span the space since none of the three has a linear term,
an $x$~term.
\end{solution}


\part
$\set{
  \begin{mat}
  1 &0 \\
  0 &-1
  \end{mat},
  \begin{mat}
  1 &2 \\
  1 &0
  \end{mat},
  \begin{mat}
  0 &0 \\
  0 &1
  \end{mat},
  \begin{mat}
  1 &1 \\
  0 &1
  \end{mat}
}\subset\matspace_{\nbyn{2}}$
\begin{solution}[1.75in]
This is a basis for $\matspace_{\nbyn{2}}$.
It is linearly independent.
\begin{equation*}
  \begin{linsys}{4}
  c_1  &+  &c_2  &   &    &+  &c_4  &=  &0  \\
       &+  &2c_2 &   &    &+  &c_4  &=  &0  \\
       &   &c_2  &   &    &+  &c_4  &=  &0  \\
  -c_1 &   &     &+  &c_3 &+  &c_4  &=  &0  
  \end{linsys}
  \grstep{\rho_1+\rho_4}
  \grstep[-(1/2)\rho_2+\rho_4]{-(1/2)\rho_2+\rho_3}
  \grstep{\rho_3\leftrightarrow\rho_4}
  \begin{linsys}{4}
  c_1  &+  &c_2  &   &    &+  &c_4       &=  &0  \\
       &   &2c_2 &   &    &+  &c_4       &=  &0  \\
       &   &     &   &c_3 &+  &(3/2)c_4  &=  &0  \\
       &   &     &   &    &   &(1/2)c_4  &=  &0  
  \end{linsys}
\end{equation*}
This system has a unique solution.

To show it spans start with
\begin{equation*}
  \begin{linsys}{4}
  c_1  &+  &c_2  &   &    &+  &c_4  &=  &a  \\
       &+  &2c_2 &   &    &+  &c_4  &=  &b  \\
       &   &c_2  &   &    &+  &c_4  &=  &c  \\
  -c_1 &   &     &+  &c_3 &+  &c_4  &=  &d  
  \end{linsys}
\end{equation*}
and the same reduction steps give
$c_1=a-c$, $c_2=b-c$, $c_3=a+b-3c$, $c_4=-b+2c$.
\end{solution}
% sage: load("../../lab/gauss_method.sage")
% sage: M = matrix(QQ, [[1,1,0,1], [0,2,0,1], [0,1,0,1], [-1,0,1,1]])
% sage: v = vector(QQ, [0, 0, 0, 0])
% sage: M_prime = M.augment(v, subdivide=True)
% sage: gauss_method(M_prime)
% [ 1  1  0  1| 0]
% [ 0  2  0  1| 0]
% [ 0  1  0  1| 0]
% [-1  0  1  1| 0]
%  take 1 times row 1 plus row 4
% [1 1 0 1|0]
% [0 2 0 1|0]
% [0 1 0 1|0]
% [0 1 1 2|0]
%  take -1/2 times row 2 plus row 3
%  take -1/2 times row 2 plus row 4
% [  1   1   0   1|  0]
% [  0   2   0   1|  0]
% [  0   0   0 1/2|  0]
% [  0   0   1 3/2|  0]
%  swap row 3 with row 4
% [  1   1   0   1|  0]
% [  0   2   0   1|  0]
% [  0   0   1 3/2|  0]
% [  0   0   0 1/2|  0]
% sage: var('c1,c2,c3,c4')
% sage: var('a,b,c,d')
% sage: eqs=[c1+c2+c4==a, 2*c2+c4==b, c2+c4==c, -c1+c3+c4==d]
% sage: solve(eqs,c1,c2,c3,c4)
% [[c1 == a - c, c2 == b - c, c3 == a + b - 3*c + d, c4 == -b + 2*c]]
\end{parts}



\question
Verify that   $S=\set{a+bx^2\suchthat a,b\in\R}$
is a subspace of $\polyspace_2$.
\begin{solution}[1.5in]
We can verify that $S$ is closed under linear combinations of two elements.
Two elements of $S$ are $\vec{v}_1=a_1+b_1x^2$ and~$\vec{v}_2=a_2+b_2x^2$.
Where $r_1,r_2\in\R$, a linear combination of those
\begin{align*}
  r_1\cdot\vec{v}_1+r_2\cdot\vec{v_2}
  &=r_1\cdot (a_1+b_1x^2)+r_2\cdot (a_2+b_2x^2) \\
  &=(r_1a_1+r_2a_2)+(r_1b_1+r_2b_2)\cdot x^2
\end{align*}
is a quadratic polynomial whose coefficient of $x$ is zero,
so it is a member of~$S$.
\end{solution}



\question
For each, find and parametrize the solution set.
\begin{parts}
\part $
  \begin{linsys}{2}
    2x &+ &y &= &0 \\
     x &- &y &= &-3 \\
    5x &+ &y &= &-3 
  \end{linsys}
  $
  \begin{solution}[1.5in]
    \begin{equation*}
      \grstep[-(5/2)\rho_1+\rho_3]{-(1/2)\rho_1+\rho_2}
      \begin{amat}{2}
        2 &1     &0 \\
        0 &-3/2  &-3  \\
        0 &-3/2   &-3 
      \end{amat}
      \grstep{-\rho_2+\rho_3}
      \begin{amat}{2}
        2 &1     &0 \\
        0 &-3/2  &-3  \\
        0 &0     &0 
      \end{amat}
    \end{equation*}
    The solution set has one element.
    \begin{equation*}
      S=\set{\colvec{-1 \\ 2}}
    \end{equation*}
  \end{solution}
% sage: M = matrix(QQ, [[2,1], [1,-1], [5,1]])
% sage: v = vector(QQ, [0,-3,-3])
% sage: M_prime = M.augment(v, subdivide=True)
% sage: gauss_method(M_prime)
% [ 2  1| 0]
% [ 1 -1|-3]
% [ 5  1|-3]
%  take -1/2 times row 1 plus row 2
%  take -5/2 times row 1 plus row 3
% [   2    1|   0]
% [   0 -3/2|  -3]
% [   0 -3/2|  -3]
%  take -1 times row 2 plus row 3
% [   2    1|   0]
% [   0 -3/2|  -3]
% [   0    0|   0]
% sage: var('x,y,z')
% (x, y, z)
% sage: eqns = [2*x+y==0, x-y==-3, 5*x+y==-3]
% sage: solve(eqns, x, y)
% [[x == -1, y == 2]]

\part $
       \begin{linsys}{3}
         2x  &  &  &+  &z  &=  &3  \\
          x  &- &y &-  &z  &=  &1  \\
         3x  &- &y &   &   &=  &4  
       \end{linsys}
      $
  \begin{solution}[1.5in]
    \begin{equation*}
      \grstep[-(3/2)\rho_1+\rho_3]{-(1/2)\rho_1+\rho_2}
      \begin{amat}{3}
        2 &     &1    &3 \\
        0 &-1   &-3/2 &-1/2 \\
        0 &-1   &-3/2 &-1/2 
      \end{amat}
      \grstep{-\rho_2+\rho_3}
      \begin{amat}{3}
        2 &     &1    &3 \\
        0 &-1   &-3/2 &-1/2 \\
        0 &0    &0    &0 
      \end{amat}
    \end{equation*}
    The parametrization is this.
    \begin{equation*}
      S=\set{\colvec{x \\ y \\ z}=
             \colvec{3/2 \\ 1/2 \\ 0}+
             \colvec{-1/2 \\ -3/2 \\ 1}z\suchthat z\in\Re}
    \end{equation*}
  \end{solution}
% sage: M = matrix(QQ, [[2,0,1], [1,-1,-1], [3,-1,0]])
% sage: v = vector(QQ, [3,1,4])
% sage: M_prime = M.augment(v, subdivide=True)
% sage: gauss_method(M_prime)
% [ 2  0  1| 3]
% [ 1 -1 -1| 1]
% [ 3 -1  0| 4]
%  take -1/2 times row 1 plus row 2
%  take -3/2 times row 1 plus row 3
% [   2    0    1|   3]
% [   0   -1 -3/2|-1/2]
% [   0   -1 -3/2|-1/2]
%  take -1 times row 2 plus row 3
% [   2    0    1|   3]
% [   0   -1 -3/2|-1/2]
% [   0    0    0|   0]

\end{parts}

\question
  Consider this subset of $\Re^3$.
  \begin{equation*}
    T=\set{\colvec{x \\ y \\ z}\suchthat x+y+2z=0}
  \end{equation*}
  \begin{parts}
  \part
  List two elements.
  \begin{solution}[1in]
  Here are two vectors whose first coordinate plus second coordinate plus
  twice the third sums to zero.
  \begin{equation*}
    \colvec{1 \\ 1 \\ -1}
    \quad
    \colvec{0 \\ 0 \\ 0}
  \end{equation*}
  \end{solution}
  \part
  Show that $T$ is closed under vector addition.
  \begin{solution}[1in]
    Consider
    \begin{equation*}
    \vec{v}_1=\colvec{x_1 \\ y_1 \\ z_1}
    \quad
    \vec{v}_2=\colvec{x_2 \\ y_2 \\ z_2}
    \end{equation*}
    subject to the restrictions $x_1+y_1+2z_1=0$
    and $x_2+y_2+2z_2=0$.
    The sum $\vec{v}_1+\vec{v}_2$ is
    \begin{equation*}
    \vec{v}_1+\vec{v}_2
    =\colvec{x_1+x_2 \\ y_1+y_2 \\ z_1+z_2}
    \end{equation*}
    and note that
    $(x_1+x_2)+(y_1+y_2)+2(z_1+z_2)
    =(x_1+y_1+2z_1)+(x_2+y_2+2z_2)=0+0=0$.
  \end{solution}
  
  \part
    In fact, $T$ is a subspace.
    Find a basis.
    You must verify that the set you give is a basis.
    \begin{solution}[1.75in]
      The $x+y+2z=0$ restriction is
      a one-equation system.
      Parametrizing $x=-y-2z$
      gives this.
      \begin{equation*}
      M=\set{\colvec{x \\ y \\ z}
              =\colvec{-1 \\ 1 \\ 0}\cdot y+\colvec{-2 \\ 0 \\ 1}\cdot z
                 \suchthat y,z\in\Re}
        \tag{$*$}
      \end{equation*}
      We now claim that the set
      \begin{equation*}
        B=\set{\colvec{-1 \\ 1 \\ 0}, \colvec{-2 \\ 0 \\ 1}}
      \end{equation*}
      is a basis for the subspace~$T$.
      Equation~($*$) explicitly writes~$T$ 
      as the span of~$B$
      so we need only show that $B$ is linearly independent.
      The linear relationship
      \begin{equation*}
        \colvec{0 \\ 0 \\ 0}
        =\colvec{-1 \\ 1 \\ 0}\cdot c_1+ \colvec{-2 \\ 0 \\ 1}\cdot c_2
      \end{equation*}
      gives that $c_2=0$ by the equation of third components, and that
      $c_1=0$ by the equation of second components.
      So $B$ is linearly independent.
    \end{solution}

  % \part 
  %   What is the dimension of the subspace?
  %   \begin{solution}[.5in]
  %     The basis has two elements, so the subspace~$T$ has dimension~$2$.
  %   \end{solution}
  \end{parts}


\question
  Decide if these two matrices are row~equivalent.
  \begin{equation*}
    \begin{mat}
      2 &4  &2  \\
      1 &1  &1  \\
      0 &-1 &0
    \end{mat}\quad
    \begin{mat}
      0  &2 &0  \\
      -1 &1 &-1 \\
      2  &3  &2
    \end{mat}
  \end{equation*}
  \begin{solution}[2.5in]
    Perform Gauss-Jordan reduction on each and see if the outcomes
    are the same.
    The first is
    \begin{equation*}
    \begin{mat}
    2 &4  &2  \\
    1 &1  &1  \\
    0 &-1 &0
    \end{mat}
    \grstep{-(1/2)\rho_1+\rho_2}
    \grstep{-\rho_2+\rho_3}
    \grstep[-\rho_2]{(1/2)\rho_1}
    \grstep{-2\rho_2+\rho_1}
    \begin{mat}
    1 &0  &1  \\
    0 &1  &0  \\
    0 &0  &0
    \end{mat}
    \end{equation*}
    and the second is this.
    \begin{equation*}
    \begin{mat}
    0  &2  &0  \\
    -1 &1  &-1  \\
    2  &3  &2
    \end{mat}
    \grstep{\rho_1\leftrightarrow\rho_2}
    \grstep{2\rho_1+\rho_3}
    \grstep{(5/2)\rho_2+\rho_3}
    \grstep[(1/2)\rho_2]{-\rho_1}
    \grstep{\rho_2+\rho_1}
    \begin{mat}
    1 &0  &1  \\
    0 &1  &0  \\
    0 &0  &0
    \end{mat}
    \end{equation*}
    So they are row equivalent.
  \end{solution}
% sage: M = matrix(QQ, [[2,4,2], [1,1,1], [0,-1,0]])
% sage: gauss_jordan(M)
% [ 2  4  2]
% [ 1  1  1]
% [ 0 -1  0]
%  take -1/2 times row 1 plus row 2
% [ 2  4  2]
% [ 0 -1  0]
% [ 0 -1  0]
%  take -1 times row 2 plus row 3
% [ 2  4  2]
% [ 0 -1  0]
% [ 0  0  0]
%  take 1/2 times row 1
%  take -1 times row 2
% [1 2 1]
% [0 1 0]
% [0 0 0]
%  take -2 times row 2 plus row 1
% [1 0 1]
% [0 1 0]
% [0 0 0]
% sage: M = matrix(QQ, [[0,2,0], [-1,1,-1], [2,3,2]])
% sage: gauss_jordan(M)
% [ 0  2  0]
% [-1  1 -1]
% [ 2  3  2]
%  swap row 1 with row 2
% [-1  1 -1]
% [ 0  2  0]
% [ 2  3  2]
%  take 2 times row 1 plus row 3
% [-1  1 -1]
% [ 0  2  0]
% [ 0  5  0]
%  take -5/2 times row 2 plus row 3
% [-1  1 -1]
% [ 0  2  0]
% [ 0  0  0]
%  take -1 times row 1
%  take 1/2 times row 2
% [ 1 -1  1]
% [ 0  1  0]
% [ 0  0  0]
%  take 1 times row 2 plus row 1
% [1 0 1]
% [0 1 0]
% [0 0 0]

  
\question
  Decide if each is set linearly independent or dependent.
  If it is independent then verify that; if dependent then state a linear
  relation among the vectors.
  \begin{parts}
    \part $\set{\colvec{1 \\ -1 \\ 2}, \colvec{3 \\ -1 \\ 0}, \colvec{-3 \\ -1 \\ 6}}$
    \begin{solution}[1.5in]
      The linear system has infinitely many solutions.
      \begin{equation*}
      \begin{linsys}{3}
        c_1  &+ &3c_2 &- &3c_3 &= &0 \\
        -c_1 &- &c_2  &- &c_3  &= &0 \\
        2c_1 &  &     &+ &6c_3 &= &0 
        \end{linsys}
        \grstep[-2\rho_1+\rho_3]{\rho_1+\rho_2}
        \grstep{3\rho_2+\rho_3}
      \begin{linsys}{3}
        c_1  &+ &3c_2 &- &3c_3 &= &0 \\
             &  &2c_2 &- &4c_3 &= &0 \\
             &  &     &  &0    &= &0 
        \end{linsys}
      \end{equation*}
      So the set of three vectors is linearly dependent.
      For an example of a dependence, take $c_3=1$.
      That leads to $c_2=2$ and $c_1=-3$.
    \end{solution}

  
    \part $\set{x^2-1,x^2+1,x+1}$
    \begin{solution}[1.25in]
      The linear relationship
      \begin{equation*}
        0+0x+0x^2=c_1\cdot(x^2-1)+c_2(x^2+1)+c_3(x+1)
      \end{equation*}
      gives $c_3=0$ by consideration of the coefficients of~$x$,
      leaving $c_1+c_2=0$ and $-c_1+c_2=0$.
      Solving that system is easy; the only solution is the trivial one.
      So the starting set is linearly independent.
    \end{solution}

    \part $\set{\begin{mat}
                  1 &-2 \\
                  1 &0 \\
                  3 &3
                \end{mat},
                \begin{mat}
                  1 &1 \\
                  2 &-2 \\
                  4 &0
                \end{mat}
          }$
          \begin{solution}[2in]
            We can observe that the second is not a multiple of 
            the first,
            or we can establish the linear relation.
            \begin{equation*}
              \begin{mat}
                0 &0 \\ 
                0 &0 \\
                0 &0
              \end{mat}
              =
              c_1\begin{mat}
                  1 &-2 \\
                  1 &0 \\
                  3 &3
                \end{mat}
              +c_2\begin{mat}
                  1 &1 \\
                  2 &-2 \\
                  4 &0
                \end{mat}
            \end{equation*}
            The equation of $2,2$~entries shows that $c_2=0$.
            The $3,2$ entries show that $c_1=0$,
            The set is linearly independent.
          \end{solution}
  \end{parts}

\question
  Do Gauss-Jordan reduction on this matrix.
  \begin{equation*}
    \begin{mat}
      1 &0  &1 &0 \\  
      2 &-1 &1 &2 \\
      2 &0  &0 &1 
    \end{mat}
  \end{equation*}
  \begin{solution}[2in]
    \begin{equation*}
      \grstep[-2\rho_1+\rho_3]{-2\rho_1+\rho_2}
      \begin{mat}
        1 &0  &1  &0 \\  
        0 &-1 &-1 &2 \\
        0 &0  &-2 &1 
      \end{mat}
      \grstep[-(1/2)\rho_3]{-\rho_2}
      \begin{mat}
        1 &0  &1  &0 \\  
        0 &1  &1  &-2 \\
        0 &0  &1  &-1/2 
      \end{mat}
      \grstep[-\rho_3+\rho_2]{-\rho_3+\rho_1}  
      \begin{mat}
        1 &0  &0  &1/2 \\  
        0 &1  &0  &-3/2 \\
        0 &0  &1  &-1/2 
      \end{mat}
    \end{equation*}
% sage: M = matrix(QQ, [[1,0,1,0], [2,-1,1,2], [2,0,0,1]])
% sage: gauss_jordan(M)
% [ 1  0  1  0]
% [ 2 -1  1  2]
% [ 2  0  0  1]
%  take -2 times row 1 plus row 2
%  take -2 times row 1 plus row 3
% [ 1  0  1  0]
% [ 0 -1 -1  2]
% [ 0  0 -2  1]
%  take -1 times row 2
%  take -1/2 times row 3
% [   1    0    1    0]
% [   0    1    1   -2]
% [   0    0    1 -1/2]
%  take -1 times row 3 plus row 1
%  take -1 times row 3 plus row 2
% [   1    0    0  1/2]
% [   0    1    0 -3/2]
% [   0    0    1 -1/2]
  \end{solution}
\end{questions}
\end{document}
