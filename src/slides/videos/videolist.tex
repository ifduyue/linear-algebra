\documentclass{article}
\usepackage{fourier}
\usepackage[T1]{fontenc}
\usepackage[margin=1in,left=1.25in,right=1.25in]{geometry}

\usepackage[style=iso]{datetime2}

\usepackage{enumitem}

\usepackage{hyperref}
\hypersetup{
    bookmarks=false,         % show bookmarks bar?
    unicode=true,           % non-Latin characters in Acrobat’s bookmarks
    pdftoolbar=true,        % show Acrobat’s toolbar?
    pdfmenubar=true,        % show Acrobat’s menu?
    pdffitwindow=false,     % window fit to page when opened
    pdfstartview={FitH},    % fits the width of the page to the window
    pdftitle={Videos for Linear Algebra by Jim Hefferon},    % title
    pdfauthor={Jim Hefferon},     % author
    pdfsubject={Linear Algebra},   % subject of the document
    pdfcreator={LaTeX},   % creator of the document
    pdfproducer={Jim Hefferon}, % producer of the document
    pdfkeywords={linear algebra, free, text, video}, % list of keywords
    pdfnewwindow=true,      % links in new PDF window
    colorlinks=true,       % false: boxed links; true: colored links
    linkcolor=blue,          % color of internal links (change box color with linkbordercolor)
    citecolor=blue,        % color of links to bibliography
    filecolor=blue,         % color of file links
    urlcolor=blue        % color of external links
}

\pagestyle{empty}
\begin{document}\thispagestyle{empty}
\begin{center}
  \LARGE\textbf{Videos for \textit{Linear Algebra}} \\[1ex]
  \Large\textbf{by Jim Hef{}feron}\\[0.3ex]
  \Large\textbf{\today}\\
\end{center}
\vspace{3ex}
\begin{center}
\begin{tabular}{|rrc|ll|}
  \multicolumn{1}{c}{\textit{Chapter}} 
  &\multicolumn{1}{c}{\textit{Section}} 
  &\multicolumn{1}{c}{\textit{Subsection}} 
  &\multicolumn{1}{c}{\textit{Title}} 
  &\multicolumn{1}{c}{\textit{Videos}} 
  \\ \hline
      &    &  &Introduction          &\href{https://youtu.be/i-aS6VrDomg}{Full}   \\
 \hline
  One &I   &1 &Solving Linear Systems             &\href{https://youtu.be/4e6AsXVIsFc}{Part One}, \href{https://youtu.be/lhkdzfWfwp0}{Part Two}   \\
      &    &2 &Describing the Solution Set        &\href{https://youtu.be/ExVBc05YQPg}{Part One}, \href{https://youtu.be/ZatFu5GVFbo}{Part Two}   \\
      &    &3 &General = Particular + Homogeneous &\href{https://youtu.be/OcayHebo62Q}{Full}   \\
      \cline{4-5}
      &II  &1  &Vectors in Space                  &\href{https://youtu.be/5UpMGlhtVf4}{Full}   \\
      &    &2  &Length and Angle Measures         &\href{https://youtu.be/-vzdOIP-a5A}{Full}   \\
      \cline{4-5}
      &III &1  &Gauss-Jordan Elimination            &\href{https://youtu.be/Q0UGClfdgCg}{Full}   \\
      &    &2  &Linear Combination Lemma          &\href{https://youtu.be/QTfFmEUhQlE}{Full}   \\
 \hline
 Two  &I   &1 &Vector Space                       &\href{https://youtu.be/LIGgeCkcr6A}{Part One}, \href{https://youtu.be/UU6FSncoWJs}{Part Two}   \\
      &    &2 &Subspaces                          &\href{https://youtu.be/SHh7sp3Tedc}{Part One}, \href{https://youtu.be/rea8WbXAVVY}{Part Two}   \\
      \cline{4-5}
      &II  &1 &Linear Independence                &\href{https://youtu.be/x27ccxIPndU}{Part One}, \href{https://youtu.be/i7tUZioOYLI}{Part Two}     \\
      \cline{4-5}
      &III &1 &Basis                              &\href{https://youtu.be/4vlyOrrESN0}{Part One}, \href{https://youtu.be/0zXzOfCQeds}{Part Two}   \\
      &    &2 &Dimension                          &\href{https://youtu.be/vnYjwQ8VV3Y}{Full}   \\
      &    &3 &Vector Spaces and Linear Systems   &\href{https://youtu.be/pPFSLeJJrd8}{Full}    \\
 \hline
 Three&I   &1 &Isomorphism                        &\href{https://youtu.be/jztZkKNiujg}{Part One}, \href{https://youtu.be/iNXBRlpR4mQ}{Part Two}   \\
      &    &2 &Dimension Characterises Isomorphism&\href{https://youtu.be/dh3C6dt53Jk}{Full}   \\
      \cline{4-5}
      &II  &1 &Homomorphism                       &\href{https://youtu.be/tS2AJqwyvFo}{Part One}, \href{https://youtu.be/gm1KIYKKbpQ}{Part Two}   \\
      &    &2 &Range Space and Null Space         &\href{https://youtu.be/1wKqFvgXHpw}{Part One}, \href{https://youtu.be/g1RFaLRNVIg}{Part Two}   \\
      &    &Extra &Transformations of the Plane         &\href{https://youtu.be/_k2eVnSrgBA}{Full}   \\
      \cline{4-5}
      &III &1 &Representing Maps with Matrices    &\href{https://youtu.be/8nIED0fu8jQ}{Part One}, \href{https://youtu.be/9DTE11VIgqg}{Part Two}   \\
      &    &2 &Any Matrix Represents a Map        &\href{https://youtu.be/gc35uHn8PdI}{Full}   \\
      \cline{4-5}
      &IV  &1 &Matrix Addition and Scalar Product &\href{https://youtu.be/gc35uHn8PdI}{Full}   \\
      &    &2 &Matrix Multiplication              &\href{https://youtu.be/7w6KUniURgw}{Part One}, \href{https://youtu.be/aWubyx5bBn4}{Part Two}   \\
      &    &3 &Mechanics of Matrix Multiplication &\href{https://youtu.be/UMTZAI2Ybus}{Full}   \\
      &    &4 &Matrix Inverses                    &\href{https://youtu.be/RF9zrRNnWBY}{Part One}, \href{https://youtu.be/tOeh88b-fR0}{Part Two}   \\
      \cline{4-5}
      &V   &1 &Changing Vector Representations    &\href{https://youtu.be/29Jzh3MeDbc}{Full}   \\
      &    &2 &Changing Map Representations       &\href{https://youtu.be/zzPDz434TGs}{Part One}, \href{https://youtu.be/-voH_B21eXc}{Part Two}    \\
      \cline{4-5}
      &VI  &1, 2 &Orthogonal Projection  &\href{https://youtu.be/AAjZJjByU3U}{Full}     \\
  \hline
 Four &I   &1, 2 &Determinants, Properties      &\href{https://youtu.be/BnyledMW6Ho}{Full}   \\
      &    &3 &Permultation Expansion             &\href{https://youtu.be/j1dYL2_Ud7E}{Part One}, \href{https://youtu.be/kWF2TzA0dU8}{Part Two}   \\
      &    &4 &Determinants Exist (optional)                 &\href{https://youtu.be/FMcTv4g7Goc}{Full}   \\
      \cline{4-5}
      &II  &1 &Geometry of Determinants           &2   \\
      \cline{4-5}
      &III &1 &Laplace's Expansion                &1   \\
 \hline
 Five &I   &1 &Complex Vector Spaces              &1   \\
      \cline{4-5}
      &II  &1 &Similarity                         &2   \\
      &    &2 &Diagonalizability                  &2   \\
      &    &3 &Eigenvalues and Eigenvectors       &2   \\
 \hline
\end{tabular}
\end{center}
\end{document}
