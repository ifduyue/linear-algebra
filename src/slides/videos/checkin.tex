\documentclass{checkin}

\begin{document}

% =====================================
% Before: One.II Vectors in Space
\begin{frame}\frametitle{Solving linear systems}
\begin{questions}
\item These are in echelon form.
How many solutions does each have? 
\begin{align*}
&\begin{linsys}{3}
  x  &+  &2y  &-  &z  &=  &4 \\
     &   &3y  &+  &z  &=  &-1 \\
     &   &    &   &0  &=  &-1 
\end{linsys}
\qquad
\begin{linsys}{3}
  x  &\   &    &-  &z  &=  &0 \\
     &    &3y  &   &   &=  &0 \\
     &    &    &   &z  &=  &-1\\
     &    &    &   &0  &=  &0\\
\end{linsys}
                                    \\
&\begin{linsys}{3}
  x  &-  &y    &   &   &=  &0 \\
     &   & y  &+   &2z   &=  &0 \\
     &   &    &   &0  &=  &0\\ 
\end{linsys}
\qquad
\begin{linsys}{3}
  2x &+   &2y &-  &2z &=  &9 \\
     &   &y  &-   &z  &=  &5 \\
     &   &    &   &0  &=  &8\\ 
     &   &    &   &0  &=  &0\\ 
\end{linsys}
\end{align*}

\pause
\item Produce a three unknowns, three equations linear system that 
has infinitely many solutions.
Produce one with no solutions.
Produce one with one solution.
\end{questions}
\end{frame}


% Before: One.II Vectors in Space, part two
\begin{frame}\frametitle{Vectors in space}
\begin{questions}
\item Geometrically, what is a linear equation such as 
$x+2y+3z=4$ or $-x+2y-z=1$?

\pause
\item What are the possibilities for the intersection of two planes?

\pause
\item Three planes?
\end{questions}
\end{frame}




% =====================================
% Before: One.III Gauss-Jordan reduction
\begin{frame}\frametitle{Geometry of a solution set}
\begin{questions}
\item This is the solution set of a linear system.
Sketch it.
\begin{equation*}
  \set{\colvec{1 \\ 1 \\ 2}
       +\colvec{1 \\ 2 \\ -1}t
       +\colvec{0 \\ 2 \\ -1}s
       \suchthat t,s\in\R}
\end{equation*}
\pause
\item Find the particular solution vector for
$t=-1$ and $s=1$.
Sketch it.
\pause
\item 
Sketch this.
\begin{equation*}
  \set{\colvec{0 \\ 1 \\ 2}
       +\colvec{1 \\ 2 \\ -1}t
       +\colvec{0 \\ 2 \\ -1}s
       \suchthat t,s\in\R}
\end{equation*}
\end{questions}
\end{frame}


\begin{frame}\frametitle{Row equivalent matrices}
\begin{questions}
\item Produce two matrices that are row equivalent.
\item Produce two that are not row equivalent.
\end{questions}
\end{frame}






% =====================================
% Before: Two.I.1 Vector spaces, 
\begin{frame}\frametitle{Linear combinations}
\begin{questions}
\item This equation involves a linear combination of vectors.
Solve it for $x$, $y$, and $z$.
\begin{equation*}
   \colvec{1 \\ 2 \\ 0}\cdot x
       +\colvec{1 \\ 1 \\ -1}\cdot y
       +\colvec{0 \\ 3 \\ 0}\cdot z
       =\colvec{0 \\ 0 \\ 0}
\end{equation*}
\pause
\item This involves a linear combination of matrices.
Solve it.
\begin{equation*}
   x\cdot\begin{mat}
     2  &1  \\
     0  &-2
   \end{mat}
   +y\cdot\begin{mat}
     1    &-1  \\
     1/2  &0
   \end{mat}
   +z\cdot\begin{mat}
     0    &0  \\
     -1  &1
   \end{mat}
  =
  \begin{mat}
   0  &0 \\
   0  &0
  \end{mat}
\end{equation*}
\end{questions}
\end{frame}


\begin{frame}\frametitle{Closure under linear combinations}
\begin{questions}
\item Consider this line through the origin in the plane.
Pick two generic vectors and make an example linear combination. 
\begin{equation*}
  \set{\colvec{x \\ y} \suchthat y=2x}
  =\set{\colvec{x \\ y} \suchthat 2x-y=0}
\end{equation*}
Check that the result is a member of the line.

\pause
\item This line does not go through the origin.
Show that it is not closed under linear combinations. 
\begin{equation*}
  \set{\colvec{x \\ y} \suchthat 2x-y=1}
\end{equation*}
\end{questions}
\end{frame}







% =====================================
% Before: Two.I.2 Subspaces, 
\begin{frame}\frametitle{Subsets}
\begin{questions}
\item This is a plane through the origin in $\R^3$.
Pick two generic vectors and make an example linear combination.
\begin{equation*}
  \set{\colvec{x \\ y \\ z}
       \suchthat x+2y+3z=0}
\end{equation*}
Check that the result is a member of the plane.

\item This is a line through the origin in $\R^3$,
which is a subset of the plane.
Pick two generic vectors and make an example linear combination.
\begin{equation*}
  \set{\colvec{x \\ y \\ z}
       \suchthat \text{$x+2y=0$ and $z=0$}}
\end{equation*}
Check that the result is a member of the line, and of the plane.
\end{questions}
\end{frame}

\end{document}
