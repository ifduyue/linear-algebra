\documentclass{checkin}

\begin{document}

% =====================================
% Before: One.II Vectors in Space
\begin{frame}\frametitle{Solving linear systems}
\begin{questions}
\item These are in echelon form.
How many solutions does each have? 
\begin{align*}
&\begin{linsys}{3}
  x  &+  &2y  &-  &z  &=  &4 \\
     &   &3y  &+  &z  &=  &-1 \\
     &   &    &   &0  &=  &-1 
\end{linsys}
\qquad
\begin{linsys}{3}
  x  &\   &    &-  &z  &=  &0 \\
     &    &3y  &   &   &=  &0 \\
     &    &    &   &z  &=  &-1\\
     &    &    &   &0  &=  &0\\
\end{linsys}
                                    \\
&\begin{linsys}{3}
  x  &-  &y    &   &   &=  &0 \\
     &   & y  &+   &2z   &=  &0 \\
     &   &    &   &0  &=  &0\\ 
\end{linsys}
\qquad
\begin{linsys}{3}
  2x &+   &2y &-  &2z &=  &9 \\
     &   &y  &-   &z  &=  &5 \\
     &   &    &   &0  &=  &8\\ 
     &   &    &   &0  &=  &0\\ 
\end{linsys}
\end{align*}

\pause
\item Produce a three unknowns, three equations linear system that 
has infinitely many solutions.
Produce one with no solutions.
Produce one with one solution.
\end{questions}
\end{frame}


% Before: One.II Vectors in Space, part two
\begin{frame}\frametitle{Vectors in space}
\begin{questions}
\item Geometrically, what is a linear equation such as 
$x+2y+3z=4$ or $-x+2y-z=1$?

\pause
\item What are the possibilities for the intersection of two planes?

\pause
\item Three planes?
\end{questions}
\end{frame}





\end{document}
