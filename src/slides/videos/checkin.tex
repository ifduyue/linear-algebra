\documentclass{checkin}

\begin{document}

% =====================================
% Before: One.II Vectors in Space
\begin{frame}\frametitle{Solving linear systems}
\begin{questions}
\item These are in echelon form.
How many solutions does each have? 
\begin{align*}
&\begin{linsys}{3}
  x  &+  &2y  &-  &z  &=  &4 \\
     &   &3y  &+  &z  &=  &-1 \\
     &   &    &   &0  &=  &-1 
\end{linsys}
\qquad
\begin{linsys}{3}
  x  &\   &    &-  &z  &=  &0 \\
     &    &3y  &   &   &=  &0 \\
     &    &    &   &z  &=  &-1\\
     &    &    &   &0  &=  &0\\
\end{linsys}
                                    \\
&\begin{linsys}{3}
  x  &-  &y    &   &   &=  &0 \\
     &   & y  &+   &2z   &=  &0 \\
     &   &    &   &0  &=  &0\\ 
\end{linsys}
\qquad
\begin{linsys}{3}
  2x &+   &2y &-  &2z &=  &9 \\
     &   &y  &-   &z  &=  &5 \\
     &   &    &   &0  &=  &8\\ 
     &   &    &   &0  &=  &0\\ 
\end{linsys}
\end{align*}

\pause
\item Produce a three unknowns, three equations linear system that 
has infinitely many solutions.
Produce one with no solutions.
Produce one with one solution.
\end{questions}
\end{frame}


% Before: One.II Vectors in Space, part two
\begin{frame}\frametitle{Vectors in space}
\begin{questions}
\item Geometrically, what is a linear equation such as 
$x+2y+3z=4$ or $-x+2y-z=1$?

\pause
\item What are the possibilities for the intersection of two planes?

\pause
\item Three planes?
\end{questions}
\end{frame}




% =====================================
% Before: One.III Gauss-Jordan reduction
\begin{frame}\frametitle{Geometry of a solution set}
\begin{questions}
\item This is the solution set of a linear system.
Sketch it.
\begin{equation*}
  \set{\colvec{1 \\ 1 \\ 2}
       +\colvec{1 \\ 2 \\ -1}t
       +\colvec{0 \\ 2 \\ -1}s
       \suchthat t,s\in\R}
\end{equation*}
\pause
\item Find the particular solution vector for
$t=-1$ and $s=1$.
Sketch it.
\pause
\item 
Sketch this.
\begin{equation*}
  \set{\colvec{0 \\ 1 \\ 2}
       +\colvec{1 \\ 2 \\ -1}t
       +\colvec{0 \\ 2 \\ -1}s
       \suchthat t,s\in\R}
\end{equation*}
\end{questions}
\end{frame}


\begin{frame}\frametitle{Row equivalent matrices}
\begin{questions}
\item Produce two matrices that are row equivalent.
\item Produce two that are not row equivalent.
\end{questions}
\end{frame}






% =====================================
% Before: Two.I.1 Vector spaces, 
\begin{frame}\frametitle{Linear combinations}
\begin{questions}
\item This equation involves a linear combination of vectors.
Solve it for $x$, $y$, and $z$.
\begin{equation*}
   \colvec{1 \\ 2 \\ 0}\cdot x
       +\colvec{1 \\ 1 \\ -1}\cdot y
       +\colvec{0 \\ 3 \\ 0}\cdot z
       =\colvec{0 \\ 0 \\ 0}
\end{equation*}
\pause
\item This involves a linear combination of matrices.
Solve it.
\begin{equation*}
   x\cdot\begin{mat}
     2  &1  \\
     0  &-2
   \end{mat}
   +y\cdot\begin{mat}
     1    &-1  \\
     1/2  &0
   \end{mat}
   +z\cdot\begin{mat}
     0    &0  \\
     -1  &1
   \end{mat}
  =
  \begin{mat}
   0  &0 \\
   0  &0
  \end{mat}
\end{equation*}
\end{questions}
\end{frame}


\begin{frame}\frametitle{Closure under linear combinations}
\begin{questions}
\item Consider this line through the origin in the plane.
Pick two generic vectors and make an example linear combination. 
\begin{equation*}
  \set{\colvec{x \\ y} \suchthat y=2x}
  =\set{\colvec{x \\ y} \suchthat 2x-y=0}
\end{equation*}
Check that the result is a member of the line.

\pause
\item This line does not go through the origin.
Show that it is not closed under linear combinations. 
\begin{equation*}
  \set{\colvec{x \\ y} \suchthat 2x-y=1}
\end{equation*}
\end{questions}
\end{frame}





% =====================================
% Before: Two.I.2 Subspaces, 
\begin{frame}\frametitle{Subsets}
\begin{questions}
\item This is a plane through the origin in $\R^3$.
Pick two generic vectors and make an example linear combination.
\begin{equation*}
  \set{\colvec{x \\ y \\ z}
       \suchthat x+2y+3z=0}
\end{equation*}
Check that the result is a member of the plane.

\item This is a line through the origin in $\R^3$,
which is a subset of the plane.
Pick two generic vectors and make an example linear combination.
\begin{equation*}
  \set{\colvec{x \\ y \\ z}
       \suchthat \text{$x+2y=0$ and $z=0$}}
\end{equation*}
Check that the result is a member of the line, and of the plane.
\end{questions}
\end{frame}


\begin{frame}\frametitle{Parametrization}
\begin{questions}
\item This is a subset of the space  $\polyspace_2$ 
of quadratic polynomials.
Name three members and three non-members.
Parametrize it.
\begin{equation*}
  \set{a+bx+cx^2
       \suchthat b+2c=0}
\end{equation*}

\item 
Parametrize this.
\begin{equation*}
  \set{a+bx+cx^2
       \suchthat a-b-c=0}
\end{equation*}
Check that it is a subspace.
\end{questions}
\end{frame}



% =====================================
% Before: Two.II Linear Independence 
\begin{frame}\frametitle{Spanning sets}
\begin{questions}
\item Pick three generic members $\vec{v}_1$, $\vec{v}_2$, and~$\vec{v}_3$ 
of this subspace
$\set{a+bx+cx^2
       \suchthat b+2c=0}\subset\polyspace_2$.
Express them as linear combinations of the members of this spanning set.
\begin{equation*}
  S=\set{1, -2x+x^2,3-4x+2x^2}
\end{equation*}

\item Observe that the third vector in~$S$ is a combination of the
other two: $3-4x+2x^2=3\cdot(1)-2\cdot(-2x+x^2)$.
Use that to express  $\vec{v}_1$, $\vec{v}_2$, and~$\vec{v}_3$ as 
combinations of just the first two members of~$S$.
\end{questions}
\end{frame}


\begin{frame}\frametitle{Linearly independent and dependent sets}
\begin{questions}
\item Show that the first is linearly dependent while the second is
linearly independent.
\begin{equation*}
  \set{\colvec{1 \\ 3 \\ 0},
       \colvec{1 \\ 2 \\ 0},
       \colvec{1 \\ -1 \\ 0}}
  \qquad
  \set{\colvec{1 \\ 3 \\ 0},
       \colvec{1 \\ 2 \\ 0},
       \colvec{1 \\ -1 \\ 1}}
\end{equation*}
\end{questions}
\end{frame}




% =====================================
% Before: Two.III.1 Basis 
\begin{frame}\frametitle{Linear independence and spanning sets}
\begin{questions}
\item Find a spanning set for this plane.
\begin{equation*}
  \set{\colvec{x \\ y \\ z}\in\R^3
       \suchthat x+2y-z=0}
\end{equation*}

\item Show that this spanning set is linearly independent.

\item If we add this vector to the spanning set, is the result 
linearly independent or linearly dependent?
\begin{equation*}
  \colvec{1 \\ 1 \\ 2}
\end{equation*}
\end{questions}
\end{frame}



\begin{frame}\frametitle{Basis}
\begin{questions}
\item Show that this is a basis for the plane.
\begin{equation*}
  \sequence{\colvec{1 \\ 1},
       \colvec{1 \\ -1}}
\end{equation*}

\item What linear combination of the basis elements makes this
plane vector?
\begin{equation*}
  \colvec{3 \\ -2}
\end{equation*}

\item Show this is a basis 
$\sequence{2-x,3+2x}\subseteq\polyspace_1$.
What combination gives $-1-x$?
\end{questions}
\end{frame}



% =====================================
% Before: Two.III.2 Dimension

\begin{frame}\frametitle{Representations}
\begin{questions}
\item This is a basis for the vector space of quadratic
polynomials, $\polyspace_2$. 
Represent $\vec{v}=1-2x-2x^2$
with respect to it. 
\begin{equation*}
  B_1=\sequence{1,1+x,1+x^2}
\end{equation*}

\item Represent $\vec{v}$ with respect to this basis.
\begin{equation*}
  B_2=\sequence{1+x+x^2,1+x,1}
\end{equation*}
\end{questions}
\end{frame}



\begin{frame}\frametitle{Dimension}
Find the dimension of each space.
\begin{questions}
\item $\set{a+bx+cx^2\suchthat a-2b=0}$
\item
  $\set{\begin{mat}
          a  &b  \\
          c  &d
        \end{mat} \suchthat \text{$a+2d=0$ and $b+c=0$}}$ 
\end{questions}
\end{frame}



% =====================================
% Before: Three.I.1 Isomorphism

\begin{frame}\frametitle{Row space and column space}
In this matrix the first row plus twice the third equals the second.
\begin{equation*}
  \begin{mat}
    1  &0  &4  \\
   -1  &4  &0  \\
   -1  &2  &-2  \\
  \end{mat}
\end{equation*}
\begin{questions}
\item Find a basis for the row space.
Give the row rank. 
\item Find a basis for the column space.
Give the column rank. 
\end{questions}
\end{frame}


\begin{frame}\frametitle{Isomorphism}
Show that $\R^2$ and $\polyspace_1$ are isomorphic via this function.
\begin{equation*} 
  \colvec{a \\ b}
  \mapsunder{f}
  (a-b)+(a+b)x
\end{equation*}
\end{frame}


\begin{frame}\frametitle{Automorphisms}
Sketch these two vectors in the plane, along with the 
parallelogram diagram giving their sum.
\begin{equation*}
  \vec{u}=\colvec{1  \\ 2}
  \quad
  \vec{v}=\colvec{3  \\ 1}
\end{equation*}
\begin{questions}
\item Where $d_2$ is the plane automorphism that dilates by a 
factor of~$2$, sketch $d_2(\vec{u})$, $d_2(\vec{v})$, and
the parallelogram diagram giving their sum. 
\item Where $t_{\pi/6}$ is the plane automorphism that rotates by
$\pi/6$ radians, sketch $t_{\pi/6}(\vec{u})$, $t_{\pi/6}(\vec{v})$, and
the parallelogram diagram giving their sum. 
\item Where $f_{\ell}$ is the plane automorphism that flips about the
line $y=x$, sketch $f_{\ell}(\vec{u})$, $f_{\ell}(\vec{v})$, and
the parallelogram diagram giving their sum. 
\end{questions}
\end{frame}


\begin{frame}\frametitle{Representation is an isomorphism}
Consider the vector spaces $\polyspace_2$ and $\R^3$.
Fix these bases.
\begin{equation*}
  B=\sequence{x^2,x^2+x,x^2+1}
  \quad
  D=\stdbasis_3=\sequence{\colvec{1 \\ 0 \\ 0},
                          \colvec{0 \\ 1 \\ 0},
                          \colvec{0 \\ 0 \\ 1} }
\end{equation*}
\begin{questions}
\item The representation map $\map{\rep{}{B}}{\polyspace_2}{\R^3}$
 associates the quadratic polynomial $-1-2+4x^2$ with what three-tall vector?
\item What quadratic polynomial is associated with this?
\begin{equation*}
  \vec{w}=\colvec{-1 \\ -1 \\ 2}
\end{equation*}
\end{questions}
\end{frame}






% =====================================
% Before: Three.II.1 Homomorphism

\begin{frame}\frametitle{Homomorphism}
Verify that each is a linear map.
\begin{questions}
\item $\map{h}{\polyspace_3}{\R^2}$
\begin{equation*}
  a+bx+cx^2+dx^3 \mapsto \colvec{a+d \\ c-2a}
\end{equation*}

\item $\map{g}{\matspace_{\nbyn{2}}}{\matspace_{\nbyn{2}}}$
\begin{equation*}
\begin{mat}
  a &b  \\
  c &d
\end{mat}
\mapsto
\begin{mat}
  a  &c \\
  b  &d
\end{mat}
\end{equation*}  
\end{questions}
\end{frame}

 

\begin{frame}\frametitle{Inverse images of range elements}
Consider this $\map{h}{\R^3}{\R}$.
\begin{equation*} 
  \colvec{x \\ y \\ z}\mapsto x-y-2z
\end{equation*}
\begin{questions}
\item Find $h^{-1}(1)$.
\item Find $h^{-1}(-1)$.
\item Find $h^{-1}(0)$.
\item Pick a vector from $h^{-1}(1)$
and a vector from $h^{-1}(-1)$. 
Verify that they add to a vector from $h^{-1}(0)$.
\end{questions}
\end{frame}

 

\begin{frame}\frametitle{Nullity plus rank is the dimension of the domain}
Consider $\map{g}{\R^3}{\polyspace_2}$ and $\map{h}{\R}{\R^3}$.
\begin{equation*} 
  \colvec{a \\ b \\ c}\mapsunder{g}0+(a+2c)\cdot x+0x^2
  \qquad
  a\mapsunder{h}\colvec{a \\ 3a \\ 0}
\end{equation*}
\begin{questions}
\item Find each null space.
\item Find each range space.
\end{questions}
\end{frame}
 







% =====================================
% Before: Three.III  Computing Linear Maps


\begin{frame}\frametitle{Null space}
Consider the derivative map $\map{d/dx}{\polyspace_3}{\polyspace_2}$.
\begin{equation*} 
  \frac{d}{dx}(\,a+bx+cx^2+dx^3\,)=b+2c\cdot x+3d\cdot x^2
\end{equation*}
\begin{questions}
\item Find its null space.
\item Find its nullity.
\item Find its rank.
\end{questions}
\end{frame}


\begin{frame}\frametitle{Computing a linear map}
Consider this map $\map{h}{\R^2}{\matspace_{\nbyn{2}}}$
\begin{equation*} 
  \colvec{x \\ y}
  \mapsto
  \begin{mat}
    3x  &0 \\
    0   &2x-y
  \end{mat}
\end{equation*}
and bases.
\begin{equation*}
  B=\sequence{\colvec{1 \\ 1},
              \colvec{1 \\ -1}}
  \quad
  D=\sequence{
    \begin{mat}
      1  &0  \\
      0  &0
    \end{mat},
    \begin{mat}
      0  &2  \\
      0  &0
    \end{mat},
    \begin{mat}
      0  &0  \\
      3  &0
    \end{mat},
    \begin{mat}
      0  &0  \\
      0  &4
    \end{mat}
   }
\end{equation*}
\begin{questions}
\item Represent the map with respect to the bases.
\item Represent $\vec{v}=\colvec{5 \\ 3}$.
\item Using those, compute the effect of the map on the vector
\end{questions}
\end{frame}
 



\begin{frame}\frametitle{Matrix-vector multiplication}
Perform, if possible, the matrix vector multiplication.
\begin{questions}
\item 
$
\begin{mat}
  1  &3  \\
  -1 &5
\end{mat}\colvec{1 \\ 2}
$
\item 
$
\begin{mat}
  1  &3  \\
  -1 &5
\end{mat}\colvec{1 \\ 2  \\ 0}
$
\item 
$
\begin{mat}
  1  &3  \\
  -1 &5
\end{mat}\colvec{1}
$
\end{questions}
\end{frame}






% =====================================
% Before: Three.IV Matrix Operations


\begin{frame}\frametitle{Computing map facts}
Consider the map $\map{f}{\R^2}{\R^2}$ represented by
\begin{equation*} 
  \rep{f}{B,D}= 
  \begin{mat}
    2  &1 \\
    -1 &3
  \end{mat}
\end{equation*}
with respect to these bases.
\begin{equation*}
  B=\sequence{
      \colvec{1 \\ 1},
      \colvec{1  \\ -1}
     }
  \quad
  D=\sequence{
      \colvec{-1 \\ 0},
      \colvec{2  \\ 1}
     }
\end{equation*}
\begin{questions}
\item Find its null space.
\item Find its range space.
\end{questions}
\end{frame}



\begin{frame}\frametitle{Matrix addition and scalar multiplication}
Consider $\map{h,g}{\polyspace_2}{\R^2}$
\begin{equation*} 
  h(a+bx+cx^2)=\colvec{b \\ a+c}
  \qquad
  g(a+bx+cx^2)=\colvec{a+b \\ -2c}
\end{equation*}
and these bases.
\begin{equation*}
  B=\sequence{1, 1+x, 1+x^2}
  \qquad
  D=\stdbasis_2=\sequence{
      \colvec{1 \\ 0},
      \colvec{0  \\ 1}
     }
\end{equation*}
\begin{questions}
\item Let $\vec{v}=3+2x+x^2$.
  Find the action on $\vec{v}$ of the functions $h$, $g$, $4h$, 
and $h+g$.
\item Represent $h$ with respect to $B,D$.
\item Represent $g$ with respect to $B,D$.
\item Also represent $4h$ and $h+g$.
\end{questions}
\end{frame}



\begin{frame}\frametitle{Matrix multiplication}
Consider $\map{h,g}{\polyspace_2}{\polyspace_2}$
given by $h(a+bx+cx^2)=(a+c)+2b\cdot x +c\cdot x^2$
and
$g(a+bx+cx^2)=b-b\cdot x+(a-3c)\cdot x^2$,
and these bases.
\begin{equation*}
  B=D=\sequence{1, 1+x, 1+x^2}
\end{equation*}
\begin{questions}
\item Let $\vec{v}=-2+3x-5x^2$.  Find 
  $h(\vec{v})$, $g(\vec{v})$,
  $\composed{g}{h}(\vec{v})$,
  and
  $\composed{h}{g}(\vec{v})$.
\item Find $\rep{\vec{v}}{B}$
  Also find $H=\rep{h}{B,B}$ and $G=\rep{g}{B,B}$.
\item Find the matrix products $GH$ and $HG$.
\item
  Use them to compute 
  $\composed{g}{h}(\vec{v})$
  and
  $\composed{h}{g}(\vec{v})$.
\end{questions}
\end{frame}



\begin{frame}\frametitle{Matrix multiplication}
Fix this matrix.
\begin{equation*}
  G=
  \begin{mat}
    1 &3 \\
    2 &4
  \end{mat}
\end{equation*}
\begin{questions}
\item Where
\begin{equation*}
  \vec{v}_1=\colvec{1 \\ -3}
  \quad
  \vec{v}_2=\colvec{4 \\ 0}
  \quad
  \vec{v}_3=\colvec{-2 \\ -2}
\end{equation*}
find the matrix-vector products $G\vec{v}_1$, 
 $G\vec{v}_2$, and~ $G\vec{v}_3$.
\item Compute this matrix-matrix product.
\begin{equation*}
  G
  \begin{mat}
  1  &4  &-2 \\
 -3  &0  &-2
  \end{mat}
\end{equation*}
\end{questions}
\end{frame}



\begin{frame}\frametitle{Mechanics of matrix multiplication}
Where
\begin{equation*}
  G=
  \begin{mat}
    1 &3 \\
    2 &4
  \end{mat}
\end{equation*}
find a matrix
\begin{equation*}
  H=
  \begin{mat}
    a  &b  \\
    c  &d
  \end{mat}
\end{equation*}
so that this holds.
\begin{equation*}
  GH=
  \begin{mat}
    1 &3 \\
    2 &4
  \end{mat}
  \begin{mat}
    a  &b  \\
    c  &d
  \end{mat}
  =
  \begin{mat}
    1  &0 \\
    0  &1
  \end{mat}
  =I
\end{equation*}
\end{frame}



\begin{frame}\frametitle{Inverses}
Find the function~$g$ inverse to $\map{f}{\R^2}{\R^2}$.
\begin{equation*}
  \colvec{x \\ y}
  \mapsunder{f}
  \colvec{x+3y \\ 2x+4y}
\end{equation*}
\end{frame}


















\end{document}
