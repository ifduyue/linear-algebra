% titlescreen.tex
%  Make the splash screens for the videos, one at the start and one at the
% end.  The list of videos is in videolist.pdf.

\documentclass{titlescreen}

\begin{document}

% Introduction
\begin{videotitle}
  Introduction
\end{videotitle}

\begin{videoend}
  Solving Linear Systems \\[1ex]
  Part One
\end{videoend}

% Videos for slides: one_i
\begin{videotitle}
  Solving Linear Systems \\[1ex]
  Part One
\end{videotitle}

\begin{videoend}
  Solving Linear Systems \\[1ex]
  Part Two
\end{videoend}

\begin{videotitle}
  Solving Linear Systems \\[1ex]
  Part Two
\end{videotitle}
\begin{videoend}
  Describing the Solution Set \\[1ex]
  Part One
\end{videoend}


% One.I.2
\begin{videotitle}
  Describing the Solution Set \\[1ex]
  Part One
\end{videotitle}
\begin{videoend}
  Describing the Solution Set \\[1ex]
  Part Two
\end{videoend}

\begin{videotitle}
  Describing the Solution Set \\[1ex]
  Part Two
\end{videotitle}
\begin{videoend}
  General = Particular + Homogeneous
\end{videoend}


% One.I.3
\begin{videotitle}
  General = Particular + Homogeneous
\end{videotitle}
\begin{videoend}
  Vectors in Space
\end{videoend}


% =================================================
% Videos for slides: one_ii
% One.II.1
\begin{videotitle}
  Vectors in Space
\end{videotitle}
\begin{videoend}
  Length and Angle Measures 
\end{videoend}

% One.II.2
\begin{videotitle}
  Length and Angle Measures 
\end{videotitle}
\begin{videoend}
  Gauss-Jordan Reduction 
\end{videoend}


% =================================================
% Videos for slides: one_iii
% One.III.1
\begin{videotitle}
  Gauss-Jordan Reduction 
\end{videotitle}
\begin{videoend}
  Linear Combination Lemma 
\end{videoend}

% One.III.2
\begin{videotitle}
  Linear Combination Lemma 
\end{videotitle}
\begin{videoend}
  Vector Spaces
\end{videoend}


% ============= Chapter Two ================

% Two.I.1
\begin{videotitle}
  Vector Spaces  \\[1ex]
  Part One
\end{videotitle}
\begin{videoend}
  Vector Spaces \\[1ex]
  Part Two 
\end{videoend}

% 
\begin{videotitle}
  Vector Spaces  \\[1ex]
  Part Two
\end{videotitle}
\begin{videoend}
  Subspaces \\[1ex]
  Part One
\end{videoend}

% Two.I.2
\begin{videotitle}
  Subspaces  \\[1ex]
  Part One
\end{videotitle}
\begin{videoend}
  Subspaces \\[1ex]
  Part Two 
\end{videoend}

% 
\begin{videotitle}
  Subspaces  \\[1ex]
  Part Two
\end{videotitle}
\begin{videoend}
  Linear Independence \\[1ex]
  Part One
\end{videoend}

% Two.II.1
\begin{videotitle}
  Linear Independence  \\[1ex]
  Part One
\end{videotitle}
\begin{videoend}
  Linear Independence \\[1ex]
  Part Two 
\end{videoend}

% 
\begin{videotitle}
  Linear Independence  \\[1ex]
  Part Two
\end{videotitle}
\begin{videoend}
  Basis \\[1ex]
  Part One
\end{videoend}


% Two.III.1
\begin{videotitle}
  Basis  \\[1ex]
  Part One
\end{videotitle}
\begin{videoend}
  Basis \\[1ex]
  Part Two 
\end{videoend}

% 
\begin{videotitle}
  Basis  \\[1ex]
  Part Two
\end{videotitle}
\begin{videoend}
  Dimension
\end{videoend}


% Two.III.2
\begin{videotitle}
  Dimension  
\end{videotitle}
\begin{videoend}
  Vector Spaces and \\[0.5ex]
  Linear Systems
\end{videoend}

 % Two.III.3
\begin{videotitle}
  Vector Spaces and \\[0.5ex]
  Linear Systems 
\end{videotitle}
\begin{videoend}
  Isomorphism \\[1ex]
  Part One
\end{videoend}




% ============= Chapter Three ================

% Three.I.1
\begin{videotitle}
  Isomorphism  \\[1ex]
  Part One
\end{videotitle}
\begin{videoend}
  Isomorphism \\[1ex]
  Part Two 
\end{videoend}

% 
\begin{videotitle}
  Isomorphism  \\[1ex]
  Part Two
\end{videotitle}
\begin{videoend}
  Dimension Characterizes \\[0.5ex]
  Isomorphism
\end{videoend}

% Three.I.2
\begin{videotitle}
  Dimension Characterizes \\[0.5ex]
  Isomorphism
\end{videotitle}
\begin{videoend}
  Homomorphism  \\[1ex]
  Part One
\end{videoend}

%  Three.II.1
\begin{videotitle}
  Homomorphism  \\[1ex]
  Part One
\end{videotitle}
\begin{videoend}
  Homomorphism  \\[1ex]
  Part Two
\end{videoend}

%
\begin{videotitle}
  Homomorphism  \\[1ex]
  Part Two
\end{videotitle}
\begin{videoend}
  Range Space  \\[0.5ex]
  and Null Space  \\[1ex]
  Part One
\end{videoend}

%  Three.II.2
\begin{videotitle}
  Range Space  \\[0.5ex]
  and Null Space  \\[1ex]
  Part One
\end{videotitle}
\begin{videoend}
  Range Space  \\[0.5ex]
  and Null Space  \\[1ex]
  Part Two
\end{videoend}

%
\begin{videotitle}
  Range Space  \\[0.5ex]
  and Null Space  \\[1ex]
  Part Two
\end{videotitle}
\begin{videoend}
  Transformations of the Plane
\end{videoend}

%
\begin{videotitle}
  Transformations of the Plane 
\end{videotitle}
\begin{videoend}
  Representing Maps  \\[0.5ex]
  with Matrices  \\[1ex]
  Part One
\end{videoend}

%  Three.III.1
\begin{videotitle}
  Representing Maps  \\[0.5ex]
  with Matrices  \\[1ex]
  Part One
\end{videotitle}
\begin{videoend}
  Representing Maps  \\[0.5ex]
  with Matrices  \\[1ex]
  Part Two
\end{videoend}

%
\begin{videotitle}
  Representing Maps  \\[0.5ex]
  with Matrices  \\[1ex]
  Part Two
\end{videotitle}
\begin{videoend}
  Any Matrix \\[0.5ex]
  Represents a Map
\end{videoend}

%
\begin{videotitle}
  Any Matrix \\[0.5ex]
  Represents a Map
\end{videotitle}
\begin{videoend}
  Matrix Addition  \\[0.5ex]
  and Scalar Product 
\end{videoend}

%  Three.IV.1
\begin{videotitle}
  Matrix Addition  \\[0.5ex]
  and Scalar Product 
\end{videotitle}
\begin{videoend}
  Matrix Multiplication  \\[1ex]
  Part One
\end{videoend}

\begin{videotitle}
  Matrix Multiplication  \\[1ex]
  Part One
\end{videotitle}
\begin{videoend}
  Matrix Multiplication  \\[1ex]
  Part Two
\end{videoend}

\begin{videotitle}
  Matrix Multiplication  \\[1ex]
  Part Two
\end{videotitle}
\begin{videoend}
  Mechanics of \\[0.5ex]
  Matrix Multiplication
\end{videoend}

\begin{videotitle}
  Mechanics of \\[0.5ex]
  Matrix Multiplication 
\end{videotitle}
\begin{videoend}
  Matrix Inverses  \\[1ex]
  Part One
\end{videoend}

\begin{videotitle}
  Matrix Inverses  \\[1ex]
  Part One
\end{videotitle}
\begin{videoend}
  Matrix Inverses  \\[1ex]
  Part Two
\end{videoend}

\begin{videotitle}
  Matrix Inverses  \\[1ex]
  Part Two
\end{videotitle}
\begin{videoend}
  Matrix Inverses  \\[1ex]
  Part Two
\end{videoend}


%  Three.V.1
\begin{videotitle}
  Changing Vector Representations
\end{videotitle}
\begin{videoend}
  Changing Map Representations \\[1ex]
  Part One
\end{videoend}

\begin{videotitle}
  Changing Map Representations \\[1ex]
  Part One
\end{videotitle}
\begin{videoend}
  Changing Map Representations \\[1ex]
  Part Two
\end{videoend}

\begin{videotitle}
  Changing Map Representations \\[1ex]
  Part Two
\end{videotitle}
\begin{videoend}
  Orthogonal Projection  
\end{videoend}


% Three.VI.1
\begin{videotitle}
  Orthogonal Projection
\end{videotitle}
\begin{videoend}
  Determinants
\end{videoend}




% ============= Chapter Four ================

% Four.I.1
\begin{videotitle}
  Determinants  
\end{videotitle}
\begin{videoend}
  Properties of Determinants \\[1ex]
  Part One
\end{videoend}

% Four.I.2
\begin{videotitle}
  Properties of Determinants \\[1ex]
  Part One
\end{videotitle}
\begin{videoend}
  Properties of Determinants \\[1ex]
  Part Two
\end{videoend}

\begin{videotitle}
  Properties of Determinants \\[1ex]
  Part Two
\end{videotitle}
\begin{videoend}
  Permutation Expansion \\[1ex]
  Part One
\end{videoend}

% Four.I.3
\begin{videotitle}
  Permutation Expansion \\[1ex]
  Part One
\end{videotitle}
\begin{videoend}
  Permutation Expansion \\[1ex]
  Part Two
\end{videoend}

\begin{videotitle}
  Permutation Expansion \\[1ex]
  Part Two
\end{videotitle}
\begin{videoend}
  Determinants Exist
\end{videoend}

% Four.I.4
\begin{videotitle}
  Determinants Exist
\end{videotitle}
\begin{videoend}
  Geometry of Determinants 
\end{videoend}

% Four.II.1
\begin{videotitle}
  Geometry of Determinants 
\end{videotitle}
\begin{videoend}
  Laplace's Expansion
\end{videoend}


% Four.III.1
\begin{videotitle}
  Laplace's Expansion
\end{videotitle}
\begin{videoend}
  Complex Vector Spaces
\end{videoend}



% ============= Chapter Five ================

% Five.I.1
\begin{videotitle}
  Complex Vector Spaces  
\end{videotitle}
\begin{videoend}
  Similarity
\end{videoend}

% Five.II.1
\begin{videotitle}
  Similarity 
\end{videotitle}
\begin{videoend}
  Diagonalizability 
\end{videoend}

% Five.II.2
\begin{videotitle}
  Diagonalizability
\end{videotitle}
\begin{videoend}
  Eigenvalues and Eigenvectors \\[1ex]
  Part One
\end{videoend}

% Five.II.3
\begin{videotitle}
  Eigenvalues and Eigenvectors \\[1ex]
  Part One
\end{videotitle}
\begin{videoend}
  Eigenvalues and Eigenvectors \\[1ex]
  Part Two
\end{videoend}

\begin{videotitle}
  Eigenvalues and Eigenvectors \\[1ex]
  Part Two
\end{videotitle}
\begin{videoend}
  No more
\end{videoend}




% Skeleton: front and back slides for each video, usually two videos.
% % Videos for slides: 
% \begin{videotitle}
%    \\[1ex]
%   part one
% \end{videotitle}
% \begin{videoend}
%    \\[1ex]
%   part two
% \end{videoend}

% \begin{videotitle}
%    \\[1ex]
%   part two
% \end{videotitle}
% \begin{videoend}
%    \\[1ex]
%   part one
% \end{videoend}


\end{document}
