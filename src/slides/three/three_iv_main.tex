% \documentclass[9pt,t]{beamer}
\usefonttheme{professionalfonts}
\usefonttheme{serif}
\PassOptionsToPackage{pdfpagemode=FullScreen}{hyperref}
\PassOptionsToPackage{usenames,dvipsnames}{xcolor}
% \DeclareGraphicsRule{*}{mps}{*}{}
\usepackage{linalgjh}
\usepackage{present}
\usepackage{directories}  % Define \mapdir, \mapmpdir, etc
\usepackage{xr}\externaldocument{\mapdir map4} % read refs from .aux file
\usepackage{xr}\externaldocument{\mapdir map2} % read refs from .aux file
\usepackage{catchfilebetweentags}
\usepackage{etoolbox} % from http://tex.stackexchange.com/questions/40699/input-only-part-of-a-file-using-catchfilebetweentags-package
\makeatletter
\patchcmd{\CatchFBT@Fin@l}{\endlinechar\m@ne}{}
  {}{\typeout{Unsuccessful patch!}}
\makeatother

\mode<presentation>
{
  \usetheme{boxes}
  \setbeamercovered{invisible}
  \setbeamertemplate{navigation symbols}{} 
}
\addheadbox{filler}{\ }  % create extra space at top of slide 
\hypersetup{colorlinks=true,linkcolor=blue} 

\title[Matrix Operations] % (optional, use only with long paper titles)
{Three.IV Matrix Operations}

\author{\textit{Linear Algebra} \\ {\small Jim Hef{}feron}}
\institute{
  \texttt{https://hefferon.net/linearalgebra}\\[0.25ex]
  \texttt{http://joshua.smcvt.edu/linearalgebra}
}
\date{}


\subject{Matrix Operations}
% This is only inserted into the PDF information catalog. Can be left
% out. 

\begin{document}
\begin{frame}
  \titlepage
\end{frame}

% =============================================
% \begin{frame}{Reduced Echelon Form} 
% \end{frame}



% ..... Three.IV.1 .....
\section{Sums and Scalar Products}
%..........
\begin{frame}{Operations on linear functions}
Recall that given 
a linear map $\map{f}{V}{W}$ then 
the scalar multiple function 
\begin{equation*}
  \vec{v}\mapsunder{r\cdot f} r\cdot (\,f(\vec{v})\,)
\end{equation*}
is also a linear map $\map{rf}{V}{W}$.

And where $\map{f,g}{V}{W}$ are 
linear then the function that is their sum 
\begin{equation*}
  \vec{v}\mapsunder{f+g} f(\vec{v})+g(\vec{v})
\end{equation*}
is again linear $\map{f+g}{V}{W}$.

We will now see how the matrix representation of
$\rep{f}{B,D}$ 
is related to that of $\rep{rf}{B,D}$, 
and how the representations of 
$\rep{f}{B,D}$ and $\rep{g}{B,D}$
combine to give the representation of $\rep{f+g}{B,D}$.
\end{frame}



\begin{frame}
\ex
Fix a domain~$V$ and codomain~$W$ with bases
$B=\sequence{\vec{\beta}_1,\vec{\beta}_2}$
and $D=\sequence{\vec{\delta}_1,\vec{\delta}_2}$.
Let $\map{h}{V}{W}$ be linear
and consider the map $\map{6h}{V}{W}$ given by 
$\vec{v}\mapsunder{6h} 6\cdot h(\vec{v})$.

Where this is the representation
of~$h(\vec{v})$
\begin{equation*}
  % \rep{\vec{v}}{B}=\colvec{v_1 \\ v_2}
  % \qquad
  \rep{\,h(\vec{v})\,}{D}=\colvec{w_1 \\ w_2}
  % =\colvec{2v_1+v_2 \\ 3v_1+4v_2} 
\end{equation*}
then
$6\cdot h(\vec{v})=6\cdot (w_1\vec{\delta}_1+w_2\vec{\delta}_2)
=(6w_1)\cdot\vec{\delta}_1+(6w_2)\cdot\vec{\delta}_2$.
So the representation of $6h\:(\vec{v})$ 
\begin{equation*}
  \rep{6h\;(\vec{v})}{D}=\colvec{6w_1 \\ 6w_2} 
\end{equation*}
is entry-by-entry bigger by a factor of~$6$.  
\end{frame}
\begin{frame}
For instance, if 
\begin{equation*}
  H=\rep{h}{B,D}=
  \begin{mat}
    2  &1  \\
    3  &4  
  \end{mat}
\end{equation*}
then this represents the action of $h$.
\begin{equation*}
  \rep{h(\vec{v})}{D}=\rep{h}{B,D}\: \rep{\vec{v}}{B}
  =
  \begin{mat}
    2  &1 \\
    3  &4
  \end{mat}
  \colvec{v_1 \\ v_2}
  =\colvec{2v_1+v_2 \\ 3v_1+4v_2}
\end{equation*}
By the prior slide we know that the output of $6h$ is $6$ times bigger.
\begin{equation*}
  \rep{6h\:(\vec{v})}{D}
  =\rep{6h}{B,D}\,\colvec{v_1 \\ v_2}
  =\colvec{12v_1+6v_2 \\ 18v_1+24v_2}              
\end{equation*}
So the matrix representing $6h$ 
\begin{equation*}
  \rep{6h}{B,D}
  =
  \begin{mat}
    12 &6 \\
    18 &24
  \end{mat}
\end{equation*}
is entry-by-entry $6$ times as large as the matrix representing~$h$.  
\end{frame}


\begin{frame}
\ex
Next consider the representation of the sum~$f+g$ of two linear maps
$\map{f,g}{V}{W}$. 
For a domain vector~$\vec{v}$ let the outputs $f(\vec{v})$ and $g(\vec{v})$ 
have these representations.
\begin{equation*}
  \rep{f(\vec{v})}{D}=\colvec{u_1 \\ u_2}
  \qquad
  \rep{g(\vec{v})}{D}=\colvec{w_1 \\ w_2}
\end{equation*}
The action of $f+g$ is this.
\begin{align*}
  \vec{v}
  \mapsunder{f+g}
  \vec{u}+\vec{v}                      
  &=(u_1\vec{\delta}_1+u_2\vec{\delta}_2)
  +
  (w_1\vec{\delta}_1+w_2\vec{\delta}_2)   \\
  &=(u_1+w_1)\cdot\vec{\delta}_1+(u_2+w_2)\cdot\vec{\delta}_2
\end{align*}
The effect on the representations
of adding the functions is to add the column vectors.
\begin{equation*}
  \rep{\,(f+g)\,(\vec{v})\,}{D}=\colvec{u_1+w_1 \\ u_2+w_2}
\end{equation*}
\end{frame}
\begin{frame}
\noindent 
For instance, 
let these be the map representations.
\begin{equation*}
  \rep{f}{B,D}=
  \begin{mat}
    2  &1  \\
    3  &4  
  \end{mat}
  \qquad
  \rep{g}{B,D}=
  \begin{mat}
    5  &8  \\
    7  &6  
  \end{mat}
\end{equation*}
The functions given have 
this effect.
\begin{align*}
  \rep{f(\vec{v})}{D}=
  \begin{mat}
    2  &1  \\
    3  &4  
  \end{mat}
  \colvec{v_1 \\ v_2}
  =\colvec{2v_1+v_2 \\ 3v_1+4v_2}
  \\
  \rep{g(\vec{v})}{D}=
  \begin{mat}
    5  &8  \\
    7  &6  
  \end{mat}
  \colvec{v_1 \\ v_2}
  =\colvec{5v_1+8v_2 \\ 7v_1+6v_2}
\end{align*}
The prior slide says that $f+g$ acts in this way
\begin{equation*}
  \rep{\,(f+g)\,(\vec{v})\,}{D}
  \colvec{v_1 \\ v_2}
  =\colvec{7v_1+9v_2 \\ 10v_1+10v_2}
\end{equation*}
so its matrix
\begin{equation*}
  \rep{f+g}{B,D}
  =\begin{mat}
    7  &9  \\
    10  &10  
  \end{mat}
\end{equation*}
is the entry-by-entry sum of the other two.
\end{frame}

\begin{frame}{Definition of matrix sum and scalar multiple}
\df[def:SumScalarProdMats]
\ExecuteMetaData[\mapdir map4.tex]{df:SumScalarProdMats}

\pause
\ex
Where
\begin{equation*}
  A=
  \begin{mat}
    1  &-1 \\
    2  &3
  \end{mat}
  \quad
  B=
  \begin{mat}
    0  &0     &2  \\
    9  &-1/2  &5
  \end{mat}
  \quad
  C=
  \begin{mat}
    1  &0 \\
    8  &-1
  \end{mat}
\end{equation*}
Then
\begin{equation*}
  A+C=
  \begin{mat}
    2  &-1  \\
    10 &2
  \end{mat}
  \qquad
  5B=
  \begin{mat}
    0  &0    &10 \\
    45 &-5/2 &25 
  \end{mat}
\end{equation*}
None of these is defined: $A+B$, $B+A$, 
$B+C$, $C+B$.

From the definition, they are not defined 
because the sizes don't match and so
the entry-by-entry sum is not possible.
But really they are not defined because the underlying function operations
are not possible.
The fact that $A$ has two columns means that
functions represented by~$A$ have two-dimensional domains.
Functions represented by~$B$ have three-dimensional domains.
Adding the two functions would be adding apples and oranges.
\end{frame}




%..........
\begin{frame}
\th[th:MatOpsRepMapOps]
\ExecuteMetaData[\mapdir map4.tex]{th:MatOpsRepMapOps}

\pause
\pf
Generalize the earlier examples.
See \nearbyexercise{exer:CorrspMapMatOps}\hspace{-0.25em}.
\qed

\pause
\medskip
\df[df:ZeroMatrix]
\ExecuteMetaData[\mapdir map4.tex]{df:ZeroMatrix}

\pause
\medskip
\ex
The zero matrix is the identity element for matrix addition.
\begin{equation*}
  \begin{mat}
    3 &1 &2 \\
    5 &0 &9
  \end{mat}
  +
  \begin{mat}
    0 &0 &0 \\
    0 &0 &0
  \end{mat}
  =
  \begin{mat}
    3 &1 &2 \\
    5 &0 &9
  \end{mat}
\end{equation*}
A zero function
$\map{Z}{V}{W}$ is the identity element for
function addition, and the matrix fact accords with the map fact. 
\end{frame}




% ..... Three.IV.2 .....
\section{Matrix Multiplication}
%..........
\begin{frame}{Representing composition}
Another function operation,
besides scalar multiplication and addition,   
is composition.

\pause
\lm[lm:CompositionOfLinearMapsIsLinear]
\ExecuteMetaData[\mapdir map4.tex]{lm:CompositionOfLinearMapsIsLinear}

\pause
\pf
\ExecuteMetaData[\mapdir map4.tex]{pf:CompositionOfLinearMapsIsLinear}
\qed

% \pause
% \medskip
% We next do an exploratory calculation to see how the matrix
% representations of the two functions combine to make
% the matrix representation of their composition.
\end{frame}




%..........
\begin{frame}
\ex
Consider two linear functions $\map{h}{V}{W}$ and $\map{g}{W}{X}$
represented as here.
\begin{equation*}
  \rep{h}{B,C}=
  \begin{mat}
    3 &1 \\
    2 &5 \\
    4 &6
  \end{mat}
  \qquad
  \rep{g}{C,D}=
  \begin{mat}
    8 &7 &11 \\
    9 &10 &12 
  \end{mat}
\end{equation*}
% The sizes of the matrices show that 
% $V$ has dimension~$2$, 
% $W$ has dimension~$3$, 
% and $X$ has dimension~$2$.
We will do an explatory computation, 
to see how these two representations combine to 
give the representation of the composition
$\map{\composed{g}{h}}{V}{X}$.

\pause
We start with the action
of~$h$ on~$\vec{v}\in V$.
\begin{align*}
  \rep{h(\vec{v})}{C}
  &=\rep{h}{B,C}\cdot\rep{\vec{v}}{B}     \\
  &=
  \begin{mat}
    3 &1 \\
    2 &5 \\
    4 &6
  \end{mat}
  \colvec{v_1 \\ v_2}  
  =
  \colvec{3v_1+v_2 \\ 2v_1+5v_2 \\ 4v_1+6v_2}
\end{align*}
\end{frame}
\begin{frame}
\noindent Next, to that apply $g$.
\begin{multline*}
  \rep{g}{C,D}\cdot \rep{h(\vec{v})}{C}
  =
  \begin{mat}
    8 &7 &11 \\
    9 &10 &12     
  \end{mat}
  \colvec{3v_1+v_2 \\ 2v_1+5v_2 \\ 4v_1+6v_2}                \\
  =
  \colvec{8(3v_1+v_2)+7(2v_1+5v_2)+11(4v_1+6v_2)  \\
          9(3v_1+v_2)+10(2v_1+5v_2)+12(4v_1+6v_2)}
\end{multline*}
Gather terms.
\begin{equation*}
  =\colvec{(8\cdot 3+7\cdot 2+11\cdot 4)v_1+(8\cdot 1+7\cdot 5+11\cdot 6)v_2 \\
           (9\cdot 3+10\cdot 2+12\cdot 4)v_1+(9\cdot 1+10\cdot 5+12\cdot 6)v_2}
\end{equation*}
Rewrite as a matrix-vector multiplication.
\begin{equation*}
  =\begin{mat}
    8\cdot 3+7\cdot 2+11\cdot 4 &8\cdot 1+7\cdot 5+11\cdot 6 \\
    9\cdot 3+10\cdot 2+12\cdot 4 &9\cdot 1+10\cdot 5+12\cdot 6
  \end{mat}
  \colvec{v_1 \\ v_2}
\end{equation*}
So here is how the two starting matrices combine.
\begin{equation*}
  \begin{mat}
    8 &7 &11 \\
    9 &10 &12 
  \end{mat}
  \begin{mat}
    3 &1 \\
    2 &5 \\
    4 &6
  \end{mat}
  =
  \begin{mat}
    8\cdot 3+7\cdot 2+11\cdot 4 &8\cdot 1+7\cdot 5+11\cdot 6 \\
    9\cdot 3+10\cdot 2+12\cdot 4 &9\cdot 1+10\cdot 5+12\cdot 6
  \end{mat}
\end{equation*}
\end{frame}




%..........
\begin{frame}{Definition of matrix multiplication}
\df[def:MatMult]
\ExecuteMetaData[\mapdir map4.tex]{df:MatMult}
\ex
\begin{equation*}
  \begin{mat}
    3 &1 &6 \\
    2 &5 &9
  \end{mat}
  \begin{mat}
    2 &0  &4 \\
    1 &-3 &5 \\
    4 &2  &7
  \end{mat}
  =
  \begin{mat}
    31  &9  &59  \\
    45  &3  &96
  \end{mat}
\end{equation*}
\end{frame}
\begin{frame}
\ex
This product 
\begin{equation*}
  \begin{mat}
    1  &3  &-1 \\
    0  &0  &0  \\
    2  &0  &0
  \end{mat}
  \begin{mat}
    5  &7  &1 \\
    2  &2  &0 
  \end{mat}
\end{equation*}
is not defined because
the number of columns on the left must equal the number of rows on the right.

\pause
\ex
Square matrices of the same size have a defined product.
\begin{equation*}
  \begin{mat}
    1  &3  &-1 \\
    0  &0  &0  \\
    2  &0  &0
  \end{mat}
  \begin{mat}
    5  &7  &1 \\
    2  &2  &0 \\
    1  &-1 &2 
  \end{mat}
  =
  \begin{mat}
    10  &14  &-1  \\
     0  &0   &0   \\
    10  &14  &2
  \end{mat}
\end{equation*}
This reflects the fact that we can compose two
functions from a space to itself $\map{g,h}{V}{V}$.
\end{frame}




%..........
\begin{frame}{Matrix multiplication represents composition}
\th[th:MatMultRepComp]
\ExecuteMetaData[\mapdir map4.tex]{th:MatMultRepComp}

\iftoggle{showallproofs}{
  \pause
  \pf
  \ExecuteMetaData[\mapdir map4.tex]{pf:MatMultRepComp0}
}{

  \bigskip
  The book has the proof, which retraces the steps of the example.
}
\end{frame}
\iftoggle{showallproofs}{
  \begin{frame}
  \ExecuteMetaData[\mapdir map4.tex]{pf:MatMultRepComp1}
  \qed
  \end{frame}
}{}



%..........
\begin{frame}{Arrow diagrams}
% \ExecuteMetaData[\mapdir map4.tex]{MatMultArrowDiag0}
This pictures the relationship between maps and matrices.
\centergraphic{\mapmpdir ch3.20}
\pause
\ExecuteMetaData[\mapdir map4.tex]{MatMultArrowDiag1}
\end{frame}

\begin{frame}
\ex 
Let $V=\Re^2$, $W=\polyspace_2$, and~$X=\matspace_{\nbyn{2}}$. 
Fix these bases.  
\begin{gather*}
  B=\sequence{\colvec{1 \\ 1}, \colvec{1 \\ -1}}    
  \qquad
  C=\sequence{x^2, x^2+x, x^2+x+1}              \\
  D=\sequence{
    \begin{mat}
      1 &0 \\
      0 &0
    \end{mat},
    \begin{mat}
      0 &2 \\
      0 &0
    \end{mat},
    \begin{mat}
      0 &0 \\
      3 &0
    \end{mat},
    \begin{mat}
      0 &0 \\
      0 &4
    \end{mat}}
\end{gather*}
Suppose that $\map{h}{\Re^2}{\polyspace_2}$ and
$\map{g}{\polyspace_2}{\matspace_{\nbyn{2}}}$ have these actions.
\begin{equation*}
  \colvec{a \\ b}\mapsunder{h} ax^2+(a+b)
  \qquad
  px^2+qx+r\mapsunder{g}
  \begin{mat}
    p &p+q \\  
    0 &r
  \end{mat}
\end{equation*}
Then the composition does this.
\begin{equation*}
  \colvec{a \\ b}\mapsunder{h}ax^2+(a+b)\mapsunder{g}
  \begin{mat}
    a &a \\
    0 &a+b
  \end{mat}
\end{equation*}
Here is the same statement in the other notation.
\begin{equation*}
  \composed{g}{h}\,(\colvec{a \\ b})=  
  \begin{mat}
    a &a \\
    0 &a+b
  \end{mat}
\end{equation*}
\end{frame}
\begin{frame}
% So far in this example we have given the maps above the arrows, $f$, $g$, 
% and $\composed{g}{h}$. 
% \centergraphic{\mapmpdir ch3.20}
We next compute the matrices representing those maps, and we will
finish by checking that the product of $G$ and~$H$ is the matrix 
representing $\composed{g}{h}$.

To find $H=\rep{h}{B,D}$, compute the action of $h$
on the domain basis vectors,
\begin{equation*}
  \colvec{1 \\ 1}\mapsunder{h}x^2+2
  \quad
  \colvec{1 \\ -1}\mapsunder{h}x^2
\end{equation*}
represent the results with respect to $D$, and make the matrix.
\begin{equation*}
  \rep{x^2+2}{C}=\colvec{1  \\ -2 \\ 2}
  \quad
  \rep{x^2}{C}=\colvec{1  \\ 0 \\ 0}         
  \qquad
  H=
  \begin{mat}
    1 &1 \\
   -2 &0 \\
    2 &0
  \end{mat}
\end{equation*}
Do the same for~$g$: see where it maps its domain basis
\begin{equation*}
  x^2\mapsto
  \begin{mat}
    1 &1 \\
    0 &0
  \end{mat}
  \quad
  x^2+x\mapsto
  \begin{mat}
    1 &2 \\
    0 &0
  \end{mat}
  \quad
  x^2+x+1\mapsto
  \begin{mat}
    1 &2 \\
    0 &1
  \end{mat}
\end{equation*}
and represent those with respect to its codomain basis.
\begin{equation*}
  G=
  \begin{mat}
    1   &1  &1  \\
    1/2 &1  &1  \\
    0   &0  &0  \\
    0   &0  &1/4
  \end{mat}
\end{equation*}
\end{frame}
\begin{frame}
Next, $\composed{g}{h}$ has this action.
\begin{equation*}
  \colvec{1 \\ 1}\mapsto
  \begin{mat}
    1 &1 \\
    0 &2
  \end{mat}
  \qquad  
  \colvec{1 \\ -1}\mapsto
  \begin{mat}
    1 &1   \\
    0 &0
  \end{mat}
\end{equation*}
This represents the composition.
\begin{equation*}
  \rep{\composed{g}{h}}{B,D}
  =
  \begin{mat}
    1   &1  \\
    1/2 &1/2 \\
    0   &0    \\
    1/2 &0
  \end{mat}
\end{equation*}

Finish by checking that the product of $G$ with~$H$
equals the matrix representing~$\composed{g}{h}$.
\begin{align*}
  \rep{g}{C,D}\cdot\rep{h}{B,C}
  &=\rep{\composed{g}{h}}{B,D}           \\
  \begin{mat}
    1   &1  &1  \\
    1/2 &1  &1  \\
    0   &0  &0  \\
    0   &0  &1/4
  \end{mat}
  \begin{mat}
    1 &1 \\
   -2 &0 \\
    2 &0
  \end{mat}                    
  &=  
  \begin{mat}
    1   &1  \\
    1/2 &1/2 \\
    0   &0    \\
    1/2 &0
  \end{mat}
\end{align*}
% \bigskip
% \centergraphic{\mapmpdir ch3.20}
\end{frame}


\begin{frame}{Order, dimensions, and sizes}
% These two functions can be composed 
% \begin{equation*}
%   \map{h}{V}{W}
%   \qquad
%   \map{g}{W}{X}
% \end{equation*}
% because the codomain of~$g$ is the domain of~$g$.

An important observation about the order in which we write these things:~in 
writing the composition~$\composed{g}{h}$,
the function $g$ is written first, that is, leftmost, 
but it is applied second.
\begin{equation*}
  \vec{v}\mapsunder{h} h(\vec{v}) \mapsunder{g} g( h(\vec{v}))
\end{equation*}
% We write $g$ first to match the definition 
% $g(\,h(\vec{v})\,)$.
That order carries over to matrices:~$\composed{g}{h}$
is represented by $GH$.

\pause
Also consider the dimensions of the spaces.
\begin{equation*}
  \text{dimension \( n \) space}
  \;\stackrel{h}{\longrightarrow}\;
  \text{dimension \( r \) space}
  \;\stackrel{g}{\longrightarrow}\;
  \text{dimension \( m \) space}
\end{equation*}
Briefly,
$\nbym{m}{r}\text{\ times\ }\nbym{r}{n}\text{\ equals\ }\nbym{m}{n}$,
as here.
\begin{align*}
  \makebox[0.65in][c]{$\nbym{2}{3}$}
  \makebox[1in][c]{$\nbym{3}{4}$}
  &=
  \makebox[1.15in][c]{$\nbym{2}{4}$}                            \\
  \begin{mat}
     2 &1 &4 \\
    -1 &0 &3
  \end{mat}
  \begin{mat}
     3 &0  &2 &1 \\
     5 &0  &0 &2 \\
     1 &-1 &4 &7
  \end{mat}
  &=
  \begin{mat}
     15 &-4  &20 &32 \\
     0  &-3  &10 &20 
  \end{mat}
\end{align*}
\end{frame}



\begin{frame}{Matrix multiplication is not commutative}
Function composition is in general not a commutative 
operation\Dash $\cos(\sqrt{x})$
is different than $\sqrt{\cos(x)}$.
This holds even in the special case of 
composition of linear functions.
\pause
\ex
Changing the order in which we multiply these matrices
\begin{equation*}
  \begin{mat}
    3  &3  \\
    0  &4
  \end{mat}
  \begin{mat}
    -2  &6  \\
    6   &5
  \end{mat}
  =
  \begin{mat}
    12  &33 \\
    24  &20
  \end{mat}
\end{equation*}
changes the result.
\begin{equation*}
  \begin{mat}
    -2  &6  \\
    6   &5
  \end{mat}
  \begin{mat}
    3  &3  \\
    0  &4
  \end{mat}
  =
  \begin{mat}
     -6  &18   \\
     18  &38  
  \end{mat}
\end{equation*}

\ex
The product of these two
is defined in one order
and not defined in the other.
\begin{equation*}
  \begin{mat}
    3  &4  \\
    0  &2
  \end{mat}
  \qquad
  \begin{mat}
    8  &12  &0 \\
   -4  &0  &1/2
  \end{mat}
\end{equation*}
% \begin{equation*}
%   \begin{mat}
%     8  &12  &0 \\
%    -4  &0  &1/2
%   \end{mat}
%   \qquad
%   \begin{mat}
%     3  &4  \\
%     0  &2
%   \end{mat}
% \end{equation*}
\end{frame}




%..........
\begin{frame}
Although the matrix operation of multiplication does not have
the property of being commutative,
it does have some nice algebraic properties.  

\th[th:MatMultWellBehaved]
\ExecuteMetaData[\mapdir map4.tex]{th:MatMultWellBehaved}
\pause
\pf
\ExecuteMetaData[\mapdir map4.tex]{pf:MatMultWellBehaved0}

% \pause
\ExecuteMetaData[\mapdir map4.tex]{pf:MatMultWellBehaved1}
\qed
\end{frame}




% ..... Three.IV.3 .....
\section{Mechanics of Matrix Multiplication}
%..........
\begin{frame}{Combinatorics of multiplication}
The striking thing about matrix multiplication is the
way rows and columns combine.
The \( i,j \) entry of the matrix product $GH$ is the dot product of
row~$i$ of the left matrix $G$ with column~$j$
of the right one~$H$.
\begin{equation*} % 
  p_{i,j}
  =
  g_{i,\text{\textcolor{red}{$1$}}}h_{\text{\textcolor{red}{$1 $}},j}
   +g_{i,\text{\textcolor{red}{$2$}}}h_{\text{\textcolor{red}{$2 $}},j}
   +\dots+g_{i,\text{\textcolor{red}{$r $}}}h_{\text{\textcolor{red}{$r $}},j}
\end{equation*}
Here a second row and a third column combine to make a $2,3$~entry.
\begin{equation*}
\setlength{\fboxsep}{1.5pt}
    \begin{mat}
       \begin{array}{@{}cc@{}} 1  &1 \end{array}                         \\ 
       {\color{blue} \begin{array}{@{}cc@{}} 0  &1 \end{array}}           \\ 
       % \text{\highlight{\( \begin{array}{@{}cc@{}}  0  &1  \end{array} \)}}   \\  
       \begin{array}{@{}cc@{}} 1  &0 \end{array}
    \end{mat}
    \begin{mat}
      \begin{array}{@{}c@{}}  4  \\  5  \end{array}
      &{\color{blue}\begin{array}{@{}c@{}}  6  \\  7  \end{array}}
      % &\text{\highlight{\(\begin{array}{@{}c@{}}  8  \\  9  \end{array}\)}}
    \end{mat}
  =
    \begin{mat}
      9   &13                           \\
      % 5  &\text{\highlight{ \(7\) }}   \\  
      5  &{\color{blue} 7 }   \\  
      4  &6
    \end{mat}
\end{equation*}
We can view this as the left matrix acting
by multiplying its rows into the columns of the right matrix.
Or we could see it as 
the right matrix using its columns to
act on the left matrix's rows.
\end{frame}




%..........
\begin{frame}
\lm[lm:ColsAndRowsInMatrixMult]
\ExecuteMetaData[\mapdir map4.tex]{lm:ColsAndRowsInMatrixMult}
\end{frame}
% \begin{frame}
% \pf
% \ExecuteMetaData[\mapdir map4.tex]{pf:ColsAndRowsInMatrixMult}
% \qed
% \end{frame}




%..........
\begin{frame}
\df[df:UnitMatrix]
\ExecuteMetaData[\mapdir map4.tex]{df:UnitMatrix}
\ex
The $2,1$ unit $\nbym{2}{3}$ matrix multiplies from the left 
\begin{equation*}
  \begin{mat}
    0 &0 &0 \\
    1 &0 &0
  \end{mat}
  \begin{mat}
    2 &1 &3 \\
    5 &6 &4 \\
    7 &8 &9
  \end{mat}
  =
  \begin{mat}
    0 &0 &0 \\
    2 &1 &3 
  \end{mat}
\end{equation*}
to copy row~$1$ of the multiplicand into row~$2$ of the result. 

\pause
\ex 
From the right the $2,1$ unit $\nbym{2}{3}$ matrix
\begin{equation*}
  \begin{mat}
    3 &4 \\
    6 &5
  \end{mat}
  \begin{mat}
    0 &0 &0 \\
    1 &0 &0
  \end{mat}
  =
  \begin{mat}
    4 &0 &0 \\
    5 &0 &0
  \end{mat}
\end{equation*}
copies column~$2$ of the first matrix into column~$1$ of the result.

\pause
\ex 
Rescaling the unit matrix rescales the result.
\begin{equation*}
  \begin{mat}
    3 &4 \\
    6 &5
  \end{mat}
  \begin{mat}
    0 &0 &0 \\
    3 &0 &0
  \end{mat}
  =
  \begin{mat}
    12 &0 &0 \\
    15 &0 &0
  \end{mat}
\end{equation*}
\end{frame}




%..........
\begin{frame}
\df[df:MainDiagonal]
\ExecuteMetaData[\mapdir map4.tex]{df:MainDiagonal}
\pause
\df[df:IdentityMatrix]
\ExecuteMetaData[\mapdir map4.tex]{df:IdentityMatrix}
\pause
Taking the product with an identity matrix returns the multiplicand. 
\ex
Multiplication by an identity from the left
\begin{equation*}
  \begin{mat}
    1  &0 \\
    0  &1
  \end{mat}
  \begin{mat}
    3  &2  \\
   -1  &5
  \end{mat}
  =
  \begin{mat}
    3  &2  \\
   -1  &5
  \end{mat}
\end{equation*}
or from the right leaves the matrix unchanged.
\begin{equation*}
  \begin{mat}
    3  &2  \\
   -1  &5
  \end{mat}
  \begin{mat}
    1  &0 \\
    0  &1
  \end{mat}
  =
  \begin{mat}
    3  &2  \\
   -1  &5
  \end{mat}
\end{equation*}
\end{frame}




%..........
\begin{frame}
\df[df:DiagonalMatrix]
\ExecuteMetaData[\mapdir map4.tex]{df:DiagonalMatrix}
\pause
\ex
Multiplication from the left by a diagonal matrix rescales the rows.
\begin{equation*}
  \begin{mat}
    2  &0 \\
    0  &3
  \end{mat}
  \begin{mat}
    3  &2  \\
   -1  &5
  \end{mat}
  =
  \begin{mat}
    6  &4  \\
   -3  &15
  \end{mat}
\end{equation*}
From the right it rescales the columns.
\begin{equation*}
  \begin{mat}
    3  &2  \\
   -1  &5
  \end{mat}
  \begin{mat}
    2  &0 \\
    0  &3
  \end{mat}
  =
  \begin{mat}
    6  &6  \\
   -2  &15
  \end{mat}
\end{equation*}
\end{frame}




%..........
\begin{frame}
\df[df:PermutationMatrix]
\ExecuteMetaData[\mapdir map4.tex]{df:PermutationMatrix}
\pause
\ex
Multiplication by a permutation matrix from the left will swap rows.
\begin{equation*}
  \begin{mat}
    0 &1 &0 \\
    1 &0 &0 \\
    0 &0 &1
  \end{mat}
  \begin{mat}
    1 &2 &3 \\
    4 &5 &6 \\
    7 &8 &9
  \end{mat}\
  =
  \begin{mat}
    4 &5 &6 \\
    1 &2 &3 \\
    7 &8 &9
  \end{mat}
\end{equation*}
From the right it swaps columns.
\begin{equation*}
  \begin{mat}
    1 &2 &3 \\
    4 &5 &6 \\
    7 &8 &9
  \end{mat}\
  \begin{mat}
    0 &1 &0 \\
    1 &0 &0 \\
    0 &0 &1
  \end{mat}
  =
  \begin{mat}
    2 &1 &3 \\
    5 &4 &6 \\
    8 &7 &9
  \end{mat}
\end{equation*}
\end{frame}




%..........
\begin{frame}
\df[df:ElementaryReductionMatrices]
\ExecuteMetaData[\mapdir map4.tex]{df:ElementaryReductionMatrices}
% \pause
% \ex
% Here are some $\nbyn{2}$ examples.
% \begin{equation*}
%   M_{2}(3)=
%   \begin{mat}[r]
%     1  &0 \\
%     0  &3
%   \end{mat}
%   \quad
%   P_{1,2}=
%   \begin{mat}[r]
%     0  &1 \\
%     1  &0
%   \end{mat}
%   \quad
%   C_{1,2}(-3)=
%   \begin{mat}[r]
%     1  &0 \\
%     -3 &1
%   \end{mat}
% \end{equation*}
% \pause
% \ex
% Some $\nbyn{3}$ examples.
% \begin{multline*}
%   M_{2}(1/2)=
%   \begin{mat}
%     1 &0   &0 \\
%     0 &1/2 &0 \\
%     0 &0   &1
%   \end{mat}
%   \quad
%   P_{2,3}=
%   \begin{mat}
%     1 &0 &0 \\
%     0 &0 &1 \\
%     0 &1 &0
%   \end{mat}
%   \\
%   C_{2,3}(-4)=
%   \begin{mat}[r]
%     1 &0  &0 \\
%     0 &1  &0 \\
%     0 &-4 &1
%   \end{mat}
% \end{multline*}
% \end{frame}




% \begin{frame}
\ex Multiplying on the left by the $\nbyn{3}$ matrix
$M_{2}(1/2)$
has the effect of the row operation $(1/2)\rho_2$. 
\begin{equation*}
  \begin{mat}
    1 &0   &0 \\
    0 &1/2 &0 \\
    0 &0   &1
  \end{mat}
  \begin{mat}
    1  &3  &0  \\
    0  &2  &5  \\
    0  &0  &0
  \end{mat}
  =
  \begin{mat}
    1  &3  &0  \\
    0  &1  &5/2  \\
    0  &0  &0
  \end{mat}
\end{equation*}
\pause
\ex
Left multiplication by $C_{1,3}(-2)$ performs the
row operation $-2\rho_1+\rho_3$.
\begin{equation*}
  \begin{mat}[r]
    1 &0   &0 \\
    0 &1   &0 \\
    -2 &0   &1
  \end{mat}
  \begin{mat}[r]
    1  &3  &0  \\
    0  &2  &5  \\
    2  &1  &1
  \end{mat}
  =
  \begin{mat}[r]
    1  &3  &0  \\
    0  &2  &5  \\
    0  &-5 &1
  \end{mat}
\end{equation*}
\end{frame}




\begin{frame}
\lm[GrByMatMult]
\ExecuteMetaData[\mapdir map4.tex]{lm:GrByMatMult}
\pf
Clear.
\qed
\co[cor:ReducViaMatrices]
\ExecuteMetaData[\mapdir map4.tex]{co:ReducViaMatrices}

\pause
\ex
You can do this Gauss-Jordan reduction
\begin{equation*}
  \begin{mat}
    1 &2 \\
    4 &3 
  \end{mat}
  \grstep{-4\rho_1+\rho_2}
  \grstep{-(1/5)\rho_2}
  \grstep{-2\rho_2+\rho_1}
  \begin{mat}
    1 &0 \\
    0 &1 
  \end{mat}
\end{equation*}
as a product; note that the order is right-to-left.
\begin{align*}
  \begin{mat}
    1 &-2 \\
    0 &1
  \end{mat}
  \begin{mat}
    1 &0 \\
    0 &-1/5
  \end{mat}
  \begin{mat}
    1 &0 \\
    -4 &1
  \end{mat}
  \cdot
  \begin{mat}
    1 &2 \\
    4 &3 
  \end{mat}
  &=
  \begin{mat}
    1 &0 \\
    0 &1 
  \end{mat}              \\[1ex]
  C_{2,1}(-2)\,M_2(-1/5)\,C_{1,2}(-4)\cdot H
  &= I
\end{align*}

\end{frame}


\begin{frame}
\ex
You can bring this augmented matrix to echelon form with matrix multiplication.
\begin{equation*}
  \begin{amat}{3}
    1  &-1  &2  &4 \\
    2  &-2  &-1 &6 \\
    0  &3   &1  &5 
  \end{amat}
\end{equation*}
% \end{frame}
% \begin{frame}
First perform $-2\rho_1+\rho_2$ via left multiplication by $C_{1,2}(-2)$.
\begin{equation*}
  \begin{mat}
    1  &0  &0  \\
    -2  &1  &0  \\
   0  &0  &1
  \end{mat}
  \begin{amat}{3}
    1  &-1  &2  &4 \\
    2  &-2  &-1 &6  \\
    0  &3   &1  &5 
  \end{amat}
  =
  \begin{amat}{3}
    1  &-1  &2  &4 \\
    0  &0   &-5 &-2 \\
    0  &3   &1  &5 
  \end{amat}
\end{equation*}
%\pause
Swap rows $2$ and $3$ with $P_{2,3}$
\begin{equation*}
  \begin{mat}
    1  &0  &0  \\
    0  &0  &1  \\
    0  &1  &0
  \end{mat}
  \begin{amat}{3}
    1  &-1  &2  &4 \\
    0  &0   &-5 &-2 \\
    0  &3   &1  &5 
  \end{amat}
  =
  \begin{amat}{3}
    1  &-1  &2  &4 \\
    0  &3   &1  &5 \\
    0  &0   &-5 &-2 
  \end{amat}
\end{equation*}
Here is the full equation.
\begin{equation*}
  \begin{mat}
    1  &0  &0  \\
    0  &0  &1  \\
    0  &1  &0
  \end{mat}
  \begin{mat}
    1  &0  &0  \\
    -2  &1  &0  \\
    0  &0  &1
  \end{mat}
  \begin{amat}{3}
    1  &-1  &2  &4 \\
    2  &-2  &-1 &6  \\
    0  &3   &1  &5 
  \end{amat}
  =
  \begin{amat}{3}
    1  &-1  &2  &4 \\
    0  &3   &1  &5 \\
    0  &0   &-5 &-2 
  \end{amat}
\end{equation*}
\end{frame}



% ..... Three.IV.3 .....
\section{Inverses}
\begin{frame}{Function inverses}
\ExecuteMetaData[\mapdir map4.tex]{FunctionInverseReview0}
Our goal is, where a linear~$h$ has an inverse, to find the
relationship between the matrices $\rep{h}{B,D}$
and $\rep{h^{-1}}{D,B}$.

\pause
\ExecuteMetaData[\mapdir map4.tex]{FunctionInverseReview1}
\end{frame}
\begin{frame}
\ExecuteMetaData[\mapdir map4.tex]{FunctionInverseReview2}

\pause
\ExecuteMetaData[\mapdir map4.tex]{FunctionInverseReview3}
\end{frame}
\begin{frame}
\ExecuteMetaData[\mapdir map4.tex]{FunctionInverseReview4}

\pause
\ExecuteMetaData[\mapdir map4.tex]{FunctionInverseReview5}
\end{frame}



%..........
\begin{frame}{Definition of matrix inverse}
\df[df:MatrixInverse]
\ExecuteMetaData[\mapdir map4.tex]{df:MatrixInverse}
\pause
\ex
This matrix 
\begin{equation*}
  H=
  \begin{mat}
    2  &5  \\
    1  &3
  \end{mat}
\end{equation*}
has a two-sided inverse.
\begin{equation*}
  H^{-1}=
  \begin{mat}
    3   &-5  \\
    -1  &2
  \end{mat}
\end{equation*}
To check that, we multiply them in both orders.  
Here is one; the other is just as easy.
\begin{equation*}
  \begin{mat}
    2  &5  \\
    1  &3
  \end{mat}
  \begin{mat}
    3   &-5  \\
    -1  &2
  \end{mat}
  =
  \begin{mat}
    1  &0  \\
    0  &1
  \end{mat}
\end{equation*}
\end{frame}
\begin{frame}
\ex One advantage of knowing a matrix inverse is that it makes solving a 
linear system easy and quick.
To solve
\begin{equation*}
  \begin{linsys}{2}
    2x &+ &5y &= &-3 \\
    x  &+ &3y &= &10
  \end{linsys}
\end{equation*}
rewrite as a matrix equation
\begin{equation*}
  \begin{mat}
    2 &5 \\
    1 &3
  \end{mat}
  \colvec{x \\ y}
  =
  \colvec{-3 \\ 10}
\end{equation*}
and multiply both sides (from the left) by the matrix inverse.
\begin{align*}
  \begin{mat}
    3  &-5 \\
    -1 &2 
  \end{mat}\cdot
  \begin{mat}
    2 &5 \\
    1 &3
  \end{mat}
  \colvec{x \\ y}
  &=
  \begin{mat}
    3  &-5 \\
    -1 &2 
  \end{mat}\cdot
  \colvec{-3 \\ 10}             \\
  \begin{mat}
    1 &0 \\
    0 &1
  \end{mat}
  \colvec{x \\ y}
  &=
  \colvec{-59 \\ 23}           \\
  \colvec{x \\ y}
  &=
  \colvec{-59 \\ 23}           
\end{align*}
\end{frame}



%..........
\begin{frame}
This specializes the arrow diagram for composition 
to the case of inverses.
\centergraphic{\mapmpdir ch3.21}  
\pause
\lm[le:LeftAndRightInvEqual]
\ExecuteMetaData[\mapdir map4.tex]{lm:LeftAndRightInvEqual}

\th[th:MatrixInvertibleIffNonsingular]
\ExecuteMetaData[\mapdir map4.tex]{th:MatrixInvertibleIffNonsingular}
\pause
\pf
\ExecuteMetaData[\mapdir map4.tex]{pf:MatrixInvertibleIffNonsingular}
\qed
\end{frame}




%..........
\begin{frame}
\lm[lem:ProdInvIsInv]
\ExecuteMetaData[\mapdir map4.tex]{lm:ProdInvIsInv}
% \pause
% \pf
% \ExecuteMetaData[\mapdir map4.tex]{pf:ProdInvIsInv0}

% \pause
% \ExecuteMetaData[\mapdir map4.tex]{pf:ProdInvIsInv1}
% \qed
% \end{frame}




% \begin{frame}
\lm[lem:ComputeInvMat]
\ExecuteMetaData[\mapdir map4.tex]{lm:ComputeInvMat}
% \pause
% \pf
% \ExecuteMetaData[\mapdir map4.tex]{pf:ComputeInvMat0}

% \pause
% \ExecuteMetaData[\mapdir map4.tex]{pf:ComputeInvMat1}

% \pause
% \ExecuteMetaData[\mapdir map4.tex]{pf:ComputeInvMat2}
% \qed  
\end{frame}




\begin{frame}
\ex
This matrix is nonsingular and so is invertible.
\begin{equation*}
  A=
  \begin{mat}
    1  &3  &1  \\
    2  &0  &-1 \\
    1  &2  &0
  \end{mat}
\end{equation*}
To ease the inverse calculation described in the prior proof, we write 
the matrix~$A$ next to the $\nbyn{3}$ identity and as we Gauss-Jordan 
reduce~the matrix on the left, 
we apply those operations also on the right.
\begin{align*}
  \begin{pmat}{rrr|rrr}
    1  &3  &1  &1  &0  &0  \\
    2  &0  &-1 &0  &1  &0  \\
    1  &2  &0  &0  &0  &1
  \end{pmat}
  &\grstep[-\rho_1+\rho_3]{-2\rho_1+\rho_2}
  \begin{pmat}{rrr|rrr}
    1  &3  &1  &1  &0  &0  \\
    0  &-6 &-3 &-2  &1  &0  \\
    0  &-1 &-1 &-1  &0  &1
  \end{pmat}                                \\
  &\grstep{-1/6\rho_2+\rho_3}
  \begin{pmat}{rrr|rrr}
    1  &3  &1    &1     &0     &0  \\
    0  &-6 &-3   &-2    &1     &0  \\
    0  &0  &-1/2 &-2/3  &-1/6  &1
  \end{pmat}                                
\end{align*}
The right-hand side is in echelon form.
We continue with the second half of Gauss-Jordan reduction on the next slide. 
\end{frame}
\begin{frame}
\begin{align*}
  &\grstep[-2\rho_3]{-1/6\rho_2}
  \begin{pmat}{rrr|rrr}
    1  &3  &1    &1     &0     &0  \\
    0  &1  &1/2  &1/3   &-1/6  &0  \\
    0  &0  &1    &4/3   &1/3   &-2
  \end{pmat}                                \\
  % &\begin{pmat}{rrr|rrr}
  %   1  &3  &1    &1     &0     &0  \\
  %   0  &1  &1/2  &1/3   &-1/6  &0  \\
  %   0  &0  &1    &4/3   &1/3   &-2
  % \end{pmat}                                \\                            
  &\grstep[-(1/2)\rho_3+\rho_2]{-\rho_3+\rho_1}    
  \begin{pmat}{rrr|rrr}
    1  &3  &0    &-1/3  &-1/3  &2  \\
    0  &1  &0    &-1/3  &-1/3  &1  \\
    0  &0  &1    &4/3   &1/3   &-2
  \end{pmat}                                \\
  &\grstep{-3\rho_2+\rho_1}    
  \begin{pmat}{rrr|rrr}
    1  &0  &0    &2/3   &2/3   &-1  \\
    0  &1  &0    &-1/3  &-1/3  &1  \\
    0  &0  &1    &4/3   &1/3   &-2
  \end{pmat}                                
\end{align*}
This is the inverse.
\begin{equation*}
  A^{-1}=
  \begin{mat}
    2/3   &2/3   &-1  \\
   -1/3   &-1/3  &1   \\
    4/3   &1/3   &-2
  \end{mat}                                 
\end{equation*}
\end{frame}
% \begin{frame}
% \ex This is the same kind of calculation applied to a $\nbyn{2}$ matrix.
% \begin{align*}
%   \begin{pmat}{cc|cc}
%     2 &1 &1 &0 \\
%     4 &3 &0 &1
%   \end{pmat}
%   &\grstep{-2\rho_1+\rho_2}
%   \begin{pmat}{cc|cc}
%     2 &1 &1 &0 \\
%     0 &1 &-2 &1
%   \end{pmat}                      \\
%   &\grstep{(1/2)\rho_1}
%   \begin{pmat}{cc|cc}
%     1 &1/2 &1/2 &0 \\
%     0 &1   &-2 &1
%   \end{pmat}                      \\
%   &\grstep{-(1/2)\rho_2+\rho_1}
%   \begin{pmat}{cc|cc}
%     1 &0 &3/2 &-1/2 \\
%     0 &1 &-2 &1
%   \end{pmat}
% \end{align*}
% The inverse is this.
% \begin{equation*}
%   \begin{mat}
%     2 &1 \\
%     4 &3
%   \end{mat}^{-1}
%   =
%   \begin{mat}
%     3/2 &-1/2 \\
%     -2  &1
%   \end{mat}
% \end{equation*}
% \end{frame}




\begin{frame}
Finding the inverse of a matrix $A$
is a lot of work but as we noted earlier,
once we have it then solving linear
systems $A\vec{x}=\vec{b}$ is easy. 
\ex
The linear system
\begin{equation*}
  \begin{linsys}{3}
    x  &+  &3y  &+  &z  &=  &2   \\
   2x  &   &    &-  &z  &=  &12  \\
    x  &+  &2y  &   &   &=  &4
  \end{linsys}
\end{equation*}
is this matrix equation.
\begin{equation*}
  \begin{mat}
    1  &3  &1  \\
    2  &0  &-1 \\
    1  &2  &0
  \end{mat}
  \colvec{x \\ y  \\ z}
  =
  \colvec{2  \\ 12  \\ 4}
\end{equation*}
Solve it by multiplying both sides from the left by 
the inverse that we found earlier.
\begin{equation*}
  % \begin{mat}
  %   1  &0  &0  \\
  %   0  &1  &0 \\
  %   0  &0  &1
  % \end{mat}
  \colvec{x \\ y  \\ z}
  =
  \begin{mat}
    2/3   &2/3   &-1  \\
   -1/3   &-1/3  &1   \\
    4/3   &1/3   &-2
  \end{mat}                                 
  \colvec{2  \\ 12  \\ 4}
  =
  \colvec{16/3 \\ -2/3 \\ -4/3}
\end{equation*}
\end{frame}
\begin{frame}
We sometimes want to repeatedly solve systems with the same left side
but different right sides.
This system equals the one on the prior slide but for one number on the
right.  
\begin{equation*}
  \begin{linsys}{3}
    x  &+  &3y  &+  &z  &=  &1   \\
   2x  &   &    &-  &z  &=  &12  \\
    x  &+  &2y  &   &   &=  &4
  \end{linsys}
\end{equation*}
The solution is this.
\begin{equation*}
  % \begin{mat}
  %   1  &0  &0  \\
  %   0  &1  &0 \\
  %   0  &0  &1
  % \end{mat}
  \colvec{x \\ y  \\ z}
  =
  \begin{mat}
    2/3   &2/3   &-1  \\
   -1/3   &-1/3  &1   \\
    4/3   &1/3   &-2
  \end{mat}                                 
  \colvec{1  \\ 12  \\ 4}
  =
  \colvec{14/3 \\ -1/3 \\ -8/3}
\end{equation*}
\end{frame}



\begin{frame}{The inverse of a $\nbyn{2}$ matrix}
\co[cor:TwoByTwoInv]
\ExecuteMetaData[\mapdir map4.tex]{co:TwoByTwoInv}
\pause
\pf
\ExecuteMetaData[\mapdir map4.tex]{pf:TwoByTwoInv}
\qed
\pause  
\ex
\begin{equation*}
  \begin{mat}
    2  &4  \\
   -1  &1
  \end{mat}^{-1}
  =
  \frac{1}{6}
  \begin{mat}
    1  &-4  \\
    1  &2
  \end{mat}
  =
  \begin{mat}
    1/6  &-2/3  \\
    1/6  &1/3
  \end{mat}
\end{equation*}
\pause
The $\nbyn{3}$ formula is much more complicated. 
We will cover it in the next chapter.
\end{frame}


%...........................
% \begin{frame}
% \ExecuteMetaData[../gr3.tex]{GaussJordanReduction}
% \df[def:RedEchForm]
% 
% \end{frame}
\end{document}
