\documentclass[answers]{examjh}
\usepackage{../sty/linalgjh}

\renewcommand{\legalesetext}{
  While you are working on these questions
  you may confer with others in the class, 
  or talk to someone who had the course before, etc.
  % or look at the book or in the library or online, etc.
  However, when you write up the answer, you must write on your own.
  Note also: \textit{you must show your work}.}
\examhead{MA~213 homework one}{Due: 2019-Sep-18}

\begin{document}

\begin{questions}
\question
Give the solution set of each system.
\begin{parts}
  \item
    $\begin{linsys}{3}
      3x &+ &2y &+ &z &= &1 \\
      x  &- &y  &+ &z &= &2 \\
      5x &+ &5y &+ &z &= &0
    \end{linsys}$
    \begin{solution}
    \begin{equation*}
      \begin{amat}{3}
        3 &2  &1  &1 \\
        1 &-1 &1  &2 \\
        5 &5  &1  &0
      \end{amat}
      \grstep[-(5/3)\rho_1+\rho_3]{-(1/3)\rho_1+\rho_2}
      \begin{amat}{3}
        3 &2    &1     &1 \\
        0 &-5/3 &2/3   &5/3 \\
        0 &5/3  &-2/3  &-5/3
      \end{amat}
      \grstep{\rho_2+\rho_3}
      \begin{amat}{3}
        3 &2    &1     &1 \\
        0 &-5/3 &2/3   &5/3 \\
        0 &0    &0     &0
      \end{amat}
    \end{equation*}
    The solution set is this.
    \begin{equation*}
      \set{\colvec{x \\ y \\ z}=\colvec{1 \\ -1 \\ 0}
                                 +\colvec{-3/5 \\ 2/5 \\ 1}z
                       \suchthat z\in\Re}
    \end{equation*}    
    \end{solution}

  \item
    $\begin{linsys}{4}
      x  &+ &y  &- &2z &= &0 \\
      x  &- &y  &  &   &= &-3 \\
      3x &- &y  &- &2z &= &-6  \\
         &  &2y &- &2z &= &3  
    \end{linsys}$
    \begin{solution}
    \begin{equation*}
      \begin{amat}{3}
        1 &1   &-2 &0 \\
        1 &-1  &0  &3  \\
        3 &-1  &-2 &-6 \\
        0 &2   &-2 &3
      \end{amat}
      \grstep[-3\rho_1+\rho_3]{-\rho_1+\rho_2}
      \begin{amat}{3}
        1 &1   &-2 &0 \\
        0 &-2  &2  &-3  \\
        0 &-4  &4  &-6 \\
        0 &2   &-2 &3
      \end{amat}
      \grstep[\rho_2+\rho_4]{-2\rho_2+\rho_3}
      \begin{amat}{3}
        1 &1   &-2 &0 \\
        0 &-2  &2  &-3  \\
        0 &0   &0  &0 \\
        0 &0   &0  &0
      \end{amat}
    \end{equation*}
    The solution set is this.
    \begin{equation*}
      \set{\colvec{-3/2 \\ 3/2 \\ 0}
           +\colvec{1 \\ 1 \\ 1}z
           \suchthat z\in\Re}
    \end{equation*}    
    \end{solution}

  \item
    $\begin{linsys}{5}
      2x  &- &y  &- &z &+ &w &= &4 \\
       x  &+ &y  &+ &z &  &  &= &-1 
    \end{linsys}$
    \begin{solution}
    \begin{equation*}
      \begin{amat}{4}
        2 &-1 &-1 &1 &4 \\
        1 &1  &1  &0 &-1
      \end{amat}
      \grstep{-(1/2)\rho_1+\rho_2}
      \begin{amat}{4}
        2 &-1   &-1   &1    &4 \\
        0 &3/2  &3/2  &-1/2 &-3
      \end{amat}
    \end{equation*}
    Here is the solution set.
    \begin{equation*}
      \set{\colvec{1 \\ -2 \\ 0 \\ 0}
             +\colvec{0 \\ -1 \\ 1 \\ 0}z
              \colvec{-1/3 \\ 1/3 \\ 0 \\ 1}w
            \suchthat z,w\in\Re}
    \end{equation*}    
    \end{solution}
  % \item 
  %   $\begin{linsys}{3}
  %     x  &+ &y  &- &2z &= &0 \\
  %     x  &- &y  &  &   &= &-3 \\
  %     3x &- &y  &- &2z &= &0    
  %   \end{linsys}$
  \end{parts}

\question 
For the second system in the first question, 
state the associated homogeneous system and give the solution set
of that homogeneous system.
\begin{solution}
    \begin{equation*}
      \begin{amat}{3}
        1 &1   &-2 &0 \\
        1 &-1  &0  &0  \\
        3 &-1  &-2 &0 \\
        0 &2   &-2 &0
      \end{amat}
      \grstep[-3\rho_1+\rho_3]{-\rho_1+\rho_2}
      \begin{amat}{3}
        1 &1   &-2 &0 \\
        0 &-2  &2  &0  \\
        0 &-4  &4  &0 \\
        0 &2   &-2 &0
      \end{amat}
      \grstep[\rho_2+\rho_4]{-2\rho_2+\rho_3}
      \begin{amat}{3}
        1 &1   &-2 &0 \\
        0 &-2  &2  &0  \\
        0 &0   &0  &0 \\
        0 &0   &0  &0
      \end{amat}
    \end{equation*}
    The solution set is this.
    \begin{equation*}
      \set{
           \colvec{1 \\ 1 \\ 1}z
           \suchthat z\in\Re}
    \end{equation*}
\end{solution}


\question 
Do Gauss-Jordan reduction.
\begin{equation*}
    \begin{linsys}{3}
      x  &+ &y &- &z &= &3 \\
      2x &- &y  &- &z &= &1 \\
      3x &+  &y  &+ &2z &= &0
    \end{linsys}
  % \item
  %   $\begin{linsys}{3}
  %     x  &+ &y  &+ &2z  &= &0 \\
  %     2x  &- &y  &+  &z &= &1 \\
  %     4x &+ &y  &+ &5z &= &1  
  %   \end{linsys}$
\end{equation*}
\begin{solution}
    \begin{multline*}
      \begin{amat}{3}
        1 &1  &-1 &3 \\
        2 &-1 &-1 &1 \\
        3 &1  &2  &0
      \end{amat}
      \grstep[-3\rho_1+\rho_3]{-2\rho_1+\rho_2}
      \begin{amat}{3}
        1 &1  &-1 &3 \\
        0 &-3 &-1 &1 \\
        0 &-2  &5  &-9
      \end{amat}
      \grstep{-(2/3)\rho_2+\rho_3}
      \begin{amat}{3}
        1 &1  &-1    &3 \\
        0 &-3 &-1    &1 \\
        0 &0  &13/3  &-17/3
      \end{amat}                                 \\
      \begin{split}
       &\grstep[(3/13)\rho_3]{-(1/3)\rho_2}
      \begin{amat}{3}
        1 &1  &-1    &3 \\
        0 &1  &-1/3    &5/3 \\
        0 &0  &1      &-17/13
      \end{amat}
      \grstep[(1/3)\rho_3+\rho_2]{\rho_3+\rho_1}
      \begin{amat}{3}
        1 &1  &0    &22/13 \\
        0 &1  &0    &16/13 \\
        0 &0  &1      &-17/13
      \end{amat}                                \\
      &\grstep{-\rho_2+\rho_1}
      \begin{amat}{3}
        1 &0  &0    &6/13 \\
        0 &1  &0    &16/13 \\
        0 &0  &1      &-17/13
      \end{amat}
      \end{split}
    \end{multline*}
\end{solution}

\question
Verify that $V=\set{a_0+a_1x+a_2x^2 \suchthat \text{$a_0,a_1,a_2\in\R$ and $a_2=3a_0$}}$ 
is a vector space by checking all ten conditions.
\begin{solution}
Consider two quadratic polynomials, $a_0+a_1x+a_2x^2$ and
$b_0+b_1x+b_2x^2$ subject to the conditions that 
the quadratic coefficient is three times the constant coefficient, 
$a_2=3a_0$ and~$b_2=3b_0$.
Their sum is  $(a_0+b_0)+(a_1+b_1)x+(a_2+b_2)x^2$ and 
adding the conditions gives $a_2+b_2=3(a_0+b_0)$,
so the quadratic coefficent of the sum is three times its constant coefficient.

The sum commutes because the sum of any two quadratic polynomials commutes,
with or without the conditions.
Specifically,
\begin{equation*}
(a_0+a_1x+a_2x^2) + (b_0+b_1x+b_2x^2)
 = (a_0+b_0)+(a_1+b_1)x+(a_2+b_2)x^2
\end{equation*}
while
\begin{equation*}
 (b_0+b_1x+b_2x^2)+(a_0+a_1x+a_2x^2) 
 = (b_0+a_0)+(b_1+a_1)x+(b_2+a_2)x^2
\end{equation*} 
and $a_0+b_0=b_0+a_0$ as that is a sum of real numbers.
The other coefficients in the sum are similar.

Addition is associative since the addition of any quadratic polynomials 
is associative.
For more, compare
\begin{multline*}
 \big((a_0+a_1x+a_2x^2) + (b_0+b_1x+b_2x^2)\big)+ (c_0+c_1x+c_2x^2)   \\
 = \big((a_0+b_0)+c_0\big)+\big((a_1+b_1)+c_1\big)x+\big((a_2+b_2)+c_2\big)x^2
\end{multline*}
with this.
\begin{multline*}
 (a_0+a_1x+a_2x^2) + \big((b_0+b_1x+b_2x^2)+ (c_0+c_1x+c_2x^2)\big)     \\
 = \big(a_0+(b_0+c_0)\big))+\big(a_1+(b_1+c_1)\big)x+\big(a_2+(b_2+c_2)\big)x^2
\end{multline*}
The constant coefficient expressions are equal
$(a_0+b_0)+c_0=a_0+(b_0+c_0)$
as that is a sum of real numbers, and
the other coefficients work the same way.

As to the zero element, consider $0+0x+0x^2$, which we can 
alternatively write as~$0$.
Obviously $0+(a_0+a_1x+a_2x^2)=a_0+a_1x+a_2x^2$.
Note also that in $0+0x+0x^2$ the quadratic coefficient is three
times the constant coefficient.

For the existence of an additive inverse, start with a member
of the set $V$,  
a quadratic polynomial $a_0+a_1x+a_2x^2$ subject to the condition that its
quadratic coefficient is three times its constant coefficient, $a_2=3a_0$.
Consider the quadratic polynomial $-a_0=a_1x-a_2x^2$.
Observe that it also satisfies that that its
quadratic coefficient is three times its constant coefficient, 
$-a_2=3\cdot(-a_0)$.
Clearly the two are additive inverses, 
$(-a_0-a_1x-a_2x^2)+(a_0+a_1x+a_2x^2)=0+0x+0x^2$.

The sixth condition, 
that the collection~$V$ is closed under scalar multiplication,
follows on starting with $a_0+a_1x+a_2x^2$ subject to $a_2=3a_0$ and
multiplying by the scalar $r\in\R$.
The result is a quadratic polynomial $(ra_0)+(ra_1)x+(ra_2)x^2$, and that
the quadratic coefficient is three times the constant coefficient
$ra_2=3ra_0$ immediately follows from the condition on the starting
polynomial.

Addition of scalars distributes over scalar multiplication as here.
\begin{align*}
  (r+s)\cdot(a_0+a_1x+a_2x^2)
    &=(r+s)a_0+(r+s)a_1x+(r+s)a_2x^2             \\
    &=(ra_0+sa_0)+(ra_1+sa_1)x+(ra_2+sa_2)x^2     \\
    &=(ra_0+ra_1x+ra_2x^2)+(sa_0+sa_1x+sa_2x^2)  \\
    &=r\cdot(a_0+a_1x+a_2x^2)+s\cdot(a_0+a_1x+a_2x^2)
\end{align*}

Similarly, scalar multiplication distributes over vector addition.
\begin{align*}
  r\cdot\big((a_0+a_1x+a_2x^2)+(b_0+b_1x+b_2x^2)\big)
    &=r\cdot \big((a_0+b_0)+(a_1+b_1)x+(a_2+b_2)x^2\big)   \\
    &=\big(r(a_0+b_0)+r(a_1+b_1)x+r(a_2+b_2)x^2\big)   \\
    &=(ra_0+rb_0)+(ra_1+rb_1)x+(ra_2+rb_2)x^2     \\
    &=(ra_0+ra_1x+ra_2x^2)+(rb_0+rb_1x+rb_2x^2)  \\
    &=r\cdot(a_0+a_1x+a_2x^2)+r\cdot(b_0+b_1x+b_2x^2)
\end{align*}

Similarly, scalar multiplication distributes over vector addition.
\begin{align*}
  (rs)\cdot(a_0+a_1x+a_2x^2)
    &=(rs)a_0+(rs)a_1x+(rs)a_2x^2          \\
    &=r\cdot (sa_0+sa_1x+sa_2x^2)   \\
    &=r\cdot\big( s\cdot (a_0+a_1x+a_2x^2) \big) 
\end{align*}

Finally, multiplication by~$1\in\R$ is
$1\cdot(a_0+a_1x+a_2x^2)=(1\cdot a_0)+(1\cdot a_1)x+(1\cdot a_2)x^2
=a_0+a_1x+a_2x^2$.
\end{solution}
\end{questions}
\end{document}
