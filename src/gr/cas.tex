% Chapter 1, Topic from _Linear Algebra_ Jim Hefferon
%  http://joshua.smcvt.edu/linalg.html
%  2001-Jun-09
\topic{Computer Algebra Systems}
\index{computer algebra systems|(}
The linear systems in this chapter are small enough that their
solution by hand is easy.
For large systems, including those involving thousands of equations,
we need a computer.
There are special purpose programs such as LINPACK\index{LINPACK} for this.
Also popular are general purpose computer algebra systems
including \Maple,\index{Maple} 
\textit{Mathematica},\index{Mathematica} or \textit{MATLAB},\index{MATLAB} 
and \Sage\index{Sage}.

For example, in the Topic on Networks, we need to solve this.
\begin{equation*}
  \begin{linsys}{7}
    i_0  &-  &i_1  &-  &i_2  &   &    &  &    &   &    &  &    &=  &0  \\
         &   &i_1  &   &     &-  &i_3 &  &    &-  &i_5 &  &    &=  &0  \\
         &   &     &   &i_2  &   &    &- &i_4 &+  &i_5 &  &    &=  &0  \\
         &   &     &   &     &   &i_3 &+ &i_4 &   &    &- &i_6 &=  &0  \\
         &   &5i_1 &   &     &+  &10i_3  &  & &   &    &  &    &=  &10  \\
         &   &     &   &2i_2 &   &    &+ &4i_4 &  &    &  &    &=  &10  \\
         &   &5i_1 &-  &2i_2 &   &    &  &    &+  &50i_5 &&    &=  &0   
  \end{linsys}
\end{equation*}
Doing this by hand would take time and be error-prone.
A computer is better.

Here is that system solved with \textit{Sage}.
(There are many ways to do this; the one here has the advantage of simplicity.)
\begin{lstlisting}
sage: var('i0,i1,i2,i3,i4,i5,i6')
(i0, i1, i2, i3, i4, i5, i6)
sage: network_system=[i0-i1-i2==0, i1-i3-i5==0, 
....:       i2-i4+i5==0, i3+i4-i6==0, 5*i1+10*i3==10,
....:       2*i2+4*i4==10, 5*i1-2*i2+50*i5==0]
sage: solve(network_system, i0,i1,i2,i3,i4,i5,i6)     
[[i0 == (7/3), i1 == (2/3), i2 == (5/3), i3 == (2/3), 
      i4 == (5/3), i5 == 0, i6 == (7/3)]] 
\end{lstlisting}
Magic.

Here is the same system solved under Maple.
We enter the array of coefficients 
and the vector of constants,
and then we get the solution.
\begin{lstlisting}
> A:=array( [[1,-1,-1,0,0,0,0],
             [0,1,0,-1,0,-1,0],
             [0,0,1,0,-1,1,0],
             [0,0,0,1,1,0,-1],
             [0,5,0,10,0,0,0],
             [0,0,2,0,4,0,0],
             [0,5,-2,0,0,50,0]] );
> u:=array( [0,0,0,0,10,10,0] );
> linsolve(A,u);
      7  2  5  2  5     7
    [ -, -, -, -, -, 0, - ]
      3  3  3  3  3     3
\end{lstlisting}

If a system has infinitely many solutions then 
the program will return a parametrization.


\begin{exercises}
  \item 
    Use the computer to solve the two problems that opened this
    chapter.
    \begin{exparts}
      \partsitem This is the Statics problem.
         \begin{align*}
            40h+15c  &= 100  \\
            25c      &= 50+50h
         \end{align*}
      \partsitem This is the Chemistry problem.
         \begin{align*} 
             7h      &= 7j  \\
             8h +1i  &= 5j+2k  \\
             1i      &= 3j  \\
             3i      &= 6j+1k
         \end{align*}
    \end{exparts}
    \begin{answer}
      \begin{exparts}
        \partsitem 
\Sage{} does this.
\begin{lstlisting}
sage: var('h,c')
(h, c)
sage: statics = [40*h + 15*c == 100,
....:            25*c == 50 + 50*h]
sage: solve(statics, h,c)
[[h == 1, c == 4]] 
\end{lstlisting}
Other Computer Algebra Systems have similar commands.
These \Maple{} commands
\begin{lstlisting}
> A:=array( [[40,15],
             [-50,25]] );
> u:=array([100,50]);
> linsolve(A,u);
\end{lstlisting}
           get the answer \lstinline![1,4]!.
        \partsitem Here there is a free variable.
\Sage{} gives this.
\begin{lstlisting}
sage: var('h,i,j,k')
(h, i, j, k)
sage: chemistry = [7*h == 7*j,
....:              8*h + 1*i == 5*j + 2*k,
....:              1*i == 3*j,
....:              3*i == 6*j + 1*k]
sage: solve(chemistry, h,i,j,k)
[[h == 1/3*r1, i == r1, j == 1/3*r1, k == r1]]          
\end{lstlisting}
Similarly, this \Maple{} session 
\begin{lstlisting}
> A:=array( [[7,0,-7,0],
             [8,1,-5,2],
             [0,1,-3,0],
             [0,3,-6,-1]] );
> u:=array([0,0,0,0]);
> linsolve(A,u);
\end{lstlisting}
         prompts the same reply (but with parameter $t_1$).
      \end{exparts}
    \end{answer}
\item 
    Use the computer to solve these systems from the
    first subsection,
    or conclude `many solutions' or `no solutions'.
    \begin{exparts*}
      \partsitem \(
               \begin{linsys}[t]{2}
                  2x  &+  &2y  &=  &5  \\
                   x  &-  &4y  &=  &0  
               \end{linsys}
             \)
      \partsitem \(
               \begin{linsys}[t]{2}
                  -x  &+  &y   &=  &1  \\
                   x  &+  &y   &=  &2  
               \end{linsys}
             \) 
      \partsitem  \(
               \begin{linsys}[t]{3}
                   x  &-  &3y  &+  &z  &=  &1  \\
                   x  &+  &y   &+  &2z &=  &14 
                \end{linsys}
             \) 
      \partsitem  \(
               \begin{linsys}[t]{2}
                  -x  &-  &y   &=  &1  \\
                 -3x  &-  &3y  &=  &2  
               \end{linsys}
             \) 
      \partsitem  \(
               \begin{linsys}[t]{3}
                      &   &4y  &+  &z  &=  &20 \\
                  2x  &-  &2y  &+  &z  &=  &0  \\
                   x  &   &    &+  &z  &=  &5  \\
                   x  &+  &y   &-  &z  &=  &10 
                \end{linsys}
             \)
      \partsitem \( \begin{linsys}[t]{4}
                 2x  &   &   &+  &z  &+  &w  &=  &5  \\
                     &   &y  &   &   &-  &w  &=  &-1 \\
                 3x  &   &   &-  &z  &-  &w  &=  &0  \\
                 4x  &+  &y  &+  &2z &+  &w  &=  &9  
               \end{linsys}
            \)
    \end{exparts*}
    \begin{answer}
      \begin{exparts}
        \partsitem The answer is \( x=2 \) and \( y=1/2 \).
          A \Sage{} session does this.
\begin{lstlisting}
sage: var('x,y')
(x, y)
sage: system = [2*x + 2*y == 5,
....:              x - 4*y == 0]
sage: solve(system, x,y)
[[x == 2, y == (1/2)]]          
\end{lstlisting}
          A \Maple{} session 
\begin{lstlisting}
> A:=array( [[2,2],
             [1,-4]] );
> u:=array([5,0]);
> linsolve(A,u);            
\end{lstlisting}
          gets the same answer, of course.
        \partsitem The answer is \( x=1/2 \) and \( y=3/2 \). 
\begin{lstlisting}
sage: var('x,y')
(x, y)
sage: system = [-1*x + y == 1,
....:              x + y == 2]
sage: solve(system, x,y)
[[x == (1/2), y == (3/2)]]          
\end{lstlisting}
        \partsitem This system has infinitely many solutions.
           In the first subsection, with $z$ as a parameter, 
           we got $x=(43-7z)/4$ and $y=(13-z)/4$.
           \Sage{} gets the same.
\begin{lstlisting}
sage: var('x,y,z')
(x, y, z)
sage: system = [x - 3*y + z == 1,
....:           x + y + 2*z == 14]
sage: solve(system, x,y)
[[x == -7/4*z + 43/4, y == -1/4*z + 13/4]]             
\end{lstlisting}
           \Maple{} responds with $(-12+7t_1, t_1, 13-4t_1)$,
           preferring $y$ as a parameter.
        \partsitem There is no solution to this system.
           \Sage{} gives an empty list.
\begin{lstlisting}
sage: var('x,y')
(x, y)
sage: system = [-1*x - y == 1,
....:           -3*x - 3*y == 2]
sage: solve(system, x,y)
[]          
\end{lstlisting}
           Similarly, 
           When the array $A$ and vector $u$ are given to \Maple{}
           and it is asked to \lstinline!linsolve(A,u)!, 
           it returns no result at all; that is, it responds with
           no solutions.
        \partsitem \Sage{} finds 
\begin{lstlisting}
sage: var('x,y,z')
(x, y, z)
sage: system = [       4*y + z == 20,
....:            2*x - 2*y + z == 0,
....:             x +        z == 5,
....:             x +    y - z == 10]
sage: solve(system, x,y,z)
[[x == 5, y == 5, z == 0]]          
\end{lstlisting}
           that the solutions is \( (x,y,z)=(5,5,0) \).
        \partsitem There are infinitely many solutions.
           \Sage{} does this.
\begin{lstlisting}
sage: var('x,y,z,w')
(x, y, z, w)
sage: system = [ 2*x +        z + w == 5,
....:                   y -       w == -1,
....:            3*x -        z - w == 0,
....:            4*x +  y + 2*z + w == 9]
sage: solve(system, x,y,z,w)
[[x == 1, y == r2 - 1, z == -r2 + 3, w == r2]]          
\end{lstlisting}
           \Maple{} gives $(1, -1+t_1, 3-t_1, t_1)$.
      \end{exparts}
    \end{answer}
  \item 
    Use the computer to solve these systems from the second subsection.
    \begin{exparts*}
      \partsitem \( \begin{linsys}[t]{2}
                  3x  &+  &6y  &=  &18  \\
                   x  &+  &2y  &=  &6   
                   \end{linsys}  \)
      \partsitem \( \begin{linsys}[t]{2}
                   x  &+  &y   &=  &1  \\
                   x  &-  &y   &=  &-1   
                    \end{linsys}  \)
      \partsitem \( \begin{linsys}[t]{3}
                   x_1  &   &     &+  &x_3   &=  &4  \\
                   x_1  &-  &x_2  &+  &2x_3  &=  &5  \\
                  4x_1  &-  &x_2  &+  &5x_3  &=  &17  
                   \end{linsys}  \)
      \partsitem \( \begin{linsys}[t]{3}
                   2a   &+  &b    &-  &c     &=  &2  \\
                   2a   &   &     &+  &c     &=  &3  \\
                    a   &-  &b    &   &      &=  &0   
                    \end{linsys}  \)
      \partsitem \( \begin{linsys}[t]{4}
                     x  &+  &2y   &-   &z   &    &    &=  &3  \\
                    2x  &+  &y    &    &    &+   &w   &=  &4  \\
                     x  &-  &y    &+   &z   &+   &w   &=  &1  
                    \end{linsys}  \)
      \partsitem \( \begin{linsys}[t]{4}
                     x  &   &     &+   &z   &+   &w   &=  &4  \\
                    2x  &+  &y    &    &    &-   &w   &=  &2  \\
                    3x  &+  &y    &+   &z   &    &    &=  &7  
                     \end{linsys}  \)
    \end{exparts*}
    \begin{answer}
      \begin{exparts}
        \partsitem This system has infinitely many solutions. 
              In the second subsection we gave the solution set as
              \begin{equation*}
              \set{\colvec{6 \\ 0}+\colvec{-2 \\ 1}y
                      \suchthat y\in\Re}
              \end{equation*}
              and \Sage{} finds 
\begin{lstlisting}
sage: var('x,y')
(x, y)
sage: system = [ 3*x + 6*y == 18,
....:              x + 2*y == 6]
sage: solve(system, x,y)
[[x == -2*r3 + 6, y == r3]]                
\end{lstlisting}
              while \Maple{} responds with $(6-2t_1,t_1)$.
        \partsitem The solution set has only one member.
          \begin{equation*}
             \set{\colvec{0 \\ 1} }
          \end{equation*}
          \Sage{} gives this.
\begin{lstlisting}
sage: var('x,y')
(x, y)
sage: system = [ x + y == 1,
....:            x - y == -1]
sage: solve(system, x,y)
[[x == 0, y == 1]]            
\end{lstlisting}
        \partsitem This system's solution set is infinite
          \begin{equation*}
            \set{\colvec{4 \\ -1 \\ 0}+\colvec{-1 \\ 1 \\ 1}x_3
                             \suchthat x_3\in\Re}
          \end{equation*}
          \Sage{} gives this
\begin{lstlisting}
sage: var('x1,x2,x3')
(x1, x2, x3)
sage: system = [   x1 +        x3 == 4,
....:              x1 - x2 + 2*x3 == 5,
....:            4*x1 - x2 + 5*x3 == 17]
sage: solve(system, x1,x2,x3)
[[x1 == -r4 + 4, x2 == r4 - 1, x3 == r4]]            
\end{lstlisting}
          and Maple gives $(t_1,-t_1+3,-t_1+4)$.
        \partsitem There is a unique solution
           \begin{equation*}
             \set{\colvec{1 \\ 1 \\ 1}}
           \end{equation*}
           and \Sage{} finds it.
\begin{lstlisting}
sage: var('a,b,c')
(a, b, c)
sage: system = [ 2*a +  b - c == 2,
....:            2*a +      c == 3,
....:              a - b      == 0]
sage: solve(system, a,b,c)
[[a == 1, b == 1, c == 1]]             
\end{lstlisting}
        \partsitem This system has infinitely many solutions; in the 
           second subsection we described the solution set with
           two parameters.
           \begin{equation*}
             \set{\colvec{5/3 \\ 2/3 \\ 0 \\ 0}
                  +\colvec{-1/3 \\ 2/3 \\ 1 \\ 0}z
                  +\colvec{-2/3 \\ 1/3 \\ 0 \\ 1}w
                  \suchthat z,w\in\Re}
           \end{equation*}
           \Sage{} does the same.
\begin{lstlisting}
sage: var('x,y,z,w')
(x, y, z, w)
sage: system = [   x + 2*y - z     == 3,
....:            2*x +   y     + w == 4,
....:              x -   y + z + w == 1]
sage: solve(system, x,y,z,w)
[[x == r6, y == -r5 - 2*r6 + 4, z == -2*r5 - 3*r6 + 5, w == r5]]             
\end{lstlisting}
           as does \Maple{} $(3-2t_1+t_2,t_1,t_2,-2+3t_1-2t_2)$.
        \partsitem The solution set is empty.
\begin{lstlisting}
sage: var('x,y,z,w')
(x, y, z, w)
sage: system = [   x +      z + w == 4,
....:            2*x +  y     - w == 2,
....:            3*x +  y + z     == 7]
sage: solve(system, x,y,z,w)
[]          
\end{lstlisting}
      \end{exparts}
    \end{answer}
  \item 
    What does the computer give for the solution of the general
    $\nbyn{2}$  system?
    \begin{equation*}
      \begin{linsys}{2}
        ax  &+  &cy  &=  &p  \\
        bx  &+  &dy  &=  &q
      \end{linsys}
    \end{equation*}
    \begin{answer}
      \Sage{} does this.
\begin{lstlisting}
sage: var('x,y,a,b,c,d,p,q')
(x, y, a, b, c, d, p, q)
sage: system = [ a*x +  c*y == p,
....:            b*x +  d*y == q]
sage: solve(system, x,y)
[[x == -(d*p - c*q)/(b*c - a*d), y == (b*p - a*q)/(b*c - a*d)]]        
\end{lstlisting}
       In response to this prompting
\begin{lstlisting}
> A:=array( [[a,c],
             [b,d]] );
> u:=array([p,q]);
> linsolve(A,u);
\end{lstlisting}
      \Maple{} gave this reply.
      \begin{equation*}
        \bigl[-\frac{-d\,p+q\,c}{-b\,c+a\,d},
          \frac{-b\,p+a\,q}{-b\,c+a\,d}\bigr]
      \end{equation*}
    \end{answer}
\end{exercises}
\index{computer algebra systems|)}
\endinput


