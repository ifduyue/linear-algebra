\documentclass{article}
\usepackage{syllabus}
\usepackage{url}
\usepackage{hyperref}
\hypersetup{
    bookmarks=false,         % show bookmarks bar?
    unicode=true,          % non-Latin characters in Acrobat’s bookmarks
    pdftoolbar=false,        % show Acrobat’s toolbar?
    pdfmenubar=false,        % show Acrobat’s menu?
    pdffitwindow=false,     % window fit to page when opened
    pdfstartview={FitH},    % fits the width of the page to the window
    pdftitle={Syllabus, Linear Algebra},    % title
    pdfauthor={Jim Hefferon},     % author
    pdfsubject={Syllabus, Linear Algebra},   % subject of the document
    pdfcreator={LaTeX},   % creator of the document
    pdfproducer={LaTeX}, % producer of the document
    pdfkeywords={Linear Algebra text, Hefferon, syllabus}, % list of keywords
    pdfnewwindow=false,      % links in new PDF window
    colorlinks=true,       % false: boxed links; true: colored links
    linkcolor=blue,          % color of internal links (change box color with linkbordercolor)
    citecolor=blue,        % color of links to bibliography
    filecolor=blue,         % color of file links
    urlcolor=blue        % color of external links
}

\begin{document} \thispagestyle{empty}
\begin{heading}
  Mathematics 213, 2021-Fall           \\[1.25ex]
  {\large\textbf{Linear Algebra}}   \\[1.25ex]
  J~Hef{}feron, \href{https://www.smcvt.edu/about-smc/directories/employee-directory/jim-hefferon/}{St Michael's College} 
\end{heading}


\section{Course Description}
We will start by solving systems of linear equations.
We will then study vector spaces, including linear independence and bases.
Next comes maps between vector spaces, linear maps, and
their computational aspect, matrices.
We will finish by learning about determinants, and then
eigenvalues and eigenvectors.


\section{Prerequisites}
Calculus~I.


\section{Text}
\textsl{Linear Algebra}, by me, edition~4, ISBN-13:~978-1944325114. 
A copy called \texttt{book.pdf} is in the Canvas files area.
Also there are the answers, in \texttt{jhanswer.pdf}.
If you download the two into 
the same directory then clicking on a question number takes you to its 
answer, and vice versa.
If you prefer a paper copy (as I do) then the bookstore carries them,
or you can get one from online booksellers; see
\href{https://hefferon.net/linearalgebra/}{the text's home page}.

Two more resources:~there are
\href{https://www.youtube.com/watch?v=37tn0z9dSDo&list=PLwF3A0R8OzMoMlE1-SaEh8h9VqUlO-r52}{matching videos},
and Canvas has a \texttt{slides.zip} of the classroom slides.
(The book, the answers, and the slides are also  
available at the home page.)



\section{Schedule}
This plan presumes that you have already studied the material of 
Section~One.II, the elements of vectors.
\newcommand{\classday}[1]{\textsc{#1}}
\newcommand{\colwidth}{1.0in}
\begin{center} \small   % George Ashline's
   \begin{tabular}{r|*{2}{p{\colwidth}}l}
      \multicolumn{1}{r}{\textsc{Week}}  
       &\textsc{Monday}          
       &\textsc{Wednesday}            
       &\textsc{Friday}        \\ \hline
       1    &One.I.1         &One.I.1, 2        &One.I.2, 3         \\
       2    &One.I.3         &One.III.1          &One.III.2         \\
       3    &Two.I.1         &Two.I.1, 2         &Two.I.2         \\
       4    &Two.II.1         &Two.III.1         &Two.III.2         \\
       5    &Two.III.2        &Two.III.2, 3         &Two.III.3        \\
       6    &\classday{Exam}   &Three.I.1         &Three.I.1       \\
       7    &Three.I.2         &Three.I.2          &Three.II.1         \\
       8    &Three.II.1        &Three.II.2          &Three.II.2          \\
       9    &Three.III.1       &Three.III.2         &Three.IV.1, 2       \\
      10    &Three.IV.2, 3   &Three.IV.4          &Three.V.1          \\
      11    &Three.V.1       &Three.V.2            &Four.I.1         \\
      12    &\classday{Exam}  &Four.I.2            &Four.III.1       \\
      13    &Five.II.1    &\multicolumn{2}{c}{\classday{--Thanksgiving break--}} \\
      14    &Five.II.1, 2     &Five.II.2          &Five.II.3        
   \end{tabular}
\end{center}

\section{Expectations}
You must come to class, if you are able, 
actively participate,
and do the assigned homeworks 
(they are listed on the back page of this sheet).
Arrange your study schedule so that you try each homework before the next class.
We will do a lot of working through problems and you
will get the most out of this if you have tried the questions already.





\section{Grading}
We will have two midterm exams and a final exam.  
There will also be take-home problem sets, 
in even-numbered weeks (when that week does
not have an exam).
That's four grades.
Each counts equally towards the overall letter grade.


% \credithoursdisclaimer %

\gradedisclaimer %


% =========================
\clearpage
\section{Homework}
These are the assigned question numbers for each subsection that we will cover.

Very important:~learning math is about doing problems.
That is how you get better.
While the complete answers to all questions are available to you,
it is crucial that before consulting an answer, you must give 
the question a good try on your own.
If you read the question and just immediately read the answer, then 
you will not learn.
\begin{center}\small
\begin{tabular}{l|l}
  \multicolumn{1}{c}{\textsc{Subsection}} &\multicolumn{1}{c}{\textsc{Exercises}} \\
  \hline
  One.I.1   &17, 19, 21, 22, 24, 27  \\
  One.I.2   &15--18, 21, 25, 29  \\
  One.I.3   &15, 17, 18, 20, 21  \\
  One.III.1 &8, 10, 12, 14  \\
  One.III.2 &11, 14, 20--22  \\[0.5ex]
  Two.I.1 &18, 19, 21, 22, 28, 29  \\
  Two.I.2 &20, 21, 23, 26, 27, 28  \\
  Two.II.1 &21, 22, 24, 25, 28  \\
  Two.III.1 &20--22, 26, 27, 31  \\
  Two.III.2 &15, 17, 20, 21   \\
  Two.III.3 &17--21, 23  \\[0.5ex]
  Three.I.1 &12, 13, 16, 17, 19, 21, 33  \\
  Three.I.2 &10, 12, 14--16, 20  \\
  Three.II.1 &18--20, 25, 26, 30, 31  \\
  Three.II.2 &21, 23--26, 30, 31, 35   \\
  Three.III.1 &13, 17, 19, 21, 22, 27  \\ 
  Three.III.2 &14--16, 19, 22     \\
  Three.IV.1 &8, 11 a--e   \\
  Three.IV.2 &14, 15, 17--19, 21, 28  \\
  Three.IV.3 &24, 25, 28, 31 \\
  Three.IV.4 &13--16, 18, 29 \\
  Three.V.1 &7, 9, 10, 12, 17  \\
  Three.V.2 &11, 13, 16--18          \\[0.5ex]
  Four.I.1 &1, 3, 4, 6, 7, 9  \\
  Four.I.2 &8, 9, 13, 18,  \\
  Four.III.1 &10, 11, 13, 15, 16, 19  \\[0.5ex]
  Five.II.1 &7, 9--11, 20 \\
  Five.II.2 &6--8, 11, 14  \\
  Five.II.3 &24, 27, 31, 33, 34  \\
\end{tabular}
\end{center}

\noindent 
(The text has checkmarked exercises for people reading the material on
their own.
I have not asked all those questions here, because
people on their own benefit from doing more 
exercises, as they have less support.)
\end{document}


