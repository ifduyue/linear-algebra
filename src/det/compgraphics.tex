% Chapter 4, Topic _Linear Algebra_ Jim Hefferon
%  http://joshua.smcvt.edu/linearalgebra
%  2020-Apr-08
\topic{Computer graphics}
\index{computer graphics|(}
The prior topic on Projective Geometry gives this model of 
how our eye, or a camera, sees the world.
\begin{center} 
  \vcenteredhbox{\includegraphics{det/mp/ch4.14}}
\end{center}
All of the points on a line through the origin project to the same 
spot.

In that topic we defined that
for any nonzero vector $\vec{v}\in\Re^3$, the associated 
\definend{point $p$ in the projective plane}\index{point!in projective plane} 
is the set $\set{k\vec{v}\suchthat \text{$k\in\Re$ and $k\neq 0$}}$.
This is the collection of nonzero vectors lying on the same line through the
origin as $\vec{v}$.

To describe a projective point we can give any representative member 
of the line. 
Thus these each represent the same projective point.
\begin{equation*}
  \colvec[r]{1 \\ 2 \\ 3}
  \qquad
  \colvec{1/3 \\ 2/3 \\ 1}
  \qquad
  \colvec[r]{-2 \\ -4 \\ -6}
\end{equation*} 
Each is a
\definend{homogeneous coordinate vector}\index{coordinates!homogeneous}%
\index{homogeneous coordinate vector}\index{vector!homogeneous coordinate}
for the point~$p$. 
Two homogeneous coordinate vectors (which are by definition nonzero)
\begin{equation*}
  \tilde{p}_1=\colvec{a_1 \\ b_1 \\ c_1}
  \qquad
  \tilde{p}_2=\colvec{a_2 \\ b_2 \\ c_2}
\end{equation*}
represent the same projective point if there is a scaling factor
$s\neq 0$ so that $s\tilde{p}_1=\tilde{p}_2$.

Of the infinitely many possible representatives,
often we use the one whose third component is~$1$.
This amounts to projecting onto the plane $z=1$.
\begin{center} 
  \vcenteredhbox{\includegraphics{det/mp/ch4.18}}
\end{center}

In this topic we will show how to use these ideas to perform the actions that 
we see in computer graphics.
For that we will take the prior picture and redraw it without the sphere,
with a movie projector at the origin, and with plane $z=1$ looking like
a movie theater screen.
\begin{center} 
  \vcenteredhbox{\includegraphics{det/mp/ch4.60}}
\end{center}
Now the projector associates vectors in three-space with
vectors in the screen plane.
\begin{equation*}
  \tilde{p}=\colvec{a \\ b \\ c} 
  \mapsto 
  p=\colvec{x \\ y}=\colvec{a/c \\ b/c}
\end{equation*}

We can adapt the things we have already seen about matrices to perform 
the transformations.
Rotation is an example.
This matrix rotates in the plane~$z=1$ about the origin by the angle~$\theta$.
\begin{equation*}
  \begin{mat}
    \cos\theta  &-\sin\theta  &0  \\
    \sin\theta  &\cos\theta   &0  \\
    0           &0            &1  
  \end{mat}
  \colvec{x \\ y \\ 1}
  =
  \colvec{cos\theta\cdot x -\sin\theta\cdot y  \\  
         \sin\theta\cdot x +\cos\theta\cdot y  \\ 
         1}
\end{equation*}
Notice that it works on any homogeneous coordinate vector;
if we apply the matrix
\begin{equation*}
  \begin{mat}
    \cos\theta  &-\sin\theta  &0  \\
    \sin\theta  &\cos\theta   &0  \\
    0           &0            &1  
  \end{mat}
  \colvec{a \\ b \\ c}
  =
  \colvec{cos\theta\cdot a -\sin\theta\cdot b  \\  
         \sin\theta\cdot a +\cos\theta\cdot b  \\ 
         c}
\end{equation*}
and then move to the $z=1$~plane
\begin{equation*}
  \colvec{cos\theta\cdot a -\sin\theta\cdot b  \\  
         \sin\theta\cdot a +\cos\theta\cdot b  \\ 
         c}
  \mapsto
  \colvec{(cos\theta\cdot a -\sin\theta\cdot b)/c  \\  
         (\sin\theta\cdot a +\cos\theta\cdot b)/c  \\ 
         1}
\end{equation*}
then we get the same result as 
if we had first moved to the plane and then applied
the matrix.
\begin{equation*}
  \begin{mat}
    \cos\theta  &-\sin\theta  &0  \\
    \sin\theta  &\cos\theta   &0  \\
    0           &0            &1  
  \end{mat}
  \colvec{a/c \\ b/c \\ 1}
  =
  \colvec{cos\theta\cdot a/c -\sin\theta\cdot b/c  \\  
         \sin\theta\cdot a/c +\cos\theta\cdot b/c  \\ 
         1}
\end{equation*}

So there is no harm in working with homogeneous coordinates.
But what is the advantage?

The computer graphic operation of translation, of sliding things from
one place to another, is not a linear transformation because it does
not leave the origin fixed.
But if we work with homogeneous coordinates then we can use matrices.
This matrix will translate points in the plane of interest by 
$t_x$ in the $x$~direction and $t_y$ in the $y$~direction.
\begin{equation*}
  \begin{mat}
    1  &0  &t_x  \\
    0  &1  &t_y  \\
    0  &0  &1  
  \end{mat}
  \colvec{a \\ b \\ c}
  =
  \colvec{a+t_x\cdot c  \\  
          b+t_y\cdot c  \\ 
          c}
  \mapsto
  \colvec{a/c+t_x  \\  
          b/c+t_y  \\ 
          1}
\end{equation*}
That is, in the plane of interest this matrix slides
$\binom{a/c}{b/c}$ to $\binom{a/c+t_x}{b/c+t_y}$.
So the homogeneous coordinates allow us to use matrices.

OK then, but what is the advantages of using these matrices?
What does the extra coordinate get us?
Suppose that we are making a movie with computer graphics.
We are at a moment where the 
camera is panning and rotating at the same time.
Every single point in the scene needs to be both translated and rotated.
Rather than have the computer perform two operations to each point,
we can just multiply the two matrices just once and
then the computer only applies one operation to each point; it multiplies
by that point the resulting matrix.
That is a tremendous speedup.

We will list some examples of the effects that we can get.
We have already talked about rotation. 
Here is the picture of rotation by twenty degrees.
\begin{center} 
  \vcenteredhbox{\includegraphics{det/mp/ch4.61}}
  \quad$\mapsto$\quad
  \vcenteredhbox{\includegraphics{det/mp/ch4.62}}
\end{center}

Next is scaling.
This matrix rescales things in the target plane 
by a factor of~$s$  in the $x$-direction, 
and  by a factor of~$t$ in the $y$~direction.
\begin{equation*}
  \begin{mat}
    s  &0  &0  \\
    0  &t  &0  \\
    0  &0  &1  
  \end{mat}
  \colvec{a/c \\ b/c \\ 1}
  =
  \colvec{s\cdot a/c  \\  
          t\cdot b/c \\ 
          1}
\end{equation*}
In this picture we rescale in the $x$~direction by a factor of $s=2.5$
and in the $y$-direction by~$t=0.75$.
\begin{center} 
  \vcenteredhbox{\includegraphics{det/mp/ch4.61}}
  \quad$\mapsto$\quad
  \vcenteredhbox{\includegraphics{det/mp/ch4.63}}
\end{center}

If we take $s=t$ then the entire shape is rescaled.
For instance, if we string together frames with $s=t=1.05$ then in
the movie it will seem that the object is getting closer to us.

We can reflect the object.
This reflects about the line $y=x$.
\begin{equation*}
  \begin{mat}
    0  &1  &0  \\
    1  &0  &0  \\
    0  &0  &1  
  \end{mat}
  \colvec{a/c \\ b/c \\ 1}
  =
  \colvec{b/c  \\  
          a/c \\ 
          1}
\end{equation*}
The dashed line here is $y=x$.
\begin{center} 
  \vcenteredhbox{\includegraphics{det/mp/ch4.61}}
  \quad$\mapsto$\quad
  \vcenteredhbox{\includegraphics{det/mp/ch4.64}}
\end{center}

This reflects about $y=-x$.
\begin{equation*}
  \begin{mat}
    0   &-1  &0  \\
    -1  &0  &0  \\
    0   &0  &1  
  \end{mat}
  \colvec{a/c \\ b/c \\ 1}
  =
  \colvec{-b/c  \\  
          -a/c \\ 
          1}
\end{equation*}
The dashed line below is $y=-x$.
\begin{center} 
  \vcenteredhbox{\includegraphics{det/mp/ch4.61}}
  \quad$\mapsto$\quad
  \vcenteredhbox{\includegraphics{det/mp/ch4.65}}
\end{center}

More complex transformations are possible.
This is a shear.
\begin{equation*}
  \begin{mat}
    1   &1  &0  \\
    0   &1  &0  \\
    0   &0  &1  
  \end{mat}
  \colvec{a/c \\ b/c \\ 1}
  =
  \colvec{a/c+b/c  \\  
          b/c \\ 
          1}
\end{equation*}
In this picture the $y$~components of points are unchanged,
but the $x$~components have added to them the value of~$y$.
\begin{center} 
  \vcenteredhbox{\includegraphics{det/mp/ch4.61}}
  \quad$\mapsto$\quad
  \vcenteredhbox{\includegraphics{det/mp/ch4.66}}
\end{center}

A major advantage of having this all be matrices is that we can do 
complex things by combining simple things.
To reflect about the line $y=-x+2$ we can find the three matrices
to slide everything to the origin, then reflect about $y=-x$, and then 
slide back.
\begin{equation*}
  \begin{mat}
    1   &0  &0  \\
    0   &1  &2  \\
    0   &0  &1  
  \end{mat}
  \begin{mat}
    0   &-1  &0  \\
    -1  &0   &0  \\
    0   &0   &1  
  \end{mat}
  \begin{mat}
    1   &0  &0  \\
    0   &1  &-2  \\
    0   &0  &1  
  \end{mat}
\end{equation*}
(Observe that, as always, the action done first is described by the matrix
on the right.
That is, the matrix on the right describes sliding all points in the plane
of interest by $2$, 
the matrix in the middle reflects about $y=-x$, 
and the matrix on the left slides all points back.)

Even more complex actions.
These are the matrices for the
general affine transformation, and the
general projective transformation.
\begin{equation*}
  \begin{mat}
    d   &e  &f  \\
    g   &h  &i  \\
    0   &0  &1  
  \end{mat}
  \qquad
  \begin{mat}
    d   &e  &f  \\
    g   &h  &i  \\
    j   &k  &1  
  \end{mat}
\end{equation*}
Description of their geometric effect is beyond our scope.




% ========================================
\begin{exercises}
  \item 
    What is the equation of this point?
    \begin{equation*}
       \colvec[r]{1 \\ 0 \\ 0}
    \end{equation*}
    \begin{answer}
      From the dot product
      \begin{equation*}
        0=\colvec[r]{1 \\ 0 \\ 0}\dotprod\rowvec{L_1 &L_2 &L_3}
         =L_1
      \end{equation*}
      we get that the equation is $L_1=0$.
    \end{answer}
  \item 
    \begin{exparts}
      \partsitem Find the line incident on these points in the 
         projective plane.
         \begin{equation*}
           \colvec[r]{1 \\ 2 \\ 3},\,\colvec[r]{4 \\ 5 \\ 6}
         \end{equation*}
      \partsitem Find the point incident on both of 
         these projective lines. 
         \begin{equation*}
           \rowvec{1 &2 &3},\,\rowvec{4 &5 &6}
         \end{equation*} 
    \end{exparts}
    \begin{answer}
      \begin{exparts}
        \partsitem This determinant
          \begin{equation*}
            0=\begin{vmat}
              1  &4  &x \\
              2  &5  &y \\
              3  &6  &z
            \end{vmat}
            =-3x+6y-3z
          \end{equation*}
          shows that the line is $L=\rowvec{-3 &6 &-3}$.
        \partsitem $\colvec[r]{-3 \\ 6 \\ -3}$
      \end{exparts}
    \end{answer}
  \item
    Find the formula for the line incident on two projective points.
    Find the formula for the point incident on two projective lines.
    \begin{answer}
      The line incident on 
      \begin{equation*}
        u=\colvec{u_1 \\ u_2 \\ u_3}
        \qquad
        v=\colvec{v_1 \\ v_2 \\ v_3}
      \end{equation*}
      comes from this determinant equation.
      \begin{equation*}
        0=\begin{vmat}
          u_1  &v_1  &x  \\
          u_2  &v_2  &y  \\
          u_3  &v_3  &z
        \end{vmat}
        =(u_2v_3-u_3v_2)\cdot x 
          + (u_3v_1-u_1v_3)\cdot y 
          + (u_1v_2-u_2v_1)\cdot z
      \end{equation*}
      The equation for the point incident on two lines is the same. 
    \end{answer}
  \item \label{exer:IncidentIndReps}
    Prove that the definition of incidence is independent of the choice of 
    the representatives of $p$ and $L$.
    That is, if $p_1$, $p_2$, $p_3$, and $q_1$, $q_2$, $q_3$ are two triples of
    homogeneous coordinates for $p$, and 
    $L_1$, $L_2$, $L_3$, and $M_1$, $M_2$, $M_3$ are two triples of 
    homogeneous coordinates for $L$, prove that  
    $p_1L_1+p_2L_2+p_3L_3=0$ if and only if 
    $q_1M_1+q_2M_2+q_3M_3=0$. 
    \begin{answer}
      If $p_1$, $p_2$, $p_3$, and $q_1$, $q_2$, $q_3$ are two triples of
      homogeneous coordinates for $p$ then the two column vectors
      are in proportion, that is, lie on the same line through the
      origin.
      Similarly, the two row vectors are in proportion.
      \begin{equation*}
        k\cdot\colvec{p_1 \\ p_2 \\ p_3}
          =\colvec{q_1 \\ q_2 \\ q_3}
        \qquad
        m\cdot\rowvec{L_1 &L_2 &L_3}
          =\rowvec{M_1 &M_2 &M_3}
      \end{equation*}
      Then multiplying gives the answer
      $(km)\cdot (p_1L_1+p_2L_2+p_3L_3)=q_1M_1+q_2M_2+q_3M_3=0$.
    \end{answer}
  \item 
    Give a drawing to show that central projection does not preserve 
    circles, that a circle may project to an ellipse.
    Can a (non-circular) ellipse project to a circle?
    %Must it (in the sense that for any ellipse there is a projection
    %such that it would project to a circle)? 
    \begin{answer}
      The picture of the solar eclipse \Dash  unless 
      the image plane is exactly perpendicular
      to the line from the sun through the pinhole \Dash  shows the circle
      of the sun projecting to an image that is an  ellipse.
      (Another example is that in many pictures in this 
      Topic, we've shown the circle that is the sphere's equator as an ellipse,
      that is, a viewer of the drawing sees a circle as an ellipse.)
      
      The solar eclipse picture also shows the converse. 
      If we picture the projection as going from left to right 
      through the pinhole
      then the ellipse $I$ projects through $P$ to a circle~$S$.
    \end{answer}
  \item \label{exer:CorrProjPlaneEucPl} 
    Give the formula for the correspondence between the 
    non-equatorial part of the antipodal modal
    of the projective plane, and the plane $z=1$.
    \begin{answer}
      A spot on the unit sphere
      \begin{equation*}
        \colvec{p_1 \\ p_2 \\ p_3} 
      \end{equation*}
      is non-equatorial if and only if $p_3\neq 0$.
      In that case it corresponds to this point on the $z=1$ plane
      \begin{equation*}
        \colvec{p_1/p_3 \\ p_2/p_3  \\ 1}
      \end{equation*}
      since that is intersection of the line containing the vector and the
      plane. 
    \end{answer}
%  \item \label{exer:ELinesCorrPLines} 
%    Consider the correspondence between the non-ideal points in
%    the projective plane and the Euclidean plane.
%    \begin{exparts}
%      \item Show that any Euclidean line corresponds to a projective line.
%      \item Show that parallel Euclidean lines correspond to projective
%        lines that meet at an ideal point.
%      \item Prove the converses of those two statements.
%    \end{exparts}
%  \item \label{exer:EuclidDesarg}
%    Give a statement of Desargue's Theorem for Euclidean geometry.
%  \item  \label{exer:DesarAntipPict}
%    Draw a picture illustrating Desargue's Theorem on the antipodal model. 
  \item 
    (Pappus's Theorem)
    Assume that $T_0$, $U_0$, and  $V_0$ are collinear and that 
    $T_1$, $U_1$, and $V_1$ are collinear. 
    Consider these three points:
    (i)~the intersection $V_2$ of the lines $T_0U_1$ and $T_1U_0$,
    (ii)~the intersection $U_2$ of the lines $T_0V_1$ and $T_1V_0$, and
    (iii)~the intersection $T_2$ of $U_0V_1$ and $U_1V_0$. 
    \begin{exparts}
      \partsitem Draw a (Euclidean) picture.
      \partsitem Apply the lemma used in Desargue's Theorem
        to get simple homogeneous coordinate vectors for the 
        $T$'s and $V_0$.
      \partsitem Find the resulting homogeneous coordinate vectors
        for $U$'s (these must each involve a parameter as, e.g., $U_0$ could
        be anywhere on the $T_0V_0$ line).
      \partsitem Find the resulting homogeneous coordinate vectors for 
        $V_1$.
        (\textit{Hint:}~it involves two parameters.)
      \partsitem Find the resulting homogeneous coordinate vectors for 
        $V_2$.
        (It also involves two parameters.)
      \partsitem Show that the product of the three parameters is $1$.
      \partsitem Verify that $V_2$ is on the $T_2U_2$ line.
    \end{exparts}
    \begin{answer}
      \begin{exparts}
        \partsitem Other pictures are possible, but this is one.
          \begin{center}
            \includegraphics{det/mp/ch4.54}
          \end{center}
          The intersections 
          $
              T_0U_1\,\intersection T_1U_0=V_2
          $, $
              T_0V_1\,\intersection T_1V_0=U_2
          $, and $
              U_0V_1\,\intersection U_1V_0=T_2
          $
          are labeled so that on each line is a $T$, a $U$, and a $V$.
        \partsitem The lemma used in Desargue's Theorem gives a 
          basis $B$ with respect to which the points have these
          homogeneous coordinate vectors.
          \begin{equation*}
            \rep{\vec{t}_0}{B}=\colvec[r]{1 \\ 0 \\ 0}
            \quad
            \rep{\vec{t}_1}{B}=\colvec[r]{0 \\ 1 \\ 0}
            \quad
            \rep{\vec{t}_2}{B}=\colvec[r]{0 \\ 0 \\ 1}
            \quad
            \rep{\vec{v}_0}{B}=\colvec[r]{1 \\ 1 \\ 1}
          \end{equation*}
        \partsitem
          First, any $U_0$ on $T_0V_0$
          \begin{equation*}
            \rep{\vec{u}_0}{B}=a\colvec[r]{1 \\ 0 \\ 0}
                               +b\colvec[r]{1 \\ 1 \\ 1}
                              =\colvec{a+b \\ b \\ b}
          \end{equation*}
          has homogeneous coordinate vectors of this form
          \begin{equation*}
            \colvec{u_0 \\ 1 \\ 1}      
          \end{equation*}
          ($u_0$ is a parameter; it depends on where on the $T_0V_0$ line 
          the point $U_0$ is, but any point on that line has
          a homogeneous coordinate vector of this form for some $u_0\in\Re$).
          Similarly, $U_2$ is on $T_1V_0$
          \begin{equation*}
            \rep{\vec{u}_2}{B}=c\colvec[r]{0 \\ 1 \\ 0}
                                +d\colvec[r]{1 \\ 1 \\ 1}
                              =\colvec{d \\ c+d \\ d}
          \end{equation*}
          and so has this homogeneous coordinate vector.
          \begin{equation*}
            \colvec{1 \\ u_2 \\ 1}
          \end{equation*}
          Also similarly, $U_1$ is incident on $T_2V_0$
          \begin{equation*}
            \rep{\vec{u}_1}{B}=e\colvec[r]{0 \\ 0 \\ 1}
                                +f\colvec[r]{1 \\ 1 \\ 1}
                              =\colvec{f \\ f \\ e+f}  
          \end{equation*}
          and has this homogeneous coordinate vector.
          \begin{equation*}
            \colvec{1 \\ 1 \\ u_1}
          \end{equation*}
        \partsitem
          Because $V_1$ is $T_0U_2\,\intersection\,U_0T_2$ we have this.
          \begin{equation*}
            g\colvec[r]{1 \\ 0 \\ 0}+h\colvec{1 \\ u_2 \\ 1}
            =i\colvec{u_0 \\ 1 \\ 1}+j\colvec[r]{0 \\ 0 \\ 1}
            \qquad\Longrightarrow\qquad
            \begin{aligned}
              g+h  &= iu_0 \\
              hu_2 &= i    \\
              h    &= i+j
            \end{aligned}
          \end{equation*}
          Substituting $hu_2$ for $i$ in the first equation 
          \begin{equation*}
            \colvec{hu_0u_2 \\ hu_2 \\ h}
          \end{equation*}
          shows that $V_1$ has this 
          two-parameter homogeneous coordinate vector.
          \begin{equation*}
            \colvec{u_0u_2 \\ u_2 \\ 1}
          \end{equation*}
        \partsitem
           Since $V_2$ is the intersection 
           $T_0U_1\,\intersection\,T_1U_0$ 
           \begin{equation*}
             k\colvec[r]{1 \\ 0 \\ 0}+l\colvec{1 \\ 1 \\ u_1}
              =m\colvec[r]{0 \\ 1 \\ 0}+n\colvec{u_0 \\ 1 \\ 1}
            \qquad\Longrightarrow\qquad
            \begin{aligned}
              k+l  &= nu_0 \\
              l    &= m+n    \\
              lu_1 &= n
            \end{aligned}
           \end{equation*}
           and substituting $lu_1$ for $n$ in the first equation 
           \begin{equation*}
             \colvec{lu_0u_1 \\ l \\ lu_1}
           \end{equation*}
           gives that 
           $V_2$ has this two-parameter homogeneous coordinate vector.
           \begin{equation*}
             \colvec{u_0u_1 \\ 1 \\ u_1}
           \end{equation*}
        \partsitem
           Because $V_1$ is on the $T_1U_1$ line its
           homogeneous coordinate vector has the form
           \begin{equation*}
             p\colvec[r]{0 \\ 1 \\ 0}+q\colvec{1 \\ 1 \\ u_1}
             =\colvec{q \\ p+q \\ qu_1}
           \tag*{($*$)}\end{equation*}
           but a previous part of this question established that $V_1$'s
           homogeneous coordinate vectors have the form
           \begin{equation*}
            \colvec{u_0u_2 \\ u_2 \\ 1}             
           \end{equation*}
           and so this a homogeneous coordinate vector for $V_1$.
           \begin{equation*}
             \colvec{u_0u_1u_2 \\ u_1u_2 \\ u_1}             
           \tag*{($**$)}\end{equation*}
           By ($*$) and ($**$), there is a 
           relationship among the three parameters:~$u_0u_1u_2=1$.
         \partsitem  
           The homogeneous coordinate vector of $V_2$ can be written
           in this way.
           \begin{equation*}
             \colvec{u_0u_1u_2 \\ u_2 \\ u_1u_2}
             =\colvec{1 \\ u_2 \\ u_1u_2}
           \end{equation*}
           Now, the $T_2U_2$ line consists of the points whose homogeneous 
           coordinates have this form.
           \begin{equation*}
             r\colvec[r]{0 \\ 0 \\ 1}+s\colvec{1 \\ u_2 \\ 1}
             =\colvec{s \\ su_2 \\ r+s}
           \end{equation*}
           Taking $s=1$ and $r=u_1u_2-1$ shows that the
           homogeneous coordinate vectors of $V_2$ have this form.
      \end{exparts}
    \end{answer}
\end{exercises}
\index{computer graphics|)}
\endinput
