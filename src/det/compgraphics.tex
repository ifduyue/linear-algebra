% Chapter 4, Topic _Linear Algebra_ Jim Hefferon
%  http://joshua.smcvt.edu/linearalgebra
%  2020-Apr-08
\topic{Computer Graphics}
\index{computer graphics|(}
The prior topic on Projective Geometry gives this model of 
how our eye, or a camera, sees the world.
\begin{center} 
  \vcenteredhbox{\includegraphics{det/mp/ch4.14}}
\end{center}
All of the points on a line through the origin project to the same 
spot.

In that topic we defined that
for any nonzero vector $\vec{v}\in\Re^3$, the associated 
\definend{point $p$ in the projective plane}\index{point!in projective plane} 
is the set $\set{k\vec{v}\suchthat \text{$k\in\Re$ and $k\neq 0$}}$.
This is the collection of nonzero vectors lying on the same line through the
origin as $\vec{v}$.

To describe a projective point we can give any representative member 
of the line. 
Thus these each represent the same projective point.
\begin{equation*}
  \colvec[r]{1 \\ 2 \\ 3}
  \qquad
  \colvec{1/3 \\ 2/3 \\ 1}
  \qquad
  \colvec[r]{-2 \\ -4 \\ -6}
\end{equation*} 
Each is a
\definend{homogeneous coordinate vector}\index{coordinates!homogeneous}%
\index{homogeneous coordinate vector}\index{vector!homogeneous coordinate}
for the point~$p$. 
Two homogeneous coordinate vectors (which are by definition nonzero)
\begin{equation*}
  \tilde{p}_1=\colvec{a_1 \\ b_1 \\ c_1}
  \qquad
  \tilde{p}_2=\colvec{a_2 \\ b_2 \\ c_2}
\end{equation*}
represent the same projective point if there is a scaling factor
$s\neq 0$ so that $s\tilde{p}_1=\tilde{p}_2$.

Of the infinitely many possible representatives,
often we use the one whose third component is~$1$.
This amounts to projecting onto the plane $z=1$.
\begin{center} 
  \vcenteredhbox{\includegraphics{det/mp/ch4.18}}
\end{center}

In this topic we will show how to use these ideas to perform some effects 
from computer graphics.
For that we will take the prior picture and redraw it without the sphere,
with a movie projector at the origin, and with plane $z=1$ looking like
a movie theater screen.
\begin{center} 
  \vcenteredhbox{\includegraphics{det/mp/ch4.60}}
\end{center}
This associates vectors in three-space on the grey line with 
$p$~in the screen plane.
\begin{equation*}
  \tilde{p}=\colvec{a \\ b \\ c} 
  \mapsto 
  p=\colvec{x \\ y}=\colvec{a/c \\ b/c}
\end{equation*}

We can adapt the things we have already seen about matrices to perform 
the transformations.
Rotation\index{rotation} 
is an example.
This matrix rotates in the plane~$z=1$ about the origin by the angle~$\theta$.
\begin{equation*}
  \begin{mat}
    \cos\theta  &-\sin\theta  &0  \\
    \sin\theta  &\cos\theta   &0  \\
    0           &0            &1  
  \end{mat}
  \colvec{x \\ y \\ 1}
  =
  \colvec{cos\theta\cdot x -\sin\theta\cdot y  \\  
         \sin\theta\cdot x +\cos\theta\cdot y  \\ 
         1}
\end{equation*}
Notice that it works on any homogeneous coordinate vector;
if we apply the matrix
\begin{equation*}
  \begin{mat}
    \cos\theta  &-\sin\theta  &0  \\
    \sin\theta  &\cos\theta   &0  \\
    0           &0            &1  
  \end{mat}
  \colvec{a \\ b \\ c}
  =
  \colvec{cos\theta\cdot a -\sin\theta\cdot b  \\  
         \sin\theta\cdot a +\cos\theta\cdot b  \\ 
         c}
\end{equation*}
and then move to the $z=1$~plane
\begin{equation*}
  \colvec{cos\theta\cdot a -\sin\theta\cdot b  \\  
         \sin\theta\cdot a +\cos\theta\cdot b  \\ 
         c}
  \mapsto
  \colvec{(cos\theta\cdot a -\sin\theta\cdot b)/c  \\  
         (\sin\theta\cdot a +\cos\theta\cdot b)/c  \\ 
         1}
\end{equation*}
then we get the same result as 
if we had first moved to the plane and then applied
the matrix.
\begin{equation*}
  \begin{mat}
    \cos\theta  &-\sin\theta  &0  \\
    \sin\theta  &\cos\theta   &0  \\
    0           &0            &1  
  \end{mat}
  \colvec{a/c \\ b/c \\ 1}
  =
  \colvec{cos\theta\cdot a/c -\sin\theta\cdot b/c  \\  
         \sin\theta\cdot a/c +\cos\theta\cdot b/c  \\ 
         1}
\end{equation*}

So there is no harm in working with homogeneous coordinates.
But what is the advantage?

The computer graphic operation of translation,\index{translation} 
of sliding things from
one place to another, is not a linear transformation because it does
not leave the origin fixed.
But if we work with homogeneous coordinates then we can use matrices.
This matrix will translate points in the plane of interest by 
$t_x$ in the $x$~direction and $t_y$ in the $y$~direction.
\begin{equation*}
  \begin{mat}
    1  &0  &t_x  \\
    0  &1  &t_y  \\
    0  &0  &1  
  \end{mat}
  \colvec{a \\ b \\ c}
  =
  \colvec{a+t_x\cdot c  \\  
          b+t_y\cdot c  \\ 
          c}
  \mapsto
  \colvec{a/c+t_x  \\  
          b/c+t_y  \\ 
          1}
\end{equation*}
That is, in the plane of interest this matrix slides
$\binom{a/c}{b/c}$ to $\binom{a/c+t_x}{b/c+t_y}$.
So the homogeneous coordinates allow us to use matrices.

OK then, but what is the advantages of using these matrices?
What does the extra coordinate get us?
Suppose that we are making a movie with computer graphics.
We are at a moment where the 
camera is panning and rotating at the same time.
Every single point in the scene needs to be both translated and rotated.
Rather than have the computer perform two operations to each point,
we can multiply the two matrices and
then the computer only applies one operation to each point; it multiplies
that point by the resulting matrix.
That is a tremendous speedup and simplification.

We will list some examples of the effects that we can get.
We have already talked about rotation. 
Here is the picture of rotation by a half radian.
\begin{center} 
  \vcenteredhbox{\includegraphics{det/mp/ch4.61}}
  \quad$\mapsto$\quad
  \vcenteredhbox{\includegraphics{det/mp/ch4.62}}
\end{center}
And here is a translation with $t_x=1.5$ and~$t_y=0.5$.
\begin{center} 
  \vcenteredhbox{\includegraphics{det/mp/ch4.61}}
  \quad$\mapsto$\quad
  \vcenteredhbox{\includegraphics{det/mp/ch4.67}}
\end{center}

Next is scaling.
This matrix rescales things in the target plane 
by a factor of~$s$  in the $x$-direction, 
and  by a factor of~$t$ in the $y$~direction.
\begin{equation*}
  \begin{mat}
    s  &0  &0  \\
    0  &t  &0  \\
    0  &0  &1  
  \end{mat}
  \colvec{a/c \\ b/c \\ 1}
  =
  \colvec{s\cdot a/c  \\  
          t\cdot b/c \\ 
          1}
\end{equation*}
In this picture we rescale in the $x$~direction by a factor of $s=2.5$
and in the $y$-direction by~$t=0.75$.
\begin{center} 
  \vcenteredhbox{\includegraphics{det/mp/ch4.61}}
  \quad$\mapsto$\quad
  \vcenteredhbox{\includegraphics{det/mp/ch4.63}}
\end{center}

If we take $s=t$ then the entire shape is rescaled.
For instance, if we string together frames with $s=t=1.10$ then in
the movie it will seem that the object is getting closer to us.
\begin{center} 
  \vcenteredhbox{\includegraphics{det/mp/ch4.61}}
  \quad
  \vcenteredhbox{\includegraphics{det/mp/ch4.68}}
  \quad
  \vcenteredhbox{\includegraphics{det/mp/ch4.69}}
  \quad
  \vcenteredhbox{\includegraphics{det/mp/ch4.70}}
\end{center}

We can reflect the object.
This reflects about the line $y=x$.
\begin{equation*}
  \begin{mat}
    0  &1  &0  \\
    1  &0  &0  \\
    0  &0  &1  
  \end{mat}
  \colvec{a/c \\ b/c \\ 1}
  =
  \colvec{b/c  \\  
          a/c \\ 
          1}
\end{equation*}
The dashed line here is $y=x$.
\begin{center} 
  \vcenteredhbox{\includegraphics{det/mp/ch4.61}}
  \quad$\mapsto$\quad
  \vcenteredhbox{\includegraphics{det/mp/ch4.64}}
\end{center}

This reflects about $y=-x$.
\begin{equation*}
  \begin{mat}
    0   &-1  &0  \\
    -1  &0  &0  \\
    0   &0  &1  
  \end{mat}
  \colvec{a/c \\ b/c \\ 1}
  =
  \colvec{-b/c  \\  
          -a/c \\ 
          1}
\end{equation*}
The dashed line below is $y=-x$.
\begin{center} 
  \vcenteredhbox{\includegraphics{det/mp/ch4.61}}
  \quad$\mapsto$\quad
  \vcenteredhbox{\includegraphics{det/mp/ch4.65}}
\end{center}

More complex transformations are possible.
This is a \definend{shear}.\index{shear}
\begin{equation*}
  \begin{mat}
    1   &1  &0  \\
    0   &1  &0  \\
    0   &0  &1  
  \end{mat}
  \colvec{a/c \\ b/c \\ 1}
  =
  \colvec{a/c+b/c  \\  
          b/c \\ 
          1}
\end{equation*}
In this picture the $y$~components of points are unchanged,
but the $x$~components have added to them the value of~$y$.
\begin{center} 
  \vcenteredhbox{\includegraphics{det/mp/ch4.61}}
  \quad$\mapsto$\quad
  \vcenteredhbox{\includegraphics{det/mp/ch4.66}}
\end{center}

A major advantage of having this all be matrices is that we can do 
complex things by combining simple things.
To reflect about the line $y=-x+2$ we can find the three matrices
to slide everything to the origin, then reflect about $y=-x$, and then 
slide back.
\begin{equation*}
  \begin{mat}
    1   &0  &0  \\
    0   &1  &2  \\
    0   &0  &1  
  \end{mat}
  \begin{mat}
    0   &-1  &0  \\
    -1  &0   &0  \\
    0   &0   &1  
  \end{mat}
  \begin{mat}
    1   &0  &0  \\
    0   &1  &-2  \\
    0   &0  &1  
  \end{mat}
  \tag{*}
\end{equation*}
(As always, the action done first is described by the matrix
on the right.
That is, the matrix on the right describes sliding all points in the plane
of interest by $-2$, 
the matrix in the middle reflects about $y=-x$, 
and the matrix on the left slides all points back.)

There are even more complex effects possible with matrices.
These are the matrices for the
\definend{general affine transformation},\index{affine transformation} 
and the
\definend{general projective transformation}.\index{projective transformation}
\begin{equation*}
  \begin{mat}
    d   &e  &f  \\
    g   &h  &i  \\
    0   &0  &1  
  \end{mat}
  \qquad
  \begin{mat}
    d   &e  &f  \\
    g   &h  &i  \\
    j   &k  &1  
  \end{mat}
\end{equation*}
However, description of their geometric effect is beyond our scope.

There is a vast literature on computer graphics, in which
linear algebra plays an important part.
An excellent source is \cite{Hughes}.
The subject is a wonderful blend of mathematics and art;
see \cite{Disney}. 



% ========================================
\begin{exercises}
  \item 
    Calculate the product in ($*$).
    \begin{answer}
      Straightforward calculation gives this for the product.
      \begin{equation*}
        \begin{mat}
          1   &0  &0  \\
          0   &1  &2  \\
          0   &0  &1  
        \end{mat}
        \begin{mat}
          0   &-1  &0  \\
          -1  &0   &0  \\
          0   &0   &1  
        \end{mat}
        \begin{mat}
          1   &0  &0  \\
          0   &1  &-2  \\
          0   &0  &1  
        \end{mat}
        =
        \begin{mat}
         0 &-1 &2 \\
        -1 &0  &2 \\
         0 &0  &1
        \end{mat}
      \end{equation*}
    \end{answer}
% sage: M0 = matrix(QQ, [[1,0,0], [0,1,2], [0,0,1]])
% sage: M0
% [1 0 0]
% [0 1 2]
% [0 0 1]
% sage: M1 = matrix(QQ, [[0,-1,0], [-1,0,0], [0,0,1]])
% sage: M2 = matrix(QQ, [[0,-1,0], [-1,0,0], [0,0,1]])
% sage: M2 = matrix(QQ, [[1,0,0], [0,1,-2], [0,0,1]])
% sage: M0*M1*M2
% [ 0 -1  2]
% [-1  0  2]
% [ 0  0  1]
  \item Find the matrix that reflects about the line $y=2x$.
    \begin{answer}
      Working in $\R^2$, let the matrix be $M$.
      We get these two.
      \begin{equation*}
        \colvec{1 \\ 2}\mapsto \colvec{1 \\ 2}
        \qquad
        \colvec{-2 \\ 1}\mapsto \colvec{2 \\ -1}
      \end{equation*}
      (For the second one, the starting vector is on the line through the 
      origin that is perpendicular to $y=2x$.)
      We have this.
      \begin{equation*}
        M
        \begin{mat}
          1  &-2  \\
          2  &1
        \end{mat}
        =
        \begin{mat}
          1  &2  \\
          2  &-1
        \end{mat}
      \end{equation*}
      Solving gives
      \begin{equation*}
        M =
            \begin{mat}
              1  &2  \\
              2  &-1
            \end{mat}
            \begin{mat}
              1  &-2  \\
              2  &1     
            \end{mat}^{-1}
          =
            \begin{mat}
              -3/5  &4/5 \\
              4/5   &3/5
            \end{mat}
      \end{equation*}
      and so for homogeneous coordinates the matrix is this.
      \begin{equation*}
      \begin{mat}
        -3/5  &4/5  &0  \\
         4/5  &3/5  &0  \\
           0  &0    &1
      \end{mat}
      \end{equation*}
    \end{answer}
  \item Find the matrix that reflects about the line $y=2x-4$.
    \begin{answer}
      Move all points over by $4$, reflect about the line $y=2x$ using the
      prior exercise, and them move them back. 
      \begin{equation*}
      \begin{mat}
        1  &0  &0  \\
        0  &1  &-4 \\
        0  &0  &1
      \end{mat}
      \begin{mat}
        -3/5  &4/5  &0  \\
         4/5  &3/5  &0  \\
           0  &0    &1
      \end{mat}
      \begin{mat}
        1  &0  &0  \\
        0  &1  &4 \\
        0  &0  &1
      \end{mat}
      =
      \begin{mat}
        -3/5  &4/5  &16/5  \\
         4/5  &3/5  &-8/5 \\
           0  &0    &1
      \end{mat}
      \end{equation*}
    \end{answer}
% sage: M0 = matrix(QQ, [[1,0,0], [0,1,-4], [0,0,1]])
% sage: M1 = matrix(QQ, [[-3/5,4/5,0], [4/5,3/5,0], [0,0,1]])
% sage: M2 = matrix(QQ, [[1,0,0], [0,1,4], [0,0,1]])
% sage: M0*M1*M2
% [-3/5  4/5 16/5]
% [ 4/5  3/5 -8/5]
% [   0    0    1]
  \item 
    Rotation and translation are rigid operations.
    What is the matrix for a rotation followed by a translation?
    \begin{answer}
    \begin{equation*}
      \begin{mat}
        1  &0  &t_x  \\
        0  &1  &t_y  \\
        0  &0  &1
      \end{mat}
      \begin{mat}
        \cos\theta &-\sin\theta  &0 \\
        \sin\theta &\cos\theta   &0 \\
        0          &0            &1
      \end{mat}
      =
      \begin{mat}
        \cos\theta  &-\sin\theta  &t_x  \\
        \sin\theta  &\cos\theta   &t_y  \\
        0           &0            &1
      \end{mat}
    \end{equation*}
    \end{answer}
  \item The homogeneous coordinates extend to manipulations of 
    three dimensional space in the obvious way:~every coordinate
    is a set of four-tall nonzero vectors that are related by
    being scalar multiples of each other.
    Give the matrix to do rotation about the $z$~axis, and the
    matrix for rotation about the $y$~axis.
    \begin{answer}
    Here are the two $\nbyn{4}$ matrices.
    \begin{equation*}
      \begin{mat}
        \cos\theta  &-\sin\theta  &0  &0  \\
        \sin\theta  &\cos\theta   &0  &0  \\
        0           &0            &0  &0  \\
        0           &0            &0  &1  
      \end{mat}
      \qquad
      \begin{mat}
        \cos\theta  &0  &-\sin\theta  &0   \\
        0           &0  &0            &0  \\
        \sin\theta  &0  &\cos\theta   &0   \\
        0           &0  &0            &1  
      \end{mat}
    \end{equation*}    
    \end{answer}
\end{exercises}
\index{computer graphics|)}
\endinput
