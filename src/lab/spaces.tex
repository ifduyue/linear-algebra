\chapter{Vector Spaces}

\Sage{} can operate with vector spaces, for example by finding a basis for
a space.

In the book's chapter on vector spaces, the scalars come
from the set of real numbers, $\Re$.
\Sage{} lets you choose from two models of the real numbers.
One is \Sagecmd{RDF}, the computer's built-in floating point
model of real numbers.
% \footnote{%
%   The computer used to make this manual has
%   IEEE~754 double-precision binary floating-point numbers;  
%   if you have programmed then you may know this number model as binary64.}  
The other is \Sagecmd{RR}, which gives arbitrary precision 
reals.
The first model runs faster
and is more widely used in practice so that is what we will use.
 




%========================================
\section{Real spaces}

\Sage{} lets you create vector spaces.
Here is $\Re^4$.
\begin{sagecommandline}
sage: V = VectorSpace(RDF,4)
sage: V
\end{sagecommandline}
You can also create subspaces.
Here is a one-dimensional subspace of $\Re^4$, described as the span
of a set with one vector.
\begin{sagecommandline}
sage: V = VectorSpace(RDF,4)
sage: V
sage: v1 = vector(RR, [1, 1, -3, 0])
sage: W = V.span([v1])
sage: W
\end{sagecommandline}

You can ask \Sage{} questions about this subspace, such as 
testing for membership. 
\begin{sagecommandline}
sage: v3 = vector(RDF, [2, 2, -6, 0])
sage: v3 in W
sage: v4 = vector(RDF, [1, 0, 0, 5])
sage: v4 in W
\end{sagecommandline}

Another example is~$\Re^3$.
\begin{sagecommandline}
sage: V = VectorSpace(RDF,3)
sage: V
sage: v1 = vector(RDF, [1, 2, 3])
sage: v1 in V
sage: v2 = vector(RDF, [1, 2, 3, 4])
sage: v2 in V
\end{sagecommandline}

We can create a subspace as a span of multiple vectors.
\begin{sagecommandline}
sage: V = VectorSpace(RDF,3)
sage: v1 = vector(RDF, [1, 0, 3])
sage: v2 = vector(RDF, [0, 1, 1])
sage: W = V.span([v1, v2])
sage: W
sage: v3 = vector(RDF, [1, 1, 4])
sage: v3 in W
\end{sagecommandline}



\subsection{Basis}
\Sage{} will retrieve a basis for a vector space or subspace.
\begin{sagecommandline}
sage: V = VectorSpace(RDF,3)
sage: V.basis()
sage: v1 = vector(RDF, [2, -1, -3])
sage: v2 = vector(RDF, [1, 1, 0])
sage: W = V.span([v1, v2])     
sage: W.basis()
\end{sagecommandline}
Notice that the basis that \Sage{} gives  
back is not the same as the set
of two vectors $\set{\vec{v}_1, \vec{v}_2}$ that we gave it.
To get the new basis, \Sage{} takes the vectors from the spanning set 
as the rows of a matrix,
brings that matrix to reduced echelon form, and reports the nonzero 
rows as the basis.
Each matrix has one and only one reduced echelon form, so each 
subspace of~$V$ has one and only one such basis;
this is the canonical basis for the subspace.

\Sage{} shows this basis when you ask for a description
of the subspace.
\begin{sagecommandline}
sage: W  
\end{sagecommandline}

Another subspace of $\Re^3$ is the plane described by the equation
$x-2y+2z=0$.
\begin{equation*}
  W=\set{\colvec{x \\ y \\ z}
    \suchthat x=2y-2z}
  =\set{\colvec{2 \\ 1 \\ 0}y+\colvec[r]{-2 \\ 0 \\ 1}z
        \suchthat y,z\in\Re}
\end{equation*}
Here is that subspace.
\begin{sagecommandline}
sage: V = VectorSpace(RDF,3)               
sage: v1 = vector(RDF, [2, 1, 0]) 
sage: v2 = vector(RDF, [-2, 0, 1]) 
sage: W = V.span([v1, v2])       
sage: W.basis()
\end{sagecommandline}

If you take that set~$\set{\vec{v}_1,\vec{v}_2}$ and add a vector~$\vec{v}_3$ that is
linearly independent of those two, 
then it generates a three dimensional subspace of $\Re^3$, 
which is all of $\Re^3$.
\begin{sagecommandline}
sage: v3 = vector(RDF, [2, 3, 0])
sage: W_prime = V.span([v1, v2, v3])
sage: W_prime.basis()
\end{sagecommandline}
If instead you add to the spanning set $\set{\vec{v}_1,\vec{v}_2}$ 
a third vector~$\vec{v}_3$ 
that is linearly dependent on the other two then
it doesn't change the span.
That is, in this case the subspace spanned by the first two
$\spanof{\vec{v}_1,\vec{v}_2}$
equals the span of all three
$\spanof{\vec{v}_1,\vec{v}_2,\vec{v}_3}$.
\begin{sagecommandline}
sage: W = V.span([v1, v2])       
sage: W.basis()
sage: v3 = vector(RDF, [0, 1, 1])
sage: W_prime = V.span([v1, v2, v3])
sage: W_prime.basis()
\end{sagecommandline}

If you are keen on using your own basis then \Sage{} will
accommodate.
\begin{sagecommandline}
sage: V = VectorSpace(RDF,3)
sage: v1 = vector(RDF, [1, 2, 3])
sage: v2 = vector(RDF, [2, 1, 3])
sage: W = V.span_of_basis([v1, v2])
sage: W.basis()
\end{sagecommandline}




\subsection{Equality}

\Sage{} can compare spaces.
Here are two descriptions of the $xy$-plane inside of $\Re^4$.
\begin{sagecommandline}
sage: V = VectorSpace(RDF,4)
sage: v1 = vector(RDF, [1, 0, 0, 0])
sage: v2 = vector(RDF, [1, 1, 0, 0])
sage: W12 = V.span([v1, v2])
sage: v3 = vector(RDF, [2, 1, 0, 0])
sage: W13 = V.span([v1, v3])  
\end{sagecommandline}
% We could test that the two spaces $W_{1,2}$ and $W_{1,3}$ 
% are equal by using set membership.
If both vectors used to make the spanning set $W_{1,2}$ are 
members of  $W_{1,3}$ then  $W_{1,2}\subseteq W_{1,3}$.
If in addition the vectors used to make the spanning set $W_{1,3}$ are 
members of  $W_{1,2}$ then the spaces are equal.
\begin{sagecommandline}
sage: v2 in W13
sage: v3 in W12
\end{sagecommandline}
Since both $\vec{v}_1,\vec{v}_3\in W_{1,2}$ and 
$\vec{v}_1,\vec{v}_2\in W_{1,3}$, the two subspaces are equal.

However, the more straightforward way to test equality is to just ask.
\begin{sagecommandline}
sage: W12 == W13
\end{sagecommandline}
This exercise of the equality operator, \inlinecode{==}, 
would be half-hearted without trying it on two spaces that are
unequal. 
\begin{sagecommandline}
sage: v4 = vector(RDF, [1, 1, 1, 1])
sage: W14 = V.span([v1, v4])
sage: v2 in W14
sage: v4 in W12
sage: W12 == W14
sage: W12 != W14
\end{sagecommandline}

There is a point here about algorithms.
\Sage{} could check for equality of two spans in a number of ways.
One way is to check whether every member of the first spanning set is in the
second space, and whether every member of the second spanning set is in the 
first space. 
But \Sage{} does something different.
It checks for equality of spaces
just by checking whether they have equal canonical bases.
\begin{sagecommandline}
sage: W12.basis()
sage: W14.basis()
\end{sagecommandline}
These two algorithms 
have the same external behavior, in that both decide whether
two spaces are equal.
But the algorithms differ internally and the second is faster, 
in part because \Sage{} can pre-compute the canonical basis for a space
and then use it as many times as desired.
Finding the fastest way to do jobs is an important research area of computing.


\subsection{Operations}
\Sage{} can find the intersection of two spaces.
Consider these members of $\Re^3$.
\begin{equation*}
  \vec{v}_1=\colvec{1 \\ 0 \\ 0}
  \quad \vec{v}_2=\colvec{0 \\ 1 \\ 0}
  \quad \vec{v}_3=\colvec{0 \\ 0 \\ 3}
\end{equation*}
Form two spans, the $xy$-plane $W_{1,2}=\spanof{\vec{v}_1, \vec{v}_2}$ 
and the $yz$-plane $W_{2,3}=\spanof{\vec{v}_2, \vec{v}_3}$.
The intersection of these two is the $y$-axis. 
\begin{sagecommandline}
sage: V = VectorSpace(RDF,3)
sage: v1 = vector(RDF, [1, 0, 0])
sage: v2 =  vector(RDF, [0, 1, 0])
sage: W12 = V.span([v1, v2])
sage: W12.basis()
sage: v3 = vector(RDF, [0, 0, 2])
sage: W23 = V.span([v2, v3])
sage: W23.basis()
sage: W = W12.intersection(W23)
sage: W.basis()
\end{sagecommandline}

If you try to intersect two spaces where the operation makes no sense,
such as if you try to intersect $\Re^3$ and~$\Re^4$, then \Sage{}
gives an error message whose final line contains the string
\inlinecode{self and other must have the same ambient space.}

Remember that the trivial space $\set{\zero}$ is the span of the empty set.
\begin{sagecommandline}
sage: V = VectorSpace(RDF,3)
sage: v1, v2 = vector(RDF, [1, 0, 0]), vector(RDF, [0, 1, 0])
sage: W12 = V.span([v1, v2])
sage: v3 = vector(RDF, [1, 1, 1])
sage: W3 = V.span([v3])
sage: W3.basis()
sage: W4 = W12.intersection(W3)
sage: W4.basis()
\end{sagecommandline}

\Sage{} will also find the sum of spaces, the span of their union.
\begin{sagecommandline}
sage: V = VectorSpace(RDF,3)
sage: v1, v2 = vector(RDF, [1, 0, 0]), vector(RDF, [0, 1, 0])
sage: W12 = V.span([v1, v2])
sage: v3 = vector(RDF, [1, 1, 1])
sage: W3 = V.span([v3])
sage: W5 = W12 + W3
sage: W5
sage: W5 == V
\end{sagecommandline}
(The text covers the sum of subspaces in an optional section.)





%========================================
\section{Other spaces}

The book has vector spaces that aren't a subspace of some $\Re^n$.
To work with them, there are more sophisticated things that you can do but
one straightforward thing is 
using a real space that is just like the desired one.\footnote{%
  The textbook's 
  third chapter, Maps Between Spaces, makes 
  ``just like'' precise.}
Consider this subspace of 
$\polyspace_2$.
\begin{equation*}
  \set{ a_2x^2+a_1x+a_0\suchthat a_2=a_0+a_1}           
   =\set{ (a_1+a_0)x^2+a_1x+a_0 \suchthat a_1,a_0\in\Re}
\end{equation*}
It is just like
this subspace of $\Re^3$.
\begin{equation*}
  \set{\colvec{a_1+a_0 \\ a_1 \\ a_0}\suchthat a_1,a_0\in\Re}
  =\set{\colvec{1 \\ 1 \\ 0}a_1+\colvec{1 \\ 0 \\ 1}a_0\suchthat a_1,a_0\in\Re}
\end{equation*}
\begin{sagecommandline}
sage: V = VectorSpace(RDF,3)
sage: v1 = vector(RDF, [1, 1, 0])
sage: v2 = vector(RDF, [1, 0, 1])
sage: W = V.span([v1, v2])
sage: W.basis()
\end{sagecommandline}

Similarly you can use \Sage{} to 
perform computation for this space of $\nbyn{2}$ matrices
\begin{equation*}
  \set{\begin{mat}
         a  &b \\
         c  &d
       \end{mat}\suchthat \text{$a-b+c=0$ and $b+d=0$}}
\end{equation*}
by parametrizing
\begin{equation*}
  \set{\begin{mat}
         a  &b \\
         c  &d
       \end{mat}\suchthat \text{$a=-c-d$ and $b=-d$}}
  =\set{\begin{mat}
         -1  &0 \\
          1  &0
       \end{mat}c
       +
       \begin{mat}
         -1  &-1 \\
          0  &1
       \end{mat}d
       \suchthat c,d\in\Re}
\end{equation*}
and then giving \Sage{} the corresponding real space.
\begin{sagecommandline}
sage: V = VectorSpace(RDF,4)
sage: v1 = vector(RDF, [-1, 0, 1, 0])
sage: v2 = vector(RDF, [-1, -1, 0, 1])
sage: W = V.span([v1, v2])
sage: W.basis()
\end{sagecommandline}
That example gets the vectors by reading across the matrix rows.
You could instead read
down the columns.  
\begin{sagecommandline}
sage: V = VectorSpace(RDF,4)
sage: v1 = vector(RDF, [-1, 1, 0, 0])
sage: v2 = vector(RDF, [-1, 0, -1, 1])
sage: W = V.span([v1, v2])
sage: W.basis()
\end{sagecommandline}
There are still other ways to produce a matching space.
They look different than each other
but the important things about the spaces, such as dimension, are 
unaffected by their look.
Much more on this is in the book's third chapter.

\endinput


TODO:
