\chapter{Matrices}

Matrix operations such as addition and multiplication
are mechanical and therefore are perfectly suited for 
a computer.



%========================================
\section{Constructing matrices}
Define a matrix using \Sage's \inlinecode{matrix} function.
Recall from the section on Gauiss's Method that it takes two inputs.
First is the set of numbers describing which scalars can 
form the entries, and second is those entries, written as lists of rows.
Here, for the first input we will use either \Sagecmd{RDF} for 
floating point number entries, 
or \Sagecmd{CDF} for complex number entries $a+bi$ where 
$a$ and~$b$ are floating points,
or we can fall back to the rational numbers, \Sagecmd{QQ}.
\begin{sagecommandline}
sage: A = matrix(RDF, [[1, 2], [3, 4]])
sage: A
sage: i = CDF(i)
sage: A = matrix(CDF, [[1+2*i, 3+4*i], [5+6*i, 7+8*i]])
sage: A
sage: A = matrix(QQ, [[1, 2], [3, 4]])
sage: A                           
\end{sagecommandline}
\Sage's default symbol for the square root of $-1$ is $i$. 
Because~$i$ is used for many things in programming,
before using it for the complex numbers you should
reset it with \inlinecode{CDF(i)}.

In this chapter, unless we have reason to do otherwise
for scalars we will use the rational numbers, \inlinecode{QQ}, 
because rationals are easier to read\Dash $1$ is easier than $1.0$\Dash and 
because the text's matrices usually have rational entries.

The \inlinecode{matrix} constructor lets you to specify the number of
rows and columns.
\begin{sagecommandline}
sage: B = matrix(QQ, 2, 3, [[1, 1, 1], [2, 2, 2]])  
sage: B
\end{sagecommandline}
One reason to do this is as a check on what you enter.
Here the specified size doesn't match the entries because
the given matrix has only two rows. 
\begin{lstlisting}[style=python]
sage: B = matrix(QQ, 3, 3, [[1, 1, 1], [2, 2, 2]])  
\end{lstlisting}
\Sage's error says
\inlinecode{Number of rows does not match up with specified number}.

Another time that you want to list the
number of rows and columns is when you are doing 
the following shortcut to get an identity matrix, putting  
a $1$ in the place of the list of rows.
\begin{sagecommandline}
sage: I = matrix(QQ, 3, 3, 1)                     
sage: I
\end{sagecommandline}
Much the same shortcut gets a zero matrix.
\begin{sagecommandline}
sage: Z = matrix(QQ, 2, 2, 0)
sage: Z
\end{sagecommandline}
The difference between this shortcut and the prior one is that 
\inlinecode{matrix(QQ, 3, 2, 1)} gives an error because 
an identity matrix must be square.
\Sage{} has a command to create an identity matrix 
that can't lead to this error.
\begin{sagecommandline}
sage: I = identity_matrix(3)
sage: I
\end{sagecommandline}

\Sage{} has many more methods on matrices.
For instance, you can transpose the rows to columns or test if the 
matrix is symmetric, that is, 
unchanged by transposition.
\begin{sagecommandline}
sage: A = matrix(QQ, [[1, 2], [3, 4]])
sage: A.transpose()
sage: A.is_symmetric()
\end{sagecommandline}
Still another example is that you can create a matrix by giving a pattern
for the entries.
Here in the created matrix the entry $a_{i,j}$ equals the sum
$i+j$.
\begin{sagecommandline}
sage: A = matrix(QQ, 2, 3, lambda x, y: x+y)
sage: A
\end{sagecommandline}
One last example:
\Sage{} will make a matrix with random entries, here
with floating points from 0 to 1.
\begin{sagecommandline}
sage: random_matrix(RDF, 3, min=0, max=1)
\end{sagecommandline}
(Note the \inlinecode{RDF}.
This is better for us because \Sage's \inlinecode{random_matrix} 
is more straightforward with floating points entries than with 
rational entries.
The \Sage{} reference has the details.)




%========================================
\section{Linear combinations}
Addition and subtraction of matrices are natural operations.
\begin{sagecommandline}
sage: A = matrix(QQ, [[1, 2], [3, 4]])
sage: B = matrix(QQ, [[1, 1], [2, -2]])
sage: A+B
sage: A-B
\end{sagecommandline}

\Sage{} knows that adding matrices with different sizes is undefined.
\begin{lstlisting}
sage: A = matrix(QQ, [[1, 2], [3, 4]])
sage: C = matrix(QQ, [[0, 0, 2], [3, 2, 1]])
sage: A+C
\end{lstlisting}
You get an error whose final line contains 
\inlinecode{unsupported operand parent(s) for +}.
In short, \inlinecode{+} is not defined between a
$\nbyn{2}$~matrix and a $\nbym{2}{3}$~matrix.

Scalar multiplication is also natural,
so you can have linear combinations.
\begin{sagecommandline}[d,0,2]
sage: A = matrix(QQ, [[1, 2], [3, 4]])
sage: B = matrix(QQ, [[1, 1], [2, -2]])
sage: 3*A
sage: 3*A-4*B
\end{sagecommandline}



%========================================
\section{Multiplication}

\subsection{Matrix-vector product}
Matrix-vector multiplication works the way that you would guess.
\begin{sagecommandline}
sage: A = matrix(QQ, [[1, 3, 5, 9], [0, 2, 4, 6]])
sage: v = vector(QQ, [1, 2, 3, 4])
sage: A, v
sage: A*v
\end{sagecommandline}
The $\nbym{2}{4}$~matrix $A$ multiplies the 
$\nbym{4}{1}$~column vector~$\vec{v}$ with the vector on the right side,
as $A\vec{v}$.
Trying this vector on the left, as below,
gives
\inlinecode{unsupported operand parent(s) for *}.
\begin{lstlisting}
sage: v*A
\end{lstlisting}

Of course, you can multiply with a vector on the left
if that vector's size  
suits the matrix.
Here, the two-row~$A$ works with a two-entry vector.
\begin{sagecommandline}
sage: w = vector(QQ, [3, 5])
sage: w*A
\end{sagecommandline}




\subsection{Matrix-matrix product}
\Sage{} is happy to multiply matrices.
\begin{sagecommandline}
sage: A = matrix(QQ, [[2, 1], [4, 3]])
sage: B = matrix(QQ, [[5, 6, 7], [8, 9, 10]]) 
sage: A*B
\end{sagecommandline}
Trying $\inlinecode{B*A}$ gives 
\inlinecode{unsupported operand parent(s) for '*'} since $BA$ is undefined.
% \begin{sagecommandline}
% sage: A = matrix(QQ, [[2, 1], [4, 3]])
% sage: B = matrix(QQ, [[5, 6, 7], [8, 9, 10]]) 
% sage: B*A
% \end{sagecommandline}

Square matrices of the same size have the product defined in either order.
\begin{sagecommandline}
sage: A = matrix(QQ, [[1, 2], [3, 4]])
sage: B = matrix(QQ, [[4, 5], [6, 7]])
sage: A*B
sage: B*A
\end{sagecommandline}
Note that the two results are
different; matrix multiplication is not commutative.
\begin{sagecommandline}[d,0,2]
sage: A*B == B*A
\end{sagecommandline}

In fact, matrix multiplication is very non-commutative 
in the sense that if you produce two $\nbyn{n}$~matrices
at random then they almost surely don't commute.
Here we multiply together a thousand random matrices to see
if any commute.
\begin{sagecommandline}
sage: number_commuting = 0 
sage: for n in range(1000):                                       
....:     A = random_matrix(RDF, 2, min=-1, max=1)
....:     B = random_matrix(RDF, 2, min=-1, max=1)
....:     if (A*B == B*A):
....:         number_commuting = number_commuting + 1 
\end{sagecommandline}
Here is the result.
\begin{sagecommandline}
sage: number_commuting
\end{sagecommandline}

% Plug a square matrix into a polynomial.
 


\subsection{Inverse}
If $A$ is an~$\nbyn{n}$ nonsingular matrix then its inverse $A^{-1}$
is the~$\nbyn{n}$ matrix such that $A^{-1}A=AA^{-1}$ is the 
$\nbyn{n}$ identity matrix. 
% For $\nbyn{2}$ matrix inverses we have a formula.
In the book, while we use a formula for do $\nbyn{2}$ inverses,
to find larger inverses 
we write the original matrix next to the identity
and then perform Gauss-Jordan reduction.
\begin{sagecommandline}
sage: A = matrix(QQ, [[1, 3, 1], [2, 1, 0], [4, -1, 0]])
sage: A.is_singular()
sage: I = identity_matrix(3)
sage: B = A.augment(I, subdivide=True)
sage: B
sage: C = B.rref()
sage: C
\end{sagecommandline}
The inverse is on the right.
To pull out the inverse, 
\Sage{} has a \inlinecode{matrix_from_columns} method 
on a matrix instance that takes a list of 
columns, extracts them, and returns the matrix composed of those columns.
\begin{sagecommandline}
sage: A_inv = C.matrix_from_columns([3, 4, 5])
sage: A_inv
sage: A_inv*A
sage: A*A_inv == identity_matrix(3)
\end{sagecommandline}

All this is very awkward, so naturally there is a
standalone command.
\begin{sagecommandline}[d,0,1]
sage: A = matrix(QQ, [[1, 3, 1], [2, 1, 0], [4, -1, 0]])
sage: A_inv = A.inverse()
sage: A_inv
\end{sagecommandline}

One reason for finding the inverse is to make solving linear systems easier.
These three systems
\begin{equation*}
  \begin{linsys}{3}
    x  &+ &3y &+ &z &= &4 \\
    2x &+ &y  &  &  &= &4 \\
    4x &- &y  &  &  &= &4 
  \end{linsys}
  \qquad\qquad
  \begin{linsys}{3}
    x  &+ &3y &+ &z &= &2 \\
    2x &+ &y  &  &  &= &-1 \\
    4x &- &y  &  &  &= &5 
  \end{linsys}
  \qquad\qquad
  \begin{linsys}{3}
    x  &+ &3y &+ &z &= &1/2 \\
    2x &+ &y  &  &  &= &0 \\
    4x &- &y  &  &  &= &12 
  \end{linsys}
\end{equation*}
have the same matrix of coefficients on the left sides 
but different right sides.
If you calculate the inverse of that matrix
then solving each system requires only a single matrix-vector product.
\begin{sagecommandline}
sage: A = matrix(QQ, [[1, 3, 1], [2, 1, 0], [4, -1, 0]])
sage: A_inv = A.inverse()
sage: v1 = vector(QQ, [4, 4, 4])
sage: v2 = vector(QQ, [2, -1, 5])
sage: v3 = vector(QQ, [1/2, 0, 12])
sage: A_inv*v1
sage: A_inv*v2
sage: A_inv*v3
\end{sagecommandline}



\section{Condition number}
For the by-hand linear systems in the text, we find exact solutions.
But in applications we use floating points and so the computation
may lose precision because of numerical issues.

Some systems are more subject to this loss than others.
The suceptibility of a system's matrix of coefficients to this problem
is measured by its \textit{condition number}.
This is a positive real number that
estimates worst-case loss of precision.\footnote{%
  The condition number's base~$10$ logarithm is a worst-case estimate of 
  how many digits are lost in solving a linear system with that matrix
  of coefficients. 
  So a condition number is large if its base~$10$ logarithm 
  is greater than or equal to the number of significant digits 
  of the entries in the matrix.} 

More precisely, we start with a linear system 
$A\vec{v}=\vec{d}$
and then ask how much an inaccuracy in~$\vec{d}$ (perhaps due to some
floating point issue or perhaps some measurement limit on the vector's entries) 
can affect the solution~$\vec{v}$.
That is, we are considering
$A(\vec{v}+\vec{\Delta v})=\vec{d}+\vec{\Delta d}$
and asking for the relationship between $\vec{\Delta d}$ and $\vec{\Delta v}$.
Linearity gives $A\vec{\Delta v}=\vec{\Delta d}$ so we are
asking:~how much is $\vec{\Delta v}$ affected by $\vec{\Delta d}$, via
the mediation of the matrix~$A$?

Define the \textit{$2$-norm} of a vector to be 
its length,
  $\norm{\vec{v}}=\sqrt{v_1^2+\cdots+v_n^2}$
(there are many norms but this is the most common).
\begin{sagecommandline}
sage: v = vector(RDF, [1, 2, 3])
sage: v.norm(2)
\end{sagecommandline}
There is some input vector that $A$ stretches
the most, a vector $\vec{v}_{\text{max}}$
where the number $M=\norm{A\vec{v}_{\text{max}}}/\norm{\vec{v}_{\text{max}}}$ 
is maximal.
There is also a vector that $A$ shrinks the most, a
vector $\vec{v}_{\text{min}}$ 
where the number
$m=\norm{A\vec{v}_{\text{min}}}/\norm{\vec{v}_{\text{min}}}$ is a minimum 
(supposing that $\vec{v}_{\text{min}}\neq\zero$).\footnote{%
  We discuss these factors more in the next chapter.}
The condition number is the ratio of the two, 
$\kappa(A)=M/m$, except 
that because a singular matrix maps some nonzero vectors to zero and
so has $m=0$, in this case we set $\kappa(A)=\infty$.

The definition of~$M$ gives $\norm{\vec{d}}\leq M\norm{\vec{v}}$ and
the definition of~$m$ gives $\norm{\vec{\Delta d}}\geq m\norm{\vec{\Delta v}}$.
Thus the relative change in the 
system's computed solution
and the relative change in the system's right side
compare in this way. 
\begin{equation*}
  \frac{\norm{\vec{\Delta v}}}{\norm{\vec{v}}}
   \leq \kappa(A)\cdot \frac{\norm{\vec{\Delta d}}}{\norm{\vec{d}}}
\end{equation*}
That is, the condition number is the worst-case factor by which the
relative error is magnified. 
Changes in the right side can result in changes 
in the computed solution that are $\kappa(A)$ times as large.
\begin{sagecommandline}
sage: A = matrix(RDF, 3, [[1, 2, 3], [4, 5, 6], [7, 8, 9.001]])
sage: A.condition(2)
\end{sagecommandline}
Matrices that are nearly singular have a higher condition number than
those that are far from singular.
\begin{sagecommandline}
sage: v = []                                                                    
sage: for i in range(300): 
....:     A = random_matrix(RDF, 3, min=0, max=100) 
....:     v.append(A.condition(2))                                                                           
\end{sagecommandline}
Now we can get an idea of what is typical.
\begin{sagecommandline}
sage: mean(v), std(v)
\end{sagecommandline}
If the condition number is large then we say
that the system for the problem is `ill-conditioned',
otherwise we say it is `well-conditioned'.





\section{Running time}
Large linear algebra problems occur frequently in science and
engineering.
Since computers are fast and accurate,
they allow us to solve problems
that we could not hope to do by hand.

But even with computers, there are limits.
One limit on just how large a problem we can do is 
how quickly the machine can find the answer.
Naturally, problems with larger matrices tend to take longer to solve.
Comparing the size of a problem to the time that an algorithm takes to 
solve it is an important way to gauge the usefulness of 
that algorithm.

The matrix inverse operation is a good illustration.
% This is an important operation; for instance, if we could do large matrix 
% inverses
% quickly then as show above we could quickly solve large linear systems.
(The entries in these matrices are floating points 
because these are the most common in applications.)
% Timing gives bad output.  It has extra mu's in there.  I don't
% know how to fix it; I suspect it is a bug in sagemath.sty or listings.sty's
% inability to do non-ASCII.
% \begin{sagecommandline}
% sage: A = matrix(RDF, [[1, 3, 1], [2, 1, 0], [4, -1, 0]])
% sage: A
% sage: A.is_singular()
% sage: timeit('A.inverse()')
% \end{sagecommandline}
\begin{lstlisting}
sage: A = matrix(RDF, [[1, 3, 1], [2, 1, 0], [4, -1, 0]])
sage: A
[ 1.0  3.0  1.0]
[ 2.0  1.0  0.0]
[ 4.0 -1.0  0.0]
sage: A.is_singular()
False
sage: timeit('A.inverse()')
625 loops, best of 3: 62.7 μs per loop
\end{lstlisting}
\Sage's \inlinecode{timeit} tells you how long
an operation takes.
It does not report just the time that the CPU
spent but rather how long the operation took if you timed
it with a the clock on the wall.
It runs the command 
a number of times, to mitigate against the
computer being slowed down by a 
disk write or other interruption.
The bottom line says it did three batches of running the command 
$625$ times each and the average time in the fastest batch
was $62.7$~microseconds.\footnote{% from cat /proc/cpuinfo
  This machine 
  identifies as: Intel(R) Core(TM) i7-6820HQ CPU @ 2.70GHz.}
That's fast, but then $A$ is only $\nbyn{3}$.

And what's more, $A$ is a particular~$\nbyn{3}$ matrix. 
For an estimate of
how long it typically takes,
you could try finding the inverse of a random matrix.
% \begin{sagecommandline}
% sage: timeit('random_matrix(RDF, 3, min=-1, max=1).inverse()')
% sage: timeit('random_matrix(RDF, 3, min=-1, max=1).inverse()')
% sage: timeit('random_matrix(RDF, 3, min=-1, max=1).inverse()')
% \end{sagecommandline}
\begin{lstlisting}
sage: timeit('random_matrix(RDF, 3, min=-1, max=1).inverse()')
625 loops, best of 3: 86 μs per loop
sage: timeit('random_matrix(RDF, 3, min=-1, max=1).inverse()')
625 loops, best of 3: 85.8 μs per loop
sage: timeit('random_matrix(RDF, 3, min=-1, max=1).inverse()')
625 loops, best of 3: 86.6 μs per loop
\end{lstlisting}

One issue with this performance data is that
we can't tell
how much of the time is spent generating
the random matrix and how much is spent finding the inverse.
The code below takes some sizes, $\nbyn{3}$, $\nbyn{10}$, etc.,
finds a single random matrix and then 
gets the time to compute the inverse of that matrix.
\begin{lstlisting}
sage: for size in [3, 10, 25, 50, 75, 100, 150, 200]:
....:     print("size = "+str(size))
....:     M = random_matrix(RR, size, min=-1, max=1)
....:     timeit('M.inverse()')
....:     
size = 3
625 loops, best of 3: 55 μs per loop
size = 10
625 loops, best of 3: 548 μs per loop
size = 25
125 loops, best of 3: 6.79 ms per loop
size = 50
5 loops, best of 3: 53 ms per loop
size = 75
5 loops, best of 3: 177 ms per loop
size = 100
5 loops, best of 3: 419 ms per loop
size = 150
5 loops, best of 3: 1.41 s per loop
size = 200
5 loops, best of 3: 3.4 s per loop
\end{lstlisting}
Some of those times are in microseconds, some are in milliseconds, and some
are in seconds.
A microsecond is one-millionth of a second,
$0.000\,001$~seconds.
A millisecond is a thousandth of a second,
$0.001$~seconds.
This table is consistently in seconds.
\begin{center}
  \begin{tabular}{r|r@{.}l}
    \multicolumn{1}{r}{\textit{size}}     &\multicolumn{2}{c}{\textit{seconds}}  \\  \hline
    $3$      &$0$ &$000\,055$ \\
    $10$     &$0$ &$000\,548$ \\
    $25$     &$0$ &$006\,79$ \\
    $50$     &$0$ &$053$ \\
    $75$     &$0$ &$177$ \\
    $100$    &$0$ &$419$ \\
    $150$    &$1$ &$41$ \\
    $200$    &$3$ &$4$ 
  \end{tabular}
\end{center}
The time grows faster than the size.
For instance, doubling the size from~$25$ to~$50$ increases the time by
more than two: $0.053/0.00679$ is about $7.8$.
Similarly, increasing the size four-fold from $50$ to~$200$ causes the time to 
increase by much more than a factor of four: $3.4/0.053\approx 64.15$.
And going from a problem size of~$10$ to a size of~$100$ 
increases the time taken by a factor of $0.419/0.000\,548\approx 764.60$. 

Get a graph by giving \Sage{} the data as a list of pairs.\footnote{% 
  The graphics in this manual are generated using 
  more drawing options than appear in the output block.
  For instance, the scatter plot here came from
  \protect\inlinecode{g = scatter_plot(d, markersize=10, facecolor='#b9b9ff')}
  and was saved in a file with
  \protect\inlinecode{g.save("graphics/mat001.pdf", figsize=[2.25,1.5], axes_pad=0.05, fontsize=7)}.
  We shall omit much of this decoration code as clutter.}
\begin{sagecommandline}
sage: d = [(3, 0.000055), (10, 0.000548), (25, 0.00679),  
....:      (50, 0.053), (75, 0.177), (100, 0.419), 
....:      (150, 1.41), (200, 3.4)]
sage: g = scatter_plot(d)  
sage: g.save("graphics/mat001.pdf")            
\end{sagecommandline}
\begin{sagesilent}
g = scatter_plot(d, markersize=10, facecolor='#b9b9ff')
g.save("graphics/mat001.pdf", figsize=[2.25,1.5], axes_pad=0.05, fontsize=7)
\end{sagesilent}
(If you enter \inlinecode{scatter_plot(d)} at the prompt, that is, 
without saving it as~$g$, then \Sage{} will pop up a window with the
graphic.)
\begin{center}
  \includegraphics{graphics/mat001.pdf}
\end{center}
The graph dramatizes that the ratio $\text{time}/\text{size}$
is not constant
since the data clearly does not lie on a line.

Here is some more data.
A caution if you are trying this yourself:~to generate this data, 
\inlinecode{timeit}
took so long that
the computer had to be left to run overnight.
\begin{lstlisting}
sage: for size in [500, 750, 1000]:                             
....:         print "size=",size
....:     M = random_matrix(RR, size, min=-1, max=1)
....:     timeit('M.inverse()')
....: 
size= 500
5 loops, best of 3: 51.4 s per loop
size= 750
5 loops, best of 3: 172 s per loop
size= 1000
5 loops, best of 3: 406 s per loop
\end{lstlisting}
Again the table is a neater way to present the data.
\begin{center}
  \begin{tabular}{r|r@{.}l}
    \multicolumn{1}{r}{\textit{size}}     &\multicolumn{2}{c}{\textit{seconds}}  \\  \hline
    $500$       &$51$ &$4$ \\
    $750$       &$172$ &   \\
    $1000$      &$406$ &   
  \end{tabular}
\end{center}
Get a graph by tacking the new data onto the existing data.
\begin{sagecommandline}
sage: d = [(3, 0.000055), (10, 0.000548), (25, 0.00679),  
....:      (50, 0.053), (75, 0.177), (100, 0.419), 
....:      (150, 1.41), (200, 3.4)]
sage: d = d + [(500, 51.4), (750, 172), (1000, 406)]
sage: g = scatter_plot(d)                           
sage: g.save("graphics/mat002.pdf")                      
\end{sagecommandline}
\begin{sagesilent}
# d = [(3, 0.000055),    
#      (10, 0.000548), 
#      (25, 0.00679),  
#      (50, 0.053), 
#      (75, 0.177), 
#      (100, 0.419), 
#      (150, 1.41), 
#      (200, 3.4)]
# d = d + [(500, 51.4), (750, 172), (1000, 406)]
g = scatter_plot(d, markersize=10, facecolor='#b9b9ff')
g.save("graphics/mat002.pdf", figsize=[2.25,2.25], axes_pad=0.05, fontsize=7)              
\end{sagesilent}
% The result is this image.
\begin{center}
  \includegraphics{graphics/mat002.pdf}
\end{center}
(Note that the two graphs have different scales;
if the graph on this page had the same vertical scale as the one on the prior
page then it would extend far off the top of the paper.)
So a practical limit to the size of a problem that we can solve with
the matrix inverse operation comes from the fact that the graph above is
not a line\Dash
the time required grows much faster than the size.

A major effort in Computer Science is to find fast algorithms to 
do practical tasks.
Many people are working on tasks in Linear Algebra in particular,
such as finding the inverse of a matrix, because
they are so common in applications.

\endinput


TODO:
