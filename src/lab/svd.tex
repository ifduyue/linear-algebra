\chapter{Singular Value Decomposition}
\label{chap:SingularValueDecomposition}

We will picture how some linear transformations $\map{t}{\Re^2}{\Re^2}$ act
(we use $\Re^2$ simply because the pictures are 
easy to draw and to understand). 


\section{Drawing the action}
A defining property of linear maps is that 
$h(r\cdot\vec{v})=r\cdot h(\vec{v})$.
Recall that a line through the origin in $\Re^n$ has the form 
$\set{r\cdot \vec{v}\suchthat r\in\Re}$ for some~$\vec{v}\in\Re^n$. 
Thus the scalar multiplication property  
imposes a uniformity condition on a linear map:~its action on any 
line through the origin is determined by its action
on any one of the nonzero vectors in that line.

For instance, consider the line~$y=2x$ in the plane
\begin{equation*}
  \set{r\cdot\colvec{1 \\ 2}\suchthat r\in\Re}
\end{equation*}
and consider the transformation $\map{t}{\Re^2}{\Re^2}$ that is
represented by this matrix.
\begin{equation*}
  \rep{t}{\stdbasis_2,\stdbasis_2}
  =
  \begin{mat}
    1 &2 \\
    3 &4
  \end{mat}
\end{equation*}
If you pick a domain vector~$\vec{v}$ then computing the effect
of the map~$t$ is easy.
\begin{equation*}
  \vec{v}=\colvec{1 \\ 2}\mapsunder{t}\colvec{5 \\ 11}
\end{equation*}
The prior paragraph notes that 
on $2\vec{v}$ the 
map has twice the effect, on  $3\vec{v}$ it has three times the
effect, etc.
\begin{equation*}
  \colvec{2 \\ 4}\mapsunder{t}\colvec{10 \\ 22}
  \qquad
  \colvec{3 \\ 6}\mapsunder{t}\colvec{15 \\ 33}
  \qquad
  \colvec{r \\ 2r}\mapsunder{t}\colvec{5r \\ 11r}
\end{equation*}

%% The $t(r\cdot\vec{v})=r\cdot t(\vec{v})$ relationship says that
%% a linear map has the same effect on all vectors in a line through the
%% origin.
So, to show a map's action we will fix 
one nonzero vector~$\vec{v}$ from each line through the origin
and show where the transformation takes it.
The natural set containing one point from each line through the origin 
is the upper half of the unit circle, shown here
(the colors are explained below).
\begin{sagesilent}
load("plot_action.sage")
p = plot_circle_action(1,0,0,1) 
p.set_axes_range(-1.5, 1.5, -0.5, 1.5) 
p.save("graphics/svd000.pdf")
\end{sagesilent}
\begin{equation*}
  U=\set{\vec{v}=\colvec{\cos(t) \\ \sin(t)}
         \suchthat 
         0\leq t<\pi}
  \qquad
  \vcenteredhbox{\includegraphics{graphics/svd000.pdf}}  
\end{equation*}
The point $(1,0)$ is included in that set, as the representative of
the $x$~axis, but
$(-1,0)$ is not, which is why $(-1,0)$ has an open circle.

The first transformation that we will illustrate is simple.
\begin{equation*}
  \colvec{x \\ y} \mapsunder{t} \colvec{2x \\ y}
\end{equation*}
The code below loads \inlinecode{plot_action.sage}
to get access to
\inlinecode{plot_circle_action(a,b,c,d)}. 
This divides the input upper half circle into a number of parts,
the red one, the orange one, etc., and then it 
multiplies the points on those curved segments 
by this matrix
\begin{equation*}
  \begin{mat}
    a &b \\
    c &d
  \end{mat}
\end{equation*}
(for this first transformation we take $a=1$, $b=0$, $c=0$, and $d=2$).
Finally it connects those output segments to get an output curve, where 
the output segments are colored red, orange, etc.
The listing of the full source of \inlinecode{plot_action.sage} is 
at the end of this chapter.

To show before and after pictures, the code first plots the circle unchanged, that is,
under the effect of the identity map, and then plots the result of applying
the transformation.
\begin{sagecommandline}
sage: load("plot_action.sage")
sage: q = plot_circle_action(1,0,0,1) 
sage: q.set_axes_range(-2, 2, -1, 2) 
sage: q.save("graphics/svd001a.pdf")
sage: p = plot_circle_action(2,0,0,1) 
sage: p.set_axes_range(-2, 2, -1, 2) 
sage: p.save("graphics/svd001b.pdf")
\end{sagecommandline}
% \begin{sagesilent}
% load("plot_action.sage")
% q = plot_circle_action(1,0,0,1) 
% q.set_axes_range(-2, 2, -1, 2) 
% q.save("graphics/svd001a.pdf")
% p = plot_circle_action(2,0,0,1) 
% p.set_axes_range(-2, 2, -1, 2) 
% p.save("graphics/svd001b.pdf")
% \end{sagesilent}
Here are the before and after pictures.
The colors are there to show the association\Dash to show which 
points on the left are mapped to which points 
on the right.
\begin{equation*}
  \vcenteredhbox{\includegraphics{graphics/svd001a.pdf}}
  \quad\mapsunder{\big (\begin{smallmatrix} 2 &0 \\ 0 &1 \end{smallmatrix}\big )}\quad
  \vcenteredhbox{\includegraphics{graphics/svd001b.pdf}}
\end{equation*}
The effect of this map is to stretch things in the $x$~direction by a factor of two.

We next show
the action of the map
\begin{equation*}
  \colvec{x \\ y} \mapsto \colvec{-x \\ 3y}
\end{equation*}
that triples the $y$~component and also multiplies the 
$x$~component by $-1$.\footnote{%
  As in other chapters, some of the graphics are drawn using 
  some options that are not shown.
  In this case, the two \protect\inlinecode{p.save(...)} commands have
  the \protect\inlinecode{ticks\_integer=True} option.
  Not showing these options just reduces clutter that isn't linear algebra.}
\begin{sagecommandline}
sage: q = plot_circle_action(1,0,0,1) 
sage: q.set_axes_range(-2, 2, -1.25, 3.25) 
sage: q.save("graphics/svd002a.pdf")
sage: p = plot_circle_action(-1,0,0,3) 
sage: p.set_axes_range(-2, 2, -1.25, 3.25) 
sage: p.save("graphics/svd002b.pdf")
\end{sagecommandline}
\begin{sagesilent}
load("plot_action.sage")
q = plot_circle_action(1,0,0,1) 
q.set_axes_range(-2, 2, -1.25, 3.25) 
q.save("graphics/svd002a.pdf", ticks_integer=True)
p = plot_circle_action(-1,0,0,3) 
p.set_axes_range(-2, 2, -1.25, 3.25) 
p.save("graphics/svd002b.pdf", ticks_integer=True)
\end{sagesilent}
\begin{equation*}
  \vcenteredhbox{\includegraphics{graphics/svd002a.pdf}}
  \quad\mapsunder{\big (\begin{smallmatrix} -1 &0 \\ 0 &3 \end{smallmatrix}\big )}\quad
  \vcenteredhbox{\includegraphics{graphics/svd002b.pdf}}
\end{equation*}
Tripling the $y$ component is easy to understand.
As for taking the negative of the $x$ component, 
this is where the colors are especially useful.
On the input circle, when you move counterclockwise, the colors go from 
red to orange, to green, blue, indigo, and then violet.
But the output does the opposite:~moving counterclockwise it
passes from violet to red.
This transformation changes the orientation,
or `sense', of the curve. 

The next transformation is rotation by $\pi/4$~radians, clockwise.
\begin{equation*}
  \colvec{x \\ y} \mapsto \colvec{\cos(-\pi/4)x-\sin(-\pi/4)y \\ \sin(-\pi/4)x+\cos(-\pi/4)y}
\end{equation*}
\begin{sagecommandline}
sage: q = plot_circle_action(1,0,0,1) 
sage: q.set_axes_range(-1.5, 1.5, -1, 1.5) 
sage: q.save("graphics/svd003a.pdf")
sage: p = plot_circle_action(cos(-pi/4),-sin(-pi/4),sin(-pi/4),cos(-pi/4)) 
sage: p.set_axes_range(-1.5, 1.5, -1, 1.5) 
sage: p.save("graphics/svd003b.pdf")
\end{sagecommandline}
\begin{equation*}
  \vcenteredhbox{\includegraphics{graphics/svd003a.pdf}}
  \quad\mapsunder{\big (\begin{smallmatrix} \cos(-\pi/4) &-\sin(-\pi/4) \\ \sin(-\pi/4) &\cos(-\pi/4) \end{smallmatrix}\big )}\quad
  \vcenteredhbox{\includegraphics{graphics/svd003b.pdf}}
\end{equation*}

The next transformation is a shear.\footnote{%
  A shear is the motion of 
  two parts of a body where they slide   
  relative to each other in a direction parallel to their 
  plane of contact.  
  Thus, when you write with a pencil the flakes of
  graphite shear off on the paper.}
\begin{equation*}
  \colvec{x \\ y} \mapsto \colvec{x+2y \\ y}
\end{equation*}
Focus on the output's first component.
Because of the addition of $2y$, 
input vectors with larger second components, that is, input vectors with $y$'s of
large absolute value,
have their first output component affected more
than do input vectors near the $x$-axis.
\begin{sagecommandline}
sage: q = plot_circle_action(1,0,0,1) 
sage: q.set_axes_range(-1.5, 2.5, -.25, 1.25) 
sage: q.save("graphics/svd004a.pdf")
sage: p = plot_circle_action(1,2,0,1) 
sage: p.set_axes_range(-1.5, 2.5, -.25, 1.25) 
sage: p.save("graphics/svd004b.pdf")
\end{sagecommandline}
\begin{sagesilent}
load("plot_action.sage")
q = plot_circle_action(1,0,0,1) 
q.set_axes_range(-1.5, 2.5, -.25, 2.25) 
q.save("graphics/svd004a.pdf", ticks_integer=True)
p = plot_circle_action(1,2,0,1) 
p.set_axes_range(-1.5, 2.5, -.25, 2.25) 
p.save("graphics/svd004b.pdf", ticks_integer=True)
\end{sagesilent}
\begin{equation*}
  \vcenteredhbox{\includegraphics{graphics/svd004a.pdf}}
  \quad\mapsunder{\big (\begin{smallmatrix} 1 &2 \\ 0 &1 \end{smallmatrix}\big )}\quad
  \vcenteredhbox{\includegraphics{graphics/svd004b.pdf}}
\end{equation*}

Next is a shear
where it is the output's 
second component that is is affected,
here by the input's distance from 
the $x$-axis.
\begin{equation*}
  \colvec{x \\ y} \mapsto \colvec{x \\ (1/2)x+y}
\end{equation*}
\begin{sagecommandline}
sage: q = plot_circle_action(1,0,0,1) 
sage: q.set_axes_range(-1.25, 1.25, -0.75, 1.5) 
sage: q.save("graphics/svd005a.pdf")
sage: p = plot_circle_action(1,0,1/2,1) 
sage: p.set_axes_range(-1.25, 1.25, -0.75, 1.5) 
sage: p.save("graphics/svd005b.pdf")
\end{sagecommandline}
% \begin{sagesilent}
% load("plot_action.sage")
% q = plot_circle_action(1,0,0,1) 
% q.set_axes_range(-2, 2, -2, 2) 
% q.save("graphics/svd005a.pdf")
% p = plot_circle_action(1,0,1/2,1) 
% p.set_axes_range(-2, 2, -2, 2) 
% p.save("graphics/svd005b.pdf")
% \end{sagesilent}
\begin{equation*}
  \vcenteredhbox{\includegraphics{graphics/svd005a.pdf}}
  \quad\mapsunder{\big (\begin{smallmatrix} 1 &0 \\ 1/2 &1 \end{smallmatrix}\big )}\quad
  \vcenteredhbox{\includegraphics{graphics/svd005b.pdf}}
\end{equation*}

And here is a generic map.
\begin{sagecommandline}
sage: q = plot_circle_action(1,0,0,1) 
sage: q.set_axes_range(-2, 4, -3, 6) 
sage: q.save("graphics/svd006a.pdf")
sage: p = plot_circle_action(1,2,3,4) 
sage: p.set_axes_range(-2, 4, -3, 6) 
sage: p.save("graphics/svd006b.pdf")
\end{sagecommandline}
% \begin{sagesilent}
% load("plot_action.sage")
% q = plot_circle_action(1,0,0,1) 
% q.set_axes_range(-2, 4, -3, 6) 
% q.save("graphics/svd006a.pdf",figsize=3)
% p = plot_circle_action(1,2,3,4) 
% p.set_axes_range(-2, 4, -3, 6) 
% p.save("graphics/svd006b.pdf",figsize=3)
% \end{sagesilent}
\begin{equation*}
  \vcenteredhbox{\includegraphics{graphics/svd006a.pdf}}
  \quad\mapsunder{\big (\begin{smallmatrix} 1 &2 \\ 3 &4 \end{smallmatrix}\big )}\quad
  \vcenteredhbox{\includegraphics{graphics/svd006b.pdf}}
\end{equation*}
Besides rotating and shearing, it also changes orientation.



\section{SVD}
In those pictures the image of the half circle is a half ellipse, and
consequently the image of the full unit circle is an ellipse.
Recall that in $\Re^2$ an ellipse has a major axis, 
the longer one, and a minor axis.\footnote{%
  If the two axes have the same length, 
  then the ellipse is a circle.
  If one axis has length zero then the ellipse is a line segment 
  and if both have length zero then it is a point.}
Write $\sigma_1$ for the length of the semi-major axis, 
the distance from the center to the furthest-away point on the ellipse,
and write $\sigma_2$ for the length of the semi-minor axis.
\begin{sagecommandline}
sage: plot.options['axes_pad'] = 0.5
sage: plot.options['fontsize'] = 4
sage: plot.options['aspect_ratio'] = 1
sage: sigma_1=3
sage: sigma_2=1
sage: E = ellipse((0,0), sigma_1, sigma_2)
sage: E.save("graphics/svd100.pdf")
\end{sagecommandline}
\begin{sagesilent}
plot.options['axes_pad'] = 0.5
plot.options['fontsize'] = 2
plot.options['aspect_ratio'] = 1
plot.options['figsize'] = 1
sigma_1=3
sigma_2=1
E = ellipse((0,0), sigma_1, sigma_2)
# E.save("graphics/svd100.pdf", axes_pad=0.075, dpi=1200, fontsize=7, ticks=([-3,-2,-1,1,2,3],[-1,1]), figsize=1.5)
E.save("graphics/svd100.pdf", ticks_integer=True, axes_pad=0.075, figsize=3)
\end{sagesilent}
\begin{center}
  \includegraphics{graphics/svd100.pdf}
\end{center}
The two axes are orthogonal.
Above, the major axis is on the $x$-axis while the
minor is on the $y$-axis.

Under any linear map $\map{t}{\Re^n}{\Re^m}$, the 
unit sphere maps to a hyperellipse.
This is a version of the \textit{Singular Value Decomposition} of
matrices:~for
any linear map $\map{t}{\Re^n}{\Re^m}$, there are bases
$B=\sequence{\vec{\beta}_1,\ldots,\vec{\beta}_n}$ for the domain and
$D=\sequence{\vec{\delta}_1,\ldots,\vec{\delta}_m}$ for the codomain
such that $t(\vec{\beta}_i)=\sigma_i\vec{\delta}_i$, for $i\leq n$.
The scalars $\sigma_i$ are called \textit{singular values}.
The next section sketches a proof
but we first illustrate this result by using an example matrix.
\Sage{} will find the two bases $B$ and~$D$ and will picture how the 
vectors $\vec{\beta}_i$ 
are mapped to the $\sigma_i\vec{\delta}_i$.

So consider again the generic matrix.
Here is its action again, now shown
on a full circle.
\begin{sagecommandline}
sage: load("plot_action.sage")
sage: q = plot_circle_action(1,0,0,1,full_circle=True) 
sage: q.set_axes_range(-3, 3, -5, 5) 
sage: q.save("graphics/svd101a.pdf")
sage: p = plot_circle_action(1,2,3,4,full_circle=True) 
sage: p.set_axes_range(-3, 3, -5, 5) 
sage: p.save("graphics/svd101b.pdf")
\end{sagecommandline}
\begin{sagesilent}
load("plot_action.sage")
q = plot_circle_action(1,0,0,1,full_circle=True) 
q.set_axes_range(-3, 3, -5, 5) 
q.save("graphics/svd101a.pdf", ticks_integer=True)
p = plot_circle_action(1,2,3,4,full_circle=True) 
p.set_axes_range(-3, 3, -5, 5) 
p.save("graphics/svd101b.pdf", ticks_integer=True)
\end{sagesilent}
\begin{equation*}
  \vcenteredhbox{\includegraphics{graphics/svd101a.pdf}}
  \quad\mapsunder{\big (\begin{smallmatrix} 1 &2 \\ 3 &4 \end{smallmatrix}\big )}\quad
  \vcenteredhbox{\includegraphics{graphics/svd101b.pdf}}
  \tag{$*$}
\end{equation*}
\Sage{} will find the SVD of this example matrix.
\begin{sagecommandline}
sage: M = matrix(RDF, [[1, 2], [3, 4]])
sage: U,Sigma,V = M.SVD()
sage: U
sage: Sigma
sage: V
sage: U*Sigma*(V.transpose())
\end{sagecommandline}
\noindent 
The Singular Value Decomposition has $M$ as the product of
three matrices, $U\Sigma\trans{V}$.
The basis vectors $\vec{\beta}_1$, $\vec{\beta}_2$, $\vec{\delta}_1$, 
and~$\vec{\delta}_2$ are the columns of $U$ and~$V$. 
The singular values are the diagonal entries of~$\Sigma$.

We can get \Sage{} to plot the effect of the transformation
on the basis vectors for the domain so that we can compare those with the
basis vectors for the codomain.
\begin{sagecommandline}
sage: M = matrix(RDF, [[1, 2], [3, 4]])
sage: U,Sigma,V = M.SVD()
sage: beta_1 = vector(RDF, [U[0][0], U[1][0]])
sage: beta_2 = vector(RDF, [U[0][1], U[1][1]])
sage: delta_1 = vector(RDF, [V[0][0], V[1][0]])
sage: delta_2 = vector(RDF, [V[0][1], V[1][1]])
sage: C = circle((0,0), 1)
sage: P = C + plot(beta_1) + plot(beta_2)
sage: P.save("graphics/svd102a.pdf")
sage: image_color=Color(1,0.5,0.5)   # color for t(beta_1), t(beta_2)
sage: Q = C + plot(beta_1*M, width=3) 
sage: Q = Q + plot(delta_1,width=1.4) 
sage: Q = Q + plot(beta_2*M,width=3) 
sage: Q = Q + plot(delta_2,width=1.4)
sage: Q.save("graphics/svd102b.pdf")
\end{sagecommandline}
\begin{sagesilent}
M = matrix(RDF, [[1, 2], [3, 4]])
U,Sigma,V = M.SVD()
beta_1 = vector(RDF, [U[0][0], U[1][0]])
beta_2 = vector(RDF, [U[0][1], U[1][1]])
delta_1 = vector(RDF, [V[0][0], V[1][0]])
delta_2 = vector(RDF, [V[0][1], V[1][1]])
plot.options['axes_pad'] = 0.05
plot.options['fontsize'] = 7
plot.options['aspect_ratio'] = 1
s = 2  # uniform arrowsize
C = circle((0,0), 1)
P = C + plot(beta_1,arrowsize=s) + plot(beta_2,arrowsize=s)
P.save("graphics/svd102a.pdf",figsize=2, ticks_integer=True)
image_color=Color(1,0.5,0.5)   # color for t(beta_1), t(beta_2)
Q = C + plot(beta_1*M, width=3,color=image_color,arrowsize=s) 
Q = Q + plot(delta_1,width=1.4,color='blue',arrowsize=s) 
Q = Q + plot(beta_2*M,width=3,color=image_color,arrowsize=s) 
Q = Q + plot(delta_2,width=1.4,color='blue',arrowsize=s)
Q.save("graphics/svd102b.pdf",figsize=4.467, ticks_integer=True)
\end{sagesilent}
Below, on the left, the domain's basis are the blue $\vec{\beta}$'s.
On the right, the codomain's basis are the blue $\vec{\delta}$'s.
Also on the right, in light red, are the images of the $\vec{\beta}$'s.
The diagonal entries of $\Sigma$ describe the relationships:~the
red vector~$t(\vec{\beta}_1)$ is about $5.5$ times $\vec{\delta}_1$
while $t(\vec{\beta}_2)$ is about $0.4$ times~$\vec{\delta}_2$.
\begin{equation*}
  % \setlength{\fboxsep}{0in} % used to set box heights by eye; surrounded each picture env with a fbox
  \setlength{\unitlength}{1in}
  \begin{picture}(1.35,1.35)
    \put(0,1.825){\includegraphics{graphics/svd102a.pdf}}
    \put(.3,1.9){\scriptsize $\vec{\beta}_1$}
    \put(0,2.7){\scriptsize $\vec{\beta}_2$}
  \end{picture}
  \quad
  \raisebox{2.65in}{$\mapsunder{\big (\begin{smallmatrix} 1 &2 \\ 3 &4 \end{smallmatrix}\big )}$}
  \qquad
  \begin{picture}(2.45,3.15)
    \put(0,0){\includegraphics{graphics/svd102b.pdf}}
    \put(1.25,2){\scriptsize $\vec{\delta}_1$}
    \put(2.3,2.1){\scriptsize $\vec{\delta}_2$}
  \end{picture}
  \tag*{\raisebox{2.5in}{($**$)}}
\end{equation*}
The two bases are orthonormal,
comprised of unit vectors that are orthogonal.
Note also, by comparing with the diagram above labeled~($*$), that
long red vector is the
semi-major axis while its short red vector is the  
semi-minor axis.



\section{Proof sketch}

Consider an~$\nbyn{n}$ matrix~$T$ that is nonsingular.
Let $\map{t}{\Re^n}{\Re^n}$ be the
nonsingular transformation represented by
$T$ with respect to the standard bases.
We will argue that there are scalars $\sigma_0$, \ldots~$\sigma_n$
and bases
$B=\sequence{\vec{\beta}_1,\ldots,\vec{\beta}_n}$ and
$D=\sequence{\vec{\delta}_1,\ldots,\vec{\delta}_n}$ for the 
domain and codomain
such that $t(\vec{\beta}_i)=\sigma_i\vec{\delta}_i.$\footnote{%
  This argument, 
  from \protect\cite{BlankKrikorianSpring89},
  is a sketch because it uses results that many readers have 
  seen in a form not as strong as needed for the full proof, 
  and because it relies on material from the book that is optional.
  In addition, we'll consider only the case of a nonsingular matrix.
  This gives the main idea, which is the point of a sketch.}

Recall Calculus~I's Extreme Value Theorem: for a continuous
function~$\map{f}{\Re}{\Re}$, if a subset $D\subset \Re$ of the domain 
is closed and bounded then
its image $f(D)=\set{f(d)\suchthat d\in D}$ 
is also closed and bounded (see \cite{wiki:ExtremeValueThm}).
A generalization gives that because the unit sphere in $\Re^n$
is closed and bounded then its image under~$t$ is closed and bounded.
Although we won't prove this, the image is a hyperellipse
so we will call it that. 

Because this hyperellipse is closed and bounded, it has a 
point furthest from the origin (if there is more than one then just pick one).
Let $\vec{w}$ be the vector extending from the origin to that furthest point.
The nonsingular map~$t$ is invertible so there is one and only one
member of the unit sphere $\vec{v}$ such that $f(\vec{v})=\vec{w}$.
Let $P$ be the hyperplane tangent to the sphere at the endpoint of~$\vec{v}$.
Let $Q$ be the image of $P$ under~$t$.
Since $t$ is one-to-one, $Q$ intersects its ellipsoid only at~$\vec{w}$
(if it intersected in two places then the inverse image of those two would be
two places on~$P$ that intersect the sphere).
\begin{center}
  \includegraphics{asy/ellipsoid1.pdf}
\end{center}

Consider sliding~$P$ along the vector~$\vec{v}$ to the origin.
This makes clear that $P$ is the subspace of~$\Re^n$
consisting of vectors perpendicular
to~$\vec{v}$ (these elements of~$P$ have their
entire bodies in the hyperplane, 
unlike the pictured vector~$\vec{v}$, which just touches~$P$ with its endpoint).
This subspace has dimension~$n-1$. 
We will argue in the next paragraph 
that similarly $Q$ contains only
vectors perpendicular to~$\vec{w}$.
With that, we can complete an argument by induction:~we 
start constructing the 
bases~$B$ and~$D$ by taking $\vec{\beta}_1$ to be~$\vec{v}$, taking
$\sigma_1$ to be the length $\absval{\vec{w}}$, and taking
$\vec{\delta}_1$ to be $\vec{w}/\absval{\vec{w}}$.
Then induction proceeds by taking the restriction of $t$ to~$P$.

So consider~$Q$.
Since~$t$ is nonsingular, $Q$ is an $n-1$~dimensional subspace of~$\Re^n$,
a hyperplane.
The hyperplane that touches the ellipsoid
only at~$\vec{w}$ is unique since if there were another then its inverse image
under $t$
would be a second hyperplane, besides~$P$, 
that touches the sphere only at~$\vec{v}$, which is impossible (since it is a
sphere).
To see that $Q$ is perpendicular to~$\vec{w}$, consider a sphere in the codomain
centered at the origin whose radius is the length~$\absval{\vec{w}}$.
This sphere has a plane tangent at the endpoint of~$\vec{w}$ 
that is perpendicular
to $\vec{w}$.
Because $\vec{w}$ ends at a point on the ellipsoid furthest from the origin,
the ellipsoid is entirely contained in this sphere, so its tangent plane,
that is, the tangent plane of the ellisoid,
touches that ellipsoid only at~$\vec{w}$.
Therefore
this tangent plane is~$Q$. 
That ends the argument.



\section{Matrix factorization}

We can express those geometric ideas in an algebraic form
(for a proof see \cite{TrefethenBau97}).

The \textit{singular value decomposition} of an $\nbym{m}{n}$ matrix~$A$
is a factorization, $A=U\Sigma \trans{V}$\!.
The $\nbym{m}{n}$ matrix $\Sigma$ 
is all zeroes except for diagonal entries, the singular values, 
$\sigma_1\geq \sigma_2 \geq \cdots \geq \sigma_r> 0$ where $r$ is the
rank of~$A$.
The $\nbyn{m}$ matrix~$U$ and the $\nbyn{n}$ matrix~$V$ are unitary, meaning
that their columns form an orthogonal basis of unit vectors, called 
the left and 
right singular vectors for~$A$. 
\begin{sagecommandline}
sage: M = matrix(RDF, [[0, 1, 2], [3, 4, 5]])
sage: U,Sigma,V = M.SVD()
sage: U
sage: Sigma
sage: V  
\end{sagecommandline}
The number of singular values is the row rank of the matrix.
Here \Sage{} gets a $\sigma_2$ that is not quite \( 1 \) because of numerical 
issues. 
% \begin{sagecommandline}
% sage: M = matrix(RDF, [[0, 1, 2], [0, 2, 4]])
% sage: U,Sigma,V = M.SVD()
% sage: U
% sage: Sigma
% sage: V  
% \end{sagecommandline}

We can simplify the product $U\Sigma\trans{V}$.
Consider the case where all three matrices are $\nbyn{2}$.
%% Write $\vec{u}_1$, $\vec{u}_2$ for the columns of~$U$
%% and $\vec{v}_1$, $\vec{v_2}$ for the columns of~$V$,
%% so that the rows of $\trans{V}$ are $\trans{\vec{v}}_1$ and
%% $\trans{\vec{v}}_2$.
\begin{align*}
  \begin{mat}
      u_{1,1}  &u_{1,2}  \\
      u_{2,1}  &u_{2,2}  
  \end{mat}
  \begin{mat}
      \sigma_1  &0  \\
      0             &\sigma_2 
  \end{mat}
  \begin{mat}
      v_{1,1}  &v_{1,2}  \\
      v_{2,1}  &v_{2,2}  
  \end{mat}                    
  &=
  \begin{mat}
      u_{1,1}  &u_{1,2}  \\
      u_{2,1}  &u_{2,2}  
  \end{mat}
  \left[
    \begin{mat}
      \sigma_1  &0  \\
      0             &0 
    \end{mat}
    +
    \begin{mat}
      0  &0  \\
      0  &\sigma_2 
    \end{mat}
  \right]
  \begin{mat}
      v_{1,1}  &v_{1,2}  \\
      v_{2,1}  &v_{2,2}  
  \end{mat}                                      \\
  &=
  \sigma_1\begin{mat}
      u_{1,1}  &u_{1,2}  \\
      u_{2,1}  &u_{2,2}  
  \end{mat}
    \begin{mat}
      1  &0  \\
      0             &0 
    \end{mat}
  \begin{mat}
      v_{1,1}  &v_{1,2}  \\
      v_{2,1}  &v_{2,2}  
  \end{mat}                                       \\ 
  &\qquad+
  \sigma_2
  \begin{mat}
      u_{1,1}  &u_{1,2}  \\
      u_{2,1}  &u_{2,2}  
  \end{mat}
    \begin{mat}
      0  &0  \\
      0  &1 
    \end{mat}
  \begin{mat}
      v_{1,1}  &v_{1,2}  \\
      v_{2,1}  &v_{2,2}  
  \end{mat}                    
\end{align*}

%% \begin{align*}
%%   U\Sigma \trans{V}
%%   &=
%%   \begin{mat}
%%     \vec{u}_1 &\vec{u}_2
%%   \end{mat}
%%   \begin{mat}
%%     \sigma_1 &0 \\
%%     0        &\sigma_2
%%   \end{mat}
%%   \begin{mat}
%%     \trans{\vec{v}}_1 \\[.75ex]
%%     \trans{\vec{v}}_2
%%   \end{mat}             \\[.5ex]
%%   &=
%%   \begin{mat}
%%     \vec{u}_1 &\vec{u}_2
%%   \end{mat}
%%   \,[
%%   \begin{mat}
%%     \sigma_1 &0 \\
%%     0        &0
%%   \end{mat}
%%   +
%%   \begin{mat}
%%     0 &0 \\
%%     0        &\sigma_2
%%   \end{mat}
%%   ]\,
%%   \begin{mat}
%%     \trans{\vec{v}}_1 \\[.75ex]
%%     \trans{\vec{v}}_2
%%   \end{mat}             \\[.5ex]
%%   &=
%%   \begin{mat}
%%     \vec{u}_1 &\vec{u}_2
%%   \end{mat}
%%   \begin{mat}
%%     \sigma_1 &0 \\
%%     0        &0
%%   \end{mat}
%%   \begin{mat}
%%     \trans{\vec{v}}_1 \\[.75ex]
%%     \trans{\vec{v}}_2
%%   \end{mat}             
%%   +
%%   \begin{mat}
%%     \vec{u}_1 &\vec{u}_2
%%   \end{mat}
%%   \begin{mat}
%%     0        &0 \\
%%     0        &\sigma_2
%%   \end{mat}
%%   \begin{mat}
%%     \trans{\vec{v}}_1 \\[.75ex]
%%     \trans{\vec{v}}_2
%%   \end{mat}             \\[.5ex]
%%   &=
%%   \sigma_1\cdot
%%   \begin{mat}
%%     \vec{u}_1 &\vec{u}_2
%%   \end{mat}
%%   \begin{mat}
%%     1 &0 \\
%%     0 &0
%%   \end{mat}
%%   \begin{mat}
%%     \trans{\vec{v}}_1 \\[.75ex]
%%     \trans{\vec{v}}_2
%%   \end{mat}             
%%   +
%%   \sigma_2\cdot 
%%   \begin{mat}
%%     \vec{u}_1 &\vec{u}_2
%%   \end{mat}
%%   \begin{mat}
%%     0        &0 \\
%%     0        &1
%%   \end{mat}
%%   \begin{mat}
%%     \trans{\vec{v}}_1 \\[.75ex]
%%     \trans{\vec{v}}_2
%%   \end{mat}
%%   \tag{$*{*}*$}
%% \end{align*}
In the first term,
right multiplication by the $1,1$~unit matrix picks out the first column of
$U$, and left multiplication by the $1,1$~unit matrix picks out first row of
$V$ so those are the only parts that remain after the product.
In short, we get this.
\begin{equation*}
  \begin{mat}
    u_{1,1} &u_{1,2} \\
    u_{2,1} &u_{2,2}
  \end{mat}
  \begin{mat}
    1 &0 \\
    0 &0
  \end{mat}
  \begin{mat}
    v_{1,1} &v_{2,1} \\
    v_{1,2} &v_{2,2}
  \end{mat}
  =
  \begin{mat}
    u_{1,1}v_{1,1} &u_{1,1}v_{2,1} \\
    u_{2,1}v_{1,1} &u_{2,1}v_{2,1}
  \end{mat}
  =\colvec{u_{1,1} \\ u_{2,1}}\rowvec{v_{1,1} &v_{2,1}}
  =\vec{u}_1\trans{\vec{v}}_1
\end{equation*}
The second term simplifies in the same way.
\begin{equation*}
  \begin{mat}
    u_{1,1} &u_{1,2} \\
    u_{2,1} &u_{2,2}
  \end{mat}
  \begin{mat}
    0 &0 \\
    0 &1
  \end{mat}
  \begin{mat}
    v_{1,1} &v_{2,1} \\
    v_{1,2} &v_{2,2}
  \end{mat}
  =
  \begin{mat}
    u_{1,2}v_{1,2} &u_{1,2}v_{2,2} \\
    u_{2,2}v_{1,2} &u_{2,2}v_{2,2}
  \end{mat}
  =\vec{u}_2\trans{\vec{v}}_2
\end{equation*}
Thus, equation~($*{*}*$) simplifies to
$U\Sigma \trans{V}=\sigma_1\cdot\vec{u}_1\trans{\vec{v}}_1
   +\sigma_2\cdot\vec{u}_2\trans{\vec{v}}_2$.
Cases other than~$\nbyn{2}$ work the same way.



\section{Application: data compression}

Suppose that a matrix is $\nbyn{n}$.
To work with it\Dash for example, 
to store it in a computer's memory or to transmit it over the Internet\Dash 
we must work with $n^2$-many floating points.
For instance, if $n=500$ then it has $500^2=250\,000$~floats.
That's a lot of data.
We will show a way to reduce it. 

We've just seen how to write the matrix as a sum,
$M=\sigma_1\cdot\vec{u}_1\trans{\vec{v}}_1
   +\sigma_2\cdot\vec{u}_2\trans{\vec{v}}_2
   +\cdots\,$,
where the vectors have unit length and the scalars decrease, 
$\sigma_1\geq \sigma_2\geq \cdots\, 0$.
But that is not yet the savings, because
each term in that sum requires 
$n$~floating points for $\vec{u}_i$, another~$n$ for~$\vec{v}_i$, and one
more for $\sigma_i$.
So if $n=500$ that's $500\cdot(2\cdot 500+1)=500\,500$~floats,
about twice the size of the original matrix.

But the sum form does have an advantage, namely that the scalars decrease.
If we throw out a lot of the terms and keep only a few with the largest $\sigma$'s,
say the first $50$ of them, 
then we get a big savings:~$50\cdot (2\cdot 50+1)=5\,050$ floats,  
which is about $2\%$ of the storage size of the full matrix.

In short, if you have data as a matrix then you can hope to compress it
by converting to the summation formula and dropping terms with small $\sigma$'s.  
The question, though, is whether you lose too much information.

The answer is that you can retain a lot.
We will illustrate with image compression.
This is \textit{The Great Wave off Kanagawa} by Hokusai, ca~1830,
\cite{wiki:GreatWave}.\footnote{%
  The image file is
  generously made freely available by The Art Institute of Chicago.}
It depicts an rogue wave threatening three boats off 
what is today Yokohama, with Mount Fuji in the background. 
\begin{center}
  \includegraphics[width=.95\textwidth]{pix/greatwave.png} % 
\end{center}

The image compression code we will use is in \inlinecode{img_squeeze.sage}, 
listed at the end of the chapter.
It breaks the picture into three matrices, for the red data, the 
green data, and the blue data.
We want to see how badly the image degrades for various cutoffs.
The code below sets the cutoff at retaining only the terms with
the top $10$\% of singular values.
For this image, that is $92$ singular values.
The code gives us some insight into their size
by printing the ones for the red matrix
(but we've edited out most of them).
\begin{lstlisting}
sage: load("img_squeeze.sage")                                 
sage: img_squeeze("pix/greatwave.png", "pix/greatwave_squeezed.png", 0.10)
image has 1497 rows and 922 columns
sigma_RD 0 =192773.40
sigma_RD 1 =26258.04
sigma_RD 2 =20055.53
sigma_RD 3 =16881.59
sigma_RD 4 =13394.80
sigma_RD 5 =12875.31
sigma_RD 6 =11064.42
sigma_RD 7 =10250.95
    :
sigma_RD 92 =2240.98
\end{lstlisting}
(For contrast, the eight smallest singular values are
  $11.22$,
  $10.80$,
  $9.00$,
  $8.26$,
  $8.08$,
  $7.29$,
  $6.55$, and
  $5.88$.)
Below is the compressed result.
\begin{center}
  \includegraphics[width=.95\textwidth]{pix/greatwave_squeezed.png}
\end{center}
This picture shows loss.
The colors are not as good, the edges are not sharp, and there are 
artifacts, including vertical lines extending from the writing in the 
upper left.
Basically, Fiji has gotten a little fuzzy.
But certainly the image is entirely recognizable\Dash not bad for omiting
$90$~percent of the summands.

Every additional term in the summation
adds to the storage and transmission requirements by about $2n$ floats.
% The starting image requires $n^2$ floats so we save space is 
% choose a cutoff value less than $0.50$.
The above cutoff of $0.10$ needs 
only two percent of the storage
and transmission requirements of the original image's full matrix,
although its quality may not be acceptable.
That is, selecting a cutoff parameter is an engineering decision where
on a fidelity versus resource-consumption continuum 
you choose a value appropriate for your application.
 
Setting
the cutoff parameter to~$0.20$ can make the output image hard to tell
from the original.
Below is the picture of Suzy 
from this manual's cover.
On the left is the original image
and on the right it is squeezed using a parameter value of~$0.20$
(both are shown at $40$\% of their full size).
\begin{center}
  \vcenteredhbox{\includegraphics[width=.4\textwidth]{pix/suzy_cover.png}}
  \quad
  \vcenteredhbox{\includegraphics[width=.4\textwidth]{pix/suzy_cover_squeezed.png}}
\end{center}
(The top singular value is 
$291533.51$.
% $39038.59$,
% $30526.84$,
% $19983.01$,
% $13946.40$,
% $11523.07$,
% $9354.02$, and
% $8067.98$.
The twenty percent cutoff is $129.23$. 
The smallest of the $1739$ singular values is $3.36$.)
The picture on the left is $4$~megabytes, 
while the right is~$3.4$.


\section{Source of plot\_action.sage}

The \inlinecode{plot_circle_action}
routine takes the four entries of the $\nbyn{2}$
matrix and returns a list of two graphics, showing the input and the
output.
Other parameters are the number of colors and a flag giving whether
to plot a full circle or just the top half circle.

The driver routine computes the list of colors and 
calls a helper \inlinecode{color_circle_list}, 
given below, which returns a list of graphics.
Finally, the routine plots those graphics.
\lstinputlisting[firstline=39,lastline=50]{plot_action.sage}

The helper routine does the heavy lifting.
There are two globals.
The variable \inlinecode{CIRCLE_THICKNESS} sets the thickness of 
the plotted curve, in points (a printer's unit, $1/72$~inch).
Similarly \inlinecode{DOT_SIZE} sets the size of the small empty circle.
This routine produces a parametrized curve $(x(t),y(t))$ and uses \Sage's
\inlinecode{parametric_plot} function to get the resulting graphic.
\lstinputlisting[firstline=5,lastline=36]{plot_action.sage}
If this routine is plotting the upper half circle then it adds the
small empty circle at the end to show that the image of $(-1,0)$
is not part of the graph.




\section{Source of img\_squeeze.sage}
We use the Python Image Library for reading and writing the graphic.
\lstinputlisting[firstline=6,lastline=6]{img_squeeze.sage}
The function
\inlinecode{img_squeeze}
takes three arguments:~the names of the two files, and
the cutoff real number between $0$ and~$1$ that gives the percentage 
of the singular values to retain in the sum.
\lstinputlisting[firstline=8,lastline=12]{img_squeeze.sage}

This function first brings the input data to a format where each
pixel is a triple 
$(\text{red}, \text{green},\text{blue})$ of integers that range from 
$0$ to~$255$.
It uses those numbers to build 
three Python arrays \inlinecode{rd}, \inlinecode{gr},
and \inlinecode{bl}, which then initialize the 
three \Sage{} matrices \inlinecode{RD},
\inlinecode{GR}, and~\inlinecode{BL}.
\lstinputlisting[firstline=13,lastline=29]{img_squeeze.sage}

The next step finds the Singular Value Decomposition of those three.
Out of curiosity, we have a look at the eight largest singular
values in the red matrix, the singular value where we make the cutoff,
and the eight smallest.
\lstinputlisting[firstline=30,lastline=41]{img_squeeze.sage}

Finally, for each matrix we compute the sum
$\sigma_1\cdot\vec{u}_1\trans{\vec{v}}_1+\sigma_2\cdot\vec{u}_2\trans{\vec{v}}_2
   +\cdots\,$,
up through the cutoff index.
\lstinputlisting[firstline=42,lastline=61]{img_squeeze.sage}

To finish, we put the data in the PNG format and save it to 
disk.
\lstinputlisting[firstline=62,lastline=70]{img_squeeze.sage}
(This part of the routine takes a long time, a number of seconds, in part because
the code is intended to be easy to read rather than fast to run.)

\endinput


