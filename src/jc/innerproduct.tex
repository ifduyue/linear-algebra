% Chapter 5, Topic _Linear Algebra_ Jim Hefferon
%  https://hefferon.net/linearalgebra
%  2021-Jul-28
\topic{Inner Product}
\index{Inner Product|(}

The first chapter covers the dot product operation. 
It inputs a pair of vectors from some $\R^n$ and outputs a scalar.
\begin{equation*}
  \colvec{-1 \\ 2 \\ 3}\dotprod\colvec{1 \\ 0 \\ 3}
  =-1\cdot 1+2\cdot 0+3\cdot 3
  =8
\end{equation*}
In that chapter we did not yet have the term `linear', but
this operation both respects addition
\begin{equation*}
 (\vec{v}_1+\vec{v}_2)\dotprod\vec{w}=\vec{v}_1\dotprod\vec{w}
                                      +\vec{v}_2\dotprod\vec{w}
\qquad
 \vec{v}\dotprod(\vec{w}_1+\vec{w}_2)=\vec{v}\dotprod\vec{w}_1
                                      +\vec{v}\dotprod\vec{w}_2
\end{equation*}
and scalar multiplication.
\begin{equation*}
 (r\vec{v})\dotprod\vec{w}=r(\vec{v}\dotprod\vec{w})
  \qquad
 \vec{v}\dotprod(r\vec{w})=r(\vec{v}\dotprod\vec{w})
\end{equation*}
Dot product describes the geometry of
$\Re^n$; for instance, two vectors are perpendicular if and only if their
dot product is zero, and the length of a vector is determined by  
$\vec{v}\dotprod\vec{v}=\absval{\vec{v}}^2$.
We now extend this idea to complex vector spaces, with the
goal of understanding the transformations of complex spaces
that preserve length, $\absval{t(\vec{v})}=\absval{\vec{v}}$.

First we need more on complex numbers than we reviewed in this
chapter's opening section.
There, we described~$\C$ as 
the collection of $z=a+bi$ where $a$ and~$b$ are real numbers,
and where $i^2=-1$. 
We also covered addition, scalar multiplication,
subtraction, and multiplication, as well as length.
As to division,  consider $(1+2i)/(3+4i)$.
The difference of squares formula
gives $(a+bi)(a-bi)=a^2-(bi)^2=a^2-(-b^2)=a^2+b^2$,
which is a real number,
so
by multiplying the quotient's numerator and denominator by $3-4i$
we get an expression of the form $a+bi$.
\begin{equation*}
  \frac{1+2i}{3+4i}\cdot\frac{3-4i}{3-4i}
  =
  \frac{(1\cdot 3-2\cdot 4)+(1\cdot 4+2\cdot 3)i}{3^2+4^2}
  =
  \frac{-5+10i}{25}
  =
  \frac{-1}{5}+\frac{2}{5}\,i
\end{equation*}
Consequently, we define the \definend{conjugate} of a complex number~$z$
to be $\compconj{z}=\compconj{a+bi}=a-bi$, so that $z\compconj{z}=\absval{z}^2$.
Dividing both sides by $\absval{z}^2$ gives $z\compconj{z}/\absval{z}^2=1$
and so  
every nonzero complex number has a multiplicative inverse,
$z^{-1}=\compconj{z}/\absval{z}^2$.
% , and thus $\absval{z}=1$ if and only if $z^{-1}=\compconj{z}$.
Observe that a complex number is a real number if and only if 
it equals its conjugate, 
that $\compconj{z_1+z_2}=\compconj{z_1}+\compconj{z_2}$,
and that $\compconj{z_1\cdot z_2}=\compconj{z_1}\cdot\compconj{z_2}$.
Plotting $z=a+bi$ on the 
complex plane\index{complex plane} 
illustrates the conjugate.
\begin{center}
  \includegraphics{jc/asy/innerproduct000.pdf}
\end{center}
It also shows the \definend{argument}, $\arg(z)$, the angle that
$z$ makes with the real axis.

We will extend the dot product to a new operation.
Instead of a dot, the traditional mathematical notation is
$\innerprod{\vec{v}}{\vec{w}}$.

The obvious guess is to try defining $\innerprod{\vec{v}}{\vec{w}}$ 
as the linear combination of the components, so that in the two-dimensional 
case
\begin{equation*}
  \vec{v}=\colvec{a_{1,1}+b_{1,1}i \\ a_{2,1}+b_{2,1}i}
  \qquad
  \vec{w}=\colvec{a_{1,2}+b_{1,2}i \\ a_{2,2}+b_{2,2}i}
\end{equation*}
it would be 
$(a_{1,1}+b_{1,1}i)\cdot(a_{1,2}+b_{1,2}i)+(a_{2,1}+b_{2,1}i)\cdot(a_{2,2}+b_{2,2}i)$.
But that won't do.
We are investigating lengths,
so we want that $\innerprod{\vec{v}}{\vec{v}}=\absval{\vec{v}}^2\!$.
Thus, we don't want that $\innerprod{\vec{v}}{\vec{v}}$ equals
$(a_1+b_1i)(a_1+b_1i)+(a_2+b_2i)(a_2+b_2i)$,
instead we want something like
$(a_1+b_1i)(a_1-b_1i)+(a_2+b_2i)(a_2-b_2i)$.
Consequently, we want this in the two-dimensional case.
\begin{equation*}
  \innerprod{\vec{v}}{\vec{w}}
  =(a_{1,1}+b_{1,1}i)\cdot(a_{1,2}-b_{1,2}i)\,+\,(a_{2,1}+b_{2,1}i)\cdot(a_{2,2}-b_{2,2}i)
\end{equation*}
This is the 
\definend{Hermitian inner product}:\index{inner product!Hermitian}\index{Hermitian inner product}
for $\vec{v},\vec{w}$ with $i$-th components $v_i$ 
and $w_i$, define
$\innerprod{v}{w}=v_1\compconj{w}_1+\cdots+v_n\compconj{w}_n$.
Observe that this extends the dot product in that
if we fall back to reals, taking the $b$'s be zero, then
we get $\vec{v}\dotprod\vec{w}$.

Here is an example.
\begin{align*}
  \innerprod{\colvec{1+2i \\ 3+4i}}{\colvec{5+6i \\ 7+8i}}
  &=(1+2i)(5-6i)+(3+4i)(7-8i)                     \\[-2ex]
  &=(17+4i)+(53+4i)=70+8i
\end{align*}
Note that this operation inputs two vectors but outputs a scalar.

This operation is linear in the first input:~checking that
$\innerprod{\vec{v}_1+\vec{v}_2}{\vec{w}}
 = \innerprod{\vec{v}_1}{\vec{w}}
   +\innerprod{\vec{v}_2}{\vec{w}}$
and
$\innerprod{z\cdot\vec{v}}{\vec{w}}
     =z\cdot\innerprod{\vec{v}}{\vec{w}}$
is straightforward.
But it is not linear in the second because
while
$\innerprod{\vec{v}}{\vec{w}_1+\vec{w}_2}
 = \innerprod{\vec{v}}{\vec{w}_1}
   +\innerprod{\vec{v}}{\vec{w}_2}$
holds, for scalar multiplication a conjugation appears,
$\innerprod{\vec{v}}{z\cdot\vec{w}}
     =\compconj{z}\cdot\innerprod{\vec{v}}{\vec{w}}$.

Many different extensions of dot product are possible but we are aiming for
one that leads to a sensible generalization of the idea of length.
An \definend{inner product} is an operation
that inputs two vectors from a real or complex vector space 
and that outputs a scalar, and that
satisfies three conditions.
Condition~(1) is that it is linear in the first input, so that
$\innerprod{\vec{v}_1+\vec{v}_2}{\vec{w}}
 = \innerprod{\vec{v}_1}{\vec{w}}
   +\innerprod{\vec{v}_2}{\vec{w}}$
and $\innerprod{z\vec{v}}{\vec{w}}
     =z\innerprod{\vec{v}}{\vec{w}}$.
(Some presentations take inner product 
to be linear in the second input; more on that
at the end.)
Condition~(2) is that
$\innerprod{\vec{w}}{\vec{v}}=\compconj{\innerprod{\vec{v}}{\vec{w}}}$.
% (this makes the operation  \definend{conjugate symmetric}).
Observe that taking both vectors to be 
$\vec{v}$ gives that $\innerprod{\vec{v}}{\vec{v}}$ equals its conjugate, and
so is a real number.
Condition~(3) says about that real number that
$\innerprod{\vec{v}}{\vec{v}}\geq 0$ for all $\vec{v}$, with 
$\innerprod{\vec{v}}{\vec{v}}= 0$ if and only if $\vec{v}=\zero$.
% (this makes the operation \definend{positive definite}).

A vector space that has an inner product operation is an
\definend{inner product space}.
The dot product operation turns any of the real vector spaces $\R^n$ into
an inner product space. 
Hermitian inner product makes any of the complex vector spaces~$\C^n$ into 
an inner product space (verifications are in the exercises).

An example which looks quite different 
is the real vector space whose members are
real valued functions that are continuous on the 
closed interval $\closedinterval{0}{1}$.
Let the inner product be this.
\begin{equation*}
  \innerprod{f}{g}=\int_{0}^1 f(t)\cdot g(t)\,dt
\end{equation*}
This definition satisfies
condition~(1) because  
$\int_{0}^1 (f_1(t)+f_2(t))\cdot g(t)\,dt$ equals 
$\int_{0}^1 f_1(t)g(t)\,dt+\int_{0}^1 f_2(t)g(t)\,dt$
by a result from elementary Calculus,
and a similar result holds for scalar multiplication. 
It satisfies condition~(3) because for real numbers conjugation has no effect,
and $f(t)g(t)$ equals $g(t)f(t)$. 
For condition~(2), if~$f$ is not the zero function then 
the square $f(t)\cdot f(t)$ is greater than~$0$ for some~$\hat{t}$.
By continuity then, the square is greater than zero for a whole subinterval
around~$\hat{t}$,
and so the area under the curve is greater than~$0$.

What we want from the definition of inner product is 
a reasonable generalization of length.
Based on dot product, 
for the extension of length it is natural to use 
the  operation 
$\norm{\vec{v}}=\sqrt{\innerprod{\vec{v}}{\vec{v}}}$.
We will show that this operation satisfies 
three conditions which we expect of a kind of length. 
The first condition, that $\norm{\vec{v}}=0$ if and only if
$\vec{v}=\zero$, we have already seen above.
The second condition is that 
$\norm{z\vec{v}}=\absval{z}\cdot\norm{\vec{v}}$.
To verify, $\norm{z\vec{v}}=\sqrt{\innerprod{z\vec{v}}{z\vec{v}}}
=\sqrt{z\compconj{z}\cdot\innerprod{\vec{v}}{\vec{v}}}
=\absval{z}\cdot\sqrt{\innerprod{\vec{v}}{\vec{v}}}$.
The third condition is the triangle inequality, that 
$\norm{\vec{v}+\vec{w}}\leq\norm{\vec{v}}+\norm{\vec{w}}$.



Together these three conditions make this operation 
a \definend{norm}\index{norm} on the
vector space (the vector space may have real or complex scalars).

We pause to give more connection between inner products, norm, and geometry.
Here is the familiar parallelogram diagram for the sum of two vectors,
where besides showing $\vec{v}+\vec{w}$ we also have $\vec{v}-\vec{w}$. 
\begin{center}
  \includegraphics{jc/asy/innerproduct001.pdf}
\end{center}
% Applied to the two labeled angles, the Law of Cosines gives 
% $\absval{\vec{v}+\vec{w}}^2=\absval{\vec{v}}^2+\absval{\vec{w}}^2+2\absval{\vec{v}}\absval{\vec{w}}\cos(A)$
% and
% $\absval{\vec{v}-\vec{w}}^2=\absval{\vec{v}}^2+\absval{\vec{w}}^2+2\absval{\vec{v}}\absval{\vec{w}}\cos(B)$.
% Recall from geometry that in a parallelogram, adjacent angles are supplementary.
% This means that $\cos(B)=-\cos(A)$.
% Therefore, adding the two equations gives  
We will prove the 
\definend{parallelogram identity},\index{parallelogram identity}
$\absval{\vec{v}+\vec{w}}^2+\absval{\vec{v}-\vec{w}}^2=2\absval{\vec{v}}^2+2\absval{\vec{w}}^2$
(this has the nice statement that in a parallelogram the sum of the squares 
of the diagonals equals the sum of the squares of the sides).
Although the drawing above shows the plane, where we usually use dot product 
notation and write length as $\absval{\vec{v}}$, 
here we will use the notation for inner product and norm.

By definition,
$
  \norm{\vec{v}+\vec{w}}^2=\innerprod{\vec{v}+\vec{w}}{\vec{v}+\vec{w}}
$.
Expanding by using additivity of the first input and then the second gives 
\begin{align*}
  \norm{\vec{v}+\vec{w}}^2
  &=\innerprod{\vec{v}+\vec{w}}{\vec{v}+\vec{w}}           \\
  &=\innerprod{\vec{v}}{\vec{v}+\vec{w}}+\innerprod{\vec{w}}{\vec{v}+\vec{w}} \\
  &=\innerprod{\vec{v}}{\vec{v}}            
   +\innerprod{\vec{v}}{\vec{w}}
   +\innerprod{\vec{w}}{\vec{v}}
   +\innerprod{\vec{w}}{\vec{w}}
  =\norm{\vec{v}}^2            
   +\innerprod{\vec{v}}{\vec{w}}
   +\innerprod{\vec{w}}{\vec{v}}
   +\norm{\vec{w}}^2
\end{align*}
and same expansion for $\innerprod{\vec{v}-\vec{w}}{\vec{v}-\vec{w}}$
gives this.
\begin{equation*}
  \norm{\vec{v}-\vec{w}}^2
  =\innerprod{\vec{v}-\vec{w}}{\vec{v}-\vec{w}}
  =\norm{\vec{v}}^2
   -\innerprod{\vec{v}}{\vec{w}}
   -\innerprod{\vec{w}}{\vec{v}}
   +\norm{\vec{w}}^2
\end{equation*}
Adding the two gives the desired result. 






With the inner product and norm operations, 
even in spaces where there is no obvious geometry
we can interpret geometric ideas. 

We can take a \definend{metric} to be the operation
$d(\vec{v},\vec{w})=\norm{\vec{v}-\vec{w}}$.








For instance we can say that
vectors $\vec{v}$ and~$\vec{w}$ are orthogonal when 
$\innerprod{\vec{v}}{\vec{w}}=0$.
Similarly, we can extend the Gram-Schmidt process
to any inner product finite dimensional space.

That brings up back to our goal, to determine which transformations
of the complex vector space~$\C^n$ preserve length.
We make it an inner product space by using the Hermitian inner product.

Given a matrix~$M$, its \definend{matrix conjugate}~$\compconj{M}$ is 
the matrix whose entries are the conjugate of the entries of~$M$.
This operation is linear, that is, 
$\compconj{M+N}=\compconj{M}+\compconj{N}$ and 
$\compconj{z\cdot M}=\compconj{z}\cdot\compconj{M}$.
This operation also respects matrix multiplication, 
$\compconj{M\cdot N}=\compconj{M}\cdot\compconj{N}$.  

The \definend{conjugate transpose} or \definend{Hermitian transpose},
$\conjtrans{M}$
is the transpose of the conjugate.
We can either conjugate and then transpose, or transpose and 
then conjugate, so that
$\conjtrans{M}=\compconj{(\trans{M})}=\trans{(\compconj{M})}$.

A complex matrix~$U$ is \definend{unitary} if $U\cdot\conjtrans{U}=I$.
We will show that a transformation preserves length if and only if its
representation is unitary.

First some results:
(1)~the conjugate transpose operation is idempotent, 
$\conjtrans{(\conjtrans{M})}=M$,
(2)~it respects addition $\conjtrans{M+N}=\conjtrans{M}+\conjtrans{N}$,
and $\conjtrans(z\cdot M)=\compconj{z}\cdot\conjtrans{M}$,
(3)~$\conjtrans{M\cdot N}=\conjtrans{N}\cdot\conjtrans{M}$,
and the dot product of complex column vectors is 
$\innerprod{\vec{v}}{\vec{w}}$ equals the 
result of treating the two vectors as one-column matrices and 
multiplying $\conjtrans{\vec{w}\,}\cdot \vec{v}$.

The verifications are straightforward.
For (1), recall that we can take a 
conjugate transpose by doign conjugation and then transposition,
or we can do them in the other order. 
That gives this.
\begin{equation*}
\conjtrans{(\conjtrans{M})}
  =\compconj{\trans{\big(\,\compconj{\trans{M}}\,\big)\!}}
  =\compconj{\compconj{\trans{\trans{M}}}}
  =\compconj{\compconj{M}}
  =M
\end{equation*}
Number~(2) follows immediately from earlier work.
For (3), recall how transpose interacts with matrix multipication,
$\conjtrans{(MN)}
=\compconj{\trans{(MN)}}
=\compconj{\trans{N}\cdot\trans{M}}
=\compconj{\trans{N}}\cdot\compconj{\trans{M}}
=\conjtrans{N}\cdot\conjtrans{M}$. 
Item~(4) is a common source of mistakes.
\begin{equation*}
  \innerprod{\vec{v}}{\vec{w}}
  =\innerprod{\colvec{v_1 \\ \vdots \\ v_n}}{\colvec{w_1 \\ \vdots \\ w_n}}
  =v_1\compconj{w_1}+\cdots +v_n\compconj{w_n}
  =\rowvec{\compconj{w_1} &\cdots &\compconj{w_n}}\colvec{v_1 \\ \vdots \\ v_n}
\end{equation*}



\begin{exercises}
\item
Show that an attempted definition of $\innerprod{\vec{v}}{\vec{w}}$
that is linear in both inputs
loses that $\innerprod{\vec{v}}{\vec{v}}$ gives the length.
\begin{answer}
If
$\innerprod{\vec{v}}{\vec{v}}$ is greater than or equal to~$0$ for all
complex vectors,
then 
$0\leq \innerprod{i\vec{v}}{i\vec{v}}
  =i\innerprod{\vec{v}}{i\vec{v}}
  =i^2\innerprod{\vec{v}}{\vec{v}}\leq 0$.
\end{answer}

\item Show that linearity of $\innerprod{\vec{v}}{\vec{w}}$ in the first
input along with conjugate symmetry implies 
\definend{sesquilinearity}:
$\innerprod{\vec{v}}{z\vec{w}}=\compconj{z}\cdot \innerprod{\vec{v}}{\vec{w}}$.
\begin{answer}
$\innerprod{\vec{v}}{z\vec{w}}
=\compconj{\innerprod{z\vec{w}}{\vec{v}}}
=\compconj{z\cdot\innerprod{\vec{w}}{\vec{v}}}
=\compconj{z}\cdot\compconj{\innerprod{\vec{w}}{\vec{v}}}
=
\compconj{z}\cdot \innerprod{\vec{v}}{\vec{w}}$
\end{answer}

\item Prove the Pythagorean Theorem,  
that if $\vec{v}$ is orthogonal to~$\vec{w}$ then 
$\norm{\vec{v}+\vec{w}}^2=\norm{\vec{v}}^2+\norm{\vec{w}}^2$.
\begin{answer}
$
\norm{\vec{v}+\vec{w}}^2
=\innerprod{\vec{v}+\vec{w}}{\vec{v}+\vec{w}}
=\innerprod{\vec{v}}{\vec{v}}
  +\innerprod{\vec{v}}{\vec{w}}
  +\innerprod{\vec{w}}{\vec{v}}
  +\innerprod{\vec{w}}{\vec{w}}
=\norm{\vec{v}}
  +\zero
  +\zero
  +\norm{\vec{w}}
$
\end{answer}

\item
Show that Hermitian inner product on the two-dimensional vector space~$\C^2$ 
satisfies these three conditions in the definition of inner product. 


\end{exercises}
\index{Inner Product|)}
\endinput
% \end{document}