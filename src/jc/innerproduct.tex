% Chapter 5, Topic _Linear Algebra_ Jim Hefferon
%  https://hefferon.net/linearalgebra
%  2021-Jul-28
\topic{Inner Product}
\index{Inner Product|(}

The first chapter covers the dot product operation. 
It inputs a pair of vectors from some $\R^n$ and outputs a scalar.
\begin{equation*}
  \colvec{-1 \\ 2 \\ 3}\dotprod\colvec{1 \\ 0 \\ 3}
  =-1\cdot 1+2\cdot 0+3\cdot 3
  =8
\end{equation*}
In that chapter we had not yet defined the term `linear' but
this operation because it respects addition
\begin{equation*}
 (\vec{v}_1+\vec{v}_2)\dotprod\vec{w}=\vec{v}_1\dotprod\vec{w}
                                      +\vec{v}_2\dotprod\vec{w}
\qquad
 \vec{v}\dotprod(\vec{w}_1+\vec{w}_2)=\vec{v}\dotprod\vec{w}_1
                                      +\vec{v}\dotprod\vec{w}_2
\end{equation*}
and scalar multiplication.
\begin{equation*}
 (r\vec{v})\dotprod\vec{w}=r(\vec{v}\dotprod\vec{w})
  \qquad
 \vec{v}\dotprod(r\vec{w})=r(\vec{v}\dotprod\vec{w})
\end{equation*}
Dot product describes the geometry of
the spaces:
$  \vec{u}\dotprod\vec{v}=
  \absval{\vec{u}\,}\,\absval{\vec{w}\,}
  \cdot \arccos(\theta)$, and in particular, the length of a vector is 
determined by $\vec{v}\dotprod\vec{v}=\absval{\vec{v}}^2$.
This Topic extends the operation to complex numbers, with the
goal of understanding the transformations of complex spaces
that preserve length, $\absval{t(\vec{v})}=\absval{\vec{v}}$.

First we need more on complex numbers.
This chapter's opening section covered addition, scalar multiplication,
subtraction, and multiplication.
As to division, note first that the difference of squares formula
gives $(a+bi)(a-bi)=a^2-(bi)^2=a^2-(-b^2)=a^2+b^2$.
Then, for instance, resolve the division in $(1+2i)/(3+4i)$
by multiplying the numerator and denominator by $3-4i$.
\begin{equation*}
  \frac{1+2i}{3+4i}\cdot\frac{3-4i}{3-4i}
  =
  \frac{(1\cdot 3-2\cdot 4)+(1\cdot 4+2\cdot 3)i}{3^2+4^2}
  =
  \frac{-5+10i}{25}
  =
  \frac{-1}{5}+\frac{2}{5}\,i
\end{equation*}
Obviously this works as long as the denominator is not zero.
Because of the difference of squares formula,
we define the \definend{conjugate} of a complex number~$z$
to be $\compconj{z}=\compconj{a+bi}=a-bi$, so that $z\compconj{z}=\absval{z}^2$
(of course, the length is defined as $\absval{a+bi}=\sqrt{a^2+b^2}$).
Dividing both sides by $\absval{z}^2$ gives that  
every nonzero complex number has a multiplicative inverse,
$z^{-1}=\compconj{z}/\absval{z}^2$,
and thus $\absval{z}=1$ if and only if $z^{-1}=\compconj{z}$.
Observe that the complex number~$z$ is a real number if and only if 
it equals its conjugate, that $\compconj{z_1+z_2}=\compconj{z_1}+\compconj{z_2}$
and that $\compconj{z_1\cdot z_2}=\compconj{z_1}\cdot\compconj{z_2}$.
\begin{center}
  \includegraphics{jc/asy/innerproduct000.pdf}
\end{center}
As shown, the \definend{argument} $\arg(z)$ is the angle 
made with the real axis.

We will define an extension of the dot product, an
operation between vectors from~$\C^n$ called the \definend{inner product}.
Instead of using a dot, the traditional notation is \definend{inner product},
$\innerprod{\vec{v}}{\vec{w}}$.
That is, dot product is the special case of inner product for real spaces.

There is an obvious thing to try.
For instance, 
where
$\vec{v}$ and $\vec{w}$ each have two complex number components
$v_1,v_2,w_1,w_2\in\C$,
we might try  
$\innerprod{\vec{v}}{\vec{w}}=v_1\cdot w_1+v_2\cdot w_2$.
But this doesn't give us what we want.
We are investigating lengths,
so we need that $\innerprod{\vec{v}}{\vec{v}}=\absval{\vec{v}}^2$.
Thus, we don't want the combination.
$(a_1+b_1i)(a_1+b_1i)+(a_2+b_2i)(a_2+b_2i)$. 
Instead, we want
$(a_1+b_1i)(a_1-b_1i)+(a_2+b_2i)(a_2-b_2i)$.
Consequently, for $\vec{v},\vec{w}\in\C^n$, with $i$-th components $v_i$ 
and $w_i$, we define the \definend{Hermitian inner product}
as $\innerprod{v}{w}=v_1\compconj{w_1}+\cdots+v_n\compconj{w_n}$.

This operation inputs two vectors and outputs a scalar.
It is \definend{positive definite}, meaning that
$\innerprod{\vec{v}}{\vec{v}}> 0$ for all $\vec{v}\in\C^n$ such that
$\vec{v}\neq \zero$.
It is also linear in the first input,
$\innerprod{\vec{v}_1+\vec{v}_2}{\vec{w}}
 = \innerprod{\vec{v}_1}{\vec{w}}
   +\innerprod{\vec{v}_2}{\vec{w}}$
and $\innerprod{z\vec{v}}{\vec{w}}
     =z\innerprod{\vec{v}}{\vec{w}}$.
And, Hermitian inner product is \definend{conjugate symmetric},
$\innerprod{\vec{w}}{\vec{v}}$ is the conjugate of 
$\innerprod{\vec{v}}{\vec{w}}$.

Any operation satisfying those three conditions is an 
\definend{inner product}.
Any vector space with a defined inner product operation is an
\definend{inner product space}.
Those three conditions together imply two more very useful ones,
that $\innerprod{\vec{v}}{\vec{v}}\geq 0$ for all vectors,
and that $\innerprod{\vec{v}}{\vec{v}}=0$ if and only if $\vec{v}=\zero$.

Note that although Hermitian inner product is linear in the first input,
it is not linear in the second, 
it is \definend{conjugate linear}:
$\innerprod{\vec{v}}{\vec{w}_1+\vec{w}_2}
 = \innerprod{\vec{v}}{\vec{w}_1}
   +\innerprod{\vec{v}}{\vec{w}_2}$
and $\innerprod{\vec{v}}{z\vec{w}}
     =\compconj{z}\innerprod{\vec{v}}{\vec{w}}$.
Observe also that the Hermitian inner product extends the dot product in that
if we fall back to reals, taking the $b_1$'s to be zero, then
we get $\vec{v}\dotprod\vec{w}$.

Finally, we are interested in lengths, so consider the operation
$\innerprod{\vec{v}}{\vec{v}}=\norm{\vec{v}}$.
The exercises show that it satisfies that $\norm{\vec{v}}=0$ if and only if
$\vec{v}=\zero$,
that $\norm{z\vec{v}}=\absval{z}\cdot\norm{\vec{v}}$, and
the triangle inequality, that 
$\norm{\vec{v}+\vec{w}}\leq\norm{\vec{v}}+\norm{\vec{w}}$.
Together these three make this operation a \definend{norm} on the
vector space~$\C^n$.

With an inner product operation, and the associated norm, 
even in spaces where there is no obvious geometry we can 
interpret geometric ideas. 
For instance we can say that
vectors $\vec{v}$ and~$\vec{w}$ are orthogonal when 
$\innerprod{\vec{v}}{\vec{w}}=0$.
Similarly, the conditions on a norm allow us to think of it as a length.
Thus, we can extend the Gram-Schmidt process
to any inner product finite dimensional space.

Given a matrix~$M$, its \definend{matrix conjugate}~$\compconj{M}$ is 
the matrix whose entries are the conjugate of the entries of~$M$.
This operation is linear, that is, 
$\compconj{M+N}=\compconj{M}+\compconj{N}$ and 
$\compconj{z\cdot M}=\compconj{z}\cdot\compconj{M}$.
This operation also respects matrix multiplication, 
$\compconj{M\cdot N}=\compconj{M}\cdot\compconj{N}$.  

The \definend{conjugate transpose} or \definend{Hermitian transpose},
$\conjtrans{M}$
is the transpose of the conjugate.
We can either conjugate and then transpose, or transpose and 
then conjugate, so that
$\conjtrans{M}=\compconj{(\trans{M})}=\trans{(\compconj{M})}$.

A complex matrix~$U$ is \definend{unitary} if $U\cdot\conjtrans{U}=I$.
We will show that a transformation preserves length if and only if its
representation is unitary.

First some results:
(1)~the conjugate transpose operation is idempotent, 
$\conjtrans{(\conjtrans{M})}=M$,
(2)~it respects addition $conjtrans{M+N}=\conjtrans{M}+\conjtrans{N}$,
and $\conjtrans(z\cdot M)=\compconj{z}\cdot\conjtrans{M}$,
(3)~$\conjtrans{M\cdot N}=\conjtrans{N}\cdot\conjtrans{M}$,
and the dot product of complex column vectors is 
$\innerprod{\vec{v}}{\vec{w}}$ equals the 
result of treating the two vectors as one-column matrices and 
multiplying $\conjtrans{\vec{w}\,}\cdot \vec{v}$.

The verifications are straightforward.
For (1), recall that we can take a 
conjugate transpose by doign conjugation and then transposition,
or we can do them in the other order. 
That gives this.
\begin{equation*}
\conjtrans{(\conjtrans{M})}
  =\compconj{\trans{\big(\,\compconj{\trans{M}}\,\big)\!}}
  =\compconj{\compconj{\trans{\trans{M}}}}
  =\compconj{\compconj{M}}
  =M
\end{equation*}
Number~(2) follows immediately from earlier work.
For (3), recall how transpose interacts with matrix multipication,
$\conjtrans{(MN)}
=\compconj{\trans{(MN)}}
=\compconj{\trans{N}\cdot\trans{M}}
=\compconj{\trans{N}}\cdot\compconj{\trans{M}}
=\conjtrans{N}\cdot\conjtrans{M}$. 
Item~(4) is a common source of mistakes.
\begin{equation*}
  \innerprod{\vec{v}}{\vec{w}}
  =\innerprod{\colvec{v_1 \\ \vdots \\ v_n}}{\colvec{w_1 \\ \vdots \\ w_n}}
  =v_1\compconj{w_1}+\cdots +v_n\compconj{w_n}
  =\rowvec{\compconj{w_1} &\cdots &\compconj{w_n}}\colvec{v_1 \\ \vdots \\ v_n}
\end{equation*}



\begin{exercises}
\item
Show that an attempted definition of $\innerprod{\vec{v}}{\vec{w}}$
that is linear in both inputs
loses that $\innerprod{\vec{v}}{\vec{v}}$ gives the length.
\begin{answer}
If
$\innerprod{\vec{v}}{\vec{v}}$ is greater than or equal to~$0$ for all
complex vectors,
then 
$0\leq \innerprod{i\vec{v}}{i\vec{v}}
  =i\innerprod{\vec{v}}{i\vec{v}}
  =i^2\innerprod{\vec{v}}{\vec{v}}\leq 0$.
\end{answer}

\item Show that linearity of $\innerprod{\vec{v}}{\vec{w}}$ in the first
input along with conjugate symmetry implies 
\definend{sesquilinearity}:
$\innerprod{\vec{v}}{z\vec{w}}=\compconj{z}\cdot \innerprod{\vec{v}}{\vec{w}}$.
\begin{answer}
$\innerprod{\vec{v}}{z\vec{w}}
=\compconj{\innerprod{z\vec{w}}{\vec{v}}}
=\compconj{z\cdot\innerprod{\vec{w}}{\vec{v}}}
=\compconj{z}\cdot\compconj{\innerprod{\vec{w}}{\vec{v}}}
=
\compconj{z}\cdot \innerprod{\vec{v}}{\vec{w}}$
\end{answer}

\item Prove the Pythagorean Theorem,  
that if $\vec{v}$ is orthogonal to~$\vec{w}$ then 
$\norm{\vec{v}+\vec{w}}^2=\norm{\vec{v}}^2+\norm{\vec{w}}^2$.
\begin{answer}
$
\norm{\vec{v}+\vec{w}}^2
=\innerprod{\vec{v}+\vec{w}}{\vec{v}+\vec{w}}
=\innerprod{\vec{v}}{\vec{v}}
  +\innerprod{\vec{v}}{\vec{w}}
  +\innerprod{\vec{w}}{\vec{v}}
  +\innerprod{\vec{w}}{\vec{w}}
=\norm{\vec{v}}
  +\zero
  +\zero
  +\norm{\vec{w}}
$
\end{answer}



\end{exercises}
\index{Inner Product|)}
\endinput
% \end{document}