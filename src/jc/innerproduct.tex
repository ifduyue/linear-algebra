% Chapter 5, Topic _Linear Algebra_ Jim Hefferon
%  https://hefferon.net/linearalgebra
%  2021-Jul-28
\topic{Inner Products}
\index{Inner Products|(}

The first chapter covers the dot product operation. 
It inputs a pair of vectors from some $\R^n$ and outputs a scalar.
\begin{equation*}
  \colvec{-1 \\ 2 \\ 3}\dotprod\colvec{1 \\ 0 \\ 3}
  =-1\cdot 1+2\cdot 0+3\cdot 3
  =8
\end{equation*}
In that chapter we had not yet defined the term, but the
dot product operation is linear because it respects addition
\begin{equation*}
 (\vec{v}_1+\vec{v}_2)\dotprod\vec{w}=\vec{v}_1\dotprod\vec{w}
                                      +\vec{v}_2\dotprod\vec{w}
\qquad
 \vec{v}\dotprod(\vec{w}_1+\vec{w}_2)=\vec{v}\dotprod\vec{w}_1
                                      +\vec{v}\dotprod\vec{w}_2
\end{equation*}
and scalar multiplication.
\begin{equation*}
 (r\vec{v})\dotprod\vec{w}=r(\vec{v}\dotprod\vec{w})
  \qquad
 \vec{v}\dotprod(r\vec{w})=r(\vec{v}\dotprod\vec{w})
\end{equation*}
Dot product is that it describes the geometry of
the spaces:
$  \vec{u}\dotprod\vec{v}=
  \absval{\vec{u}\,}\,\absval{\vec{w}\,}
  \cdot \arccos(\theta)$, and in particular, the length of a vector is 
determined by $\vec{v}\dotprod\vec{v}=\absval{\vec{v}}^2$.
This Topic extends the operation to complex numbers, with the
goal of understanding the transformations of complex spaces
that preserve length, $\absval{t(\vec{v})}=\absval{\vec{v}}$.

First we need more on complex numbers.
This chapter's opening section covereded addition, scalar multiplication,
subtraction, and multiplication.
As to division, note first that the difference of squares formula
gives $(a+bi)(a-bi)=a^2-(bi)^2=a^2-(-b^2)=a^2+b^2$.
Then, for instance, resolve the division in $(1+2i)/(3+4i)$
by multiplying the numerator and denominator by $3-4i$.
\begin{equation*}
  \frac{1+2i}{3+4i}\cdot\frac{3-4i}{3-4i}
  =
  \frac{(1\cdot 3-2\cdot 4)+(1\cdot 4+2\cdot 3)i}{3^2+4^2}
  =
  \frac{-5+10i}{25}
  =
  \frac{-1}{5}+\frac{2}{5}\,i
\end{equation*}
The usefulness of the difference of squares formula
suggests the definition that the \definend{conjugate} of a complex number~$z$
is $\compconj{z}=\compconj{a+bi}=a-bi$, so that $z\compconj{z}=\absval{z}^2$.


Complex number, complex length, complex conjugation, complex arg.

Why the obvious extension won't work.

The length calculation suggests using the conjugation.




% \begin{exercises}
% \item Use the formula for the cosine of a sum to give an even more
%   general formula for simple harmonic motion.
%   \begin{answer}
%     The angle sum formula for the cosine function is
%     $\cos(\alpha+\beta)=cos(\alpha)\cos(\beta)-\sin(\alpha)\sin(\beta)$.
%     Expand $A\cos(\omega t+\phi)$ to $A\cdot[cos(\omega
%     t)\cos(\phi)-\sin(\omega t)\sin(\phi)]$.  Then $\cos(\phi)$ and
%     $\sin(\phi)$ do not vary with~$t$ so we get the general solution
%     $x(t)=B\cos(\omega t)+C\sin(\omega t)$.
%   \end{answer}
% \end{exercises}
\index{Inner Products|)}
\endinput
% \end{document}